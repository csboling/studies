\documentclass{letter}

\usepackage{amsfonts}

\begin{document}

The tangent space $T_p M$ is defined as the vector space of equivalence classes
of $\gamma : \mathbb{R} \to M$ with $\gamma(0) = p$
given by $\eta = [\gamma^\prime(0)]$, i.e., two curves
$\gamma, \tilde{\gamma}$ are equivalent if
$(\varphi \circ \gamma)^\prime(0) = (\varphi \circ \tilde{\gamma})^\prime(0)$
on a coordinate patch $\varphi : M \to \mathbb{R}^n$.

$$
T_{*}M \equiv
  \sum_{p : M}
    \frac{\prod_{\gamma : \mathbb{R} \to M} \gamma(0) =_M p}
         {\lambda (f, g : \mathbb{R} \to M) .
            (\varphi \circ f)^\prime(0) =_M (\varphi \circ g)^\prime(0)}
$$
Thus by a \emph{vector tangent to $M$ at $p$} we mean an equivalence class
of curves $\gamma : \mathbb{R} \to M$ such that $\gamma(0) = p$.
Use of the coproduct means that an element of the tangent bundle $T_{*}M$
is a pair $(p, V)$, tagging tangent vectors $V$ with the point $p \in M$
at which they are defined.


\end{document}
