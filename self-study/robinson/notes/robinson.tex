\documentclass{letter}

\usepackage{amsfonts}
\usepackage{amsmath}

\begin{document}
  \section*{Exercise 3.2}
  Using attachments for the real line
  $$
  \cdots \longleftarrow \{n\} \longrightarrow \{(n, n+1)\} \longleftarrow \{n+1\} \longrightarrow \cdots
  $$
  assign stalks
  $$
  \cdots \longleftarrow \mathbb{R}^3 \longrightarrow \mathbb{R}^2
         \longleftarrow \mathbb{R}^3 \longrightarrow \cdots
  $$
  so that
  $$
  \overbrace{\longleftarrow}^{(y, m_1, m_2)_n \mapsto \left(\frac{1}{2}(m_1)_n + y, (m_1)_n\right)}
  \mathbb{R}^{3}_n
  \overbrace{\longrightarrow}^{(y, m_1, m_2)_n \mapsto \left(\frac{1}{2}(m_2)_n + y, (m_2)_n\right)}
  \mathbb{R}^{2}_{(n, n+1)}
  \overbrace{\longleftarrow}^{(y, m_1, m_2)_{n+1} \mapsto \left(\frac{1}{2}(m_1)_{n+1} + y, (m_1)_{n+2}\right)}
  \mathbb{R}^{3}_{n+1}
  $$
  that is, each attachment map is either of the form
  $$
  \left[\begin{array}{c}
  y + \frac{1}{2}m_2 \\ m_2
  \end{array}\right] =
  \left[\begin{array}{c c c}
  1 & 0 & \frac{1}{2} \\
  0 & 0 & 1
  \end{array}\right]
  \left[\begin{array}{c}
  y \\ m_1 \\ m_2
  \end{array}\right]
  $$
  or
  $$
  \left[\begin{array}{c c c}
  1 & \frac{1}{2} & 0 \\
  0 & 0 & 1
  \end{array}\right]
  $$
  Then to go from
  $$
  (y, m_1, m_2)_n \mapsto (y, m_1, m_2)_{n+1}
  $$
  uses the map
  $\mathscr{S}(n+1 \rightsquigarrow (n, n+1))^{-1}_{(m_1)_n} \circ
     \mathscr{S}(n \rightsquigarrow (n, n+1))$
  where
  $$
  S(n+1 \rightsquigarrow (n,n+1))^{-1}_{(m_1)_n}
    :       (\frac{1}{2} (m_2)_{n+1} + y_{n+1}, (m_2)_{n+1})
    \mapsto (y, (m_1)_n, (m_2)_{n+1})
  $$

  \begin{defn}[Presheaf]
    Let $C$ be a category. A \emph{presheaf} on $C$ is a functor
    $C^{\mathrm{op}} \to V$, where $V$ is another category.

    Let $(X, \tau)$ be a topological space. The category with
    open sets of $\tau$ as objects and inclusions as morphisms
    is the category of open sets $\mathrm{Op}(X)$.
  \end{defn}

  A sheaf is a covariant functor from the face category of a simplicial
  complex to the category of vector spaces. Given a simplicial complex
  $X$ with $k$-skeletons $X_k$ and a sheaf $\mathscr{F}$, define
  $$
  C^k(X; \mathscr{F}) = \oplus_{a \in X_k} \mathscr{F}(a)
  $$
  and the coboundary maps
  $d^k : C^k(X; \mathscr{F}) \to C^{k+1}(X; \mathscr{F})$ by
  $$
  d^k(s(b)) = \sum_{a \in X_k} [b:a] \mathscr{F}(a \leadsto b) s(a)
  $$
  for any map $s : X_k \to C^k(X; \mathscr{F})$.

  A simplicial map $f : X \to Y$ can (?) be regarded as a functor
  between the face categories of two simplices, so that a
  sheaf morphism $\mathscr{F} \to \mathscr{G}$ between a sheaf
  $\mathscr{F}$ on $X$ and a sheaf $\mathscr{G}$ on $Y$ is a natural
  transformation  $\mathscr{G} \to \mathscr{F} \circ f$ between sheaf functors
  from the face category of $Y$ to the category of vector spaces.

% \section*{Exercise 3.4}
%                 (-1, 1, 1)
% (y, m_1, m_2)    -> (-y, m_1, m_2)
%    |                     |
%    v                     v
% (m_1/2 + y, m_1) -> (m_1/2 + y, m_1
%                  id
\end{document}
