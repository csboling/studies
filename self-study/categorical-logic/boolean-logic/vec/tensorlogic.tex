
\documentclass{article}

\usepackage{amsmath}
\usepackage{amsfonts}
\usepackage{mathrsfs}
\usepackage{tabularx}

\begin{document}
Let $V$ be an inner product space and let $\vec{\bot}, \vec{\top} \in V$ such that
$\|\vec{\bot}| = \|\vec{\top}\| = 1$ and 
$\langle \vec{\bot}, \vec{\top} \rangle = 0$. Construct a Boolean algebra 
$\mathrm{Bool}(V)$ over elements of $V$ by assigning $\forall \vec{x}, \vec{y} \in V$

\begin{tabular}{c | c}
$\mathrm{Bool}(V)$ & $V$ \\
\hline
$\bot$                   & $\vec{\bot}$ \\
$\top$                   & $\vec{\top}$ \\
$\neg \vec{x}$           & $\!\begin{aligned} 
                                \phantom{+} & \langle \vec{\top}, \vec{x} \rangle \vec{\bot}
                                            + \langle \vec{\bot}, \vec{x} \rangle \vec{\top}
                              \end{aligned}$ \\
$\vec{x} \wedge \vec{y}$ & $\!\begin{aligned}
                             & \langle \vec{\bot} \otimes \vec{\bot}, 
                                       \vec{x} \otimes \vec{y} 
                               \rangle \vec{\bot} \\
                           + & \langle \vec{\bot} \otimes \vec{\top}, 
                                       \vec{x} \otimes \vec{y} 
                               \rangle \vec{\bot} \\
                           + & \langle \vec{\top} \otimes \vec{\perp}, 
                                       \vec{x} \otimes \vec{y} 
                               \rangle \vec{\bot} \\
                           + & \langle \vec{\top} \otimes \vec{\top}, 
                                       \vec{x} \otimes \vec{y} 
                               \rangle \vec{\top}
                           \end{aligned}$ \\
\\
$\vec{x} \vee \vec{y}$   & $\!\begin{aligned}
                             & \langle \vec{\bot} \otimes \vec{\bot}, 
                                       \vec{x} \otimes \vec{y} 
                               \rangle \vec{\bot} \\
                           + & \langle \vec{\bot} \otimes \vec{\top}, 
                                        \vec{x} \otimes \vec{y} 
                               \rangle \vec{\top} \\
                           + & \langle \vec{\top} \otimes \vec{\bot},
                                       \vec{x} \otimes \vec{y} 
                               \rangle \vec{\top} \\
                           + & \langle \vec{\top} \otimes \vec{\top}, 
                                       \vec{x} \otimes \vec{y} 
                               \rangle \vec{\top}
                           \end{aligned}$
\end{tabular}

The terms of $\neg, \wedge, \vee$ can be read off of the truth table 
for the corresponding logic operator. The reason this works
is that
\begin{align*}
  \left\langle
    t_1 \otimes \cdots \otimes t_N,
    x_1 \otimes \cdots \otimes x_N
  \right\rangle
&= \prod_{i=1}^{N} \langle t_i, x_i \rangle,
\end{align*}
which is, in this case, $1$ when $x_i = t_i \forall i$ and
otherwise $0$. Then we can construct a function that takes
a truth table $V^{2^N}$ and $N$ arguments and produces the
appropriate result from the truth table by
\begin{align*}
\mathrm{fromTruth} : V^{2^N}$ \to V^{\otimes N} \to V & \\
\mathrm{fromTruth}((f_1, \cdots, f_{2^N}))(x_1 \otimes \cdots \otimes x_N)
  &= \sum_{n=1}^{2^N} f_n \prod_{i=1}^N \langle t_i^n , x_i \rangle \\
  &= \sum_{n=1}^{2^N} f_n \left(\bigotimes_{i=1}^N t_i  \right)
                        \left(\bigotimes_{i=1}^N G   \right)
                        \left(\bigotimes_{i=1}^N x_i \right), \\
\mathrm{fromTruth}((f_1, \cdots, f_{2^N})) &= \sum_{n=1}^{2^N} f_N \bigotimes_{i=1}^N t_i G
\end{align*}
where $t \in M_{N \times 2^{N}}(V)$ contains the $i$th bit of
the $N$-bit binary representation of $n$ in position $(i,n)$.

More generally, given a Boolean
algebra $B$, we examine the type of an $N$-ary operation
$f_N : B^N \to B$ to see

\begin{align*}
B^N \to B & \simeq (N \to B) \to B
\end{align*}
so
\begin{align*}
f : \prod_{n : \mathbb{N}}
      \left(\left(\sum_{i : \mathbb{Z}_n} 1\right) \to B\right)
      \to B
\end{align*}
and the function which takes a truth table and produces a
Boolean connective has type
\begin{align*}
\mathrm{fromTruth} : B^{2^N} \to B^N \to B
  & \simeq B^{2^N} \to (B^{B^N}) \\
  & \simeq (B^{B^N})^{B^{2^N}} \\
  & \simeq B^{B^N B^{2^N}} \\
  & \simeq B^{B^{N + 2^N}} \\
  & \simeq B^{N + 2^N} \to B \\
  & \simeq (N + 2^N \to B) \to B.
\end{align*}


\begin{align*}
F\left(x_1 \otimes \cdots \otimes x_N\right)
  &= \sum_{n} \left\langle
                T(n),
                x_1 \otimes \cdots \otimes x_N
              \right\rangle
              t^{\prime}(n)
\end{align*}
or in components
\begin{align*}
  \left(F\left(x_1 \otimes \cdots \otimes x_N\right)\right)_k
  &= \sum_{n} \left(T(n)\right)^{i_1 \cdots i_N}
              \left(\otimes^{N} G\right)_{i_1 \cdots i_{2N}}
              \left(x_1 \otimes \cdots \otimes x_{N}\right)^{i_{N+1} \cdots i_{2N}}
              \left(t^{\prime}(n)\right)^k e_k \\
  &= \sum_{n} \left(T(n) \otimes t^{\prime}(n)\right)_{i_{N+1} \cdots i_{2N}}^{k}
              e_k
              \left(x_1 \otimes \cdots \otimes x_{N}\right)^{i_{N+1} \cdots i_{2N}} \\
\end{align*}
so
\begin{align*}
  F &= \left[\sum_{n=1}^{2^{N} - 1} T(n) \otimes t^{\prime}(n)\right]
\end{align*}
where $T : \mathbb{Z}_{2^{N} - 1} \to V^{\otimes N}$ is given by
$$
T(n) = \left(\mathcal{V}^{\otimes N} \circ \mathrm{Bits}_N\right)(n)
$$
and $t^{\prime} : \mathbb{Z}_{2^{N} - 1} \to V$ is
$$
t^{\prime}(n) = \left(\mathcal{V} \circ f \circ \mathrm{Bits}_N\right)(n).
$$

Here $G$ is the metric tensor on $V^{\otimes N}$ and
$\mathcal{V} : \{\bot, \top\} \to V$ is the map
$\bot \mapsto \vec{\bot}, \top \mapsto \vec{\top}$.
$\mathrm{Bits}_N : \mathbb{Z}_{2^N - 1} \to B^{\otimes N}$
decomposes a number $n$ into a product of elements of $B$
which correspond to the binary representation of $n$.
That is, it is given by
$$
\mathrm{Bits}_N = \mathcal{B}_{\mathbb{Z}_{2}}^{\otimes N}
            \circ \left(\bigotimes_{i=1}^N \sigma_i\right)
$$
where $\mathcal{B}_{\mathbb{Z}_{2}} : \mathbb{Z}_{2} \to B$ is
given by $0 \mapsto \bot, 1 \mapsto \top$ and $\sigma_i$ form the
dual basis of the power basis $\mathscr{B}_2$ of
$\mathbb{Z}_{2^N - 1}$.

\end{document}
