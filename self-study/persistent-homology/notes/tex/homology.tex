\begin{defn}[Homotopy, homotopy type]
Let $X, Y$ be topological spaces, and let $f : X \to Y$ and
$g : X \to Y$. A \emph{homotopy} is a continuous function
$h : [0,1] \to (X \to Y)$ such that $h(0) = f$ and $h(1) = g$.

Two functions $f : X \to Y$ and $g : Y \to X$ are said to be
\emph{homotopy equivalent} or to possess the same \emph{homotopy type}
if $f \circ g$ is homotopic to $\mathrm{id}_Y$ and $g \circ f$ is
homotopic to
$\mathrm{id}_X$.
\end{defn}

\begin{defn}[Simplicial complex]
A finite set $K$ of simplices is called a \emph{simplicial complex} if
\begin{enumerate}
  \item{if $\sigma \in K$ and $\tau \leq \sigma$ then
        $\tau \in K$,}
  \item{for $\sigma, \tau \in K$, either $\sigma \cap \tau =
      \varnothing$ or $\sigma \cap \tau \leq \sigma$ and $\sigma \cap
      \tau \leq \tau$.
       }
\end{enumerate}
\end{defn}

\begin{defn}[Polyhedron, triangulable space]
The \emph{polyhedron} of a simplicial complex $K$ is the union of all
simplices in the complex, denoted $|K|$.

A topological space $X$ is
said to be \emph{triangulable} if there exists a simplicial complex
$K$ and a homeomorphism $f : |K| \to X$, in which case the pair
$(K, f)$ is called a \emph{triangulation} of $X$.
\end{defn}

\begin{defn}[Oriented simplex, $r$-chain group]
An \emph{oriented $r$-simplex} is an element
$(p_0 p_1 \cdots p_r)$ of the symmetric group
over the $0$-skeleton of a simplicial complex $K$ such that
$\{p_0, p_1, \dots, p_r\}$ is a simplex in $K$. The
\emph{$r$-chain group} $C_r(K)$ of a simplicial complex is a free
abelian group generated by the oriented $r$-simplices of $K$. Elements
of $C_r(K)$ are called \emph{$r$-chains}.
\end{defn}

\begin{defn}[Boundary, chain complex]
The boundary operator $\partial_r$ on an oriented $r$-simplex
$\sigma_r$ is given by
$$
\partial_r \sigma_r
= \sum_{i = 0}^r (-1)^i(p_0 \cdots \hat{p}_i \cdots p_r),
$$
where $\hat{p}_i$ is taken to mean that vertex $i$ is omitted from the
$i$-th term.

If $\partial_r c = 0$ for some $c \in C_r(K)$, $c$ is called an
$r$-cycle (not to be confused with

The \emph{chain complex} of $K$ is the sequence of
free abelian groups and homomorphisms
$$
0      \xrightarrow{i}
C_n(K) \xrightarrow{\partial_n}
\cdots
C_0(K) \xrightarrow{\partial_0}
0.
$$
We have also the \emph{$r$-cycle group}
$Z_r(K) \triangleq \ker \partial_r$ and the
\emph{$r$-boundary-group}
$B_r(K) \triangleq \mathrm{Im} \partial_{r+1}$. Note that
$B_r(K) < Z_r(K)$.
\end{defn}

\begin{lemma}
$\partial_r \circ \partial_{r+1}$ is the zero map.
\end{lemma}
\begin{proof}
\begin{align*}
(\partial_r \circ \partial_{r+1})(c)
&= \partial_r \left(
     \partial_{r+1}
       (p_0 \cdots p_r)
   \right) \\
&=
\end{align*}
\end{proof}

\begin{defn}[Homology Groups]
Let $K$ be an $n$-dimensional simplicial complex. The
\emph{$r$th homology group} is
$$
H_r(K) \triangleq Z_r(K) / B_r(K).
$$
(For this definition we need also to show that
 $B_r(K) \triangleleft Z_r(K)$.) Two $r$-cycles are said to be
 \emph{homologous} if they belong to the same \emph{homology class},
 i.e. the same coset in this quotient group.
\end{defn}

\begin{theorem}
If topological spaces $X$ and $Y$ are homeomorphic, and if $(K, f)$
triangulates $X$ and $(L, g)$ triangulates $Y$, then
$H_r(K) \simeq H_r(L)$ $\forall r$.
\end{theorem}
