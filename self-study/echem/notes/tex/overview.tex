\section{Mass Transfer}

A common assumption or use case is that all reactions are faster
than mass transfer. For instance, under the conditions:
\begin{enumerate}
  \item{
    The heterogeneous reactions (those at the electrode surface)
    are dominated charge-transfer (fast)
  }
  \item{
    The homogeneous reactions (which are the same throughout the
    bulk solution) are mobile and reversible
  }
\end{enumerate}
it follows that
\begin{enumerate}
  \item{
    Homogeneous reactions are approximately at equilibrium
  }
  \item{
    Surface concentrations of electrochemically active redox
    reactants have a Nernstian relationship with the electrode potential
  }
  \item{
    The reaction rate is limited by the mass transfer, i.e.
    $$
    v_{mt} \approx v_{rxn} = \frac{i}{nFA}
    $$
    where $v_{rxn}$ is the net reaction rate, $i$ is the electrode current,
    $n$ is the net number of electrons exchanged, $F$ is the number of
    Coulombs of charge in one mole of electrons, and $A$ is the reactive
    surface area of the electrode.
  }
\end{enumerate}

This is the \emph{reversible} case, and in the one-dimensional case the
flux $J_i$ of a species $i$ at a distance $x$ from the electrode is given by
the Nernst-Planck equation
\begin{align*}
J_i(x) &= \text{diffusion} + \text{migration} + \text{convection} \\
       &= -D_i \frac{\partial C_i}{\partial x}
          -\frac{z_i F}{RT} D_i C_i \frac{\partial \phi}{\partial x}
          +C_i v(x),
\end{align*}
where $D_i$ is the diffusion coefficient, $z_i$ is the charge, $C_i$ is
the concentration, and $v(x)$ is the velocity of a volume element.
Note that $n F A J = i$, since the flux describes the flow rate of
ions and each has a charge, so this corresponds to a current density.

Assume for instance that there is an excess of supporting electrolyte
(eliminating the diffusion term), there is no vibration or agitation in
the solution (eliminating the convection term), and the concentration
gradient of a species $O$ is linear. Then
$$
v_{mt} = \frac{D_O}{\delta_O}(C_O^\ast - C_O(0))
$$
where $C_O^\ast$ is the bulk concentration so that
$$
\frac{i}{nFA} = \frac{D_O}{\delta_O}(C_O^\ast - C_O(0))
$$
for a net anodic reaction (converting anions into oxidants O) and
$$
\frac{i}{nFA} = \frac{D_R}{\delta_R}(C_R(0) - C_O^\ast)
$$
for a net cathodic reaction (converting cations into reductants R).

When the concentration at the electrode is zero (i.e. all
faradaic reactants have been consumed) the current has a maximum
value, the limiting current
$$
i_l = n F A m_O C_O^\ast,
$$
which is rate-limited by mass transport.
