\documentclass[10pt]{article}

\usepackage[a4paper,margin=1cm]{geometry}
\usepackage{amsmath,amsfonts,amssymb}
\newcommand\dif{\mathop{}\!\mathrm{d}}

\pagestyle{empty}
\setlength{\parindent}{0pt}

\begin{document}
The radius of a power series may be found by
$$
  \frac{1}{\lim |a_n|^{1/n}}
= \lim \frac{|a_n|}{|a_{n+1}|}
= \lim \frac{1}{\limsup |a_n|^{1 / n}}.
$$

The limit of the geometric series is
$$
\sum_{n=0}^\infty z^n = \frac{1}{1 - z}, \quad |z| < 1
$$
with partial sum
$$
\sum_{n=0}^{N-1} z^n = \frac{1 - z^N}{1 - z}.
$$
Useful for estimating path integrals if $f$ is continuous:
$$
\left|
  \int_\gamma f
\right|
\leq
\| f \|_\gamma L(\gamma)
$$

The Cauchy theorem: Let $f$ be holomorphic on a domain $U$
and $\gamma$ be a contour in $U$ such that $W(\gamma, z)$
for all $z \in \mathbb{C} \backslash U$. Then
$\int_\gamma f = 0$.

The Cauchy theorem implies that
for $f$ holomorphic on $\mathrm{Int}(J) \cup J$,
$$
  f(w)
= \frac{1}{2 \pi i}
  \int_J
    \frac{f(z)}{z - w}
    \dif z.
$$
This hinges on the fact that
$$
\int_{|z - w| = r}
  \frac{f(z)}
       {z - w}
  \dif z
= 2 \pi i f(w)
$$
while more generally
$$
\frac{1}{2 \pi i}
\int_\gamma
  \frac{f(z)}{z - z_0}
  \dif z
= W(\gamma, z_0) f(z_0).
$$

Observing
$$
  \frac{f(z)}{z - w}
= \frac{f(z)}{z - w_0}
  \sum_{n=0}^\infty
    \frac{(w - w_0)^n}
         {(z - w_0)^n}
$$
by a geometric expansion gives
$$
  f(w)
= \sum_{n=0}^\infty
    (w - w_0)^n
    \frac{1}{2 \pi i}
    \int_J
      \frac{f(z)}{(z - w_0)^{n+1}}
      \dif z, \quad
  |w - w_0| < \mathrm{dist}(w_0, J)
$$
for any $w \in \mathrm{Int}(J)$
so that
$$
  f^{(n)}(w_0)
= \frac{n!}{2 \pi i}
  \int_J
    \frac{f(z)}{(z - w_0)^{n+1}}
    \dif z.
$$

The mean value theorem for harmonic functions
(and thus for holomorphic functions) is
$$
  u(z_0)
= \frac{1}{2 \pi}
  \int_0^{2 \pi}
    u(z_0 + re^{i\theta})
    \dif \theta
= \frac{1}{\pi r^2}
  \int_{D(z_0, r)}
    u(z)
    dx dy
$$

The winding number is given by
$$
  W(\gamma, z_0)
= \frac{1}{2 \pi i}
  \int_\gamma
    \frac{1}{z - z_0}
    \dif z
$$
and $2\pi W(\gamma, \alpha)$ is equal to the increment
of $\arg (z - \alpha)$ along $\gamma$. Crossing the curve
from right to left increments the winding number.

\end{document}
