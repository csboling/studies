\documentclass{article}

\usepackage{amsmath}
\usepackage{amsfonts}
\usepackage{mathrsfs}
\usepackage{amssymb}
\usepackage{amsthm}

\usepackage{enumerate}
\usepackage{hyperref}

\newcommand\dif{\mathop{}\!\mathrm{d}}
\renewcommand{\Im}{\mathrm{Im}}
\renewcommand{\Re}{\mathrm{Re}}
\newcommand{\ord}{\mathrm{ord}}
\newcommand{\id}{\mathrm{id}}
\newcommand{\Res}{\mathrm{Res}}
\renewcommand{\Im}{\mathrm{Im}}

\begin{document}

\section{Basic computations with complex numbers}

\subsection{Arithmetic}

\subsection{Triangle Inequality}
Useful inequalities for a metric
$\| x - y \|$ include
\begin{align*}
      \| x + y \|
&\leq \| x \| + \| y \| \\
      \| x - y \|
&\geq \| x \| - \| y \|
\end{align*}
For the complex modulus specifically,
\begin{align*}
      | z |
&\leq \Re~z, \Im~z
\end{align*}

\subsection{Polar and rectangular form}

We have
\begin{align*}
   \frac{r e^{i\theta}}
        {s e^{i\varphi}}
&= \frac{r}{s} e^{i(\theta - \varphi)}, \\
   \frac{x + iy}
        {u + iv}
&= (x + iy)
   \frac{u - iv}
        {u^2 + v^2}
 = \frac{ux - ixv + iyu + yv}
        {u^2 + v^2} \\
&= \frac{(xu + yv) - i(xv + yu)}
        {u^2 + v^2}
\end{align*}
and
\begin{align*}
   r e^{i \theta} + s e^{i \varphi}
&= (r \cos \theta
 +  s \cos \varphi)
 + (ir \sin \theta
 +  is \sin \theta).
\end{align*}
In particular
\begin{align*}
   \frac{1 + i}
        {1 - i}
&= \frac{\sqrt{2} e^{i \frac{\pi}{4}}}
        {\sqrt{2} e^{-i\frac{\pi}{4}}
 = e^{i \frac{\pi}{2}} = i \\
&= h_i(1),
\end{align*}
whereas $h_i(i) = \infty$, $h_i(0) = -1$.

\subsection{Powers and $n$-th roots (polar form)}
A complex number
$$
z = r e^{i (\theta + 2 \pi k)}
$$
has $n$-th roots
$$
  z^{\frac{1}{n}}
= \sqrt[n]{r} e^{\frac{\theta}{n} + 2 \pi \frac{k}{n}}
$$
and $n$-th power
$$
  z^n
= r^n (\cos n\theta + i \sin n\theta).
$$
Note that
$$
\frac{\theta + 2 \pi k}{n} = 2 \pi m
$$
when
$$
\theta = 2 \pi (mn - k)
$$
for some $k, m \in \mathbb{Z}$.
We can also write
$$
  z^{\frac{1}{n}}
= e^{\frac{1}{n} \log z}
$$
so that this function is dependent on a choice of branch cut
of $\log z$.

\subsection{Complex exponential, logarithm,
                  trigonometric and hyperbolic functions}

We can define
\begin{align*}
   \cosh z
&= \frac{e^z + e^{-z}}{2}
 = \frac{e^{-i(iz)} + e^{i(iz)}}{2} \\
&= \cos iz, \\
   \sinh z
&= \frac{e^z - e^{-z}}{2}
 = \frac{e^{-i(iz)} - e^{i(iz)}}{2} \\
&= -i \sin iz.
\end{align*}

\section{Topology on $\mathbb{C}$}

The topology on $\mathbb{C}$ is generated by open disks.
A closed set $\bar{U}$ is the complement of an open set
and satisfies
\begin{enumerate}
  \item{
    that it contains its boundary and contains all its
    limit points, i.e. it is closed under the limit operation
  }
  \item{
    that every bounded sequence in $\bar{U}$ contains a
    convergent subsequence
  }
\end{enumerate}

A subset $A \subset U$ is said to be relatively open in $U$ if
for each $z_0 \in A$, there exists an $r$ such that
$D(z_0, r) \cap U$ lies in $U$. Informally, each point in $A$ has an
open neighborhood also in $A$, which may be truncated by the boundary
of $U$.
A subset $A \subset U$ is said to be relatively closed in $U$
if $U - A$ is relatively open in $U$.

Continuous functions on compact sets are uniformly continuous and have
compact images. The uniform limit of continuous functions is continuous.

\section{Complex differentiability, Cauchy-Riemann equations}

Let $U \subset \mathbb{C}$ be open. We say that $f$ is
holomorphic on $U$ if, for each $z_0 \in U$, the limit
$$
  \lim_{z \to z_0}
    \frac{f(z) - f(z_0)}
         {z - z_0}
$$
exists. This is equivalent to total differentiability of
$f(x + i y) = u(x, y) + i v(x, y)$, which is satisfied if and
only if
$$
  \left[
    \begin{array}{r r}
       \frac{\partial}{\partial x}
    & -\frac{\partial}{\partial y} \\
       \frac{\partial}{\partial x}
    &  \frac{\partial}{\partial y}
    \end{array}
  \right]
  \left[
    \begin{array}{c}
      u \\
      v
    \end{array}
  \right]
= \mathbf{0}.
$$

We can also write
$$
f^\prime = u_x + i v_x = v_y - i u_y.
$$

\subsection{Branches of the logarithm (primitive of $\frac{1}{z}$),
                  complex powers}

A branch of the logarithm in an open set
$U \subset \mathbb{C} \setminus \{ 0 \}$ is defined to be
a continuous function $L$ that satisfies
$\exp \circ L = \mathrm{id}$ on $U$.

A branch of the logarithm with $\theta \in (0, 2 \pi)$ is
given by
$$
  \log z
= \log |z| + i \arg z, \quad
  \arg z \in (0, 2 \pi)
$$
but we may also choose a branch cut along any half-line
$$
  t e^{i \theta_0}, \quad
  t \in [0, \infty),
$$
in which case
$$
  \log z
= \log |z| + i \arg z, \quad
  \theta_0 < \arg z < \theta_0 + 2 \pi.
$$

\subsubsection{Properties of the logarithm}
Any function that is a primitive of $\frac{1}{z}$ on $U$
differs from a branch of the logarithm by an additive
constant. We may define a branch of the logarithm on
any simply connected domain that excludes 0.

More generally, given a
$f: U \to \mathbb{C} \setminus \{ 0 \}$,
holomorphic on an open set $U$
we can find a function $L$ that satisfies
$\exp \circ L \circ f = f$ and
$$
  \frac{\dif}{\dif z}
  (L \circ f)(z)
= L^\prime(f(z)) \cdot f^\prime(z)
= \frac{f^\prime(z)}
       {f(z)}
$$
on $U$, and we say that $\log f$ is a primitive of
$\frac{f^\prime}{f}$.

A logarithm on an arbitrary subset of
$\mathbb{C} \setminus \{ 0 \}$ can be found by
$$
  L(z)
= \log z_0 + \int_\gamma \frac{1}{w} \dif w
$$
where $\gamma(0) = z_0$ and $\log z_0$ is taken over a chosen
branch.

\subsection{Uniqueness theorem for holomorphic functions}

If two functions $f$, $g$ are holomorphic on an open set
$U$, and equal on a subset $S \subset U$ which has an
accumulation point in $U$, then $f = g$ on $U$. This is a
consequence of the fact that zeros of a holomorphic function
must be isolated, which can be seen by a power series
argument.

\section{Power Series}

\subsection{Useful formulas}
Finite geometric progression sum:
$$
  \sum_{k=0}^n z^k
= \frac{z^{n+1}} - 1}
       {z - 1}.
$$

Rewriting a double sum involving a finite convolution:
$$
  \sum_{n=0}^\infty
  \sum_{k=0}^n
    a_{k,n-k}
= \sum_{m=0}^\infty
  \sum_{n=0}^\infty
    a_{n,m}
$$

Binomial theorem:
$$
  (a + b)^n
= \sum_{k=0}^n
    {n \choose k}
    a^k
    b^{n-k}
$$

Expanding a difference of powers:
$$
  (a - b)^n
= (a - b)
  \sum_{k=0}^{n-1}
    a^{n - k - 1} b^k
$$

\subsection{Comparison tests}
If $|a_n| \leq c_n$ for a real nonnegative
sequence $c_n$ and $\sum c_n$ converges, then
$\sum a_n$ converges absolutely.

If $\| f_n \|_U \leq c_n$ for some nonnegative real sequence
$c_n$ and $\sum c_n$ converges, then $\sum f_n$
converges uniformly and absolutely on $U$.

If for a fixed $r > 0$, $\sum |a_n| r^n$ converges, then
$\sum a_n z^n$ converges absolutely and uniformly for
$|z| \leq r$.

\subsection{Radius of convergence}
Define $\limsup a_n$ to be the supremum of the set of accumulation
points of the sequence $(a_n)$.
When $t = \limsup |a_n|^{1 / n}$,
$|a_n| \leq (t + \varepsilon)^n$ for cofinitely many $n$
for every $\varepsilon > 0$. Then
when $|z| < \frac{1}{t + \varepsilon}$,
$|a_n| |z|^n < 1$, so $\sum a_n z^n$ converges absolutely.

From real analysis,
$ \lim \frac{|a_{n+1}|}{|a_n|}
= \lim |a_n|^{1 / n}
= \limsup |a_n|^{1 / n}$ if the first two limits exist.

\subsection{Derivative of power series}

\subsection{Difference equations}

\subsection{Coefficient expressions using derivatives}
Since $f(z_0) = a_0$, $f^{(n)}(z_0) = n! a_n$, etc.
we have the usual Taylor expansion for an analytic function.


\section{Integration on curves}

\subsection{Using definitions, using primitives}
By definition,
$$
  \int_\gamma
    f(z)
    \dif z
= \int_a^b
    f(\gamma(t)) \cdot \gamma^\prime(t)
    \dif t,
$$
and
$$
  g(\gamma(b)) - g(\gamma(a))
= \int_\gamma
    f(z) \dif z
$$
if (and only if) $g^\prime = f$. Notably, the function $(z - z_0)^n$ has the
primitive $\frac{(z - z_0)^{n+1}}{n + 1}$ for $n \neq -1$, so a Laurent series
at a point $z_0$ with finitely many negative terms can be integrated
term-by-term around a closed curve to give $\int_\gamma f = 2 \pi i a_{-1}$.


\section{Cauchy's theorem and formula}

Path-independence of integrals, existence of a primitive,
and zero integral around closed paths are equivalent.

Cauchy's theorem for Jordan curves says that for a Jordan
curve $\gamma$ and a function $f$ holomorphic on
$\gamma \cup \mathrm{Int}(\gamma)$,
$$
  \oint_\gamma
    f(z)
    \dif z
= 0.
$$

By expanding $g(w) = \frac{f(z)}{z - w}$ in a geometric series in $(w
- w_0)$, we can see
$$
  f(w)
= \frac{1}{2 \pi i}
  \int_J
    \sum_{n=0}^\infty
      \frac{f(z)}{(z - w_0)^{n+1}}
      (w - w_0)^n
$$
and after showing that this series converges uniformly it follows that
$$
  f^{(k)}(w)
= \frac{n!}{2 \pi i}
  \int_J
    \frac{f(z)}
         {(z - w)^{n+1}}, \quad w \in \mathrm{Int}(J).
$$

The radius of a power series expansion at $z_0$ of a holomorphic function on a
domain $U$ is at least $\mathrm{dist}(z_0, \mathbb{C} \setminus U)$.

\subsection{Liouville's theorem, fundamental theorem of algebra}

Any bounded entire function is constant. If a polynomial $P$ has no
zero, then $\frac{1}{P}$ is entire, and
$\lim_{z \to \infty} \frac{1}{P} = 0$ so $\frac{1}{P}$ is bounded on
$\mathbb{C} \setminus \bar{D}(0, R)$ for some $R$. Since
$\left(\frac{1}{P}\right)(\bar{D}(0, R))$ is compact we can conclude
that $\frac{1}{P}$ is bounded, thus constant, so $P$ is constant.

\subsection{Simply connected domains}
We say that a domain $U$ (i.e. an open connected subset of $\mathbb{C}$)
is simply connected if the interior of every Jordan curve in $U$ lies
in $U$. Equivalently, every closed curve in $U$ is homotopic to a point.
Note that a star domain is simply connected.

\subsubsection{Existence of primitives}
A function is holomorphic on a simply connected domain if and only if
it has a primitive on that domain.
On a simply connected open set,
$g(z) = \int_{z_0}^z f(\zeta) \dif \zeta$ is a primitive for $f$,
where $z_0$ is any point in the set.

\subsubsection{Branch of $\log z$}
If $U \subset \mathbb{C} \setminus \{ 0 \}$ is simply connected then
$\frac{1}{z}$ has a primitive, which differs from a branch of the
logarithm by an additive constant. If $\frac{f^\prime}{f}$ has a
primitive in $U$, then it differs from a branch of $\log f$ by a
constant, i.e. a continuous function $L$ such that
$\exp \circ L = f$.

\subsubsection{Harmonic conjugates}
Given a harmonic function $u$, $f = u_x - i u_y$ is holomorphic, and so if
the domain of definition is simply connected then $f$ has a primitive
$F$ there. $\Re~F$ shares both its derivatives
with $u$ and thus differs from $u$ by a real constant $C$, so
$\Im(F - C)$ is a harmonic conjugate for $u$.

To find the harmonic conjugate explicitly, we solve the
Cauchy-Riemann equations $v_y = u_x$ and $v_x = -u_y$. The first gives
$v = k(x, y) + h(x)$, the second then gives
$-u_y = k^\prime(x, y) + h^\prime(x)$, and if $u$ is harmonic it is
possible to solve this for $h$.

\subsubsection{Riemann Mapping Theorem}



\section{Harmonic functions}
A harmonic function $f$ is one that satisfies
$$
  \nabla^2 f
= \sum_i
    \frac{\partial^2 f}
         {\partial x_i^2}
= 0,
$$
which is satisfied for holomorphic functions by the Cauchy-Riemann
equations. A function $f$ is harmonic on an open set $U$
if and only if $f_x - i f_y$ is holomorphic on $U$.

\subsection{Mean value theorems}
If $f$ is harmonic (and thus also if $f$ is holomorphic) then
$$
  f(z_0)
= \frac{1}{2\pi}
  \int_{0}^{2\pi}
    f(z_0 + re^{i\theta})
    d \theta
= \frac{1}{\pi r^2}
  \int_{D(z_0, r)}
    f(z) \dif x \dif y
$$

\subsection{Maximum principle}



\section{Residue calculus}

\subsection{Winding number}
The winding number is defined by
$$
  W(\gamma; z_0)
= \frac{1}{2 \pi i}
  \int_\gamma
    \frac{1}{z - z_0}
    \dif z.
$$
The winding number of a curve at a point outside the curve is
0. Crossing the curve from the right to the left increases the winding
number by 1.

\subsection{General Cauchy's formula}
Let $\gamma$ be a closed contour. Then
$$
  2 \pi i W(\gamma; z_0) \cdot f^{(k)}(z_0)
= \int_\gamma
    \frac{f(z)}{(z - z_0)^{k+1}}
    \dif z
$$

\subsection{Laurent series}

The coefficients of a Laurent series on the annulus $A(z_0, r, R)$
are given by
$$
  a_n
= \frac{1}{2 \pi i}
  \int_{|z - z_0| = t}
    \frac{f(z)}{(z - z_0)^{n+1}},
n \in \mathbb{Z},
t \in (r, R)
$$

\subsubsection{Product and ratio}
We can compute
$$
  \left(
    \sum_{n=0}^\infty
      a_n
      (z - z_0)^n
  \right)
  \left(
    \sum_{n=0}^\infty
      b_n
      (z - z_0)^n
  \right)
= \sum_{n=0}^\infty
    \left(
      \sum_{k=0}^n
        a_k
        b_{n-k}
    \right)
    (z - z_0)^n
$$
and can use this to recursively solve $\frac{f}{g} \cdot g = f$ for
$\frac{f}{g}$. The first few terms may also be computed by polynomial
long division.

\subsection{Residue formula}
Let $z_i$ be the poles of $f$, which is holomorphic on
Jordan curve $J$ and its interior except for
at these points. Then
$$
  \int_J f
= 2 \pi i
  \sum_i
  \Res_{z_0}~f.
$$
This implies Cauchy's theorem since the sum is empty for a holomorphic
function as well as Cauchy's formula since
$$
  \frac{1}{2 \pi i}
  \int_\gamma
    \frac{f(z)}
         {(z - z_0)^{n+1}}
    \dif z
= \Res_{z_0}
    \frac{f(z)}{(z - z_0)^{n+1}}
= f^{(n)}(z_0).
$$

\section{Singularities}

\subsection{Classification}

\begin{enumerate}
  \item{
    A singularity $z_0$ of $f$ is said to be removable if the Laurent
    expansion about $z_0$ has no negative terms. In this case, the
    function
    $$
      g(z)
    = \left\{
        \begin{array}{l l}
          f(z), & \quad z \neq z_0 \\
          a_0,  & \quad z = z_0
        \end{array}
    $$
    where $a_0$ is the 0 coefficient of the Laurent series. This is
    because the limit of $f$ exists at $z_0$.

    Sufficiently near a removable singularity, $f$ is bounded,
    explicitly by $|\lim_{z \to z_0} f| + \varepsilon$ for any
    $\varepsilon > 0$, for $z \in D(z_0, r(\varepsilon))$.
  }
  \item{
    A singularity $z_0$ of $f$ is said to be a pole of order $m$ if
    $a_{-m} \neq 0$ and $a_{-n} = 0$ for $n > m$. In this case
    $$
      f(z)
    = \frac{g(z)}
           {(z - z_0)^m}
    $$
    where $g$ is holomorphic.

    In this case $\lim_{z \to z_0} |f| = \infty$. Therefore
    $\lim_{z \to z_0} \left|\frac{1}{f}\right| = 0$, so $z_0$ is a
    removable singularity of $\frac{1}{f}$.
  }
  \item{
    A singularity $z_0$ of $f$ is said to be essential if it is not
    removable or a pole.

    In this case
    $f(D(z_0, r)) = \mathbb{C}$ or $\mathbb{C} \setminus \{ w_0 \}$
    (for some $w_0$) for any $r > 0$. In some ways essential
    singularities behave like poles of infinite order.
  }
\end{enumerate}

\subsection{Finding Laurent series of $\frac{1}{z - z_0}$}
Let
$$
  f(z)
= \frac{1}{z - z_0}
$$
This function is already its own
Laurent series around $z_0$.

When $z_0 \neq 0$, and $\left| \frac{z}{z_0} \right| < 1$,
we have
\begin{align*}
   -\frac{1}{z_0} f(z)
&=  \frac{\frac{1}{z_0}}
         {1 - \frac{z}{z_0}}
 = -\frac{1}{z_0}
    \sum_{n=0}^\infty
      \left(\frac{z}{z_0}\right)^n
\end{align*}
which is a Laurent series around 0.

\subsection{Computing residues}

If $g$, $f$ are holomorphic and
$h = \frac{g}{f}$, and $\ord_{z_0} f = 1$, then
$\Res_{z_0} h = \frac{g(z_0)}{f^\prime(z_0)}$.

A useful manipulation for results like this is to express
a function as $\frac{1}{z - z_0} h(z)$ or
$\frac{C}{z - z_0} + h(z)$ for some holomorphic function
$h(z)$. In the latter case the residue is of course $C$, and in the
former case it is $a_0$, where $a_0 = h(z_0)$ is the first term of the
Taylor series of the function $h$ which is analytic at $z_0$. More
generally the idea is to ``displace'' a holomorphic function by some
number of powers of $(z - z_0)$ to compute the residue more easily;
this is analogous to the computation of say
$$
  \Res_0 \frac{\sin z}{z^6}
= \frac{\sin^{(5)}(0)}{5!}.
$$
Another useful example of this approach is the observation that when
$f(z) = (z - z_0)^m g(z)$,
$$
  \Res_{z_0} \frac{f^\prime}{f}
= \Res_{z_0}
  \left\{
    \frac{m}{z - z_0} + \frac{g^\prime(z)}{g(z)}
  \right\}
= \ord_{z_0} f
$$
from which it follows that
$$
  \frac{1}{2 \pi i}
  \int_\gamma f
= \mathrm{Zeros}_{\mathrm{Int}(\gamma)}(f)
- \mathrm{Poles}_{\mathrm{Int}(\gamma)}(f).
$$

\subsection{Rouch\'e's theorem}
If $|f(z) - g(z)| < |f(z)|$ on the boundary of a disk
$D$ and $f$, $g$ are analytic on the closed disk, then they have the
same number of zeros within the disk.

If a zero $z_0$ of $P(z)$ is not a simple zero, then it is also a zero
of $P^\prime(z)$. We can compute $Q = \mathrm{gcd}(P, P^\prime)$ by
polynomial long division, and the roots of $Q$ are the multiple roots
of $P$.

\subsection{Open mapping theorem, inverse mapping theorem}
If a holomorphic function is non-constant on every open disk in its
domain, then it is an open mapping. Similarly, if $f^\prime(z_0) \neq
0$, $f$ is locally non-constant and indeed injective, since in this
case $\ord_{z_0} (f - f(z_0)) = 1$ and there is an
$r$ such that for any $w \in D(f(z_0), r)$, $f - w$ has the same
number of zeros as $f - f(z_0)$.

More generally, if $m = \ord_{z_0} (f - f(z_0)) \in \mathbb{N}$, then
$$
f(z) = (\sqrt[m]{a_m} (z - z_0)(1 + h(z)))^m = f_0(z)^m
$$
on some disk $D(z_0, r)$, where $h(z_0) = 0$ and so $f_0^\prime(z_0)
\neq 0$, whence $f_0$ is not constant anywhere on this disk. Therefore
$f_0$ is open, and the image of a disk under the power map is a disk.

If $|f(z_0)|$ is a local maximum for $|f|$, then $f$ is locally
constant at $z_0$. Indeed if $f$ is not locally constant, then
$f(D(z_0, r))$ is open, so $D(f(z_0), s) \subset f(D(z_0, r))$, so
$|f(z_1)| = |f(z_0)| + s > |f(z_0)|$ for some $z_1 \in D(z_0, r)$.

If $f^\prime(z_0) = 0$, then $f^\prime$ is either locally constant or
$z_0$ is an isolated zero of $f^\prime$: it is not true in this case
that $f$ is locally constant, consider $f(z) = z^2 - 1$ at zero.


\section{Contour integration}

\subsection{Half-disk}

\subsection{Half-disk minus half-disk}

\subsection{Taking real parts}

\subsection{Rectangular contours}

\subsection{Trigonometric integrands}

\subsection{Integrals involving branch cuts}


\section{Conformal maps}

\subsection{Automorphisms}
The automorphisms of the disk are all of the form $M_c \circ g_\alpha$ where
$$
  g_\alpha(z)
= \frac{\alpha - z}
       {1 - \bar{\alpha} z}.
$$
The automorphisms of the upper half-plane are all of the form
$h_{z_2}^{-1} \circ M_c \circ h_{z_1}$, where $z_1, z_2 \in
\mathbb{H}$ and
$$
  h_{z_0}(z)
= \frac{z - z_0}
       {z - \bar{z_0}}
\in \mathrm{Iso}(\mathbb{H}, \mathbb{D}).
$$

\subsection{The Schwarz lemma}
Let $f : \mathbb{D} \to \mathbb{D}$ be holomorphic and $f(0) =
0$. Then $|f^\prime(z)| \leq |z|$ and $|f^\prime(0)| \leq 1$, and if
$|f^\prime(z_0)| = |z_0|$ for any $z_0 \in \mathbb{D}$ or
$|f^\prime(0)| = 1$ then $f$ is a rotation. Furthermore if
$f$ has a nonzero fixed point or $f^\prime(0) = 1$
then $f$ is identity. As a consequence of this, any automorphism of
the unit disk that fixes zero is a rotation.

\section{M\"obius transformations}
A M\"obius transformation can be decomposed as
$$
f = T_{\frac{a}{c}} \circ J \circ T_{\frac{d}{b^\prime}} \circ M_{\frac{c}{b^\prime}}
$$
where $b^\prime = b - \frac{ad}{c}$.

\subsection{Finding the transformation between given circles}
The transformation $F$ mapping $z_1, z_2, z_3$ to $w_1, w_2, w_3$ can be
found by setting $a = 1$ and solving the system of equations $F(z_i) =
w_i$ or by computing
$F = F_{w_1, w_2, w_3}^{-1} \circ F_{z_1, z_2, z_3}$,
where $F_{a, b, c}$ maps $a$ to $0$, $b$ to $\infty$ and $c$ to 1,
explicitly
$$
  F_{a, b, c}
= \frac{c - b}
       {c - a}
  \frac{z - a}
       {z - b},
$$
and so maps the circle defined by these three points onto the real
line. If any $a, b, c$ is $\infty$, we omit the terms containing it.

recognizing that
$$
  [F(z_1), F(z_2), F(z_3), F(z_4)]
= F_{w_2, w_1, w_3}(w_4)
= [z_1, z_2, z_3, z_4]
= F_{z_2, z_1, z_3}(z_4)
$$
so that
$$
w_4 = (F_{w_2, w_1, w_3}^{-1} \circ F_{z_2, z_1, z_3})(z_4)
$$

\subsection{Cross ratio}
The cross-ratio is given by
$$
  [z_1, z_2, z_3, z_4]
= \frac{z_1 - z_3}
       {z_2 - z_3}
  \frac{z_2 - z_4}
       {z_1 - z_4}
= F_{z_2, z_1, z_3}(z_4).
$$


\section{Riemann mapping theorem}
Every nonempty open simply connected subset of $\mathbb{C}$ has a
unique (up to a rotation) isomorphism to the unit disk such that
$f(z_0) = 0$.

\subsection{Compact convergence}
A sequence is said to converge compactly on an open set $U$ if it
converges uniformly on compact subsets of $U$. The limit of a
compactly convergent sequence of holomorphic functions is holomorphic,
which allows us to define new holomorphic functions by showing that a
given series or integral converges compactly (typically by comparison)
and that the partial sums are holomorphic.

\subsection{Normal families}
A collection of analytic functions $\Phi$ is called a normal family if every
sequence in $\Phi$ contains a compactly convergent subsequence.
$\Phi$ is a normal family if and only if $\Phi$ is uniformly bounded
on compact sets (Montel's theorem). If a sequence $(f_n)$ of analytic
functions or its derivative sequence is uniformly bounded on an open
set $U$, then $(f_n)$ is equicontinuous on each compact subset of $U$,
and hence contains a subsequence which converges uniformly on $K$
(Arzel\`a-Ascoli theorem).

\subsection{Finding biholomorphisms}

\end{document}
