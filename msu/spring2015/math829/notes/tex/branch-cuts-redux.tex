\section{Differentiability of the Logarithm}

Let $U$ be an open set with $0 \notin U$.
Recall that $L(z)$ is called a branch of $\log z$ in $U$ if $L$
is continuous on $U$ and $e^{L(z)} = z$ for all $z \in U$. We will now
show that $L$ is holomorphic and $L^\prime = \frac{1}{z}$.

\begin{theorem}
Suppose $f$ is a primitive of $\frac{1}{z}$ in a domain $U$. Then
$\exists C \in \mathbb{C}$ such that $f + C$ is a branch of $\log z$ on $U$.
\end{theorem}

\begin{proof}
Let $g(z) = \frac{e^{f(z)}}{z}$. Then
$$
  g^\prime(z)
= \frac{e^{f(z)}f^\prime(z)}{z} - \frac{e^{f(z)}}{z^2}
= 0.
$$
Since $U$ is connected this means $g$ is constant, so
$g(z) = C_0$.

There exists $C \in mathbb{C}$ such that $e^C = \frac{1}{C_0}$. Then
$$
e^{f(z) + C} = \frac{e^{f(z)}}{C_0} = \frac{e^{f(z)}}{g(z)} = z,
$$
so $f + C$ is a branch of $\log z$ in $U$.
\end{proof}

\begin{theorem}
Let $U \subset \mathbb{C} \backslash \{ 0 \}$ be a
simply connected domain. Then there is a branch of $\log z$ in $U$
which is a primitive of $\frac{1}{z}$ on $U$.
\end{theorem}

\begin{proof}
Since $\frac{1}{z}$ is holomorphic on $U$ and $U$ is simply connected,
there exists a primitive for $\frac{1}{z}$ on $U$. Let $f$ be such a
primitive. Then there exists a $C$ such that $g = f + C$ is a branch of
the logarithm in $U$, and we are done because $g^\prime = f^\prime = \frac{1}{z}$.
\end{proof}

\begin{lemma}
If $f$ is a continuous function on a domain $U$ such that
$f(z) \in \mathbb{Z}$ for all $z \in U$ then $f$ is constant.
\end{lemma}

\begin{proof}
Let $z_0 \in U$ and $n_0 = f(z_0) \in \mathbb{Z}$. Let
$A = f^{-1}(\{n_0\})$. Since $\{n_0\}$ is closed, $A$ is
relatively closed in $U$. Since $f(z) \in \mathbb{Z}$ for all
$z \in U$, $A = f^{-1}((n_0 - \frac{1}{2}, n_0 + \frac{1}{2}))$.
Since $(n_0 - \frac{1}{2}, n_0 + \frac{1}{2})$ is an open subset
of $\mathbb{R}$, we conclude that $A$ is also relatively open.
Since $U$ is connected, this means either $A = \varnothing$ or
$A = U$. Since $z_0 \in A$ it cannot be that $A = \varnothing$,
so $A = U$, i.e. $f \equiv n_0$.
\end{proof}

\begin{theorem}
If $L$ is a branch of the logarithm on an open set $U$, then
$L$ is a primitive of $\frac{1}{z}$ in $U$.
\end{theorem}

\begin{proof}
Let $z_0 \in U$. Then $\exists r > 0$ such that $D(z_0, r) \subset U$.
Since $D(z_0, r)$ is simply connected, there is a branch of $\log z$
in $D(z_0, r)$ which is a primitive of $\frac{1}{z}$ in $D(z_0, r)$.
Call this branch $M(z)$. Then
$$
e^{L(z)} = z = e^{M(z)},
$$
so we must have $\frac{L(z) - M(z)}{2 \pi i} \in \mathbb{Z}$ for
every $z \in D(z_0, r)$. Since $\frac{L - M}{2 \pi i}$ is continuous
on this disk we must have $\frac{L - M}{2 \pi i}$ is constant, so
$L - M$ is constant on $D(z_0, r)$. Since $M$ is a primitive of
$\frac{1}{z}$ on $D(z_0, r)$, so is $L$, since they differ by a constant.
Then $L^\prime(z_0) = \frac{1}{z_0}$ for all $z_0 \in U$, so $L$ is a
primitive of $\frac{1}{z}$ in $U$.
\end{proof}

\begin{defn}[Branch of the Logarithm of a Function]
Let $f$ be an analytic function on an open set $U$
which does not take the value $0$. We say that $g$ is a branch of
$\log f$ if
\begin{enumerate}
  \item{
    $g$ is continuous on $U$
  }
  \item{
    $e^{g(z)} = f(z)$ for all $z \in U$.
  }
\end{enumerate}
If $g$ is holomorphic then differentiating gives
$$
g^\prime(z) = \frac{f^\prime(z)}{f(z)}.
$$
We can also show that $g$ is holomorphic and a primitive
of $\frac{f^\prime}{f}$, using an appropriately adapted argument
to the process followed above. Essentially
we substitute $\log f$ and $\frac{f^\prime}{f}$ for $\log z$ and
$\frac{1}{z}$, respectively.
\end{defn}

\begin{theorem}[Mean Value Theorem]
Let $f$ be holomorphic on a closed disk $\bar{D}(z_0, r)$.
Then
\begin{align*}
   f(z_0)
&= \frac{1}{2 \pi}
   \int_0^{2\pi}
     f(z_0 + r e^{i \theta})
     \dif \theta \\
&= \frac{1}{\pi r^2}
     \int_{|z - z_0| \leq r}
       f(z) \dif x \dif y.
\end{align*}
\end{theorem}

\begin{proof}
Cauchy's formula gives
\begin{align*}
   f(z_0)
&= \frac{1}{2 \pi i}
   \oint_{|z - z_0| = r}
     \frac{f(z)}{z - z_0}
     \dif z \\
&= \frac{1}{2 \pi i}
   \int_0^{2 \pi}
     \frac{f(z_0 + r e^{i\theta})}
          {re^{i \theta}}
     i r e^{i \theta}
     \dif \theta \\
&= \frac{1}{2 \pi}
   \int_0^{2\pi}
     f(z_0 + r e^{i \theta})
     \dif \theta.
\end{align*}
Next,
$$
  2 \pi s f(z_0)
= \int_0^{2 \pi}
    f(z_0 + s e^{i\theta}) s
    \dif \theta, \quad
0 \leq s \leq r
$$
so
\begin{align*}
   \pi r^2 f(z_0)
&= \int_0^r
     2 \pi s f(z_0)
     \dif s
&= \int_0^r \int_0^{2\pi}
     f(z_0 + s e^{i\theta}) s
     \dif s \dif \theta \\
&= \int_{|z - z_0| \leq r}
     f(z)
     \dif x \dif y.
\end{align*}
\end{proof}

\begin{theorem}
Let $f$ be holomorphic on $\bar{D}(z_0, r)$.
Suppose $|f(z_0)| \geq |f(z)|$ for every $z \in D(z_0, r)$.
Then $f$ is constant on $D(z_0, r)$ (and thus on
$\bar{D}(z_0, r)$ since $f$ is continuous).
\end{theorem}

\begin{proof}
We first show $|f|$ is constant on $D(z_0, r)$, from which it
follows that $f$ is constant on this disk. If $|f|$ is not
constant then $\exists z_1 \in D(z_0, r)$ such that
$|f(z_1)| < |f(z_0)|$. Let $\varepsilon = |f(z_0)| - |f(z_1)| > 0$.
Since $|f|$ is continuous, $\exists \delta > 0$ such that
$D(z_1, \delta) \subset D(z_0, r)$ and
$|f(z)| < |f(z_1)| + \frac{\varepsilon}{2}$ for any $z \in D(z, \delta)$.
Let $D_0 = D(z_0, r)$ and $D_1 = D(z_1, \delta)$.
\end{proof}

\begin{corol}
Let $f$ be holomorphic on a domain $U$. Suppose $|f|$ attains a local
maximum at some $z_0 \in U$, i.e. $\exists r > 0$ such that
$|f(z_0)| \geq |f(z)|$, $\forall z \in D(z_0, r)$. Then $f$ is constant
on $U$.
\end{corol}

\begin{proof}
From the previous theorem $f$ is constant on $D(z_0, r)$, so
$f - f(z_0) \equiv 0$ on $D(z_0, r)$. From the uniqueness theorem,
$f - f(z_0) \equiv 0$ on $U$.
\end{proof}

\begin{theorem}[Maximum Modulus Principle]
Let $U$ be a bounded domain and let $\bar{U}$ be its closure.
Suppose $f$ is holomorphic on $U$ and continuous on $\bar{U}$.
Then $\exists z_0 \in \partial U$ such that $|f(z_0)| \geq |f(z)|$
$\foral z \in \bar{U}$, i.e. the maximum modulus of the function is
attained at some point on the boundary.
\end{theorem}

\begin{proof}
$U$ is bounded, so $\bar{U}$ is bounded. Since $\bar{U}$ is closed,
$\bar{U}$ is compact. Then $f$ is continuous on a compact set, so
$\exists z_0 \in \bar{U}$ such that $|f|$ attains its maximum at
$z_0$. If $z_0 \in \partial U$, we are done. If $z_0 \in U$, then
$f$ is constant on $U$ and thus constant on $\bar{U}$, so we may
take any point on the boundary to satisfy the theorem.
\end{proof}

\begin{corol}
Suppose $f$ and $g$ are holomorphic on $U$ and continuous on $\bar{U}$.
Suppose $f = g$ on $\partial U$. Then $f = g$ on $U$.
\end{corol}

\begin{proof}
Let $h = f - g$. Then $h$ is holomorphic on $U$ and continuous on
$\bar{U}$. Since $f = g$ on $U$, $h \equiv 0$ on $\partial U$, and from
the maximum modulus principle this means $h \equiv 0$ on $U$.
\end{proof}
