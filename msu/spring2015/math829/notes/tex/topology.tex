\section{Topology of Complex Numbers}

The topology of complex numbers is generated by the metric
\begin{align*}
  d(z_1, z_2) 
&= |z_1 - z_2|
 = |(x_1 + i y_1) - (x_2 + i y_2)| \\
&\triangleq \sqrt{(x_1 - x_2)^2 + (y_1 - y_2)^2}
\end{align*}
which obeys the triangle inequality
$$
d(z_1, z_3) \leq d(z_1, z_2) + d(z_2, z_3).
$$

The topology on $\mathbb{C}$ is thus the same as the topology on
$\mathbb{R}^2$.

\begin{defn}[Sequence]
Let $U \subset \mathbb{C}$. A \emph{sequence in U}
is a function $\mathbb{N} \to U$ given by 
$n \mapsto z_n$ and denoted $( z_n )$.
\end{defn}

\begin{defn}[Limit, convergence]
Let $( z_n )$ be a sequence, $w \in \mathbb{C}$.
We say that $w$ is the \emph{limit} of the sequence $( z_n )$
if $|z_n - w| \to 0$, i.e. $\forall \varepsilon > 0$ there is an
$N \in \mathbb{N}$ such that $n \geq N \implies |z_n - w| < \varepsilon$.
In this case we say that $(z_n)$ is \emph{convergent} and
\emph{converges} to $w$.
\end{defn}

\begin{defn}[Open set]
The \emph{open sets} in $\mathbb{C}$ are unions of the open disks
$$
D(z_0, R) = \{ z \in \mathbb{C} : |z - z_0| < R \},
$$
i.e. a subset $U$ of $\mathbb{C}$ is open iff. $\forall z \in U$,
$D(z, r(z)) \subset U$ for some $r : \mathbb{C} \to \mathbb{R}$.
\end{defn}

\begin{defn}[Boundary]
Given a set $U \subset \mathbb{C}$, its \emph{boundary} is the set
$$
  \partial U 
= \{ z \in \mathbb{C} 
   : \forall \varepsilon > 0, 
     D(z, \varepsilon) \cap U \neq \varnothing
     \and
     D(z, \varepsilon) \cap (\mathbb{C} \backslash U) \neq \varnothing \}.
$$
\end{defn}

\begin{defn}[Closed set, closure]
A set $U$ is \emph{closed} if $\partial U \subset U$. 
The \emph{closure} of $U$ is $\bar{U} = U \cup \bar{U}$.
\end{defn}

\begin{obsv}
  We observe the following about closed sets. Let $U \subset \mathbb{C}$.
  \begin{enumerate}
    \item{
      $z \in \bar{U}$ if and only if $\forall r > 0$, 
      $D(z, r) \cap U \neq \varnothing$
    }
    \item{
      $\bar{U}$ is closed.
    }
    \item{
      $U$ is closed if and only if $\mathbb{C} \backslash U$ is open.
    }
    \item{
      The closed subsets of $\mathbb{C}$ are closed under 
      arbitrary intersection and finite union.
    }
    \item{
      $U$ is closed iff. it contains the limits of all convergent sequences
      in $U$.
    }
  \end{enumerate}
\end{obsv}

\begin{defn}[Dense subset]
We say that $U \subset \mathbb{C}$ is \emph{dense} in
$\mathbb{C}$ if $\bar{U} = \mathbb{C}$.
\end{defn}

\begin{theorem}
  $z_n \to w$ iff. $\mathrm{Re} z_n \to \mathrm{Re} w$ and
  $\mathrm{Im} z_n \to \mathrm{Im} w$.
\end{theorem}

Using this theorem we get $z_n \to z_0$, $w_n \to w_0$
implies that $z_n \pm w_n \to z_0 \pm w_0$,
$z_n w_n \to z_0 w_0$, and $z_n / w_n \to z_0 / w_0$ if
$w_n, w_0 \neq 0$.

\begin{xmpl}
  To show $z_n w_n \to z_0 w_0$, we write
  $$
  z_n = x_n + i y_n, \quad w_n = u_n + i v_n
  $$
  and assume $x_n \to x_0$, $y_n \to y_0$,
  $u_n \to u_0$, $v_n \to v_0$. Then
  $$
      z_n w_n
  =   (x_n u_n - y_n v_n) + i (x_n v_n + y_n u_n)
  \to (x_0 u_0 - y_0 v_0) - i (x_0 v_0 + y_0 u_0)
  =   z_0 w_0
  $$
\end{xmpl}

\begin{defn}[Cauchy sequence]
  A sequence of complex numbers $(z_n)$ is said to be
  \emph{Cauchy} if
  $$
  \forall \varepsilon > 0, \exists N . \forall m, n \geq N,
  |z_m - z_n| < \varepsilon.
  $$
\end{defn}

\begin{obsv}
  \begin{enumerate}
    \item{
      If $(z_n)$ is convergent, then it is Cauchy.
    }
    \item{
      If $(z_n)$ is Cauchy, then $(\mathrm{Re}~z_n)$
      and $(\mathrm{Im}~z_n)$ are also Cauchy. This is because
      $$
           | \mathrm{Re}~z_n - \mathrm{Re}~z_m|
      =    |\mathrm{Re}(z_n - z_m)|
      \leq |z_n - z_m|.
      $$
      It follows that $(z_n)$ is Cauchy iff. $(z_n)$ is convergent.
    }
  \end{enumerate}
\end{obsv}

\begin{defn}[Boundedness]
  We say that $S$ is \emph{bounded} if
  $\exists R \in \mathbb{R}$, $R > 0$ such that $|z| \leq R$,
  $\forall z \in S$.
\end{defn}

\begin{xmpl}
  \begin{enumerate}
    \item{
      A finite set $\{ z_1, \dots, z_n \}$ is bounded by
      $R = \max \{ |z_1|, \dots, |z_n| \}$.
    }
    \item{
      $D(\alpha, r)$ and $\bar{D}(\alpha, r)$ are bounded
      by $R = |\alpha| + r$.
    }
    \item{
      If $S_1$ and $S_2$ are bounded then $S_1 \cup S_2$ is bounded,
      since we may take $R = \max \{ R_1, R_2 \}$.
    }
    \item{
      If $S^\prime$ is bounded and $S \subset S^\prime$, then $S$ is bounded.
    }
  \end{enumerate}
\end{xmpl}

\begin{theorem}
  Every convergent sequencei s bounded.
\end{theorem}

\begin{proof}
  Suppose $z_n \to z_0$. Then there exists an $N \in \mathbb{N}$ such that
  $|z_n - z_0| < 1$ if $n > N$. Thus
  $$
  \{ z_n, n \in \mathbb{N} \} \subset \{ z_1, \dots, z_N \} \cup D(z_0, 1),
  $$
  which is the union of a finite set and a disk.
\end{proof}

\begin{theorem}[Bolzano-Weierstrass]
  Every bounded sequence of complex numbers contains a convergent
  subsequence.
\end{theorem}
\begin{proof}
  Let $(z_n)$ be a bounded sequence of complex numbers. Then
  $(x_n) = (\mathrm{Re}~z_n)$ and $(y_n) = (\mathrm{Im}~z_n)$ are bounded
  sequences of real numbers. Applying the Bolzano-Weierstrass theorem
  for real numbers, we have a convergent subsequence of $(x_n)$
  denoted $(x_{n_k})$. Then the subsequence $(y_{n_k})$ is still bounded
  and thus has a convergent subsequence $(y_{n_{k_l}})$.

  Now $(x_{n_{k_l}})$ is a subsequence of $(x_{n_k})$ and therefore convergent.
  Thus $(z_{n_{k_l}}) = (x_{n_{k_l}} + i y_{n_{k_l}})$ is a convergent subsequence of
  $(z_n)$.
\end{proof}

\begin{defn}[Compact set]
  There are several equivalent definitions:
  \begin{enumerate}
    \item{
      A subset $S \subset \mathbb{C}$ is said to be compact if
      whenever $\{ U_\alpha : \alpha \in A \}$ is a family of
      open sets such that $S \subset \bigcup_{\alpha \in A} U_\alpha$,
      then there exists $\alpha_1, \dots, \alpha_n$ such that
      $S \subset \bigcup_{k = 1}^n \alpha_k$,
      i.e. every open cover of $S$ admits a finite subcover.
    }
    \item{
      $S$ is compact if every sequence in $S$ contains a convergent
      subsequence whose limit is in $S$. This is valid for all metric
      spaces.
    }
    \item{
      $S \subset \mathbb{C}$ is compact if it is both bounded and closed.
      This is valid only in finite-dimensional Euclidean spaces.
    }
  \end{enumerate}
\end{defn}

\begin{theorem}
  Definition 2 is equivalent to Definition 3 for complex numbers.
\end{theorem}
\begin{proof}
  Suppose $S$ is compact in the sense of Def. 2. If $S$ is not
  bounded, then for every $n \in \mathbb{N}$ we can find
  $z_n \in S$ with $|z_n| > n$. Then $|z_n| \to \infty$, so
  every subsequence $(z_{n_k})$ also has $|z_{n_k}| \to \infty$,
  so $z_{n_k}$ cannot converge. This contradicts our assumption,
  so $S$ is bounded.

  Suppose $S$ is not closed. Then there exists a $z_n \in S$ with
  $z_n \to \alpha$ and $\alpha \notin S$. Any subsequence of
  $z_n$ also converges to $\alpha$, but this contradicts Definition 2.
  Therefore $S$ is closed.

  If $S$ is closed and bounded, then the Bolzano-Weierstrass theorem
  shows that every sequence in $S$ contains a convergent subsequence.
  Since $S$ is closed, the limit of every such subsequence is in $S$,
  as desired.
\end{proof}

\begin{theorem}
  Let $(S_n)_{n=1}^\infty$ be a sequence of nonempty compact subsets of
  $\mathbb{C}$ such that $S_n \supset S_{n+1}$. Then
  $\bigcap_{n=1}^{\infty} S_n \neq \varnothing$.
\end{theorem}
\begin{proof}
  Since $S_n \neq \varnothing$, pick any $z_n \in S_n$. Then
  $(z_n)$ is a sequence in $S_1$, since $S_n \subset S_1$. Since
  $S_1$ is compact, $(z_n)$ contains a convergent subsequence
  $(z_{n_k})$. Call its limit $v$. We have $v \in S_1$. Fix any
  $m \in \mathbb{N}$. If $k \geq m$ then $n_k \geq k \geq m$.
  We see that
  $$
    v
  =   \lim_{k \to \infty} z_{n_k}
  =   \lim_{\substack{k \geq m \\ k \to \infty}} z_{n_k}
  \in S_m
  $$
  since $S_m$ is closed, so $v \in \bigcap_{m=1}^\infty S_m$.
\end{proof}

Let $A, B \subset \mathbb{C}$ such that $A, B \neq \varnothing$.
The distance between $A$ and $B$ is
$$
\mathrm{dist} (A, B) = \inf \{ |z - w| : z \in A, w \in B \}.
$$
We note that this distance is symmetric and nonnegative.
If $A \cap B \neq \varnothing$, then the distance is zero.
However, if the distance is zero, this does not imply that
$A$ and $B$ are disjoint. For example, consider $A = (0, \infty)$
and $B = \{ 0 \}$.

\begin{theorem}
  Let $A$, $B$ be nonempty subsets of $\mathbb{C}$. If $A$ is
  compact and $B$ is closed, then there exists $z_0 \in A$
  and $w_0 \in B$ such that $|z_0 - w_0| = \mathrm{dist}(A, B)$.
  In other words $\min \{ |z - w| : z \in A, w \in B \}$ exists.
\end{theorem}
\begin{proof}
  By definition, we can find two sequences
  $(z_n)$ in $A$ and $(w_n)$ in $B$
  such that $|z_n - w_n| \to \mathrm{dist}(A, B)$.
  Since $A$ is compact, $A$ is bounded and so $(z_n)$ is bounded.
  Since $(|z_n - w_n|)$ is convergent it is bounded. This implies
  that $(w_n)$ is bounded, because $|w_n| \leq |z_n| + |z_n - w_n|$.
  $A$ is compact and so $(z_n)$ has a convergent subsequence
  $(z_{n_k})$ such that $z_0 = \lim z_{n_k} \in A$. Applying the
  Bolzano-Weierstrass theorem to the bounded subsequence $(w_{n_k})$
  we find a convergent subsequence $(w_{n_{k_l}})$. Let
  $w_0 = \lim w_{n_{k_l}}$. Then $w_0 \in B$ since $B$ is closed.

  Now $z_{n_{k_l}} \to z_0 \in A$, $w_{n_{k_L}} \to w_0$ in $B$,
  and $|z_{n_{k_l}} - w_{n_{k_l}}| \to \mathrm{dist}(A, B)$.
  From the triangle inequality,
  $$
       |z_{n_{k_l}} - w_{n_{k_l}}|
  \leq |z_{n_{k_l}} - z_0| + |z_0 - w_0| + |w_0 - w_{n_{k_l}}|
  $$
  so that
  $$
       |z_{n_{k_l}} - w_{n_{k_l}}| - |z_0 - w_0|
  \leq |z_{n_{k_l}} - z_0| + |w_{n_{k_l}} - w_0|.
  $$
  Additionally
  $$
       |z_0 - w_0|
  \leq |z_0 - z_{n_{k_l}}| + |z_{n_{k_l}} - w_{n_{k_l}}| + |w_{n_{k_l}} - w_0|
  $$
  so that
  $$
       |z_0 - w_0| - |z_{n_{k_l}} - w_{n_{k_l}}|
  \leq |z_{n_{k_l}} - z_0| + |w_{n_{k_l}} - w_0|,
  $$
  and therefore
  $$
  \left||z_0 - w_0| - |z_{n_{k_l}} - w_{n_{k_l}}|\right|
  \leq |z_{n_{k_l}} - z_0| + |w_{n_{k_l}} - w_0|.
  $$
  Therefore $|z_{n_{k_l}} - w_{n_{k_l}}| \to |z_0 - w_0|$,
  and since $|z_{n_{k_l}} - w_{n_{k_l}}| \to \mathrm{dist}(A, B)$,
  it follows that $\mathrm{dist}(A, B) = |z_0 - w_0|$.

  If $A$ is compact, $B$ is closed, and $\mathrm{dist}(A,B) = 0$,
  then $A \cap B \neq \varnothing$, which implies that
  $\mathrm{dist}(A, B) > 0$.
\end{proof}

\begin{xmpl}
  The theorem does not hold if $A$ is only assumed to be closed.
  Let $A = \mathbb{N}$, $B = \{ n + \frac{1}{2n} : n \in \mathbb{N} \}$.
  These have empty intersection and zero distance.
\end{xmpl}

Let $S \subset \mathbb{C}$, $\alpha \in \bar{S}$, $f : S \to \mathbb{C}$.
We say that
$$
w = \lim_{\substack{z \to \alpha \\ z \in S}} f(z)
$$
if for every sequence $(z_n)$ in $S$, we have $f(z_n) \to w$. Equivalently,
$$
\forall \varepsilon > 0 .
\exists \delta > 0 .
\forall z \in D(\alpha, \delta) . |f(z) - w| < \varepsilon.
$$
If $\alpha \in S$ and
$f(\alpha) = \lim_{\substack{z \to a \\ z \in S}} f(z)$, then
we say that $f$ is continuous at $\alpha$. If $f$ is continuous at every
$\alpha \in S$, we say that $f$ is continuous.

\begin{theorem}
  Let $S, T \subset \mathbb{C}$.
  \begin{enumerate}
    \item{
      If $f : S \to \mathbb{C}$ and
      $g: S \to \mathbb{C}$ are continuous, then $f + g$, $f - g$,
      $f \cdot g$ are continuous. If $g(z) \neq 0$, then
      $f / g$ is continuous at $z$.
    }
    \item{
      If $f : S \to \mathbb{C}$ and $g : T \to \mathbb{C}$ are continuous
      and $f(S) \subset T$, then $g \circ f$ is continuous.
    }
  \end{enumerate}
\end{theorem}

\begin{xmpl}
  \begin{enumerate}
    \item{
      $f(z) = z$ is continuous on $\mathbb{C}$.
      $f(z) = C$ is continuous on $\mathbb{C}$ for constant
      $C \in \mathbb{C}$.
    }
    \item{
      Let $a_0, a_1, \dots, a_n \in \mathbb{C}$. A function
      $P(z) = \sum_{k=0}^n a_k z^k$, i.e. a polynomial with
      complex coefficients, is continuous. If $a_n \neq 0$,
      we say that $P$ has degree $n$. If $a_n = 1$ we say
      that $P$ is monic.
    }
    \item{
      $\mathrm{Re}~z$ and $\mathrm{Im}~z$ are continuous on $\mathbb{C}$.
      Fix $\alpha \in \mathbb{C}$. Given $\varepsilon > 0$, let
      $\delta = \varepsilon > 0$. If $|z - \alpha| < \delta$, then
      $$
           |\mathrm{Re}~z - \mathrm{Re}~\alpha|
      =    |\mathrm{Re}~(z - \alpha)|
      \leq |z - \alpha| < \delta = \varepsilon.
      $$
    }
  \end{enumerate}
\end{xmpl}

\begin{theorem}
  Let $S \subset \mathbb{C}$ be compact. Let $f : S \to \mathbb{C}$ be
  continuous. Then
  \begin{enumerate}
    \item{
      $f(S) = \{ f(z) : z \in S \}$ is compact.
    }
    \item{
      $f$ is bounded on $S$, i.e. $\exists M \in \mathbb{R}$,
      $M \geq 0$ such that $|f(z)| \leq M, \forall z \in S$.
    }
  \end{enumerate}
\end{theorem}

\begin{proof}
  \begin{enumerate}
    \item{
      Let $(w_n)$ be a sequence in $f(S)$. Then $w_n = f(z_n)$
      for some $z_n \in S$. Since $S$ is compact, $(z_n)$ has a convergent
      subsequence $(z_{n_k})$ whose limit, say $z_0$, is in $S$. Since
      $f$ is continuous, $w_{n_k} = f(z_{n_k}) \to f(z_0) \in f(S)$,
      so $(w_n)$ has a convergent subsequence.
    }
    \item{
      This follows from the first part since $f(S)$ is bounded.
    }
  \end{enumerate}
\end{proof}

\begin{defn}[Relatively open and closed]
  Let $U \subset S \subset \mathbb{C}$. We say that $U$ is
  \emph{relatively open} in $S$ if $\forall z_0 \in U$,
  $\exists r > 0$ such that $D(z, r) \cap S \subset U$.
  Equivalently, there exists an open set $V \subset \mathbb{C}$
  such that $U = V \cap S$.

  A set $K \subset S$ is \emph{relatively closed} if
  $S \backslash K$ is relatively open in $S$.
  Equivalently, $K$ is relatively closed in $S$ if there exists a closed set
  $F \subset \mathbb{C}$ such that $K = F \cap S$.

\end{defn}

The set of relatively open subsets of $S$ gives a topology of $S$,
noting that $\varnothing$ and $S$ are both relatively open and closed.
If $S$ is open, then $U$ is relatively open in $S$ if and only if $U$
is open and $U \subset S$. Note that a set relatively open in
$\mathbb{R}$ agrees with an open set in $\mathbb{R}$ under the usual
topology.

\begin{theorem}
  Let $f : S \to T \subset \mathbb{C}$. Then $f$ is continuous on
  $S$ if and only if for all relatively open sets $U$ in $T$,
  the preimage $f^{-1}(T)$ is relatively open in $S$.
\end{theorem}

Note that for $f(z) = |z - z_0|$, $D(z, z_0) = f^{-1}((-\infty, r))$
and $\mathbb{C} - D(z, z_0) = f^{-1}((r, \infty))$.

\begin{defn}[Connected set]
  We say that $S \subset \mathbb{C}$ is \emph{connected} if the only
  relatively clopen sets in $S$ are $\varnothing$ and $S$.

  $S$ is \emph{path-connected} if $\forall z_0, w_0 \in S$ there exists
  a continuous function $\gamma : [0, 1] \to S$ such that $\gamma(0) = z_0$,
  $\gamma(1) = w_0$. $S$ is path-connected implies that $S$ is connected.
  If $S$ is open, then connectedness implies path-connectedness.
\end{defn}

\begin{defn}[Complex domain]
  A \emph{complex domain} or \emph{domain} is a nonempty connected open
  subset of $\mathbb{C}$.
\end{defn}
