\section{Laurent Series}

\begin{defn}[Laurent Series]
A \emph{Laurent series} is a series of the form
$$
\sum_{n=-\infty}^\infty a_n z^n, \quad a_n \in \mathbb{C}
$$
which converges if both
$$
\sum_{n=0}^\infty a_n z^n
$$
and
\begin{align*}
   \sum_{n=-\infty}^{-1} a_n z^n
&= \sum_{m=1}^\infty a_{-m} z^{-m} \\
&= \sum_{m=1}^\infty a_{-m} \left(\frac{1}{z}\right)^m.
\end{align*}
\end{defn}

Suppose $\sum a_n z^n$ has radius $R_+$ and
$\sum a_{-m} w^m$ is $R_-$. Then the Laurent series
converges when $|z| < R_+$ and $\left|\frac{1}{z}\right| < R_-$,
i.e. $\frac{1}{R_-} < |z| < R_+$. Therefore the series converges
on an annulus.

Let
$$
f(z) = \sum_{n=-\infty}^\infty a_n z^n = f_+(z) + f_-(z)
$$
on $A$ where
$$
f_-(z) = \sum_{n=-\infty}^{-1} a_n z^n = g\left(\frac{1}{z}\right)
$$
and
$$
g(w) = \sum_{m=1}^\infty a_{-m} w^m.
$$
Then $f_+$ is holomorphic on $\{ |z| < R_+ \}$ and
$g$ is holomorphic on $\{ |z| < R_- \}$, so
$f$ is holomorphic on this annulus. Furthermore
$$
  f^\prime(z)
= \sum_{m=1}^\infty n a_n z^{n-1}
+ f_-^\prime(z)
$$
where
\begin{align*}
   f_-^\prime(z)
&= \sum_{m=1}^\infty
     m a_{-m}
     \left(\frac{1}{z}\right)^{m-1}
     \frac{-1}{z^2} \\
&= \sum_{m=1}^\infty (-m) a_{-m} z^{-(m+1)} \\
&= \sum_{n=-\infty}^{-1} n a_n z^{n-1}
\end{align*}
so that
$$
f^\prime(z) = \sum_{n=-\infty}^\infty n a_n z^{n-1}
$$
on the annulus $A$.

\begin{theorem}
Let $0 \leq r < R \leq \infty$. Suppose that $f$ is holomorphic
on the annulus $A(0, r, R)$. Then $f$ has a Laurent series expansion
where
$$
a_n = \frac{1}{2 \pi i} \int_{|z| = t} \frac{f(z)}{z^{n+1}} \dif z
$$
for $n \in \mathbb{Z}$, $t \in (r, R)$.
\end{theorem}

\begin{proof}
Observe from Cauchy's formula
that the value of $a_n$ does not depend on the choice of $t$.
Let $w \in A$. Choose $s, S$ such that
$$
r < s < |w| < S < R.
$$
Let $\varepsilon = \frac{\min \{ |w| - s, S - |w| \}}{2}$.
Now let $J_1 = \{ |z| = S \}$, $J_2 = \{ |z| = s \}$,
$J_3 = \{ |z - w| = \varepsilon \}$.

The function $z \mapsto \frac{f(z)}{z - w}$ is holomorphic
on each of these circles and the domain inside $J_1$
and outside $J_2, J_3$. From Cauchy's formula,
$$
  \int_{J_1}
    \frac{f(z)}
         {z - w}
    \dif z
- \int_{J_2}
    \frac{f(z)}
         {z - w}
    \dif z
= 2 \pi i f(w).
$$
Now expand $\frac{1}{z - w}$ as a Laurent series
in $w$. If $z \in J_1$ then $|z| = S > |w|$, so
\begin{align*}
   \frac{1}{z - w}
&= \frac{\frac{1}{z}}
        {1 - \frac{w}{z}}
 = \frac{1}{z}
   \sum_{n=0}^\infty
     \left(
       \frac{w}{z}
     \right)^n \\
&= \sum_{n=0}^\infty
     \frac{w^n}{z^{n+1}}.
\end{align*}
If $z \in J_2$ then
\begin{align*}
   \frac{1}{z - w}
&= \frac{\frac{1}{w}}
        {\frac{z}{w} - 1}
 = -\frac{1}{w}
    \sum_{n=0}^\infty
      \left(
        \frac{z}{w}
      \right)^n \\
&= -\sum_{n=0}^\infty
      \frac{z^n}{w^{n+1}}
 = -\sum_{m=-\infty}^{-1} \frac{w^n}{z^{n+1}}
\end{align*}
under the change of index $m =-1 - n$.
Then
\begin{align*}
   2 \pi i f(w)
&= \int_{J_1}
     \sum_{n=0}^\infty
       \frac{f(z)}
            {z^{n+1}}
       w^n
       \dif z
 + \int_{J_2}
     \sum_{n=-\infty}^{-1}
       \frac{f(z)}
            {z^{n+1}}
       w^n
       \dif z.
\end{align*}
If the sums commute with the integrals, then
we have
\begin{align*}
2 \pi i f(w)
&= \sum_{n=0}^\infty
     \int_{J_1}
       \frac{f(z)}
            {z^{n+1}}
       w^n
       \dif z
 + \sum_{n=-\infty}^{-1}
     \int_{J_2}
       \frac{f(z)}
            {z^{n+1}}
       w^n
       \dif z \\
&= \sum_{n=0}^\infty
     a_n w^n
 + \sum_{n=-\infty}^{-1}
     a_n w^n
 = \sum_{n=-\infty}^\infty a_n w^n.
\end{align*}
To show that this is true, we need to show that
$$
\sum_{n=0}^\infty \frac{f(z)}{z^{n+1}} w^n
$$
converges uniformly on $J_1$ and
$$
\sum_{n=-\infty}^{-1} \frac{f(z)}{z^{n+1}} w^n
$$
converges uniformly on $J_2$. Note that
\begin{align*}
      \left|
        \frac{f(z)}{z^{n+1}} w^n
      \right|
&\leq \frac{\| f \|_{J_1}}
           {S^{n+1}}
      |w|^n
 =    \frac{\| f \|_{J_1}}
           {S^{n+1}}
      \left(
        \frac{|w|}{S}
      \right)^n,
\end{align*}
and since $\frac{|w|}{S} < 1$ the series converges
from the comparison test.

Next
\begin{align*}
       \left|
        \frac{f(z)}
             {z^{n+1}}
        w^n
       \right|
& \leq \frac{\| f \|_{J_2}}
            {S^{n+1}}
  =    \left(
         \frac{S}{|w|}
       \right)^{-n}
\end{align*}
and $\left|\frac{S}{|w|}\right| < 1$.
\end{proof}

We will later see that the Laurent series expansion is unique,
and use this theorem to calculate integrals.

Recall that
$$
\frac{1}{1 - z} = \sum_{n=0}^\infty z^n, \quad |z| < 1.
$$
For $|z| < |z_0|$, $\left|\frac{z}{z_0}\right| < 1$ so that
\begin{align*}
   \frac{1}{z - z_0}
&= \frac{-\frac{1}{z_0}}{1 - \frac{z}{z_0}}
 = -\frac{1}{z_0} \sum_{n=0}^\infty \left(\frac{z}{z_0}\right)^n \\
&= \sum_{n=0}^\infty -\frac{1}{z_0^{n+1}} z^n.
\end{align*}
If $|z| > |z_0|$, then $\left|\frac{z_0}{z}\right| < 1$ so
\begin{align*}
   \frac{1}{z - z_0}
&= \frac{\frac{1}{z}}{1 - \frac{z_0}{z}}
 = \frac{1}{z}
   \sum_{n=0}^\infty
     \left(\frac{z_0}{z}\right)^n \\
&= \sum_{n=0}^\infty
     \frac{z_0^n}{z^{n+1}} \\
&= \sum_{m=-\infty}^{-1} \frac{z^m}{z_0^{m+1}}.
\end{align*}

To find the Laurent series of
$$
  f(z)
= \frac{1}{(z-1)(z-2)}
= \frac{1}{z - 2} - \frac{1}{z - 1}
$$
and we can then find a Laurent series of $f$
on the annulus $A(0, 1, 2)$.

Suppose $f$ is analytic at $z_0$. The power series coefficients can
be recovered from the Taylor series expansion. How can we recover
the Laurent series coefficients on an annulus when $f$ is not
holomorphic at $z_0$?

\begin{xmpl}
  \begin{itemize}
    \item{
      The Laurent expansion of $e^{\frac{1}{z}}$ at $0$ is
      $$
        e^{\frac{1}{z}}
      = \sum_{n=0}^{\infty}
          \frac{\left(\frac{1}{z}\right)^n}
               {n!}
      = \sum_{n=-\infty}^0 \frac{z^n}{(-n)!}
      $$
      so that
      $$
        \int_{|z| = 1}
          e^{\frac{1}{z}} \dif z
      = 2 \pi i a_{-1} = 2 \pi i.
      $$
    }
    \item{
      If $f$ is holomorphic on $A(z_0, r, R)$ then
      there exists a unique $(a_n)_{n \in \mathbb{Z}}$ such that
      $$
      f(z) = \sum_{n=-\infty}^{\infty} a_n (z - z_0)^n
      $$
      on that annulus where
      $$
      a_n = \frac{1}{2 \pi i}
            \int_{|z - z_0| = t}
              \frac{f(z)}{(z - z_0)^{n+1}}
              \dif z,
      $$
      $n \in \mathbb{Z}$.
    }
  \end{itemize}
\end{xmpl}

\section{Singularities}
\begin{defn}[Isolated singularity, principal part]
Suppose $f$ is holomorphic on $U$,
$z_0 \notin U$, but $\exists r > 0$ such that
$D(z_0, r) \backslash \{ z_0 \} \subset U$. Then
we say that $z_0$ is an \emph{(isolated) singularity} of $f$.
Then $f$ has a Laurent expansion in $\{ 0 < |z - z_0| < r\}$
given by
$$
\sum_{n=-\infty}^\infty a_n (z - z_0)^n
$$
where
$$
a_n = \frac{1}{2 \pi i}
      \int_{|z - z_0| = t}
        \frac{f(z)}{(z - z_0)^{n+1}}
        \dif z.
$$
We call
$$
\sum_{n=-\infty}^{-1} (z - z_0)^n
$$
the \emph{principal part} of $f$ at $z_0$.
\end{defn}

\begin{defn}[Types of singularities]
Singularities may have the following forms:
\begin{itemize}
  \item{
    If $a_n = 0$, $\forall n \leq -1$ then the
    principal part vanishes. If we define $f(z_0) = a_0$,
    then we extend $f$ to be holomorphic on
    $U \cup \{ z_0 \}$. Such $z_0$ is called a \emph{removable
    singularity}.
  }
  \item{
    Not all $n \leq -1$ are zero, and there are only
    finitely many that are nonzero. Then
    $\exists m \in \mathbb{N}$ such that $a_{-m} \neq 0$
    and $\forall n < -m$, $a_n = 0$, so that
    $f(z) = \sum_{n=-m}^\infty a_m (z - z_0)^n$. In this case
    $z_0$ is called a \emph{pole} of $f$, and the
    \emph{order} of the pole is $m$. A pole of order 1 is called
    a \emph{simple pole}. For example, $\frac{1}{z}$ has a simple
    pole at $0$.
  }
  \item{
    There are infinitely many $n \leq -1$ such that $a_n \neq 0$.
    In this case $z_0$ is called an \emph{essential singularity}.
    For example, $e^{\frac{1}{z}}$ has an essential singularity at
    $0$.
  }
\end{itemize}

Throughout the following, let $z_0$ be a singularity of $f$ and the
Laurent expansion of $f$ at $z_0$ be
$$
f(z) = \sum_{n=-\infty}^{\infty} a_n (z - z_0)^n, \quad
0 < |z - z_0| < r.
$$
Assume the following:
\begin{enumerate}
  \item{
    The $a_n$ are not all 0.
  }
  \item{
    There are at most finitely many $n < 0$ such that
    $a_n \neq 0$, i.e. $z_0$ is not an essential singularity.
  }
\end{enumerate}
Then there exists a unique $m \in \mathbb{Z}$ such that
$a_m \neq 0$ and $a_n = 0$, $\forall n < m$. We write
$$
f(z) = \sum_{n=m}^{\infty} a_m (z - z_0)^n.
$$
We say that the order of $f$ at $z_0$ is $m$ and write
$\ord_{z_0} f = m$.

\begin{itemize}
  \item{
    If $m \geq 0$, then $z_0$ is a removable singularity.
  }
  \item{
    If $m \geq 1$, then after removing the singularity
    $z_0$ is a zero of $f$ since $a_0 = 0$ in the holomorphic
    continuation. We say that such a zero has order $m$. A
    zero of order 1 is called a simple zero.
    $z_0$ is a zero of $f$ of order $m$ if and only if
    $$
    f^{(n)}(z_0) = 0, \quad 0 \leq n \leq m - 1
    $$
    and $f^{(m)}(z_0) \neq 0$.
  }
  \item{
    If $m < 0$, then $z_0$ is a pole of order $|m|$.
    Note that $\ord z_0 f = m$ if and only if there exists
    some function $g$ holomorphic on $D(z_0, r)$ for some $r$
    such that $f(z) = (z - z_0)^m g(z)$ on this disk.
    To see this, write
    \begin{align*}
       f(z)
    &= \sum_{n=m} a_n (z - z_0)^n
     = \sum_{k=0}^\infty a_{m+k} (z - z_0)^{m+k} \\
    &= (z - z_0)^m \sum_{k=0}^\infty a_{m+k} (z - z_0)^k \\
    &= (z - z_0)^m g(z).
    \end{align*}
    Note also that $g(z_0) = a_m \neq 0$
  }
\end{itemize}
\end{defn}

Note that if $g, h$ are holomorphic at $z_0$ such that
$g(z_0)$ and $h(z_0)$ are not 0, then $g \cdot h$ and
$\frac{g}{h}$ are also holomorphic at $z_0$, and
$(g \cdot h)(z_0)$ and $\frac{g}{h}(z_0)$ are not 0.
The radius of the holomorphic domain for the product or quotient
may be different.

If $\ord z_0 f = m$ and $\ord z_0 g = n$, then there exists
$F$ and $G$ holomorphic at $z_0$, with $F(z_0), G(z_0) \neq 0$
such that
$f(z) = (z - z_0)^m F(z)$ and $g(z) = (z - z_0)^m G(z)$
in a disc centered at $z_0$ except at $z_0$. Then
$$
f(z) g(z) = (z - z_0)^{m + n} F(z) G(z)
$$
and
$$
\frac{f(z)}{g(z)} = (z - z_0)^{m - n} \frac{F(z)}{G(z)}.
$$
Therefore
$$
  \ord_{z_0} (f \cdot g)
= \ord_{z_0} f + \ord_{z_0} g, \quad
  \ord_{z_0} \left(\frac{f}{g}\right)
= \ord_{z_0} f - \ord_{z_0} g.
$$
Furthermore, for any nonzero constant $C$,
$\ord z_0 C = 0$, so it follows that
$\ord_{z_0} \frac{1}{f} = -\ord_{z_0} f$.

That is, there is a group homomorphism
$\mathrm{\ord}_{z_0} : \mathrm{Mer}(\{z_0\})^\times \to \mathbb{Z}$.

\begin{xmpl}
$\ord_0 z = 1$ since $z = 0$ at 0, $z^\prime = 1 \neq 0$ at 0.
$\ord_0 \sin z = 1$ since $\sin 0 = 0$, $\cos 0 = 1 \neq 0$.
Then $\ord_0 \frac{\sin z}{z} = 0 = \ord \frac{z}{\sin z}$.
Since $f(z) = \frac{\sin z}{z}$ is holomorphic on
$\mathbb{C} \backslash \{ 0 \}$, and
$\ord_0 \frac{\sin z}{z} = 0$, 0 is a removable singularity of
$f$, and so $f$ can be extended to be holomorphic at 0.
\end{xmpl}

\begin{defn}[Meromorphic function]
Let $U$ be open. Suppose that $S \subset U$ has no accumulation points
in $U$. If $f$ is holomorphic on $U \backslash S$ and every $z_0 \in
S$
is a pole of $f$, we say that $f$ is \emph{meromorphic} on $U$.
\end{defn}

We may construct a meromorphic function by the quotient of two
holomorphic functions. Assume $U$ is a domain and $f, g$ are
holomorphic on $U$. Suppose $g$ is not constant 0. Let $Z$ denote
the zeros of $g$. Then $Z$ has no accumulation points in $U$.
Let $h \frac{f}{g}$. Then $h$ is holomorphic on $U \backslash Z$,
and every $z_0 \in Z$ is a singularity of $h$. Furthermore since
$\ord_{z_0} h = \ord_{z_0} f - \ord_{z_0} g$. $z_0$ is either
removable (if $\ord z_0 h \geq 0$ or a pole (if $\ord z_0 h < 0$).
After removing all removable singularities, we get a meromorphic
function on $U$.

\begin{xmpl}
$\tan z = \frac{\sin z}{\cos z}$, $\cot z = \frac{\cos z}{\sin z}$
are meromorphic on $\mathbb{C}$.

The zeros of $\cos z$ are at $n \pi + \frac{\pi}{2}$, $n \in
\mathbb{Z}$, and at these zeros
$$
\cos(n \pi + \frac{\pi}{2}) = 0, \quad
-\sin(n \pi + \frac{\pi}{2}) \neq 0
$$
and so the order of $\cos z$ is 1 at each such $z_0$ is
zero, and the order of $\sin z$ is 0 at each $z_0$, so
$\ord_{z_0} \tan z = -1$. Therefore every singularity of
$\tan z$ (at $n \pi + \frac{\pi}{2}$) is a simple pole.
Similarly the singularities of $\cot z$ (at $n \pi$)
are simple poles.
\end{xmpl}

\begin{defn}[Rational function]
If $P$, $Q$ are polynomials, then $R = \frac{P}{Q}$ is called a
rational function, which is meromorphic on $\mathbb{C}$.
\end{defn}

\begin{theorem}
  The following are equivalent.
  \begin{enumerate}[(i)]
    \item
      {
        $z_0$ is a removable singularity.
      }
    \item
      {
        $\lim_{z \to z_0} f(z)$ exists and is finite.
      }
    \item
      {
        $\exists r > 0$ such that $f$ is bounded on
        $D(z_0, r) \backslash \{ z_0 \}$.
      }
  \end{enumerate}
\end{theorem}

\begin{proof}
  \begin{itemize}
    \item[(i) $\implies$ (ii)]
      {
        After removing the singularity $z_0$ we have a holomorphic
        function $g$, so that $\lim_{z \to z_0} f(z) = g(z_0) \in \mathbb{C}$.
      }
    \item[(ii) $\implies$ (iii)]
      {
        Since $\lim_{z \to z_0} f(z) = C \in \mathbb{C}$, there exists
        an $r > 0$ such that $|f(z) - f(z_0)| < 1$ if $|z - z_0| < r$.
        Then $|f(z)| < |f(z_0)| + 1$ on $D(z_0, r) \backslash \{ z_0 \}$.
      }
    \item[(iii) $\implies$ (i)]
      {
        Suppose $|f(z)| \leq M \in \mathbb{R}$ for all
        $z \in D(z_0, r) \backslash \{ z_0 \}$.
        Recall that
        $$
          a_n
        = \frac{1}{2 \pi i}
          \int_{|z - z_0| = t}
            \frac{f(z)}{(z - z_0)^{n+1}}
            \dif z
        $$
        so that
        \begin{align*}
              |a_n|
        &\leq \frac{1}{2 \pi} \frac{M}{t^{n+1}} 2 \pi t \\
        &=    M t^{-n}, \forall 0 < t < r.
        \end{align*}
        If $n < 0$, $-n > 0$, so
        $\lim_{t \to 0} t^{-n} = 0$. Since
        $|a_n| \leq M t^{-n}$, we have $a_n = 0$.
      }
  \end{itemize}
\end{proof}

\begin{theorem}
$z_0$ is a pole of $f$ if and only if $\lim_{z \to z_0} |f(z)| = \infty$.
\end{theorem}

\begin{proof}
$z_0$ is a pole of $f$ if and only if $z_0$ is a zero of $\frac{1}{f}$
after analytic extension, which is true if and only if
$\lim_{z \to z_0} \frac{1}{f} = 0$, which is true if and only if
$\lim_{z \to z_0} |f(z)| = \infty$. Recall that $S \subset \mathbb{C}$
is dense in $\mathbb{C}$ if $\bar{S} = \mathbb{C}$, which is true
if and only if $\forall w_0 \in \mathbb{C}$, $\forall r > 0$,
$D(w_0, r) \cap S \neq \varnothing$.
\end{proof}

\begin{theorem}
$z_0$ is an essential singularity of $f$ if and only if
$\forall t \in (0, r)$, $f(D(z_0, t) \backslash \{ z_0 \})$ is
dense in $\mathbb{C}$.
\end{theorem}

\begin{proof}
\begin{itemize}
  \item[($\impliedby$)]{
    Assume that $f(D(z_0, t) \backslash \{ z_0 \})$ is dense in
    $\mathbb{C}$, for all $t \in (0, r)$.

    If $z_0$ is
    removable, then $\exists t \in (0, r)$ such that $f$ is
    bounded on $D(z_0, t) \backslash \{ z_0 \}$. Then
    $f(D(z_0, t) \backslash \{ z_0 \}) \subset \bar{D}(0, M)$
    for some $M > 0$, so $f(U)$ is not dense in $\mathbb{C}$
    (contrad.).

    If $z_0$ is a pole, then from
    $\lim_{z \to z_0} |f(z)| = \infty$ we see $\exists t \in (0, r)$
    such that $|f(z)| > 1$ on $D(z_0, t) \backslash \{ z_0 \}$.
    Then $f(D(z_0, t) \backslash \{ z_0 \})
    \subset \{ z \in \mathbb{C} : |z| > 1 \}$, a contradiction.
  }
  \item[($\implies$)]{
    Assume $z_0$ is an essential singularity. Suppose $\exists t \in
    (0, r)$ such that $f(D(z_0, t) \backslash \{ z_0 \})$ is not dense
    in $\mathbb{C}$. Then $\exists w_0, R > 0$ such that
    $
    D(w_0, R) \cap f(D(z_0, t) \backslash \{ z_0 \}) = \varnothing.
    $
    Then $\forall z \in D(z_0, t) \backslash \{ z_0 \}$,
    $|f(z) - w_0| \geq R$. Let $g(z) = \frac{1}{f(z) - w_0}$ on
    $D(z_0, t) \backslash \{ z_0 \}$. Then $g$ is holomorphic on
    $D(z_0, t) \backslash \{ z_0 \}$, and $|g(z)| \leq \frac{1}{R}$ on
    this punctured disk. So $z_0$ is a removable singularity of $g$.
    We have $f(z) = w_0 + \frac{1}{g(z)}$, $z \in D(z_0, t) \backslash
    \{ z_0 \}$. If $g(z_0) \neq 0$, then $z_0$ is a removable
    singularity of $f$. If $g(z_0) = 0$, then $z_0$ is a pole of $f$.
    This completes the contradiction.
  }
\end{itemize}
\end{proof}
