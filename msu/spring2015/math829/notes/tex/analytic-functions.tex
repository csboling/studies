\begin{defn}[Analytic function]
  Let $U \subset \mathbb{C}$ be open and $f : U \to \mathbb{C}$.
  We say that $f$ is \emph{analytic} on $U$ if
  $$
  \forall z_0 \in U, \exists r > 0, (a_n)_{n=0}^\infty .
  f(z) = \sum_{n=0}^\infty a_n (z - z_0)^n
  $$
  on $D(z_0, r) \subset U$.
\end{defn}

From the differentiability of power
series, we see that $f$ is holomorphic on $D(z_0, r)$ and
$f^\prime$ is also the sum of the power series in $D(z_0, r)$.
It follows that $f$ is holomorphic on $U$ and $f^\prime$ is also
analytic on $U$. We conclude that
$f \in C^\infty(U, \mathbb{C})$. Moreover,
$$
a_n = \frac{f^{(n)}(z_0)}{n!}.
$$
We will prove later that holomorphic functions are analytic,
and thus every complex-differentiable function is infinitely
differentiable. Such a phenomenon does not exist in real analysis.

\begin{defn}[Zeros of a function]
Let $f$ be analytic on $U$ and let $z_0 \in U$. A
\emph{zero} of $f$ is some $z \in U$ such that $f(z) = 0$.
\end{defn}

Let the power series expansion of $f$ at $z_0$ be
$$
\sum_{n=0}^\infty a_n (z - z_0)^n, z \in D(z_0, r).
$$
There are 3 possibilities:
\begin{enumerate}
  \item{
    $a_0 \neq 0$. Then $f(z_0) = a_0 \neq 0$. Since $f$ is
    continuous, there exists an $r^\prime \in (0, r)$ such that
    $f$ has no zero on $D(z_0, r^\prime)$. In this case
    $U \backslash Z = U$.
  }
  \item{
    $a_0 = 0$ but not all $a_n = 0$. Let $m$ be the first
    $n$ such that $a_n \neq 0$. Then
    $$
    f(z) = \sum_{n=m}^\infty a_n (z - z_0)^n, |z - z_0| < r.
    $$
    Let
    $$
    g(z) = \sum_{n=0}^\infty a_n (z - z_0)^{n - m} a_n (z - z_0)
         = \sum_{k=0}^\infty a_{m+n} (z - z_0)^k, |z - z_0| < r.
    $$
    Then $g$ is continuous on $D(z_0, r)$ and
    $f(z) = (z - z_0)^m g(z)$ on $D(z_0, r)$.
    $g(z_0) = a_m \neq 0$, so $g$ is nonvanishing on $D(z_0, r^\prime)$
    for some $r^\prime \in (0, r)$. Since $(z - z_0)^m = 0$ only at
    $z = z_0$, we see that $f$ has only one zero in
    $D(z_0, r^\prime)$, which is at $z_0$.

    In this case $U \backslash Z = U \backslash \{ z_0 \}$
    is relatively open in $U$.
  }
  \item{
    All $a_n = 0$. Then $f = 0$ on $D(z_0, r)$. In this case
    $U \backslash Z = \varnothing$.
  }
\end{enumerate}
It follows that the set of zeros of $f$ is relatively closed in
$\mathrm{dom} f$.

Recall that an \emph{accumulation point} of a set $S$ is some
$z_0 \in \mathbb{C}$ such that $\forall r > 0$,
$D(z_0, r) \cap S$ is infinite. This implies $z_0 \in \bar{S}$.
If $S = \mathbb{N}$, $S$ has no accumulation point.

\begin{lemma}
  $z_0$ is an accumulation point of $S$ iff. exists a
  sequence $(z_n)$ in $S \backslash \{ z_0 \}$ such that $z_n \to z_0$.
\end{lemma}

\begin{proof}
  \begin{itemize}
    \item[($\implies$)]{
      $\forall n \in \mathbb{N}$, the set
      $S \cap (D(z_0, \frac{1}{n}) \backslash \{ z_0 \})$ is not empty.
      Pick $z_n$ in this set. Then $(z_n)$ is a sequence in
      $S \backslash \{z_0\}$, and $|z_n - z_0| < \frac{1}{n}$.
    }
    \item[($\impliedby$)]{
      Suppose $z_0$ is not an accumulation point of $S$. Then
      $\exists r > 0$ such that $D(z_0, r) \cap S$ is finite.
      Then $D(z_0, r) \cap \{ z_n : n \in \mathbb{N} \}$ is
      finite, so $\inf_{n \in \mathbb{N}} |z_n - z_0| > 0$,
      so $z_n {\not \to} z_0$.
    }
  \end{itemize}
\end{proof}

\begin{theorem}[Uniqueness Theorem]
  Let $U$ be a domain (a connected nonempty open set).
  \begin{enumerate}[(a)]
    \item{
      Suppose $f$ is analytic on $U$, and is not constantly 0.
      Then the set of zeros of $f$ has no accumulation point on $U$.
    }
    \item{
      Supose $f$, $g$, are analytic on $U$ and there is $S \subset U$
      with an accumulation point in $U$ such that $f(z) = g(z)$
      for all $z \in S$. Then $f = g$ on $U$.
    }
  \end{enumerate}
\end{theorem}

\begin{proof}
  \begin{enumerate}[(a)]
    \item{
      Let $Z \subset U$ denote the set of zeros of $f$. Then $Z$ is
      relatively closed in $U$. Let $A$ be the set of
      accumulation points of $Z$ in $U$. Then $A \subset Z$.
      Since $f$ is not zero everywhere, $Z \subsetneq U$.

      We now show that $A$ is relatively closed.
      Let $(z_n)$ be a sequence in $A$ such that $z_n \to z_0 \in U$.
      If some $z_n = z_0$, then $z_0 \in A$. If
      $\forall n, z_n \neq z_0$, then there is a sequence in
      $A \backslash \{ z_0 \} \subset Z$ such that $z_n \to z_0$.
      We have seen that $z_0$ is an accumulation point of $Z$.
      Again we get $z_0 \in A$. Therefore $A$ is relatively closed in
      $U$.

      Next, we show that $A$ is relatively open. Suppose $z_0 \in A$. Then for any
      $r > 0$, $D(z_0, r)$ contains infinitely many zeros of $f$, and so
      we must have $f = 0$ on some disk around $z_0$, i.e.
      $D(z_0, r^\prime) \subset Z$. Then $D(z_0, r^\prime) \subset A$,
      since every one of its points is an accumulation point, so
      $A$ is relatively open.

      Since $A$ is relatively closed and relatively open in $U$, and since
      $U$ is connected, we have either $A = \varnothing$ or $A = U$. But
      $A \subset Z \subsetneq U$, so $A \neq U$. Therefore $A = \varnothing$,
      as desired.
    }
    \item{
      Let $h = f - g$. Then $h$ is analytic. The assumption implies that
      $S \subset Z$, where $Z$ is the set of zeros of $h$. Since $S$ has
      an accumulation point on $U$, it follows that $h = 0$, so $f = g$.
    }
  \end{enumerate}
\end{proof}

\begin{remark}
  \begin{enumerate}
    \item{
      We will show later that holomorphic functions are analytic,
      so that the uniqueness theorem holds for analytic functions.
    }
    \item{
      We know that the complex exponential function is an extension
      of the real exponential function. The uniqueness theorem shows
      that there is only one extension so that the function is
      holomorphic on $\mathbb{C}$. In fact, if $f$ is holomorphic on
      $\mathbb{C}$ such that $f(x) = e^x$ for all $x \in \mathbb{R}$,
      then $f$ and $g(z) = e^z$ are both holomorphic on $\mathbb{C}$
      and agree on $\mathbb{R}$, which has an accumulation point.
      Therefore the theorem shows that $f = g$.
    }
    \item{
      This also works for $\sin z$ and $\cos z$. Another proof of
      $\sin^2 z + \cos^2 z = 1$ on $\mathbb{C}$ is that
      $\sin^2 x + \cos^2 x = 1$ on $\mathbb{R}$ and that
      $\sin^2 z$ and $\cos^2 z$ are holomorphic.
    }
  \end{enumerate}
\end{remark}
