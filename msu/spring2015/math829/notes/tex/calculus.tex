\section{Complex Differentiability}

Let $U$ be open and $z_0 \in U$, and let
$f : U \to \mathbb{C}$. If the limit
$$
\lim_{z \to z_0} \frac{f(z) - f(z_0)}{z - z_0}
$$
exists, we say that $f$ is complex-differentiable at
$z_0$ and denote this limit by $f^\prime(z_0)$. We may also say
$$
f^\prime(z_0) = \lim_{h \to 0} \frac{f(z_0 + h) - f(z_0)}{h}.
$$

\begin{defn}[Holomorphic and Entire Functions]
If $f$ is complex differentiable at every $z_0 \in U$, then
we say that $f$ is complex differentiable on $U$, we also say
that $f$ is \emph{holomorphic} on $U$. If $f$ is holomorphic on
$\mathbb{C}$, we say that $f$ is an \emph{entire function}.
\end{defn}

\begin{xmpl}
  \begin{enumerate}
    \item{
      Let $w_0 \in \mathbb{C}$, $f(z) = w_0$. Such $f$ is
      differentiable and $f^\prime(z) = 0$.
    }
    \item{
      Let $f(z) = z$. Then $\frac{f(z) - f(z_0)}{z - z_0} = 1$
      and so $f$ is differentiable and $f^\prime(z) = 1$.
    }
  \end{enumerate}
\end{xmpl}

\begin{obsv}
  Assume $f$ and $g$ are differentiable at $z_0$.
  \begin{enumerate}
    \item{ Then $f$ and
      $f$ and $g$ are continuous at $z_0$.
    }
    \item{
      $(f + g)^\prime(z_0) = f^\prime(z_0) + g^\prime(z_0)$.
    }
    \item{
      $$
      (f\cdot g)^\prime(z_0) = f^\prime(z_0) g(z_0) + f(z_0) g^\prime(z_0).
      $$
    }
    \item{
      If $g(z_0) \neq 0$, then
      $$
        \left(\frac{f}{g}\right)^\prime(z_0)
      = \frac{f^\prime(z_0) g(z_0) - f(z_0) g^\prime(z_0)}{g(z_0)^2}.
      $$
      In particular, when $f(z) = 1$,
      $$
      \left(\frac{1}{g}\right)^\prime(z_0) = -\frac{g^\prime(z_0)}{g(z_0)^2}.
      $$
    }
    \item{
      If $g$ is differentiable at $f(z_0)$, then
      $$
      (g \circ f)^\prime(z_0) = g^\prime(f(z_0)) f^\prime(z_0).
      $$
    }
  \end{enumerate}
\end{obsv}

\begin{xmpl}
  \begin{enumerate}
    \item{
      $\frac{d}{dz} (z^n) = n z^{n-1}$ for any $z \in \mathbb{C}$, $n \in \mathbb{N}$.
    }
    \item{
      The polynomials $P(z) = \sum_{k=0}^n a_k z^k$ are entire, and
      $\deg P^\prime \deg P - 1$ for $\deg P \geq 1$. If $\deg P = 0$ then
      $P^\prime(z) = 0$.
    }
    \item{
      For $n \in \mathbb{C}$, $f(z) = z^{-n}$ is holomorphic on $\mathbb{C} - \{ 0 \}$.
      We see from the quotient rule that
      $$
      f^\prime(z) = \frac{-n z^{n-1}}{(z^n)^2} = -n z^{-n - 1},
      $$
      so the derivative rule for $z^n$ holds for any $n \in \mathbb{Z}$.
    }
  \end{enumerate}
\end{xmpl}

\section{Cauchy-Riemann Equations}

\begin{theorem}
  Let $U \subset \mathbb{C}$ be open and $z_0 = x_0 + i y_0 \in U$.
  Let $f : U \to \mathbb{C}$. Write $f(x + i y) = u (x, y) + i v(x, y)$
  where $u, v : \mathbb{R}^2 \to \mathbb{R}$.
  $f$ is complex differentiable at $z_0$ if and only if:
  \begin{enumerate}
    \item{
      Both $u$ and $v$ are totally differentiable at $(x_0, y_0)$,
      meaning $\exists a, b \in \mathbb{R}$ such that
      $$
      \lim_{(x, y) \to (x_0, y_0)}
        \frac{u(x,y) - (u(x_0, y_0) + a(x - x_0) + b(y - y_0))}
             {\sqrt{(x - x_0)^2 + (y - y_0)^2}}
      = 0
      $$
      and similarly for $v$.
      This means that near $(x_0, y_0)$, $u(x,y)$ can be approximated
      by $f(x, y) = u(x_0, y_0) + a(x - x_0) + b(y - y_0)$, which has
      the tangent plane of the graph of $u$ at $(x_0, y_0)$ as its
      image. This means that for $y = y_0$ and letting $x \to x_0$,
      $u_x(x_0, y_0) = a$ and for $x = x_0$ letting $y \to y_0$,
      $u_y(x_0, y_0) = b$. Total differentiability implies the
      existence of partial derivatives, but not vice-versa.
      Furthermore any directional derivative
      $$
      \frac{d}{dt}u(x_0 + t \cos \theta, y_0 + t \sin \theta)
      $$
      exists.
    }
    \item{
      The partial derivatives of $u$ and $v$ at $(x_0, y_0)$ satisfy
      \begin{align*}
            \frac{\partial u}{\partial x} (x_0, y_0)
        &=  \frac{\partial v}{\partial y} (x_0, y_0) \\
            \frac{\partial u}{\partial y} (x_0, y_0)
        &= -\frac{\partial v}{\partial y} (x_0, y_0)
      \end{align*}
    }
  \end{enumerate}
\end{theorem}

\begin{proof}
  $$
       w_0
  =    f^\prime(z_0)
  \iff \lim_{z \to z_0} \frac{f(z) - f(z_0) - w(z - z_0)}{z - z_0}
  =    0
  \iff \lim_{z \to z_0} \frac{f(z) - f(z_0) - w(z - z_0)}{|z - z_0|}
  =    0,
  $$
  recalling that $z_n \to 0$ if and only if $|z_n| \to 0$.

  Write $z = x + yi$, $f = u + iv$, $w_0 = a + bi$ so that
  \begin{align*}
     f(z) - f(z_0) - w_0(z - z_0)
   &= u(x, y) + i v(x, y) - u(x_0, y_0) - i v(x_0, y_0)
    - (a + bi)(x + yi - x_0 - y_0i) \\
   &=  [u(x, y) - u(x_0, y_0) - a(x - x_0) + b(y - y_0)]
    + i[v(x, y) - v(x_0, y_0) - b(x - x_0) - a(y - y_0)].
  \end{align*}
  Then
  \begin{align*}
    \lim_{(x, y) \to (x_0, y_0)}
      \frac{u(x, y) - u(x_0, y_0) - a(x - x_0) + b(y - y_0)}
           {\sqrt{(x - x_0)^2 + (y - y_0)^2}
   = 0
  \end{align*}
  and
  \begin{align*}
    \lim_{(x, y) \to (x_0, y_0)}
      \frac{v(x, y) - v(x_0, y_0) - b(x - x_0) - a(y - y_0)}
           {\sqrt{(x - x_0)^2 + (y - y_0)^2}
   = 0
  \end{align*},
  which is equivalent to $u$ being totally differentiable at
  $(x_0, y_0)$ and $u_x(x_0, y_0) = a$, $u_y(x_0, y_0) = -b$,
  as well as $v$ being totally differentiable at
  $(x_0, y_0)$ and $v_x(x_0, y_0) = b$, $v_y(x_0, y_0) = a.
\end{proof}

\begin{prop}
  Let $U$ be open in $\mathbb{R}^2$ and $f : U \to \mathbb{R}$.
  and are continuous on $U$. Then $f$ is totally differentiable at
  every $(x, y) \in U$.
\end{prop}

\begin{xmpl}
  \begin{enumerate}
    The exponential function is entire since
    $$
    f(z) = e^z = e^x \cos y + i e^x \sin y
    $$
    and so $u_x = e^x \cos y$, $u_y = -e^x \sin y$,
    $v_x = e^x \sin y$, $v_y = e^x \cos y$, which are
    continuous on $\mathbb{R}^2$ and satisfy the
    Cauchy-Riemann equations. Furthermore
    $f^\prime = f$. From the chain rule,
    $\frac{d}{dz} e^{rz} = r e^{rz}$ for any $r \in \mathbb{C}$.
    The usual derivatives of the trigonometric functions follow.

    We will see later that any branch of $\log z$ is holomorphic.
  \end{enumerate}
\end{xmpl}
