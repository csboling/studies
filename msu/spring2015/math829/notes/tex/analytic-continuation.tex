\section{Analytic Continuation}

\begin{defn}
A sequence of functions $(f_n)$ on an open set $U$ is said
to \emph{converge compactly} to a function $f$ in $U$ if
for every compact set $K \subset U$, we have
$f_n \to f$ uniformly on $K$.
\end{defn}

\begin{remark}
  \begin{enumerate}
    \item{
      Using the open covering definition of compact sets,
      we see that $f_n \to f$ compactly in $U$ if and only
      if $\forall z_0 \in U$, $\exists r > 0$ so that
      $f_n \to f$ uniformly on $D(z_0, r) \subset U$.
      For the latter condition we say that
      $f_n \to f$ locally uniformly. That is, compact
      convergence is equivalent to locally uniform convergence.
    }
    \item{
      If every $f_n$ is continuous, then the limit $f$ is
      continuous.
    }
  \end{enumerate}
\end{remark}

\begin{theorem}
Let $(f_n)$ be a sequence of analytic functions on an open set $U$
which converges compactly to $f$. Then $f$ is analytic on $U$, and
$(f_n^\prime)$ converges compactly to $f^\prime$.
\end{theorem}

\begin{proof}
Let $z_0 \in U$. Pick $r > 0$ such that $\bar{D}(z_0, r) \subset U$.
Let $\gamma$ be any closed curve in $D(z_0, r)$. Since each $f_n$ is
analytic on $D(z_0, r)$, we have $\int_\gamma f_n = 0$. Since
$\gamma$ is compact, $f_n \to f$ uniformly on $\gamma$. Thus
$$
  \int_f \gamma
= \lim_{n \to \infty} \int_\gamma f_n
= 0.
$$
Since every $f_n$ is continuous, so is $f$. Then from Morera's
theorem, $f$ is holomorphic on $D(z_0, r)$. Since $z_0$ was chosen
arbitrarily, $f$ is holomorphic on $U$.

To prove $f_n^\prime \to f^\prime$ compactly on $U$, it suffices to
show that $f_n^\prime \to f^\prime$ uniformly on the closed disk
$\bar{D}(z_0, r)$, i.e. to show locally uniform convergence.
Pick $R > r$ such that $\bar{D}(z_0, R) \subset U$, and let
$J = \partial D(z_0, R)$. From Cauchy's formula,
$$
  f_n^\prime(z) - f^\prime(z)
= \frac{1}{2 \pi i}
  \int_J
    \frac{f_n(w) - f(w)}
         {(w - z)^2}
    \dif w, \quad
\forall z \in D(z_0, R).
$$
Therefore
$$
     |f_n^\prime(z) - f^\prime(z)|
\leq \frac{1}{2 \pi}
     \frac{\| f_n - f \|_J}
          {(\mathrm{dist}(z, J))^2}
     \cdot 2 \pi R.
$$
If $z \in \bar{D}(z_0, r)$, then
$\mathrm{dist}(z, J) \geq R - r$, so
$$
     \|f_n^\prime - f^\prime\|_{\bar{D}(z_0, r)}
\leq \frac{R}{(R - r)^2}
     \|f_n - f\|_J
\to 0
$$
since $f_n \to f$ uniformly. So
$f_n^\prime \to f$ uniformly on $\bar{D}(z_0, r)$.
\end{proof}

\begin{defn}
We say a series of functions $\sum_{n=m}^\infty f_n$ converges
compactly to $f$ in $U$ if the partial sum sequence converges to $f$
compactly in $U$.
\end{defn}

Recall the comparison theorem: given $K \subset U$, if there exists
$(c_n)$ depending on $K$ such that $\|f_n\|_K \leq c_n$, $\forall n$
and $\sum c_n < \infty$ then $\sum f_n$ converges uniformly on $K$.
If this is true for every compact $K \subset U$ then $\sum f_n$
converges compactly on $U$.

\begin{xmpl}
  \begin{enumerate}
    \item{
      Suppose $\sum_{n=0}^\infty a_n z^n$ has radius $R > 0$. Recall that
      this series converges uniformly on $\bar{D}(0, R_1)$ for all
      $R_1 \in (0, R)$. If $K \subset D(0, R)$ is compact, then
      $\exists R_1 \in (0, R)$ such that $K \subset D(0, R_1)$. Therefore
      this series converges uniformly on $K$.
    }
    \item{
      Consider
      $$
        \tilde{\zeta}(z)
      = \sum_{n=1}^\infty \frac{1}{n^z}
      = \sum_{n=1}^\infty e^{-z \log n}.
      $$
      Note that $n \in \mathbb{R}$ so no branch cut is needed.
      We see that
      $$
        \left|
          \frac{1}{n^z}
        \right|
      = e^{-\log n \cdot \Re(z)}
      = \frac{1}{n^{\Re(z)}}.
      $$
      We know
      $$
      \sum_{n=1}^\infty
        \frac{1}{n^p}
      < \infty, \quad
      p > 1
      $$
      so from the comparison  principle $\tilde{\zeta}$ converges uniformly
      in $\{z : \Re(z) \geq p\}$, $\forall p > 1$. Since for
      every compact subset $K$ of $\{ z : \Re(z) > 1 \}$ there is $p >
      1$ such that $K \subset \{ z : \Re z \geq p \}$, we see that
      $\tilde{\zeta}$ converges compactly on $\{ z : \Re z > 1 \}$.

      From the theorem, the limit $\tilde{\zeta}(z)$ is analytic and
      $$
        \tilde{\zeta}^\prime(z)
      = \sum_{n=1}^\infty
          -\frac{\log n}{n^z}.
      $$
      Such $f$ has an analytic extension to
      $\mathbb{C} \setminus \{ 1 \}$, namely the famous Riemann zeta
      function. This function has trivial zeros at negative even
      integers. The Riemann Hypothesis states that every nontrivial
      zero of $\zeta$ lies on the line $\{ z : \Re z = \frac{1}{2} \}$.
    }
    \item{
      Let $a, b \in \mathbb{R}$ and $U \subset \mathbb{C}$ be open.
      Suppose $f : [a, b] \times U \to \mathbb{C}$ is a homotopy of
      analytic functions on $U$. Then
      $$
      F(z) = \int_a^b f(t, z) \dif t
      $$
      is analytic on $U$ since the Riemann sum
      $$
        F_n(z)
      = \sum_{k=1}^n
          f(t_k, z)
          (t_k - t_{k-1})
      $$
      is analytic on $U$ and $F_n \to F$ compactly.
    }
    \item{
      Suppose $f$ is continuous on $\mathbb{R}$.
      For any $n \in \mathbb{N}$,
      $$
        F_n(z)
      = \int_{-n}^n
          f(t) e^{i t z}
          \dif t
      $$
      is analytic on $\mathbb{C}$. If there exists an open set $U$
      such that
      $$
          F_n
      \to \hat{f}(z)
      =   \int_{-\infty}^{\infty}
            f(t) e^{i t z}
            \dif t
      $$
      compactly in $U$, then $\hat{f}$ is analytic on $U$.
    }
    \item{
      Define
      $$
        \Gamma(t)
      = \int_0^\infty
          x^{t-1}
          e^{-x}
          \dif x
      = \lim_{\varepsilon \to 0^+}
        \lim_{R \to \infty}
        \int_\varepsilon^R
          x^{t-1}
          e^{-x}
          \dif x
      $$
      where $t \in \mathbb{C}$ and
      $x^{t-1} = e^{(t-1)\log x}$. One can show that the limit
      converges compactly in
      $H_R = \{z \in \mathbb{C} : \Re z > 0\}$. Thus
      $\Gamma$ is analytic on $H_R$. $\Gamma$ can be extended
      analytically to
      $\mathbb{C} \setminus \{ n \in \mathbb{Z} : n \leq 0 \}$
      and every $n \in \mathbb{N} \cup \{ 0 \}$ is a simple pole
      of $\Gamma$. This function obeys the functional equation
      $\Gamma(n) = (n - 1)!$ for nonnegative integers.
    }
  \end{enumerate}
\end{xmpl}

In summary we have the following methods to construct analytic
functions or determine analyticity:
\begin{enumerate}
  \item{
    Compute the complex derivative from the definition.
  }
  \item{
    Construct a function from addition, multiplication, division, or
    composition of analytic functions.
  }
  \item{
    Check that a function satisfies the Cauchy-Riemann equation.
  }
  \item{
    Use power series and Laurent series.
  }
  \item{
    The primitive or local primitive of an analytic function is always analytic.
  }
  \item{
    The derivative of an analytic function is always analytic.
  }
  \item{
    The inverse or local inverse of an analytic function is analytic.
  }
  \item{
    The limit of a compactly convergent sequence or series of analytic
    functions is analytic.
  }
  \item{
    The integral of a family of analytic functions is analytic.
  }
\end{enumerate}