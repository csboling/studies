\section{Riemann Mapping Theorem}

\begin{defn}
A sequence of functions $(f_n)$ on an open set $U$ is said
to \emph{converge compactly} to a function $f$ in $U$ if
for every compact set $K \subset U$, we have
$f_n \to f$ uniformly on $K$.
\end{defn}

\begin{remark}
  \begin{enumerate}
    \item{
      Using the open covering definition of compact sets,
      we see that $f_n \to f$ compactly in $U$ if and only
      if $\forall z_0 \in U$, $\exists r > 0$ so that
      $f_n \to f$ uniformly on $D(z_0, r) \subset U$.
      For the latter condition we say that
      $f_n \to f$ locally uniformly. That is, compact
      convergence is equivalent to locally uniform convergence.
    }
    \item{
      If every $f_n$ is continuous, then the limit $f$ is
      continuous.
    }
  \end{enumerate}
\end{remark}

\begin{theorem}
Let $(f_n)$ be a sequence of analytic functions on an open set $U$
which converges compactly to $f$. Then $f$ is analytic on $U$, and
$(f_n^\prime)$ converges compactly to $f^\prime$.
\end{theorem}

\begin{proof}
Let $z_0 \in U$. Pick $r > 0$ such that $\bar{D}(z_0, r) \subset U$.
Let $\gamma$ be any closed curve in $D(z_0, r)$. Since each $f_n$ is
analytic on $D(z_0, r)$, we have $\int_\gamma f_n = 0$. Since
$\gamma$ is compact, $f_n \to f$ uniformly on $\gamma$. Thus
$$
  \int_f \gamma
= \lim_{n \to \infty} \int_\gamma f_n
= 0.
$$
Since every $f_n$ is continuous, so is $f$. Then from Morera's
theorem, $f$ is holomorphic on $D(z_0, r)$. Since $z_0$ was chosen
arbitrarily, $f$ is holomorphic on $U$.

To prove $f_n^\prime \to f^\prime$ compactly on $U$, it suffices to
show that $f_n^\prime \to f^\prime$ uniformly on the closed disk
$\bar{D}(z_0, r)$, i.e. to show locally uniform convergence.
Pick $R > r$ such that $\bar{D}(z_0, R) \subset U$, and let
$J = \partial D(z_0, R)$. From Cauchy's formula,
$$
  f_n^\prime(z) - f^\prime(z)
= \frac{1}{2 \pi i}
  \int_J
    \frac{f_n(w) - f(w)}
         {(w - z)^2}
    \dif w, \quad
\forall z \in D(z_0, R).
$$
Therefore
$$
     |f_n^\prime(z) - f^\prime(z)|
\leq \frac{1}{2 \pi}
     \frac{\| f_n - f \|_J}
          {(\mathrm{dist}(z, J))^2}
     \cdot 2 \pi R.
$$
If $z \in \bar{D}(z_0, r)$, then
$\mathrm{dist}(z, J) \geq R - r$, so
$$
     \|f_n^\prime - f^\prime\|_{\bar{D}(z_0, r)}
\leq \frac{R}{(R - r)^2}
     \|f_n - f\|_J
\to 0
$$
since $f_n \to f$ uniformly. So
$f_n^\prime \to f$ uniformly on $\bar{D}(z_0, r)$.
\end{proof}

\begin{defn}
We say a series of functions $\sum_{n=m}^\infty f_n$ converges
compactly to $f$ in $U$ if the partial sum sequence converges to $f$
compactly in $U$.
\end{defn}

Recall the comparison theorem: given $K \subset U$, if there exists
$(c_n)$ depending on $K$ such that $\|f_n\|_K \leq c_n$, $\forall n$
and $\sum c_n < \infty$ then $\sum f_n$ converges uniformly on $K$.
If this is true for every compact $K \subset U$ then $\sum f_n$
converges compactly on $U$.

\begin{xmpl}
  \begin{enumerate}
    \item{
      Suppose $\sum_{n=0}^\infty a_n z^n$ has radius $R > 0$. Recall that
      this series converges uniformly on $\bar{D}(0, R_1)$ for all
      $R_1 \in (0, R)$. If $K \subset D(0, R)$ is compact, then
      $\exists R_1 \in (0, R)$ such that $K \subset D(0, R_1)$. Therefore
      this series converges uniformly on $K$.
    }
    \item{
      Consider
      $$
        \tilde{\zeta}(z)
      = \sum_{n=1}^\infty \frac{1}{n^z}
      = \sum_{n=1}^\infty e^{-z \log n}.
      $$
      Note that $n \in \mathbb{R}$ so no branch cut is needed.
      We see that
      $$
        \left|
          \frac{1}{n^z}
        \right|
      = e^{-\log n \cdot \Re(z)}
      = \frac{1}{n^{\Re(z)}}.
      $$
      We know
      $$
      \sum_{n=1}^\infty
        \frac{1}{n^p}
      < \infty, \quad
      p > 1
      $$
      so from the comparison  principle $\tilde{\zeta}$ converges uniformly
      in $\{z : \Re(z) \geq p\}$, $\forall p > 1$. Since for
      every compact subset $K$ of $\{ z : \Re(z) > 1 \}$ there is $p >
      1$ such that $K \subset \{ z : \Re z \geq p \}$, we see that
      $\tilde{\zeta}$ converges compactly on $\{ z : \Re z > 1 \}$.

      From the theorem, the limit $\tilde{\zeta}(z)$ is analytic and
      $$
        \tilde{\zeta}^\prime(z)
      = \sum_{n=1}^\infty
          -\frac{\log n}{n^z}.
      $$
      Such $f$ has an analytic extension to
      $\mathbb{C} \setminus \{ 1 \}$, namely the famous Riemann zeta
      function. This function has trivial zeros at negative even
      integers. The Riemann Hypothesis states that every nontrivial
      zero of $\zeta$ lies on the line $\{ z : \Re z = \frac{1}{2} \}$.
    }
    \item{
      Let $a, b \in \mathbb{R}$ and $U \subset \mathbb{C}$ be open.
      Suppose $f : [a, b] \times U \to \mathbb{C}$ is a homotopy of
      analytic functions on $U$. Then
      $$
      F(z) = \int_a^b f(t, z) \dif t
      $$
      is analytic on $U$ since the Riemann sum
      $$
        F_n(z)
      = \sum_{k=1}^n
          f(t_k, z)
          (t_k - t_{k-1})
      $$
      is analytic on $U$ and $F_n \to F$ compactly.
    }
    \item{
      Suppose $f$ is continuous on $\mathbb{R}$.
      For any $n \in \mathbb{N}$,
      $$
        F_n(z)
      = \int_{-n}^n
          f(t) e^{i t z}
          \dif t
      $$
      is analytic on $\mathbb{C}$. If there exists an open set $U$
      such that
      $$
          F_n
      \to \hat{f}(z)
      =   \int_{-\infty}^{\infty}
            f(t) e^{i t z}
            \dif t
      $$
      compactly in $U$, then $\hat{f}$ is analytic on $U$.
    }
    \item{
      Define
      $$
        \Gamma(t)
      = \int_0^\infty
          x^{t-1}
          e^{-x}
          \dif x
      = \lim_{\varepsilon \to 0^+}
        \lim_{R \to \infty}
        \int_\varepsilon^R
          x^{t-1}
          e^{-x}
          \dif x
      $$
      where $t \in \mathbb{C}$ and
      $x^{t-1} = e^{(t-1)\log x}$. One can show that the limit
      converges compactly in
      $H_R = \{z \in \mathbb{C} : \Re z > 0\}$. Thus
      $\Gamma$ is analytic on $H_R$. $\Gamma$ can be extended
      analytically to
      $\mathbb{C} \setminus \{ n \in \mathbb{Z} : n \leq 0 \}$
      and every $n \in \mathbb{N} \cup \{ 0 \}$ is a simple pole
      of $\Gamma$. This function obeys the functional equation
      $\Gamma(n) = (n - 1)!$ for nonnegative integers.
    }
  \end{enumerate}
\end{xmpl}

In summary we have the following methods to construct analytic
functions or determine analyticity:
\begin{enumerate}
  \item{
    Compute the complex derivative from the definition.
  }
  \item{
    Construct a function from addition, multiplication, division, or
    composition of analytic functions.
  }
  \item{
    Check that a function satisfies the Cauchy-Riemann equation.
  }
  \item{
    Use power series and Laurent series.
  }
  \item{
    The primitive or local primitive of an analytic function is always analytic.
  }
  \item{
    The derivative of an analytic function is always analytic.
  }
  \item{
    The inverse or local inverse of an analytic function is analytic.
  }
  \item{
    The limit of a compactly convergent sequence or series of analytic
    functions is analytic.
  }
  \item{
    The integral of a family of analytic functions is analytic.
  }
\end{enumerate}

\subsection{Normal Families}
\begin{defn}
Let $U \subset \mathbb{C}$ be open. Let $\Phi$ be a family of analytic
functions defined on $U$. We say $\Phi$ is a normal family if every
sequence in $\Phi$ contains a subsequence which converges compactly in
$U$. The limit does not have to lie in $\Phi$.
\end{defn}

\begin{remark}
Let $\Sigma$ denote the set of analytic functions on $U$. It is
possible to define a metric on $\Sigma$ such that $f_n \to f$
compactly in $U$ iff. $d(f_n, f) \to 0$. We first construct a sequence
of compact subsets $(K_n)$ of $U$ such that
$K_n$ is contained in the interior of $K_{n+1}$ (the union of all open
subsets of $K_{n+1}$ and $\bigcup_{n=1}^\infty K_n = U$.

For example, let
$$
  K
= \{ z \in U : |z| \leq n,
     \mathrm{dist}(z, U^c) \geq \frac{1}{n} \}
$$
and define
$$
  d(f, g)
= \sum_{n=1}^\infty
    \frac{1}{2^n}
    \frac{\|f - g\|_{K_n}}
         {1 + \|f - g\|_{K_n}}.
$$
Observe that $d(f_n, f) \to 0$ if and only if
$\|f_n - f\|_{K_n} \to 0$ for every $n$.
Observe also that for all compact $K \subset U$, there exists an $n$
such that $K \subset K_n$. Thus $\|f_n - f\|_{K_n} \to 0$ if and only
if $f_n \to f$ compactly in $U$. Then $\Phi$ is a normal family if and
only if $\Phi$ is relatively compact with respect to this metric, i.e.
the closure of $\Phi$ is compact.
\end{remark}

\begin{defn}[Equicontinuous function]
Let $A$ be some indexing set, and $K$, $L$ metric spaces.
We say that a family $(f_\alpha)_{\alpha \in A}$
of functions $f_\alpha : K \to L$ is equicontinuous if
$\forall \varepsilon > 0$, $\exists \delta > 0$ such that
$\forall z, w \in K$ with $d_K(z, w) < \delta$,
$d_L(f_n(z), f_n(w)) < \varepsilon$ for all $n$.
\end{defn}

\begin{theorem}[Arzela-Ascoli Theorem]
Let $K$, $L$ be compact metric spaces. Let $f_n : K \to L$ be an
equicontinuous set of functions. Then $(f_n)$ has a subsequence which
converges uniformly in $K$.
\end{theorem}

\begin{proof}[(sketch)]
Since $K$ is compact, $\exists (z_m)$ in $K$ which is dense in $K$.
Let $n_k^0 = k$. Consider the values of $(f_{n_k^0})$ at $z_1$. Since
$L$ is compact, $\exists$ a subsequence of $(f_{n_k^0})$, call it
$(f_{n_k^1})$, such that $(f_{n_k^1}(z_1))$ converges. Then we look at
$(f_{n_k^1})$ valued at $z_2$. Since $L$ is compact, there exists a
subsequence of $(f_{n_k^2})$ of $(f_{n_k^1})$ such that
$(f_{n_k^2}(z_2))$ converges. Repeating this argument gives
$$
        (f_k)
=       (f_{n_k^0})
\supset (f_{n_k^1})
\supset (f_{n_k^2})
\supset \cdots
$$
such that for all $m \in \mathbb{N}$, $(f_{n_k^m}(z_m))$ converges.
Now let $n_k = n_k^k$, i.e. consider the diagonal of
\begin{align*}
  f_{n_1^1} & \supset & f_{n_1^2} & \supset & \cdots \\
  f_{n_2^1} & \supset & f_{n_2^2} & \supset & \cdots \\
  \vdots  & \supset & \cdots   & \supset & \ddots \\
  \to z_1 &         & \to z_2  &         &
\end{align*}
so that $(f_{n_k})$ is a subsequence of $(f_k)$. Then
$(f_{n_k}(z_m))$ converges for all $m$ because
$(f_{n_k})$ is a subsequence of $(f_{n_k^m})$ except for the first
$m-1$ items.

Now $(f_{n_k})$ is a subsequence of $(f_k)$ and converges at every
$z_m$. Using the denseness of $z_m$ and the equicontinuity of
$(f_{n_k})$, we can conclude that $(f_n)$ converges uniformly on $K$.
\end{proof}

\begin{lemma}
Suppose $K$ is a compact of an open set $U \subset\mathbb{C}$. Let
$(f_n)$ be a sequence of analytic function on $U$. If $(f_n^\prime)$
or $(f_n)$ is uniformly bounded on $U$, i.e. there exists $M > 0$
such that $\| f_n^\prime \|$ or $\| f_n \| \leq M$ for all $n$,
then $(f_n)$ is equicontinuous on $K$.
\end{lemma}

\begin{proof}
Assume $(f_n^\prime)$ is uniformly bounded on $U$. Since $K \subset U$
is compact, $\exists r > 0$ such that $\forall z_0 \in K$,
$D(z_0, r) \subset U$.

If $z, w \in K$ and $|z - w| < r$, then
$[z, w] \subset D(z, r) \subset U$. So
$$
  |f_n(z) - f_n(w)|
=    \left|
       \int_{[w, z]}
          f_n^\prime
     \right|
\leq \| f_n^\prime \| |w - z|
\leq M |w - z|.
$$
For any $\varepsilon > 0$, let
$\delta = \min \{ r, \frac{\varepsilon}{M} \}$. If
$z, w \in K$ and $|z - w| < \delta$, then
$$
     |f_n(z) - f_n(w)|
\leq M \delta
<    \varepsilon.
$$

Next assume $(f_n)$ is uniformly bounded on $U$. Then
$\exists M > 0$ such that $\| f_n \|_U \leq M$ for all $n$.
Let $r > 0$ be as above and define
$$
  U^\prime
= \bigcup_{z_0 \in K} D(z_0, \frac{r}{2})
$$
so that $K \subset U^\prime \subset U$ and $U^\prime$ is open
and $\forall w_0 \in U^\prime$,
$\bar{D}(w_0, \frac{r}{2}) \subset U$.

If $z \in \bar{D}(w_0, \frac{r}{2})$, then
$|z - w_0| \leq \frac{r}{2}$. Since
$w_0 \in D(z_0, \frac{r}{2})$ for some $z_0 \in K$,
$$
|z - z_0| \leq |z - w_0| + |w_0 - z_0| = \frac{r}{2} + \frac{r}{2} = r
$$
so that $z \in D(z_0, r)$.

Then
$$
  f_n^\prime(z_0)
= \frac{1}{2 \pi i}
  \int_{|z - z_0| = \frac{r}{2}}
    \frac{f_n(z)}
         {(z - z_0)^2}
    \dif z
$$
so that
$$
     |f_n^\prime(z_0)|
\leq \frac{1}{2 \pi}
     \frac{M}{(r / 2)^2}
     2 \pi
     \frac{r}{2}
=    \frac{2M}{r}.
$$
Thus $(f_n^\prime)$ is uniformly bounded by $\frac{2M}{r}$ on
$U^\prime$. Since $K \subset U^\prime$, the first part of the proof
implies that $(f_n)$ is equicontinuous on $K$.
\end{proof}

\begin{corol}
Suppose $K$ is a compact subset of an open set $U \subset \mathbb{C}$.
Let $(f_n)$ be a uniformly bounded sequence of analytic functions on
$U$.
Then $(f_N)$ contains a subsequence which converges uniformly on $K$.
\end{corol}
\begin{proof}
$(f_n)$ is equicontinuous on $K$, and since $(f_n)$ is uniformly
bounded, $\exists M > 0$ so that $\| f_n \|_K \leq M$.
Let $L = \bar{D}(0, M)$, so $L$ is compact and $f_n : K \to L$. Then
from the A-A theorem, the desired consequent is shown.
\end{proof}

\begin{theorem}[Montel's Thoerem]
Let $U \subset \mathbb{C}$ be open. Let $\Phi$ be a family of
analytic functions on $U$. Then $\Phi$ is a normal family if and only
if $\Phi$ is uniformly bounded on every compact subset of $U$.
\end{theorem}

\begin{proof}
  \begin{itemize}
    \item[($\implies$)]{
      Suppose $\Phi$ is not uniformly bounded on some compact set $K
      \subset U$. Then there exists a sequence $(f_n)$ in $K$ such
      that $\| f_n \|_K \to \infty$, which contradicts that $(f_n)$
      converges uniformly on $K$, which was required by the assumption
      that $\Phi$ is a normal family.
    }
    \item[($\impliedby$)]{
      For $m \in \mathbb{N}$, let
      $$
        K_m
      = \{ z \in U : |z| \leq M, \mathrm{dist}(z, U^C) \geq
      \frac{1}{m} \}.
      $$
      Then each $K_m$ is a compact subset of $U$, $K_m$ lies in the
      interior of $K_{m+1}$, and $U = \cup_m K_m$.

      Let $U_m = \cup_{z \in K_m} D(z, \frac{1}{m} - \frac{1}{m +
        1})$.
      Then each $U_m$ is open, and $K \subset U_m \subset K_{m+1}$, so
      $U = \cup U_m$.

      We have a sequence
      $$
      K_1 \subset U_1 \subset K_2 \subset U_2 \subset \cdots
      $$
      such that each $K_m$ is compact and each $U_m$ is open.
      By assumption, $\Phi$ is uniformly bounded on each $K_m$.
      Since $U_m \subset K_{m+1}$, this means $\Phi$ is uniformly
      bounded on each $U_m$. Since $K_m$ is a compact subset of
      $U_m$, the corollary above shows that every sequence in
      $\Phi$ contains a subsequence which converges uniformly on
      $K_m$. We use another diagonal argument to show that every
      sequence in $\Phi$ contains a subsequence which converges
      uniformly on every $K_m$.

      Let $(f_m) \in \Phi$ and let $n^0_k = k$. Then by the argument
      above, $(f_{n^0_k})$ contains a subsequence $(f_{n_k}^1)$ which
      converges uniformly on $K_1$. Similarly $(f_{n^1_k})$ has a
      subsequence $(f_{n^2_k})$ which converges uniformly on $K_2$.
      Then we have
      $$
      (f_{n^0_k}) \supset (f_{n^1_k}) \supset \cdots
      $$
      such that for each $m$, $(f_{n^m_k})$ converges uniformly on
      $K_m$. Let $n_k = n^k_k$. Then $(f_{n_k}) \subset (f_n)$ which
      converges uniformly on each $K_m$.

      Let $K \subset U$ be compact. Then the $U_m$ forms an open cover
      of $U$, so there exists an $M$ such that $K \subset U_M \subset
      K_{M+1}$.
      Since $(f_{n_k})$ converges uniformly on $K_{M+1}$, it also
      converges uniformly on $K$.
    }
  \end{itemize}
\end{proof}

\begin{remark}
$Phi$ is uniformly bounded on every compact subset of $U$ if and only
if $\Phi$ is locally uniformly bounded in $U$, i.e.
$\forall z_0 \in U$, $\exists r > 0$ such that $\Phi$ is uniformly
bounded on $D(z_0, r)$.
\end{remark}

\begin{theorem}[Riemann Mapping Theorem]
Let $U \subsetneq \mathbb{C}$ be simply connected and $z_0 \in U$.
Then there exists $f \in \mathrm{Iso}(U, \mathbb{D})$ such that
$f(z_0) = 0$. If another $g$ has this property, then
$\exists c \in \mathbb{C}$ with $|c| = 1$ such that $g = M_c \cdot f$.
Moreover, if we require that $f^\prime(z_0) > 0$, then such $f$ is unique.
\end{theorem}

\begin{proof}
\begin{enumerate}
  \item{
    First we prove the uniqueness statements. Suppose
    $f_1, f_2 \in \mathrm{Iso}(U, \mathbb{D})$ and
    $f_1(z_0) = 0 = f_2(z_0)$. Then
    $f = f_1 \circ f_2^{-1} \in \ mathrm{Aut}(\mathbb{D})$ and $f(0) = 0$,
    so $f = M_c$ with $|c| = 1$. If $f_1^\prime(z_0), f_2^\prime(z_0) >
    0$,
    then $f^\prime(0) > 0$. Then $c = 1$, whence $f$ is identity.
  }
  \item{
    Let $\Phi$ be the set of injective analytic functions
    $f : U \to \mathbb{D}$ such that $f(z_0) = 0$. Let
    $c \in \mathbb{C} \setminus U$. Since $U$ is simply connected,
    there is a branch of $\log(z - c)$ in $U$. Call it $L(z)$. Such
    $L$ is an analytic isomorphism out of $U$, since
    $L(z_1) = L(z_2)$ implies $e^{L(z_1)} = e^{L(z_2)}$  and so $z_1 =
    z_2$, whence $L$ is injective.

    Let $V = L(U)$. Then $V$ is open, and
    $V \cap (V + 2 \pi i) = \varnothing$. Let
    $z_0 \in V \cap (V + 2 \pi i)$. Then $z_0, z_0 - 2 \pi i \in V$,
    so $z_0 = L(w_1)$, $z_0 - 2\pi i = L(w_2)$ for $w_1, w_2 \in U$,
    so $w_1 = e^{z_0} = e^{z_0 - 2 \pi i} = w_2$, a contradiction.
    Now $V + 2 \pi i$ is also open, so we can find
    $D(w_0, r) \subset V + 2 \pi i$, so $D(w_0, r) \cap V = \varnothing$.

    Let $h(z) = \frac{r}{L(z) - w_0}$. Then $h$ is injective and
    $|h(z)| < 1$ on $U$, so
    $h \in \mathrm{Iso}(U, \mathbb{D})$.
    Let $f = g_{h(z_0)} \circ h$. Then $f(z_0) = 0$, so $f \in \Phi$,
    so $\Phi \neq \varnothing$.
  }
  \item{
    Suppose $f_0 \in \Phi \cap \mathrm{Iso}(U, \mathbb{D})$. Then
    $\forall f \in \Phi$, we can define
    $F = f \circ f_0^{-1} : \mathbb{D} \to \mathbb{D}$, and from
    the Schwarz lemma $|F^\prime(0)| \leq 1$.
    $f = F \circ f_0$, so
    $f^\prime(z_0) = F^\prime(0) \cdot f_0^\prime(z_0)$, whence
    $|f^\prime(z_0)| \leq |f_0^\prime(z_0)|$. Thus
    $|f_0^\prime(z_0)| = \max_{f \in \Phi} |f^\prime(z_0)|$, so we
    want to find $f_0 \in \Phi$ such that $f_0^\prime(z_0)$ is
    maximized.
  }
  \item{
    Set $S = \sup_{f \in \Phi} |f^\prime(z_0)|$. Then there exists a
    sequence $(f_n)$ in $\Phi$ such that $|f_n^\prime(z_0)| \to S$.
    Since $\Phi$ is a family of analytic functions defined on $U$
    which is uniformly bounded (since $|f| < 1$), by Montel's theorem
    it is a normal family. Thus $(f_n)$ has a subsequence $(f_{n_k})$ which
    converges compactly in $U$. Let $f_0 = \lim f_{n_k}$. Then
    $f_0$ is analytic and $f_{n_k}^\prime \to f_0^\prime$ compactly
    in $U$. Especially, $f_{n_k}^\prime(z_0) \to f_0^\prime(z_0)$.
    Thus $|f_0^\prime(z_0)| = S > 0$,
    so $f_0$ is not constant, so $|f_0(z)| < 1$. Therefore
    $f_0(U) \subset \mathbb{D}$.
  }
  \item{
    Suppose $f_0$ is not injective. Then we have $z_1, z_2 \in U$
    such that $f_0(z_1) = f_0(z_2) = w_0$. Since $f_0$ is not constant,
    $f_0 - w_0$ is not constantly zero, but $z_1$ and $z_2$ are zeros
    of $f_0 - w_0$. Then $\exists r > 0$ such that
    $\bar{D}(z_1, r) \cap \bar{D}(z_2, r) = \varnothing$, and
    $\bar{D}(z_j, r) \subset U$. Moreover, $f_0 - w_0$ has no zeros
    on $C_j = \partial \bar{D}(z_j, r)$. Since $|f_0 - w_0|$ is
    continuous and $C_j$ are compact, $\exists \varepsilon > 0$
    such that $|f_0(z) - w_0| \geq \varepsilon$ for all
    $z \in C_1 \cup C_2$. Since $f_{n_k} \to f$ compactly,
    $f_{n_k} \to f_0$ uniformly on $C_1 \cup C_2$, so there exists
    a $k_0$ such that $\| f_{n_{k_0}} - f_0 \|_{C_1 \cup C_2} <
    \varepsilon$.
    For $z \in C_1$,
    \begin{align*}
          |(f_{n_{k_0}}(z) - w_0) - (f_0(z) - w_0)|
    &=    |f_{n_{k_0}}(z) - f_0(z)|
     <    \varepsilon
     \leq |f_0(z) - w_0|,
    \end{align*}
    so from Rouche's theorem $f_{n_{k_0}} - w_0$ also has at least one
    zero on $D(z_1, r)$. Similarly, $f_{n_{k_0}} - w_0$ has a zero in
    $D(z_0, r)$, and this contradicts that $f_{n_{k_0}}$ is injective,
    which is assumed since $f_{n_{k_0}} \in \Phi$.
    Therefore $f_0$ is injective.
  }
  \item{
    Finally we show that $f_0(U) = \mathbb{D}$. Suppose
    $a \in \mathbb{D} \setminus f(U)$. Let
    $h_1 = g_a \circ f_0$, where $g_a$ is the automorphism of
    $\mathbb{D}$ that swaps 0 and $a$. Then $h_1$ is injective
    and analytic and has no zero in $\mathbb{U}$. Since
    $U$ is simply connected, there exists an analytic $h_2$
    $U$ such that $h_1 = h_2^2$, since on a simply connected open
    set we may find an analytic branch of $\log \circ h_1$.  Then
    $h_2$ is also injective and
    $$
      |h_2(z)|
    = |h_1(z)|^{1 / 2}
    < 1,
    $$
    so $h_2 : U \to \mathbb{D}$.

    We have $a = h_1(z_0)$, so let $b = h_2(z_0)$, so that $a = b^2$.
    Let $f_1 = g_b \circ h_2$. Then $f_1 : U \to \mathbb{D}$ is
    injective and analytic, and
    $$
      f_1(z_0)
    = g_b(h_2(z_0))
    = g_b(b)
    = 0,
    $$
    so $f_1 \in \Phi$.

    Let $S(z) = z^2$. Then
    \begin{align*}
       f_0
    &= g_a \circ h_1
     = g_a \circ S \circ h_2 \\
    &= g_a \circ S \circ g_b \circ f_1
    \end{align*}
    and so
    \begin{align*}
       f_0^\prime(z_0)
    &= f_1^\prime(z_0)
       g_b^\prime(0)
       S^\prime(b)
       g_a^\prime(a).
    \end{align*}
    Therefore
    \begin{align*}
       \frac{|f_0^\prime(z_0)|}
            {|f_1^\prime(z_0)|}
    &= |g_b^\prime(0)|
       |S^\prime(b)|
       |g_a^\prime(a)|.
    \end{align*}
    Recall that
    $$
      g_c(z)
    = \frac{c - z}
           {1 - \bar{c} z}
    $$
    so that
    $$
      g_c^\prime(z)
    = - \frac{1}
             {1 - \bar{c} z}
      + \frac{\bar{c}(c - z)}
             {(1 - \bar{c} z)^2}
    = \frac{|c|^2 - 1}
           {(1 - \bar{c} z)^2}
    $$
    so that
    $$
      |g_c^\prime(0)|
    = 1 - |c|^2
    $$
    and
    $$
      |g_c^\prime(c)|
    = \frac{1}{1 - |c|^2}.
    $$
    We also have $S^\prime(z) = 2z$, so we see that
    \begin{align*}
       \frac{|f_0^\prime(z_0)|}
            {|f_1^\prime(z_0)|}
    &= (1 - |b|^2) 2|b| \frac{1}{1 - |a|^2} \\
    &= \frac{2|b|}{1 + |b|^2}
     < 1
    \end{align*}
    since
    $$
      1 - |a|^2
    = 1 - |b|^4
    = (1 - |b|^2)(1 + |b|^2)
    $$
    and thus $|f_0^\prime(z_0)| < |f_1^\prime(z_0)|$, which
    contradicts that
    $$
      |f_0^\prime(z_0)|
    = \sup \{ |f^\prime(z_0) : f \in \Phi \}.
    $$
    Hence $f(U) = \mathbb{D}$.
  }
\end{enumerate}
\end{proof}

\begin{xmpl}
  \begin{enumerate}
    \item{
      Let $S(z) = z^2$. Then
      $$
        |S(z)|
      = |z|^2 \arg(S(z))
      = 2 \arg z.
      $$
      $S(z_1) = S(z_2) \iff  z_1 = \pm z_2$.
      If $U \cap (-U) = \varnothing$, then $S$ is an analytic
      isomorphism in $U$.
    }
    \item{
      We denote
      \begin{align*}
         \mathbb{H}
      &= \{ z \in \mathbb{C} : \Im~z > 0 \}
       = \{ z \in \mathbb{C} : \arg z \in (0, \pi) \}, \\
         \mathbb{H}_R
      &= \{ z \in \mathbb{C} : \Re~z > 0 \}
       = \{ z \in \mathbb{C}
          : \arg z \in \left(-\frac{\pi}{2}, \frac{\pi}{2}\right)
         \},
      \end{align*}
      and $M_i(z) = i z \in \mathrm{Iso}(\mathbb{H}_R, \mathbb{H})$.
    }
    \item{
      We can see that
      $S \in \mathrm{Iso}(\mathbb{H} \cap \mathbb{H}_R, \mathbb{H}$,
      i.e. squaring maps the first quadrant to the upper half plane
      isomorphically.
    }
    \item{
      Similarly
      $S \in \mathrm{Iso}(
               \mathbb{D} \cap \mathbb{H} \cap \mathbb{H}_R,
               \mathbb{D} \cap \mathbb{H})$.
    }
    \item{
      Let $F(z) = \frac{1 + z}{1 - z}$. If
      $x \in \mathbb{R} \cup \{ \infty \}$, then
      $F(x) \in \mathbb{R} \cup \{ \infty \}$.
      We can check that
      $$
        F(i)
      = \frac{1 + i}
             {1 - i}
      = i \in \mathbb{H},
      $$
      and see that $F \in \mathrm{Aut}(\mathbb{H})$.
      Furthermore
      $$
        F^{-1}(z)
      = \frac{z - 1}
             {z + 1}
      $$
      and
      $$
        F^{-1} \circ M_i^{-1}(z)
      = \frac{\frac{z}{i} - 1}
             {\frac{z}{i} + 1}
      = \frac{z - i}
             {z + i}
      = h_i(z).
      $$
      Therefore
      $M_i \circ F \in \mathrm{Iso}(\mathbb{D}, \mathbb{H})$,
      so $F \in \mathrm{Iso}(\mathbb{D}, \mathbb{H}_R$, and thus
      $F \in \mathrm{Iso}(\mathbb{D} \cap \mathbb{H},
                          \mathbb{H}_R \cap \mathbb{H})$.
    }
    \item{
      From the above we can see
      $$
          G
      =   S \circ F \circ S
      \in \mathrm{Iso}(\mathbb{D} \cap \mathbb{H} \cap \mathbb{H}_R,
                       \mathbb{H}).
      $$
    }
    \item{
      We can see that $z \mapsto \sqrt{z^2 - 1}$ is an isomorphism
      from $\mathbb{H}_R \setminus [0, 1]$ to $\mathbb{H}_R$.
      First we see that
      $$
          S
      :   \mathbb{H}_R \setminus [0, 1]
      \to \mathbb{C} \setminus (-\infty, 1]
      $$
      and
      $$
          T_{-1}
      :   \mathbb{C} \setminus (-\infty, 1]
      \to \mathbb{C} \setminus (-\infty, 0].
      $$
    }
    \item{
      $\exp$ is injective on $U$ if
      $U \cap (U + 2 k \pi i) = \varnothing$ for all
      $k \in \mathbb{Z}$.
    }
    \item{
      Let
      $$
        S_y
      = \{ z \in \mathbb{C} : 0 < \Im~z < y \}
      $$
      for $y > 0$. Then
      $\exp \in \mathrm{Iso}(S_\pi, \mathbb{H})$
      and
      $\exp in \mathrm{Iso}(S_\pi \cap (-\mathbb{H}), \mathbb{H} \cap
      \mathbb{D})$.
    }
    \item{
      The inversion map $J(z) = \frac{1}{z}$ is an automorphism of
      $\mathbb{C} \setminus \{ 0 \}$, and
      $J \in \mathrm{Iso}(\mathbb{D} \setminus \{ 0 \},
                          \mathbb{C} \setminus \bar{\mathbb{D}})$.
    }
  \end{enumerate}
\end{xmpl}
