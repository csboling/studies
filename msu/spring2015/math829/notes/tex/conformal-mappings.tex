\begin{defn}[Conformal Mapping]
A conformal mapping is an analytic isomorphism.
For open sets $U$ and $V$, let $\mathrm{Iso}(U, V)$ denote the
set of analytic isomorphisms $f$ such that $f(U) = V$. Let
$\mathrm{Aut}(U) = \mathrm{Iso}(U, U)$. Observe that
$\mathrm{Aut}(U)$ has an obvious group structure given by
function composition.
\end{defn}

The name comes from the following observation. Let $U$ be open
and $f$ be holomorphic on $U$. Suppose $\gamma : [a, b] \to U$
is $C^1$ and let $\beta = f \circ \gamma$. Then $\beta$ is a
$C^1$ curve in $f(U)$ and
$\beta^\prime(t) = f^\prime(\gamma(t)) \gamma^\prime(t)$. If
$\gamma^\prime(t) \neq 0$, then $\arg \gamma^\prime(t)$ gives the
direction of the curve $\gamma$ at $\gamma(t)$. If
$f^\prime(\gamma(t)) \neq 0$ then the direction of $\beta$ at
$t$ is
$$
  \arg \beta^\prime(t)
= \arg f^\prime(\gamma(t)) + \arg \gamma^\prime(t).
$$

Now suppose $f$ is an analytic isomorphism. Then $f^\prime$ is never
zero. Suppose two $C^1$ curves in $U$, $\gamma$ and $\eta$, intersect
at $z_0 = \gamma(t_0) = \eta(t_1)$. The angle between $\gamma$ and
$\eta$ at $z_0$ is
$$
  \theta
= \arg \gamma^\prime(t_0) - \arg \eta^\prime(t_1),
$$
while the angle between $f \circ \gamma$ and $f \circ \eta$ at
$f(z_0)$ is
$$
  \arg (f \circ \gamma)^\prime(t_0)
- \arg (f \circ \eta)^\prime(t_0).
$$

\begin{lemma}[Schwarz Lemma]
Let $\mathbb{D} = D(0, 1)$. Write $M_c$ for the function
$M_c(z) = c z$. If $|c| = 1$, then $M_c$ is a rotation by
$\arg c$ and so in this case $M_c \in \mathrm{Aut}(\mathbb{D})$.

Let $f : \mathbb{D} \to \mathbb{D}$ be analytic such that $f(0) = 0$.
Then
\begin{enumerate}[(i)]
  \item{
    $|f(z)| \leq |z|$, $\forall z \in \mathbb{D}$ and $|f^\prime(0)| \leq 1$.
  }
  \item{
    If $|f^\prime(0)| = 1$ or
    $\exists z_0 \in \mathbb{D} \setminus \{ 0 \}$ such that
    $|f(z_0)| = z_0$, then $f = M_c$ for some $c \in \mathbb{C}$ with
    $|c| = 1$. In particular, if $f^\prime(0) = 1$ or $f(z_0) = z_0$
    for some $z_0 \in \mathbb{D} \setminus \{ 0 \}$, then $c = 1$,
    i.e. $f = \id$.
  }
\end{enumerate}
\end{lemma}

\begin{proof}

\begin{enumerate}[(i)]
  \item{
    Define $h$ on $\mathbb{D}$ by
    $$
      h(z)
    = \left\{
        \begin{array}{c c}
          \frac{f(z)}{z}, & \quad z \in \mathbb{D} \setminus \{ 0 \} \\
          f^\prime(0),     & \quad z = 0
        \end{array}\right.
    $$
    First, $0$ is a singularity of $\frac{f(z)}{z}$, and since
    $\lim_{z \to 0} \frac{f(z) - f(0)}{z - 0} = f^\prime(0)$, $0$ is a
    removable singularity. Therefore $h$ is analytic on $\mathbb{D}$.
    Fix $r \in (0, 1)$. On the circle $\{|z| = r\}$,
    $|h(z)| = \frac{|f(z)|}{|z|} < \frac{1}{r}$, so from the maximum
    modulus principle $|h(z)| < \frac{1}{r}$ for any $|z| \leq r$.

    We will show that $|h(z)| \leq 1$. If this is not true, then
    there is a $z_0 \in \mathbb{D}$ such that $|h(z_0)| > 1$.
    We know $|z_0| < 1$ and $\frac{1}{|h(z_0)|} < 1$, so we may pick
    $r \in (\max \{ |z_0|, \frac{1}{|h(z_0)|} \}, 1)$, so $r \in(0, 1)$
    $|z_0| < r$ and $|h(z_0)| > \frac{1}{r}$, a contradiction of the
    previous observation. Therefore $|h(z)| \leq 1$ for any $z$, from
    which the conclusion holds.
  }
  \item{
    If $|f^\prime(0)| = 1$, then $|h(0)| = 1$.
    If $|f(z_0)| = |z_0|$ for some
    $z_0 \in \mathbb{D} \setminus \{ 0 \}$,
    then $|h(z_0)| = 1$. We already know $|h(z)| \leq 1$,
    so $h$ attains its maximum on $\mathbb{D}$, which happens
    only if $h \equiv c = 1$. If $f^\prime(0) = 1$, then if
    $f(z_0) = z_0$ then $h(z_0) = 1$. In either case $c = 1$ and so
    $f = M_1 = \id$.
  }
\end{enumerate}
\end{proof}

\begin{remark}
If $f \in \mathrm{Aut}(\mathbb{D})$ and $f(0) = 0$, then
$f^{-1}(0) = 0$. From Schwarz' lemma part 1,
$$
|f(z)| \leq |z|, \quad |f^{-1}(z)| \leq |z|
$$
on $\mathbb{D}$. Then $|f(z)| = |z|$ and so $f = \id$.
\end{remark}

\section{Classification of Analytic Isomorphisms}
\subsection{Automorphisms of the Disk}
We will completely characterize $\mathrm{Aut}(\mathbb{D})$.
Let $\alpha \in \mathbb{D}$ and define
$$
g_\alpha(z) = \frac{\alpha - z}{1 - \bar{\alpha} z}.
$$
Then $g_\alpha$ has one pole at $\frac{1}{\bar{\alpha}}$, which has
modulus greater than 1 so $g_\alpha$ is holomorphic on $\bar{\mathbb{D}}$.

Now
\begin{align*}
   (g_\alpha \circ g_\alpha)(z)
&= \frac{\alpha - \frac{\alpha - z}
                       {1 - \bar{\alpha} z}}
        {1 - \bar{\alpha} \frac{\alpha - z}
                               {1 - \bar{\alpha} z}}
 = \frac{\alpha(1 - \bar{\alpha} z) - (\alpha - z)}
        {(1 - \bar{\alpha} z) - \bar{\alpha}(\alpha - z)} \\
&= \frac{z(1 - \alpha \bar{\alpha})}
        {1 - \alpha \bar{\alpha}}
 = z,
\end{align*}
so $g_\alpha^{-1} = g_\alpha$.

Suppose $|z| = 1$. Then
\begin{align*}
   |g_\alpha(z)|
&= \frac{|\alpha - z|}
        {|1 - \bar{\alpha} z|}
 = \frac{|\alpha - z|}
        {\left|\frac{1}{z} - \bar{\alpha}\right| |z|} \\
&= \frac{|\alpha - z|}
        {|\bar{z} - \bar{\alpha}|}
 = \frac{|\alpha - z|}
        {|\overline{z - \alpha}|}
 = 1,
\end{align*}
so from the maximum modulus principle $|g_\alpha(z)| \leq 1$ for
any $z$ in the disk. In fact we must have $|g_\alpha(z)| < 1$, since
otherwise $|g_\alpha(z)| = 1$ and thus $g_\alpha$ is constant, which
contradicts that $g_\alpha$ is one-to-one. Thus
$g_\alpha \in \mathrm{Aut}(\mathbb{D})$.

\begin{theorem}
Every $f \in \mathrm{Aut}(\mathbb{D})$ factors as
$f = M_c \circ g_\alpha$ for some $c \in \{ |z| = 1 \}$ and
$\alpha \in \mathbb{D}$.
\end{theorem}

\begin{proof}
$M_c, g_\alpha \in \mathrm{Aut}(D)$, so
$M_c \circ g_\alpha \in \mathrm{Aut}(D)$.
If $f \in \mathrm{Aut}(\mathbb{D})$, let
$\alpha = f^{-1}(0) \in D$ and let
$h = f \circ g_\alpha$. Then
$h \in \mathrm{Aut}(\mathbb{D})$ and
$h(0) = f(g_\alpha(0)) = f(\alpha) = 0$, so
$h = M_c$ for some $c \in \partial \mathbb{D}$.
Hence $f = h \circ g_\alpha = M_c \circ g_\alpha$.
\end{proof}

\begin{remark}
Applying Schwarz' lemma with the map $g_\alpha$, we
can obtain, we can obtain some inequalities of
analytic maps $f \in \mathrm{End}(\mathbb{D})$ which
may not fix $0$.
If $z \in \mathbb{D}$, $w = f(z) \in \mathbb{D}$, then
$h : g_w \circ f \circ g_z$ takes 0 to 0, so
we can get an inequality for $f$.
\end{remark}

\subsection{Automorphisms of the Upper Half-Plane}
Let $\mathbb{H} = \{ z \in \mathbb{C} : \Im z > 0 \}$.
\begin{theorem}
$$
  f(z)
= \frac{z - i}{z + i}
\in \mathrm{Iso}(\mathbb{H}, \mathbb{D}).
$$
\end{theorem}
\begin{proof}
$f$ has only one singularity, at $-i$, which is outside $\mathbb{H}$.
Let $z = x + iy \in \mathbb{H}$. Then $y > 0$ so
$$
  |f(z)|^2
= \frac{|x + i(y - 1)|^2}
       {|x + i(y + 1)|^2}
= \frac{x^2 + (y - 1)^2}
       {x^2 + (y + 1)^2}
< 1.
$$
If $w = f(z) = \frac{z - i}{z + i}$ then
$w (z + i) = z - i$ so that
$i w + i = z - z w$, whence
$z = \frac{i (1 + w)}{1 - w}$. Then $f$ is injective.
It remains to show that $f(\mathbb{H}) = \mathbb{D}$. To see this,
we need to show
$$
g(w) = i \frac{1 + w}{1 - w} \in \mathbb{H}.
$$

We have found $f^{-1}(w) = -i \frac{w + 1}{w - 1}$, and it remains
to show that $f^{-1}(w) \in \mathbb{H}$ for all $w \in \mathbb{D}$.
Let $w \in \mathbb{D}$ and write
$$
w = r \cos \theta + i r \sin \theta
$$
where $r \in [0, 1)$ and $\theta \in [0, 2 \pi)$. Then
\begin{align*}
   f^{-1}(w)
&= -i \frac{r \cos \theta + 1 + i r \sin \theta}
           {r \cos \theta - 1 + i r \sin \theta} \\
 = -i \frac{r \cos \theta + 1 + i r \sin \theta}
           {r \cos \theta - 1 + i r \sin \theta}
      \frac{r \cos \theta + 1 - i r \sin \theta}
           {r \cos \theta - 1 - i r \sin \theta} \\
&= -\frac{i}{|c|^2}
    [(r \cos \theta)^2 - (1 + i r \sin \theta)^2] \\
&= -\frac{i}{|c|^2}
    [ r^2 \cos^2 \theta - 1 + r^2 \sin^2 \theta
    - 2 i r \sin \theta] \\
&= -\frac{i}{|c|^2}
   (r^2 - 1 - i 2 r \sin \theta)
 = \frac{-2 r \sin \theta + i(1 - r^2)}
        {|c|^2}
\end{align*}
so that
$$
  \Im(f^{-1}(w))
= \frac{1 - r^2}
       {|r \cos \theta - 1 + i r \sin \theta|^2}
> 0
$$
so $f^{-1}(w) \in \mathbb{H}$.
\end{proof}

Let $z_0 = x_0 + i y_0 \in \mathbb{H}$ and
$$
  h_{z_0}(z)
= \frac{z - z_0}{z - \overline{z_0}}.
$$
We have shown $h_i \in \mathrm{Iso}(\mathbb{H}, \mathbb{D})$.
We may write
\begin{align*}
   h_{z_0}(z)
&= \frac{z - x_0 - i y_0}
        {z - x_0 + i y_0}
 = \frac{\frac{z - x_0}{y_0} - i}
        {\frac{z - x_0}{y_0} + i},
\end{align*}
and the map $z \mapsto \frac{z - x_0}{y_0}$ is a translation by
a real number followed by a scaling by a positive real number, so
this map is an automorphism of $\mathbb{H}$. Thus
$h_{z_0} \in \mathrm{Iso}(\mathbb{H}, \mathbb{D})$.

\begin{remark}
  \begin{enumerate}
    \item{
      Every $f \in \mathrm{Aut}(\mathbb{H})$ can be expressed as
      $f = h_{z_2}^{-1} \circ M_c \circ h_{z_1}$ for some
      $z_1, z_2 \in \mathbb{H}$ and $c$ with $|c| = 1$.
      Choose any $z_1 \in \mathbb{H}$ and let
      $z_2 = f(z_1) \in \mathbb{H}$. Then
      $$
          g = h_{z_2} \circ f \circ h_{z_1}^{-1}
      \in \mathrm{Aut}(\mathbb{D}).
      $$
      Then
      $$
        g(0)
      = (h_{z_2} \circ f \circ h_{z_1}^{-1})(0)
      = h_{z_2} (z_2) = 0,
      $$
      so $g$ is an automorphism of the disk that fixes 0 and thus
      $g$ is a rotation. From
      $$
      M_c = h_{z_2} \circ f \circ h_{z_1}^{-1}
      $$
      for some $c$, we get the desired result.
    }
    \item{
      Combining Schwarz lemma with the maps $h_z$, we can obtain some
      useful inequalities (see HW \#13).
    }
  \end{enumerate}
\end{remark}

\section{Riemann Sphere}
\begin{defn}[Riemann Sphere]
Add an additional element, denoted $\infty$, to $\mathbb{C}$, and let
$\hat{\mathbb{C}} = \mathbb{C} \cup \{ \infty \}$. We call this the
extended complex plane or the \emph{Riemann sphere}. We define a
topology on $\hat{\mathbb{C}}$ with
$$
           D(\infty, r)
\triangleq \{ \infty \} \cup \{z \in \mathbb{C} : |z| > \frac{1}{r} \}.
$$
The open subsets of $\hat{\mathbb{C}}$ are then the open disks, i.e.
if $\infty \in U$ then the definition of open set coincides with the
definition of open sets for $\mathbb{C}$. If $U$ is open in
$\hat{\mathbb{C}}$ and $\infty \in U$, then $U \setminus \{ \infty \}$
is open in $\mathbb{C}$. We see that $z \to \infty \iff |z| \to
\infty$.
\end{defn}

There is a homeomorphism between $\hat{\mathbb{C}}$ and the 2-sphere
$S^2$. Let $h : S^2 \to \hat{\mathbb{C}}$ be a map where
$h(x, y, z)$ is the intersection between the $xy$ plane and
the line connecting the point $(x, y, z) \in \mathbb{R}^3$ with
the north pole of the sphere, i.e. $(0, 0, 1)$, for $(x, y, z) \neq
(0, 0, 1)$, and $h(0, 0, 1) = \infty$. Explicitly,
$$
  h(x, y, z)
= \left\{
    \begin{array}{c c}
      \frac{x + i y}{1 - z}, & \quad z \neq 1 \\
      \infty,                & \quad z = 1
    \end{array}
  \right..
$$
This viewpoint gives a new description of simply connected domains.
For a domain $D \subset \mathbb{C}$, $D$ is simply connected if and
only if $\hat{\mathbb{C}} \setminus D$ is connected.

\begin{xmpl}
  \begin{enumerate}
    \item{
      Let $D = \mathbb{C} \setminus \{ 0 \}$. Then
      $\mathbb{C} \setminus D = \{ 0 \}$ is connected, but
      $\hat{\mathbb{C}} \setminus D = \{ 0, \infty \}$ is disconnected.
    }
    \item{
    }
  \end{enumerate}
\end{xmpl}

\begin{defn}[$n$-connectedness]
  We say that $D \subset \mathbb{C}$ is \emph{$n$-connected} if
  $\hat{\mathbb{C}} \setminus D$ has $n$ connected components.
  A 2-connected domain is also called doubly connected. For
  example, every annulus is 2-connected.
\end{defn}

The Riemann sphere $\hat{\mathbb{C}}$ is a 1-dimensional complex
manifold.

\begin{defn}[Extended analytic function]
Let $U \subset \mathbb{C}$ be an open set.
A map
$f : U \to \hat{\mathbb{C}}$ is called an
\emph{extended analytic function} if for every $z_0 \in U$,
either of the following is true:
\begin{itemize}
  \item{
    $f(z_0) \subset \mathbb{C}$ and $f$ is analytic at $z_0$ in the
    usual sense,
  }
  \item{
    $f(z_0) = \infty$ and $\frac{1}{f}$ is analytic at $z_0$ with the
    convention that $\frac{1}{\infty} = 0$.
  }
\end{itemize}
\end{defn}

This condition means that if $f(z_0) \neq \infty$, then $\exists r > 0$ such
that $f(z) \neq \infty$ on $D(z_0, r)$. If $f(z_0) = \infty$, then
either $\exists r > 0$ such that $f(z) = \infty$ for all
$z \in D(z_0, r)$, or $\exists r > 0$ such that
$f(z) \neq \infty \forall z \in D(z_0, r) \setminus \{ 0 \}$.
In the last case, $z_0$ is a pole of $f$. Thus if $U$ is a domain,
then either $f$ is constant $\infty$ or $f^{-1}(\infty)$ has no
accumulation point in $U$. In the latter case,
$f|_{U \setminus f^{-1}(\infty)}$ is meromorphic on $U$. On the other
hand, if $f$ is meromorphic on $U$, then by defining
$f(z) = \infty$ for every pole $z$ of $f$, we get an extended
analytic function.

\begin{defn}
Suppose $U \subset \hat{\mathbb{C}}$ is open. Then a map
$f : U \to \hat{\mathbb{C}}$ is called an extended analytic
function if:
\begin{enumerate}[(i)]
  \item{
    $f|{U \cap \mathbb{C}}$ is an extended analytic function
    as defined before.
  }
  \item{
    if $\infty \in U$, then the function $g$ defined by
    $g(z) = f\left(\frac{1}{z}\right)$ is extended analytic
    at 0.
  }
\end{enumerate}
In the same spirit we may talk about the
\emph{singularity at $\infty$}. We say that $\infty$ is
a singularity of $f$ if $f$ is defined on a neighborhood of infinity
$\{ z \in \mathbb{C} : |z| > R \}$ for some $R > 0$.
For instance, if $f$ is entire then $\infty$ is a
singularity. Defining $g(z) = f\left(\frac{1}{z}\right)$,
0 is a singularity of $g$. The type of the singularity at $\infty$
of $f$ is defined to be the type of the singularity at 0 of $g$.
\end{defn}

Suppose the Laurent series of $f$ in $\{ R < |z| < \infty \}$ is
$\sum_{n=-\infty}^\infty a_n z^n$.. Then
$$
  g(z)
= \sum_{n=-\infty}^\infty a_n z^{-n}
= \sum_{n=-\infty}^\infty a_{-n} z^n, \quad
  0 < |z| < \frac{1}{R}.
$$
We see that
\begin{itemize}
  \item{
    $\infty$ is a removable singularity if $a_n = 0$, $\forall n > 0$,
  }
  \item{
    $\infty$ is a pole if there exist finitely many $n > 0$ such that
    $a_n \neq 0$.
  }
  \item{
    $\infty$ is a pole if there exist infinitely many $n > 0$ such
    that $a_n \neq 0$.
  }
\end{itemize}

Suppose $f$ is entire. Then $\infty$ is removable if and only if
$f \equiv a_0$, $\infty$ is a pole if and only if $f$ is a polynomial
of degree $\geq 1$, and $\infty$ is an essential singularity
otherwise.

\begin{lemma}
Every analytic isomorphism $f$ of $\mathbb{C}$ has the form
$$
f(z) = a z + b
$$
for some $a, b \in \mathbb{C}$ with $a \neq 0$.
\end{lemma}

\begin{proof}
$f$ is entire. Consider the type of singularity of $f$ at $\infty$.
We know that $f$ is not constant since $f$ is surjective,
so $\infty$ is not removable. If $\infty$ is an essential singularity
then $f(\{ |z| > 1 \})$ is dense in $\mathbb{C}$. But
$f(\{ |z| > 1 \})$ is disjoint from $f(\{ |z| < 1 \})$ since $f$ is
injective, and $f(\{ |z| < 1 \})$ is open. Therefore $f(\{|z| < 1\})$
cannot be dense, a contradiction. Therefore $\infty$ is a pole, so
$f$ is a polynomial.

Next we see that since $f$ is an isomorphism, $f^\prime \neq 0$ for
any $z \in \mathbb{C}$. $f^\prime$ is also a polynomial and nonzero,
so from the fundamental theorem of algebra, $f^\prime$ is constant.
Therefore $f(z) = az + b$.

We can conclude also that
$\mathrm{Iso}(\mathbb{C}, U) = \varnothing$ if $U \neq \mathbb{C}$
and
$$
  \mathrm{Aut}(\mathbb{C})
= \{ az + b : a, b \in \mathbb{C}, a \neq 0 \}.
$$
\end{proof}

Let $P, Q$ be polynomials with $Q \nequiv 0$. Let
$R = \frac{P}{Q}$. We know that $R$ is meromorphic on $\mathbb{C}$,
so $R$ is extended analytic. We consider
$$
  R\left(\frac{1}{z}\right)
= \frac{P\left(\frac{1}{z}\right)}
       {Q\left(\frac{1}{z}\right)}
= \frac{\tilde{P}(z)}
       {\tilde{Q}(z)}
$$
with $\tilde{P}$, $\tilde{Q}$ some polynomials. If we suitably
define $R(\infty)$, then $R$ is extended analytic on
$\hat{\mathbb{C}}$. In fact, every function that is extended
analytic on all of $\hat{\mathbb{C}}$ is a rational function,
but we will not prove this.

If $f$, $g$ are extended analytic functions such that
$f(U) \subset V$, then $g \circ f$ is also extended analytic.
If $f$ is injective on $U$ and $f(U) = V$, then we say
$f$ is an extended analytic isomorphism of $U$ with $V$.
In this case $f^{-1}$ is an extended analytic function on $V$.
If $V = U$, we say that $f$ is an extended analytic automorphism.

\subsection{M\"obius Transformation}
There is a group homomorphism
$\psi : \mathrm{GL}_2(\mathbb{C}) \to \mathrm{Aut}(\hat{\mathbb{C}})$
given by
$$
  \psi\left(
    \left[
      \begin{array}{c c}
        a & b \\
        c & d
      \end{array}
    \right]
  \right)(z)
= \frac{a z + b}{c z + d}
= f_M(z).
$$
We note that $f_{rM} = f_M$
for any $r \in \mathbb{C} \setminus \{ 0 \}$.
If $c = 0$, $f_M(z) = \frac{a}{d} z + \frac{b}{d}$
and $f_M(\infty) = \infty$. If $c \neq 0$, then
$f(-\frac{d}{c}) = \infty$, $f(\infty) = \frac{a}{c}$.
These transformations are called \emph{M\"obius transformations}
or \emph{linear fractional transformations}.

\begin{xmpl}
  \begin{itemize}
    \item{
      For
      $$
        M
      = \left[
          \begin{array}{c c}
            a & b \\
            0 & 1
        \right]
      $$
      we have $f_M(z) = a z + b$.
    }
    \item{
      For
      $$
        M
      = \left[
          \begin{array}{c c}
            0 & 1 \\
            1 & 0
        \right]
      $$
      we have $f_M(z) = \frac{1}{z}$.
    }
    \item{
      For
      $$
        M
      = \left[
          \begin{array}{c c}
            -1            & \alpha \\
            -\bar{\alpha} & 1
        \right]
      $$
      we have $f_M = g_\alplha$.
    }
    \item{
      For
      $$
        M
      = \left[
          \begin{array}{c c}
            1 & -z_0 \\
            1 & -\overline{z_0}
        \right]
      $$
      we have $f_M(z) = \frac{z - z_0}{z - \overline{z_0}}$.
    }
    \item{
      We can see
      \begin{align*}
         (f_{M_1} \circ f_{M_2})(z)
      &= f_{M_1 \cdot M_2}(z)
      \end{align*}
      as long as $c_2 z + d_2 \neq 0$ and
      $(c_1 a_2 + d_1 c_2)z + (c_1 b_2 + d_1 d_2) \neq 0$
      so that this is a group homomorphism for all
      $z \in \hat{\mathbb{C}}$ with a few exceptions. But
      these exceptions do not in fact exist since
      $f_{M_1}$, $f_{M_2}$, and $f_{M_1 \cdot M_2}$ are continuous
      on $\hat{\mathbb{C}}$. Indeed, for any
      $M \in \mathrm{GL}_2(\mathbb{C})$ we have
      $$
      f_M \circ f_{M^{-1}} = f_{M^{-1}} \circ f_M = f_I = \id,
      $$
      so $f_M \in \mathrm{Aut}(\hat{\mathbb{C}})$ and
      $f_M^{-1} = f_{M^{-1}}$.
    }
  \end{itemize}
\end{xmpl}

\begin{theorem}
Every $f \in \mathrm{Aut}(\mathbb{D})$ or
$\mathrm{Aut}(\mathbb{H})$ is a M\"obius transformation.
\end{theorem}

\begin{proof}
Let $f \in \mathrm{Aut}(\mathbb{D})$. Recall that we may
express $f$ as $M_c \circ g_a$, both of which are Mobius
transformations.

Let $f \in \mathrm{Aut}(\mathbb{H})$. Recall that
$$
  h_i(z)
= \frac{z - i}{z + i}
\in \mathrm{Iso}(\mathbb{H}, \mathbb{D})
$$
so that
$g = h_i \circ f \circ h_i^{-1} \in \mathrm{Aut}(\mathbb{D})$.
But then $g$ is a M\"obius transformation and so is $h_i$,
so $f$ is as well.
\end{proof}

\begin{defn}
We define the following \emph{simple M\"obius transformations}:
\begin{enumerate}
  \item{
    $M_a(z) = a z$, multiplication by $a \in \mathbb{C} \setminus 0$.
  }
  \item{
    $T_b(z) = z + b$, translation by $b \in \mathbb{C}$.
  }
  \item{
    $J(z) = \frac{1}{z}$, inversion.
  }
\end{enumerate}
\end{defn}

\begin{lemma}
The simple M\"obius transformations generate the M\"obius group
$\mathrm{PGL}_2(\mathbb{C})$.
\end{lemma}

\begin{proof}
Suppose $c = 0$. Then
\begin{align*}
   f(z)
&= \frac{az + b}{d}
 = \frac{a}{d} z + \frac{b}{d} \\
&= (T_{\frac{b}{a}} \circ M_{\frac{a}{d}})(z).
\end{align*}
Suppose $c \neq 0$. Then
\begin{align*}
   f(z) - \frac{a}{c}
&= \frac{a z + b}{c z + d}
 - \frac{a(z + \frac{d}{c})}
        {c(z + \frac{d}{c})} \\
&= \frac{b - \frac{ad}{c}}
        {c z + d}
 = \frac{b^\prime}{c z + d}
\end{align*}
so that
$$
  \frac{1}{f(z) - \frac{a}{c}}
= \frac{cz + d}{b^\prime}
$$
or
$$
  f(z)
= \frac{a}{c}
+ \frac{1}
       {\frac{c}{b^\prime} z + \frac{d}{b}}
=     T_{\frac{a}{c}}
\circ J
\circ T_{\frac{d}{b^\prime}}
\circ M_{\frac{c}{b}}.
$$
\end{proof}

\begin{defn}[Generalized circle]
A \emph{generalized circle} is either a circle in
$\mathbb{C}$ or the union of $\infty$ with a
straight line in $\mathbb{C}$.

A \emph{generalized disk} is either a disk in
$\mathbb{C}$, or the exterior of a circle together
with $\infty$, or a half-plane.
\end{defn}

\begin{remark}
  \begin{enumerate}
    \item{
      A straight line can be viewed as a circle with
      radius $\infty$.
    }
    \item{
      Every generalized circle divides $\hat{\mathbb{C}}$
      into two generalized disks.
    }
    \item{
      The stereographic projection generates a 1-1
      correspondence between circles on $S^2$ and generalized
      circles on $\hat{\mathbb{C}}$. The circles on $S^2$ that
      pass through the north pole correspond to straight lines.
    }
  \end{enumerate}
\end{remark}

\begin{theorem}
A M\"obius transformation maps generalized circles to generalized
circles.
\end{theorem}

\begin{proof}
It suffices to show that each simple M\"obius transformation
maps generalized circles to generalized circles. This is obvious
in the case of translation. Suppose $a = |a|e^{i \theta}$. Then
$M_a = M_{|a|} \circ M_{e^{i \theta}}$, which is a rotation followed
by a dilation, both of which preserve circles and straight lines.

Consider $J(z) = \frac{1}{z}$. The equation of a circle or line
in the $(x, y)$ plane has the form
$$
A(x^2 + y^2) + B x + C y + D = 0,
$$
since the equation for a circle is
$$
  0
= (x - x_0)^2 + (y - y_0)^2
= x^2 + y^2 - 2x_0 x - 2 y_0 y + (x_0^2 + y_0^2 - R^2).
$$
Suppose $u + i v = J(x + i y)$ and $(x, y)$ satisfies this equation.
Then $J(u + iv)$ satisfies this equation since $J$ is its own inverse.
Thus we have
$$
  x + iy
= \frac{1}{u + iv}
$$
 so that
$$
  x
= \frac{u}{u^2 + v^2}, \quad
  y
= \frac{-v}{u^2 + v^2}
$$
so that
$$
  A\left(
    \left(
      \frac{u}{u^2 + v^2}
    \right)^2
  + \left(
      -\frac{v}{u^2 + v^2}
    \right)^2
  \right)
+ B \frac{u}{u^2 + v^2}
+ C \frac{-v}{u^2 + v^2}
+ D
= 0
$$
or
$$
A + Bu + C(-v) + D(u^2 + v^2) = 0.
$$
\end{proof}

Let $f$ be a M\"obius transformation and $C$ be a generalized
circle. Let $C^\prime = f(C)$. $C$ divides $\hat{\mathbb{C}}$ into
two generalized disks $D_1, D_2$, as does $C^\prime$ into
$D_1^\prime$ and $D_2^\prime$. Then either
$f(D_1) = D_1^\prime$ and $f(D_2) = D_2^\prime$ or
$f(D_1) = D_2^\prime$ and $f(D_2) = D_1^\prime$. To find out which,
we may pick $z_0 \in D_1$ and look at $f(z_0$. On the other hand,
if $f$ maps a generalized disk onto a generalized disk, then
$f$ maps a boundary of the disk onto the boundary of the other
disk. For example, if $f \in \mathrm{Aut}(\mathbb{H})$, then it maps
$\mathbb{R} \cup {\infty} \to \mathbb{R} \cup {\infty}$

\begin{theorem}
Given three distinct points $z_1, z_2, z_3$ and three distinct points
$w_1, w_2, w_3$ in the Riemann sphere, there exists a unique M\"obius
transformation such that $f(z_j) = w_j$, $j = 1, 2, 3$.
\end{theorem}

\begin{proof}
For existence, it suffices to show that such $f$ exists if $w_1 = 0$,
$w_2 = \infty$, $w = 1$. In fact, if we let $F_{z_1, z_2, z_3}$ denote
a transformation that takes $z_1 \to 0$, $z_0 \to \infty$, and
$z_3 \to 1$, then
$$
f = F_{w_1, w_2, w_3}^{-1} \circ F_{z_1, z_2, z_3}
$$
is the transformation that we need. Suppose $w_1 = 0$,
$w_2 = \infty$, $w_3 = 1$. We note that
$\frac{z - z_1}{z - z_2}$ takes $z_1 \to 0$, $z_2 \to \infty$,
but not $z_3 \to 1$. But
$$
  f(z)
= \frac{z_3 - z_2}
       {z_3 - z_1}
  \frac{z - z_1}
       {z - z_2}
$$
and we can check directly that this proves the existence of
such a transformation.

Suppose $f, g$ are two M\"obius transformations meeting these
criteria. Define
$$
h = F_{z_1,z_2,z_3} \circ g^{-1} \circ f \circ F_{z_1, z_2, z_3}^{-1}.
$$
Then $h$ is a M\"obius transformation that fixes $0, \infty, 1$.
Since $h(\infty) = \infty$, $h$ is a polynomial. Since $h(0) = 0$,
$b = 0$. Since $h(1) = 1$, $h(z) = z$, and thus $g = f$.
\end{proof}

Given $z_1, z_2, z_3$ and $w_1, w_2, w_3$, we can find a M\"obius
transformation $f$ such that $f(z) = \frac{az + b}{cz + d}$.
We can solve a group of linear equations to find $a, b, c, d$
using $f(z_j) = w_j$. We may assume one of them is 1. If it is
in fact 0, we get a contradiction, and then we may assume another
is 1.

\begin{defn}
Let $z_1, z_2, z_3, z_4$ be distinct points. Their \emph{cross-ratio}
is defined to be
$$
  [z_1, z_2, z_3, z_4]
= \frac{z_1 - z_3}{z_2 - z_3}
: \frac{z_1 - z_4}{z_2 - z_4}
= \frac{(z_1 - z_3)(z_2 - z_4)}
       {(z_2 - z_3_(z_1 - z_4)}.
$$
If any of these factors is $\infty$, we have instead e.g.
$$
  [z_1, z_2, z_3, \infty]
= \frac{z_1 - z_3}{z_2 - z_3}.
$$
The cross-ratio satisfies some symmetry relations
such as
$$
  [z_2, z_1, z_3, z_4]
= [z_1, z_2, z_3, z_4]^{-1}
= [z_1, z_3, z_4, z_3]
$$
and
$$
[z_1, z_2, z_3, z_4] = [z_3, z_4, z_1, z_2].
$$
\end{defn}

\begin{theorem}
M\"obius transformations preserve the cross-ratio,
i.e.
$$
  [F(z_1), F(z_2), F(z_3), F(z_4)
= [z_1, z_2, z_3, z_4]
$$
for any M\"obius transformation $F$.
\end{theorem}
\begin{proof}
Consider the map $z_4 \mapsto [z_1, z_2, z_3, z_4]$.
We observe that this is a M\"obius transformation that maps
$z_1 \to \infty$, $z_2 \to 0$, $z_3 \to 1$. Thus this is the
map $F_{z_2, z_1, z_3} (z_4)$, and
\begin{align*}
   F_{z_2, z_1, z_3}
&= F_{z_2^\prime, z_1^\prime, z_3^\prime} \circ F(z_4)
 = F_{z_2^\prime, z_1^\prime, z_3^\prime}(z_4^\prime) \\
&= [z_1^\prime, z_2^\prime, z_3^\prime, z_4^\prime].
\end{align*}
\end{proof}

\subsection{Riemann Mapping Theorem}

\begin{defn}[Conformal equivalence]
Let $D_1, D_2$ be two complex domains. We say that
$D_1$ is \emph{conformally equivalent} to $D_2$ and write
$D_1 \simeq D_2$ if
$\mathrm{Iso}(D_1, D_2) \neq \varnothing$. This is
an equivalence relation.
\end{defn}

\begin{theorem}[Riemann Mapping Theorem]
Let $U \subsetneq \mathbb{C}$ be a simply connected domain.
Let $z_0 \in U$. Then there exists an
$f \in \mathrm{Iso}(U, \mathbb{D})$ such that $f(z_0) = 0$,
and such $f$ is unique up to rotation, i.e. if
$g \in \mathrm{Iso}(U, \mathbb{D})$ and $g(z_0) = 0$ then
$g = M_c \circ f$ for some $c$ with $|c| = 1$.

If we require that $f^\prime(z_0) = 0$, then $f$ is unique.
\end{theorem}
