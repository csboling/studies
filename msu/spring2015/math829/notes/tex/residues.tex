\section{Residue Formula}
\begin{defn}[Residue]
Let $z_0$ be an isolated singularity of $f$, and
$$
f(z) = \sum_{n=-\infty}^\infty a_n (z - z_0)^n, \quad
z \in D(z_0, r) \backslash \{ z_0 \}.
$$
We call $a_{-1}$ the \emph{residue} of $f$ at $z_0$ and write
$\mathrm{Res}_{z_0} f = a_{-1}$.
\end{defn}

\begin{lemma}
If $t \in (0, r)$ then
$$
  \int_{\{|z - z_0| = t \}} f
= 2 \pi i a_{-1}.
$$
\end{lemma}
\begin{proof}
Recall
$$
  a_{-1}
= \frac{1}{2 \pi i}
  \int_{|z - z_0| = t}
    \frac{f(z)}{(z - z_0)^0}
    \dif z.
$$
\end{proof}

\begin{theorem}
$f$ has a primitive in $D(z_0, r) \backslash \{ z_0 \}$
if and only if $\mathrm{Res}_{z_0} f = 0$.
\end{theorem}
\begin{proof}
If $f$ has a primitive then it is clear from the lemma that
$a_{-1} = 0$. If $a_{-1} = 0$, then a primitive of $f$ is
$$
F(z) = \sum_{n=-\infty}^\infty \frac{a_n}{n + 1} (z - z_0)^{n+1}.
$$
\end{proof}

\begin{theorem}[Residue Formula for Jordan Curves]
Let $J$ be a positively oriented Jordan curve. Suppose that
$f$ is holomorphic on $\mathrm{Int}(J) \cup J$ except at finitely
many points $z_i \in \mathrm{Int}(J)$. Then
$$
\int_J f = 2 \pi i \sum_{j=1}^n \mathrm{Res}_{z_j} f.
$$
\end{theorem}
\begin{proof}
Pick $r > 0$ such that the closed disks $\bar{D}(z_j, r)$ are
mutually disjoint and are all contained in $\mathrm{Int}(J)$.
Then from Cauchy's theorem,
$$
  \int_J f
= \sum_{j=1}^n \int_{|z - z_j| = r} f
= \sum_{j=1}^n 2 \pi i \mathrm{Res}_{z_0} f.
$$
\end{proof}

\begin{remark}
Cauchy's Theorem and Cauchy's Formula may be viewed as special cases
of the residue formula. If there is no singularity of $f$ in
$\mathrm{Int}(J)$,
then the residue formula reduces to Cauchy's Theorem. Suppose $f$ is
holomorphic on $\mathrm{Int}(J) \cup J$, $z_0 \in \mathrm{Int}(J)$,
and $m \in \mathbb{N} \cup \{0\}$. Let $g(z) = \frac{f(z)}{(z -
  z_0)^{m+1}}$.
Suppose
$$
f(z) = \sum_{n=0}^\infty \frac{f^{(n)}(z_0)}{n!}(z - z_0)^n.
$$
Then
$$
g(z) = \sum_{n=0}^\infty \frac{f^{(n)}(z_0)}{n!} (z - z_0)^{n - m - 1}.
$$
Then
$$
  \mathrm{Res}_{z_0} g
= \frac{f^{(m)}(z_0)}{m!}
= \frac{1}{2 \pi i}
  \int_J
    \frac{f(z)}
         {(z - z_0)^{m+1}}
    \dif z.
$$
\end{remark}

\begin{xmpl}
We will see how to calculate the residues of a function.
\begin{enumerate}
  \item{
    If $f$ is holomorphic at $z_0$, then
    $$
    f(z) = \sum_{n=0}^\infty a_n (z - z_0)^n
    $$
    near $z_0$ so
    $$
    \frac{f(z)}{(z - z_0)^m} = \sum_{n=0}^\infty a_n (z - z_0)^{n-m}
    $$
    and then
    $$
    \mathrm{Res}_{z_0} \frac{f(z)}{(z - z_0)^n}
    $$
    is the coefficient $a_n$ such that $n - m = 1$.
  }
  \item{
    We wish to find $\mathrm{Res}_{0} \frac{\sin z}{z^6}$.
    If
    $$
    \sin z = \sum_{n=0}^\infty a_n z^n,
    $$
    then
    $$
      \mathrm{Res}_0 \frac{\sin z}{z^6}
    = a_5
    = \frac{\sin^{(5)}(0)}{5!}
    = \frac{\cos(0)}{5!}
    = \frac{1}{120}.
    $$
  }
  \item{
    Let
    $$
      f(z)
    = \frac{z^2}{(z+1)(z-1)^2}.
    $$
    Find $\mathrm{Res}_1 f$.

    First let $g(z) = \frac{z^2}{z+1}$, so $f(z) =
    \frac{g(z)}{(z-1)^2}$. If
    $$
    g(z) = \sum_{n=0}^\infty a_n (z - 1)^n
    $$
    near 1, then $\mathrm{Res}_1 f = a_1$. But
    $$
      a_1
    = g^\prime(1)
    = \left.
        \frac{2z}{z + 1} - \frac{z^2}{(z + 1)^2}
      \right|_1
    = \frac{2}{2} - \frac{1}{2} = \frac{3}{4}.
    $$

    Next we wish to find $\int_{|z - 1| = 1} f$. Note that
    $f$ has singularities at $\pm 1$, of which only 1 lies
    inside this circle. Then
    $$
    \int_{|z - 1| = 1} f = 2 \pi i \mathrm{Res}_1 f = \frac{3}{2} \pi i.
    $$
  }
\end{enumerate}
\end{xmpl}

\begin{lemma}
  Let $f, g$ be holomorphic at $z_0$. Suppose $f(z_0) = 0$ and
  $f^\prime(z_0) \neq 0$, i.e. $\ord_{z_0} f = 1$. Then
  $$
    \mathrm{Res}_{z_0} \frac{g}{f}
  = \frac{g(z_0)}{f^\prime(z_0)}.
  $$
\end{lemma}
\begin{proof}
  We may write $f(z) = F(z) (z - z_0)$ near $z_0$, where $F$ is
  holomorphic at $z_0$ and $F(z_0) = f^\prime(z_0) \neq 0$, since
  $$
  f(z) = \sum_{n=1}^\infty a_n (z - z_0)^n, \quad
  F(z) = \sum_{n=1}^\infty a_n (z - z_0)^{n-1}.
  $$
  Since $F, g$ are holomorphic at $z_0$ and $F(z_0) \neq 0$,
  $\frac{g}{F}$ is holomorphic at $z_0$. But
  $$
  \frac{g(z)}{f(z)} = \frac{\frac{g(z)}{f(z)}}{z - z_0}.
  $$
  Then
  $$
    \mathrm{Res}_{z_0} \frac{g}{f}
  = \left.\frac{g}{F}\right|_{z_0}
  = \frac{g(z_0)}{F(z_0)}
  = \frac{g(z_0)}{f^\prime(z_0)}
  $$
\end{proof}

\begin{xmpl}
  \begin{enumerate}
    \item{
      Consider $\cot z = \frac{\cos z}{\sin z}$, which has
      singularities at zeros of $\sin z$, namely $k \pi$ for
      $k \in \mathbb{Z}$.
      At $k\pi$, $\sin k\pi = 0$ and $\sin^\prime(k \pi) = \pm 1 \neq
      0$. Applying the lemma, we see
      $$
        \mathrm{Res}_{k \pi} \cot z
      = \frac{\cos k \pi}{\sin^\prime(k \pi)}
      = 1,
      $$
      so
      $$
        \int_{|z| = 5}
          \cot z
      = 2 \pi i
          ( \mathrm{Res}_0 \cot z
          + \mathrm{Res}_\pi \cot z
          + \mathrm{Res}_{-\pi} \cot z
          )
      = 6 \pi i.
      $$
    }
    \item{
      To find $\mathrm{Res}_0 \frac{e^z}{\sin z}$ we see that
      $\sin(0) = 0$, $\sin^\prime(0) = 1$, so
      $\mathrm{Res}_0 \frac{e^z}{\sin z} = 1$.
    }
    \item{
      Let $f(z) = z^2 - 2z + 3$,
      $R = [-1, 3] \times [-2, 2]$, and find
      $\int_{\partial R} \frac{1}{f}$.

      We see that $\frac{1}{f}$ has singularities at
      $1 \pm \sqrt{2} i$, both inside $\partial R$.
      Then
      $$
        \int_{\partial R} \frac{1}{f}
      = 2 \pi i
        (
          \mathrm{Res}_{z_1} \frac{1}{f}
        + \mathrm{Res}_{z_2} \frac{1}{f}
        ).
      $$
      Since $f = 0$ at $z_1, z_2$ and
      $f^\prime = 2z - 2 = \pm 2 \sqrt{2} i \neq 0$
      at $z_1, z_2$ we can apply the lemma to find
      $$
        \mathrm{Res}_{z_1}
          \frac{1}{f}
      = \frac{1}{f^\prime(z_1)} = \frac{1}{2 \sqrt{2} i}
      $$
      and
      $$
        \mathrm{Res}_{z_1}
          \frac{1}{f}
      = \frac{1}{f^\prime(z_2)} = \frac{1}{-2 \sqrt{2} i}
      $$
      so that $\int_{\partial R} \frac{1}{f} = 0$.
    }
  \end{enumerate}
\end{xmpl}

\begin{theorem}[General Residue Formula]
Let $\gamma$ be a contour in a domain $U$ such that
$W(\gamma, \alpha) = 0$ for all
$\alpha \in \mathbb{C} \backslash U$. Suppose that $f$ is
holomorphic on $U$ except on a set $S$, which has no
accumulation point in $U$, and does not intersect $\gamma$.
Then
$$
  \int_\gamma f
= 2 \pi i
  \sum_{w \in S}
    W(\gamma, w)
    \mathrm{Res}_w f.
$$
\end{theorem}

\begin{proof}
  Although $S$ can be infinite, there are at most finitely many
  $w \in S$ such that $W(\gamma, w) \neq 0$. Let $S^\prime$
  denote the set of such $w$ and define a new contour by
  $$
    \eta
  = \gamma
  - \sum_{w \in S^\prime}
      W(\gamma, w)
      \{ |z - w| = r \}
  $$
  where $r$ is chosen such that $\bar{D}(w, r)$,
  $w \in S^\prime$ are mutually disjoint and contained in
  $U \backslash \gamma$. From the general Cauchy theorem,
  $\int_\eta f = 0$., so that
  $$
    \int_\gamma f
  = \sum_{w \in S}
      W(\gamma, w)
      \int_{|z - w| = r} f
  = 2 \pi \mathrm{Res}_w f.
  $$
\end{proof}
