\section{Evaluating Definite Integrals}

We have the definite of an improper integral on the real line as
$$
           \int_{-\infty}^\infty
             f(x)
             \dif x
\triangleq \lim_{R \to \infty}
             \int_{-R}^0
               f(x)
               \dif x
         + \lim_{R \to \infty}
             \int_0^R
               f(x)
               \dif x.
$$
Assume $f$ is holomorphic on the closed upper half plane
${\Im z > 0}$ except at finitely many singularities $z_k$
in the open upper half plane. Choose $R > 0$ such that $|z_k| < R$
for $1 \leq k \leq n$. Let $S_R$ denote the semicircle
$\gamma(t) = R e^{it}$, $0 \leq t \leq \pi.$ Then we have a
positively oriented Jordan curve
$$
\Gamma := [-R, R] \oplus S_R
$$
so that
$$
  2 \pi i \sum_{k=1}^n Res_{z_k} f
= \int_{\Gamma} f
= \int_{[-R, R]} f + \int_{S_R} f
= \int_{-R}^{R}
    f(x)
    \dif x
$$
since
$$
  \int_{[-R, R]} f
= \int_{-R}^R f(-R(1 - t) + R t) 2 R \dif t
= \int_{-R}^R f(u) \dif u.
$$
If $\lim_{R \to \infty} \int_{S_R} f = 0$ then this means
$$
  \lim_{R \to \infty}
  \int_{-R}^R
    f(x) \dif x
= 2 \pi i
  \sum_{k=1}^n
    \Res_{z_k} f.
$$
If this limit exists, we denote it by
$$
  \lim_{R \to \infty}
  \int_{-R}^R
    f(x) \dif x
= \mathrm{P.V.}
    \int_{-\infty}^\infty
      f(x)
      \dif x,
$$
the \emph{principal value} of the improper integral.

\begin{enumerate}
  \item{
    If $\int_{-\infty}^\infty f(x) \dif x$ exists, then its
    principal value also exists and they are equal.
  }
  \item{
    If the principal value exists, the improper integral may not
    exist. For instance, consider $f(x) = x$. This is an odd function,
    so $\int_{-R}^R x \dif x = 0$. But
    $\int_0^R x \dif x = \frac{x^2}{2} \to \infty$ and
    $\int_{-R}^0 x \dif x = -\frac{x^2}{2} \to -\infty$, and
    $-\infty + \infty$ is not defined.
  }
  \item{
    If $f \geq 0$ or is even, then the existence of the principal
    value implies the existence of the improper integral. Moreover, if
    $f$ is even and $\int_{-\infty}^\infty f(x) \dif x$ exists, then
    $$
      \int_0^\infty f(x) \dif x
    = \int_{-\infty}^0 f(x) \dif x
    = \frac{1}{2}
      \int_{-\infty}^\infty f(x) \dif x.
    $$
  }
\end{enumerate}

\begin{xmpl}
  Consider the integral
  $$
  \int_{-\infty}^\infty \frac{\dif x}{x^4 + 1}.
  $$
  The function $f(z) = \frac{1}{z^4 + 1}$ is meromorphic and has
  simple poles at the 4th roots of -1, two of which lie in the
  upper half plane. Since the integrand is even, we proceed by:
  \begin{enumerate}
    \item{
       computing the residues of $f$ at
       $\frac{1}{\sqrt{2}}(\pm 1 + i)$, namely
       $$
         \Res_{z_i} f
       = \frac{1}{\left.(z^4 + 1)^\prime\right|_{z_i}}
       = \frac{1}{4z_i^3}
       = -\frac{z_i}{4}
       $$
       so that
       $$
         2 \pi i (\Res_{z_1} f + \Res_{z_2} f)
       = -\frac{2\pi i}{4}(z_1 + z_2) = \frac{\pi}{\sqrt{2}}.
       $$
     }
     \item{
       showing that $\int_{S_R} f \to 0$ by estimating
       $$
            \left|
              \int_{S_R} f
            \right|
       \leq \| f \|_{S_R} L(S_R)
       =    \frac{\pi R}{R^4 - 1} \to 0
       $$
       since $|z^4 + 1| \geq |z^4| - 1 = R^4 - 1$.
     }
     \item{
       concluding that
       $\int_{-\infty}^\infty f(x) \dif x = \frac{\pi}{\sqrt{2}}$.
     }
  \end{enumerate}
\end{xmpl}

\begin{xmpl}
  Consider
  $$
  \int_0^\infty \frac{\sin x}{x} \dif x.
  $$
  We know that $\frac{\sin z}{z}$ extends to an entire function
  since its only singularity $z$ is removable. However,
  $\int_{S_R} \frac{\sin z}{z} \dif z$ is hard to work with since
  $\sin z$ is unbounded on this semicircle. Instead let
  $$
  f(z) = \frac{e^{iz}}{z} = \frac{\cos z + i \sin z}{z},
  $$
  which has exactly one pole at $z = 0$. We cannot integrate $f$ over
  the $\Gamma$ defined previously, since this function is not defined
  at 0. Instead we choose
  $$
    \Gamma
  = [\varepsilon, R]
  \oplus S_R
  \oplus [-R, -\varepsilon]
  \oplus S_{\varepsilon}^{-}
  $$
  and $f$ has no pole inside $\Gamma$. Therefore
  \begin{align*}
     0
  &= \int_{\Gamma} f
   = \int_{\varepsilon}^R
       \frac{e^{ix}}{x}
       \dif x
   + \int_{-R}^{-\varepsilon}
       \frac{e^{ix}}{x}
       \dif x
   + \int_{S_R} f
   - \int_{S_\varepsilon} f.
  \end{align*}
  Notice that
  \begin{align*}
    \int_\varepsilon^R
      \frac{e^{ix}}{x}
      \dif x
  + \int_{-R}^{-\varepsilon}
      \frac{e^{ix}}{x}
      \dif x
  = \int_\varepsilon^R
      \frac{e^{ix}}{x}
      \dif x
  - \int_{varepsilon}^R
      \frac{e^{-ix}}{x}
      \dif x
  = 2i
    \int_\varepsilon^R
      \frac{\sin x}{x}
      \dif x.
  \end{align*}
  Next,
  \begin{enumerate}
    \item{
      $$
           \left|
             \int_{S_R} f
           \right|
      \leq \| f \|_{S_R} L(S_R)
      =    \pi
      $$
      since
      $$
        |f(z)|
      = \frac{|e^{iz}|}{|z|}
      = \frac{e^{\Re (i z)}}{R}
      = \frac{e^{-\Im z}}{R}
      \leq \frac{1}{R},
      $$
      so this inequality cannot be used to show
      the integral over the semicircle goes to zero.
      Instead use
      \begin{align*}
            \left|
              \int_{S_R} f
            \right|
      &=    \left|
              \int_0^\pi
                \frac{e^{i R e^{it}}}{Re^{it}}
                i R e^{it}
                \dif t
            \right|
       =    \left|
              \int_0^\pi
                e^{i R e^{it}}
                \dif t
            \right| \\
      &\leq \int_0^\pi
              |e^{i R e^{it}}|
              \dif t
       =    \int_0^\pi
              e^{-R \sin t}
              \dif t
       =    2 \int_0^{\pi / 2}
                e^{-R \sin t}
                \dif t.
      \end{align*}
      Now $\sin x$ is concave on $[0, \pi]$ since
      $\sin^{\prime\prime} x = - \sin x \leq 0$ on $[0, \pi]$,
      so $\sin x \geq \frac{x}{\pi / 2}$ on $[0, \pi / 2]$.
      Therefore
      \begin{align*}
            2 \int_0^{\pi / 2}
                e^{-R \sin t}
                \dif t
      &\leq 2 \int_0^{\pi / 2} e^{-\frac{R t}{\pi / 2}}
       =    2 \frac{\pi / 2}{R}
            \int_0^R
              e^{-s}
              \dif s
       \leq \frac{\pi}{R}
            \int_0^\infty
              e^{-s}
              \dif s \\
      &=    \frac{\pi}{R}.
      \end{align*}
      Therefore $\left| \int_{S_R} f \right| \to 0$ as $R \to \infty$.
    }
    \item{
      We compute $\lim_{\varepsilon \to 0} \int_{S_\varepsilon}
      \frac{e^{iz}}{z}$.
      Recall that the curve $S_\varepsilon$ is parameterized by
      $\gamma(t) = \varepsilon e^{it}$, $0 \leq t \leq 2 \pi$.
      For $\varepsilon > 0$, $\Theta \in (0, 2 \pi)$, let
      $\gamma_{\varepsilon, \theta}$ denote the curve
      $\gamma(t) = \varepsilon e^{it}$, $0 \leq t \leq \theta$.

      \begin{lemma}
        Suppose $f$ has a simple pole at 0. Then
        $$
          \lim_{\varepsilon \to 0}
            \int_{\gamma \varepsilon \theta} f
        = i \theta
          \Res_0 f.
        $$
      \end{lemma}
      \begin{proof}
        Let $a = \Res_0 f$. We may write
        $$
        f(z) = \frac{a}{z} + h(z),
        $$
        where $h(z)$ is holomorphic at 0. We compute
        $$
          \int_\gamma
            \frac{a}{z}
        = \int_0^\theta
            \frac{a}{\varepsilon e^{it}}
            i \varepsilon e^{it}
            \dif t
        = i a \theta.
        $$
        Since $h$ is bounded near 0 and
        $L(\gamma) = \varepsilon \theta \to 0$,
        we get $\int_\gamma h \to 0$ as $\varepsilon \to 0$.
        Then
        $$
          \lim_{\varepsilon \to 0}
            \int_{\gamma_{\varepsilon, \theta}} f
        = \lim_{\varepsilon \to 0}
            \int_{\gamma_{\varepsilon, \theta}}
            \frac{a}{z}
        = i \theta a.
        $$
      \end{proof}
      If $\theta = \pi$ and $f(z) = \frac{e^{i z}}{z}$, we get
      $$
        \lim_{\varepsilon \to 0}
          \int_{S_\epsilon}
            \frac{e^{iz}}{z}
      = i \pi
      $$
      and so
      \begin{align*}
         0
      &= \int_\varepsilon^R
           \frac{e^{ix}}{x}
           \dif x
       + \int_{-R}^{-\varepsilon}
           \frac{e^{ix}}{x}
           \dif x
       + \int_{S_R}
           \frac{e^{iz}}{z}
           \dif z
       - \int_{S_\varepsilon}
           \frac{e^{iz}}{z}
           \dif z
       = 2 i
         \int_0^\infty
           \frac{\sin x}{x}
           \dif x
       + 0
       - i \pi
      \end{align*}
      as $R \to \infty$, $\varepsilon \to 0$. So
      $\int_0^\infty \frac{\sin x}{x} \dif x = \frac{\pi}{2}$.
    }
\end{enumerate}
\end{xmpl}

\begin{xmpl}
Compute
$$
\int_{-\infty}^\infty
  \frac{\cos (a x)}{1 + x^2}
  \dif x
$$
for $a \in \mathbb{R}$. First assume $a \geq 0$.
Let $f(z) = \frac{e^{iaz}}{1 + z^2}$, which has poles
at $\pm i$. Let $R > 1$ and let
$\Gamma = [-R, R] \oplus S_R$. $f$ has 1 singularity
inside $\Gamma$, so $\int_\Gamma f = 2 \pi i \Res_i f$.
Then
\begin{align*}
  \int_\Gamma f
&= \int_{-R}^{R}
     \frac{e^{i a x}}
          {1 + x^2}
 + \int_{S_R}
     \frac{e^{i a z}}{1 + z^2}
     \dif z
 = \int_{-R}^{R}
     \frac{\cos(ax)}
          {1 + x^2}
     \dif x
 + i
   \int_{-R}^{R}
     \frac{\sin(ax)}
          {1 + x^2}
     \dif x
 + \int_{S_R}
     \frac{e^{i a z}}{1 + z^2}
     \dif z \\
&= \int_{-R}^{R}
     \frac{\cos(ax)}
          {1 + x^2}
     \dif x
 + \int_{S_R}
     \frac{e^{i a z}}{1 + z^2}
     \dif z.
\end{align*}

Note that for $z \in S_R$,
$|e^{i a z}| = e^{-a \Im z} \leq 1$ and
$|1 + z^2| \geq |z|^2 - 1 = R^2 - 1$
so that
$$
  \left|
    \int_{S_R}
      \frac{e^{i a z}}{1 + z^2}
      \dif z
  \right|
\leq \frac{1}{R^2 - 1} \pi R \to 0.
$$
Using the fact that the integrand is even,
$$
  \int_{-\infty}^\infty
    \frac{\cos(ax)}{1 + x^2}
    \dif x
= 2 \pi i \Res_i f
= 2 \pi i \frac{e^{a i i}}{(z^2 + 1)^\prime|_i}
= \pi e^{-a}.
$$
If $a < 0$, we may use $\cos(-ax) = \cos (ax)$.
\end{xmpl}

\section{Fourier Transform}
Given $f$ on $\mathbb{R}$ satisfying certain properties,
the Fourier transform of $f$ is defined to be
$$
  \hat{f}(t)
= \int_{-\infty}^{\infty}
    e^{i t x} f(x) \dif x
= \int_{-\infty}^\infty
    \cos(tx) f(x) \dif x
+ i
  \int_{-\infty}^\infty
    \sin(tx) f(x) \dif x, \quad t \in \mathbb{R}.
$$

\begin{xmpl}
  \begin{enumerate}
    \item{
      If $f(x) = \frac{1}{1 + x^2}$,
      $$
      \hat{f}(a) = \pi e^{-|a|}.
      $$
    }
    \item{
      $$\hat{f}(0) = \int_{-\infty}^\infty f(x) \dif x.$$
    }
    \item{
      Consider $f(x) = e^{-\frac{x^2}{2}}$. Then
      let $I = \int_{-\infty}^\infty e^{-\frac{x^2}{2}} \dif x$.
      We write
      \begin{align*}
         I^2
      &= \int_{-\infty}^\infty
           e^{-\frac{x^2}{2}}
           \dif x
         \cdot
         \int_{-\infty}^\infty
           e^{-\frac{y^2}{2}}
           \dif y
       = \int_{-\infty}^\infty
         \int_{-\infty}^\infty
           e^{-\frac{x^2 + y^2}{2}}
           \dif x \dif y \\
      &= \int_0^{2\pi}
         \int_0^\infty
           e^{-\frac{r^2}{2}} r \dif r \dif \theta
       = 2 \pi (-e^{-\frac{r^2}{2}})|_0^\infty
       = 2 \pi
      \end{align*}
      so that $I = \sqrt{2 \pi}$. Now we compute
      $$
        \hat{f}(a)
      = \int_{-\infty}^\infty
          e^{i a x}
          e^{-\frac{x^2}{2}}
          \dif x
      = \int_{-\infty}^\infty
          e^{-\frac{x^2}{2} + i a x} \dif x.
      $$
      Since
      $$
        -\frac{x^2}{2} + i a x
      = -\frac{1}{2}(x - i a)^2 - \frac{a^2}{2}
      $$
      we have
      $$
        \hat{f}(a)
      = e^{-\frac{a^2}{2}}
        \int_{-\infty}^\infty
          e^{-\frac{1}{2}(x - ia)^2}
          \dif x.
      $$
      Now let $f = e^{-\frac{z^2}{2}}$ and note that
      $$
        \int_{-R}^R
          e^{-\frac{1}{2}(x - ia)^2}
          \dif x
      = \int_{[-R - i a, R - i a]} f.
      $$
      Let $\Gamma$ be the boundary of the rectangle given by
      $-R \leq \Re z \leq R$ and $-a \leq \Im z \leq 0$.
      Then
      \begin{align*}
         0
      &= \int_\Gamma f
       =  \int_{-R}^R e^{-\frac{x^2}{2}} \dif x
        - \int_{[-R-ia, R - ia]} f
        + \int_{[R, R - i a]} f
        + \int_{[-R-ia, -R]} f.
      \end{align*}
      We will show that
      $$
      \int_{[\pm R, \pm R - i a]} f \to 0.
      $$
      $$
           \left|
             \int_{[\pm R, \pm R - i a]} f
           \right|
      \leq \| f \|_{[\pm R, \pm R - ia]} |a|.
      $$
      But
      $$
           |e^{-\frac{z^2}{2}}|
      \leq e^{-\Re(\frac{z^2}{2})}
      =    e^{-\frac{1}{2}(x^2 - y^2)}
      \leq e^{\frac{1}{2}(a^2 - R^2)}
      $$
      on $[\pm R, \pm R - ia]$, so this integral goes to 0
      as $R \to \infty$. Therefore
      $$
      \lim_{R \to \infty}
        \int_{-R}^R
          e^{-\frac{1}{2}(x - i a)^2}
          \dif x
      = \sqrt{2 \pi}
      $$
      so that
      $$
      \lim_{R \to \infty}
        \int_{-R}^R
          e^{i a x}
          e^{-\frac{1}{2}x^2}
          \dif x
      = \sqrt{2 \pi} e^{-\frac{a^2}{2}}
      $$
      so that $\hat{f} = \sqrt{2 \pi} e^{-\frac{x^2}{2}}$, i.e.
      the normal distribution is an eigenfunction of the Fourier
      transform.
    }
  \end{enumerate}
\end{xmpl}

\subsection{Trigonometric Integrals}
We wish to integrate
$$
\int_0^{2\pi} Q(\cos \theta, \sin \theta) \dif \theta
$$
where $Q(x, y)$ is a rational function. We see that
$$
   \int_0^{2\pi}
     Q(\cos \theta, \sin \theta)
     \dif \theta
=  \int_{|z| = 1}
     Q\left(
       \frac{z + z^{-1}}{2},
       \frac{z - z^{-1}}{2 i}
      \right)
     \frac{\dif z}
          {i z}.
$$
We may then use the residue formula to compute such integrals.

\begin{xmpl}
  Compute
  \begin{align*}
  I &= \int_0^{2\pi}
         \frac{1}{a + \sin \theta}
         \dif \theta
     = \int_{|z| = 1}
         \frac{1}{a + \frac{z - z^{-1}}{2i}}
         \frac{\dif z}{i z} \\
    &= \int_{|z| = 1}
         \frac{2}{z^2 - 1 + 2iaz}
         \dif z
  \end{align*}
  with $a > 1$. We rewrite the
  denominator as $(z+ia)^2 + a^2 - 1$, so that the
  zeros are at $z_1 = -ia \pm i \sqrt{a^2 - 1}$. We
  see that $|z_2| = a + \sqrt{a^2 - 1} > a > 1$ and
  $z_1 z_2 = -1$, so $|z_1| < 1$. Then we compute
  \begin{align*}
     2 \pi i \cdot \Res_{z_1} \frac{2}{f(z)}
  &= 2 \pi i \frac{2}{f^\prime(z_1)}
   = 4 \pi i \frac{1}{(2z + 2ia)|_{z = z_1}} \\
  &= \frac{4 \pi i}{2 i \sqrt{a^2 - 1}}
   = \frac{2 \pi}{\sqrt{a^2 - 1}}.
  \end{align*}
\end{xmpl}

Now consider an integrand with a branch cut.
\begin{xmpl}
  Compute
  $$
    \int_0^\infty
      \frac{1}{1 + x^a}
      \dif x, \quad
    a > 1.
  $$
  Let
  $$
    f(z)
  = \frac{1}{1 + z^a}
  = \frac{1}{1 + e^{a L(z)}}
  $$
  where $L(z)$ is a branch of the logarithm.

  Let $A_r$ be a curve parameterized by
  $$
  \gamma(t) = r e^{it}, \quad 0 \leq t \leq \frac{2 \pi}{a}.
  $$
  Fix $R > 1 > \varepsilon > 0$. Let
  $$
    \Gamma
  =      [\varepsilon, R]
  \oplus A_R
  \oplus [R e^{i \frac{2 \pi}{a}}, \varepsilon e^{i \frac{2 \pi}{a}}]
  oplus  A_\varepsilon^{-}.
  $$
  Choose the branch $L(z)$ to be a branch of $\log z$ in
  $\mathbb{C} \backslash \{ r e^{i \theta_0} : r \geq 0 \}$
  where $\theta_0 = \frac{\pi}{a} - \pi$, such that
  $L(x) = \log x$ if $x \in \mathbb{R}$ and $x > 0$. Then
  $L$ is well-defined and holomorphic on this domain, and
  $L^\prime(z) = \frac{1}{z}$. Namely
  $L(z) = \log |z| + i \theta(z)$ where
  $\theta(z) \in \arg z$ such that
  $\frac{\pi}{a} - \pi < \theta(z) < \frac{\pi}{a} + \pi$.

  Then $f$ is holomorphic on
  this domain except at the singularities of the denominator.
  We next calculate the zeros of $1 + e^{a L(z)}$. Since
  $-1 = e^{i \pi}$ we have $a L(z) = i \pi + 2 n \pi i$ for any
  $n \in \mathbb{Z}$, or where $L(z) = i \pi \frac{2n + 1}{a}$.

  On $\Gamma$ and $\mathrm{Int}(\Gamma)$,
  $\Im(L(z)) \in \left[0, \frac{2 \pi}{a}\right]$ The only zero
  is then at $n = 0$, i.e. $z_0 = e^{\frac{i \pi}{a}}$.

  Observe now
  \begin{itemize}
    \item{
      $$
        \int_{[\varepsilon, R]} f
      = \int_\varepsilon^R
          \frac{1}{1 + x^a}
          \dif x.
      $$
    }
    \item{
      $$
         \int_{[R e^{i \frac{2\pi}{a}},
                \varepsilon e^{i \frac{2\pi}{a}}]}
           f
       = -\int_{\varepsilon}^R
            f(te^{i \frac{2\pi}{a}}) e^{i \frac{2\pi}{a}}
            \dif t
      $$
      and $L(te^{i \frac{2\pi}{a}}) = \log(t) + i \frac{2\pi}{a}$
      so that
      $$
        f(t e^{i \frac{2 \pi}{a}})
      = \frac{1}{1 + e^{a \log t + i 2 \pi}}
      = \frac{1}{1 + t^a}.
      $$
    }
    \item{
      Therefore
      \begin{align*}
          \int_{\gamma} f
       &= \int_{A_R} f
        - \int_{A_\varepsilon} f
        + \int_{[\varepsilon, R]} f
        + \int_{[R e^{i \frac{2\pi}{a}},
                \varepsilon e^{i \frac{2 \pi}{a}}]}
             f \\
       &= \int_{A_R} f
        - \int_{A_\varepsilon} f
        + (1 - e^{i \frac{2\pi}{a}})
          \int_\varepsilon^R
            \frac{1}{1 + x^a}
            \dif x.
      \end{align*}
      Since
      $$
          \Res_{z_0} f
        = \frac{1}
               {e^{a L(z)} a L^\prime(z)|z_0}
        = \frac{1}
               {-\frac{a}{z_0}} = -\frac{z_0}{a},
      $$
      if $\int_{A_R} f \to 0$ and $\int_{A_\varepsilon} f \to 0$
      then
      $$
        \int_0^\infty
          \frac{\dif x}{1 + x^a}
      = -2 \pi i \frac{e^{i\frac{\pi}{a}}}
                      {a(1 - e^{i\frac{2\pi}{a}})}
      = \frac{\frac{\pi}{a}}
             {\sin \frac{\pi}{a}}.
      $$
    }
    \item{
      Since $|e^{a L(z)}| = |z|^a$, on $A_R$ we have
      $$
      |1 + e^{a L(z)}| \geq |e^{a L(z)}| - 1 = R^a - 1
      $$
      and so
      $$
           \left|
             \int_{A_R} f
           \right|
      \leq \frac{2 \pi R}{R^a - 1}
      \to  0.
      $$
      On $A_\varepsilon$,
      $$
      |1 + e^{a L(z)}| \geq 1 - |e^{a L(z)}| = 1 - \varepsilon^a
      $$
      so
      $$
      \frac{2 \pi \varepsilon}{1 - \varepsilon^a} \to 0
      $$
      as $\varepsilon \to 0$, since $\varepsilon < 1$.
    }
  \end{itemize}
\end{xmpl}
