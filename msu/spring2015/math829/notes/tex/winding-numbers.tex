\section{Winding Number}

\begin{defn}[Winding number]
Let $\gamma$ be a closed curve and
$\alpha \in \mathbb{C} \backslash \gamma$. The \emph{winding number}
of $\gamma$ with respect to $\alpha$ is
$$
           W(\gamma, \alpha)
\triangleq \frac{1}{2 \pi i}
           \int_\gamma
             \frac{1}{z - \alpha}
             \dif z.
$$
\end{defn}

\begin{xmpl}
Suppose that $\gamma$ is a Jordan curve. If $\alpha \in \mathrm{Ext}(\gamma)$,
applying Cauchy's theorem gives $W(\gamma, \alpha) = 0$. If
$\alpha \in \mathrm{Int}(\gamma)$, then applying Cauchy's formula gives
$$
W(\gamma, \alpha) = \pm 1
$$
depending on whether $\gamma$ is positively or negatively oriented.
\end{xmpl}

\begin{lemma}
$W(\gamma, \alpha) \in \mathbb{Z}$.
\end{lemma}

\begin{proof}
Suppose $\gamma : [a, b] \to \mathbb{C}$. Define
$$
  F(t)
= \int_a^t
    \frac{\gamma^\prime(s)}{\gamma(s) - \alpha}
    \dif s, \quad
a \leq t \leq b.
$$
Then $F$ is continuous on $[a, b]$, $F(a) = 0$, and
$F(b) = 2 \pi i W(\gamma, \alpha)$. Furthermore
$$
F^\prime(t) = \frac{\gamma^\prime(t)}{\gamma(t) - \alpha}
$$
for $t \in [a, b]$ except at partition points
$$
a = x_0 < x_1 < \cdots < x_n = b.
$$
Compute
\begin{align*}
   \frac{d}{dt}
     e^{-F(t)} \cdot (\gamma(t) - \alpha)
&= e^{-F(t)} \gamma^\prime(t)
 + e^{-F(t)} (-F^\prime(t))(\gamma(t) - \alpha) \\
&= 0
\end{align*}
and so $e^{-F(t)} (\gamma(t) - \alpha)$ is constant on
each subinterval $[x_{k-1}, x_k]$ and so constant throughout
$[a,b]$. Therefore $\exists C \in \mathbb{C}$ such that
$$
e^{F(t)} = C(\gamma(t) - \alpha), \quad a \leq t \leq b.
$$
Since $\gamma$ is closed, this means $e^{F(b)} = e^{F(a)} = 1$ and
thus $F(b) \in 2 \pi i \mathbb{Z}$.
\end{proof}

\begin{remark}
Let $\theta_0 \in \arg \frac{1}{z}$. From
$\gamma(t) - \alpha = \frac{1}{C}e^{F(t)}$ we see that
$\mathrm{Im} F(t) + \theta_0$ is an argment of
$\gamma(t) - \alpha$. Suppose $h$ is continuous on
$[a, b]$ such that $h(t) \in \arg(\gamma(t) - \alpha)$.
Then
$$
\frac{h(t) - (\mathrm{Im} F(t) + \theta_0)}{2 \pi i} \in \mathbb{Z}
$$
and is continuous on $[a, b]$. Therefore
$h - (\mathrm{Im} F + \theta_0)$ is constant. We conclude that
\begin{align*}
   W(\gamma, \alpha)
&= \frac{F(b) - F(a)}{2 \pi i}
 = \frac{i \mathrm{Im} F(b) - i \mathrm{Im} F(a)}
        {2 \pi i} \\
&= \frac{h(b) - h(a)}{2 \pi i},
\end{align*}
which means that the winding number is the total increment of
$\arg (z - \alpha)$ along $\gamma$.
\end{remark}

\begin{lemma}
$\alpha \mapsto W(\gamma, \alpha)$ is continuous on
$\mathbb{C} \backslash \gamma$.
\end{lemma}

\begin{proof}
Fix $\alpha_0 \in \mathbb{C} \backslash \gamma$. Let
$\alpha_n \in \mathbb{C} \backslash \gamma$ such that
$\alpha_n \to \alpha$. We want to show that
$$
    \int_\gamma \frac{\dif z}{z - \alpha_n}
\to \int_\gamma \frac{\dif z}{z - \alpha_0},
$$
which is true if $\frac{1}{z - \alpha_n} \to \frac{1}{z - \alpha_0}$
uniformly.

Let $r = \mathrm{dist}(\alpha_0, \gamma) > 0$. Note
$$
\left|
  \frac{1}{z - \alpha_n}
- \frac{1}{z - \alpha_0}
\right|
=
\frac{|\alpha_n - \alpha_0}
     {|z - \alpha_n||z - \alpha_0|}.
$$
Suppose $|\alpha_n - \alpha_0| < \frac{r}{2}$. Then
\begin{align*}
      |z - \alpha_n|
&\geq |z - \alpha_0| - |\alpha_n - \alpha_0| \\
&>    r - \frac{r}{2} = \frac{r}{2}.
\end{align*}
Then
$$
\left|
  \frac{1}{z - \alpha_n}
- \frac{1}{z - \alpha_0}
\right|
\leq
\frac{|\alpha_n - \alpha_0|}{r \frac{r}{2}}
$$
when $|\alpha_n - \alpha_0| < \frac{r}{2}$.
\end{proof}

\begin{corol}
$\alpha \mapsto W(\gamma, \alpha)$ is constant on each connected component
of $\mathbb{C} \backslash \gamma$.
\end{corol}

\begin{proof}
This follows since the restriction of this function to each connected
component is constant and integer-valued on a domain.
\end{proof}

\begin{lemma}
$W(\gamma, \alpha) = 0$ if $\alpha$ lies on the unbounded component of
$\mathbb{C} \backslash \gamma$.
\end{lemma}

\begin{proof}
Since $\gamma$ is bounded, there exists an $M$ such that $|z| \leq M$ for all
$z \in \gamma$. If $|\alpha| \geq 2 M$, then
$$
|z - \alpha| \geq |\alpha| - |z| \geq \frac{M}{2},
$$
so $\left|\frac{1}{z - \alpha}\right| \leq \frac{2}{M}$ if
$|\alpha| \geq 2 M$. So $\frac{1}{z - \alpha} \to 0$ as
$\alpha \to \infty$ uniformly for $z \in \gamma$. Therefore
$$
\lim_{\alpha \to \infty} W(\gamma, \alpha) = 0.
$$
Since $W(\gamma, \dot)$ is constant on the unbounded component
of $\mathbb{C} \backslash \gamma$, the constant has to be 0.
\end{proof}

\begin{defn}[Contour]
A \emph{contour} is a ``sum'' of finitely many closed curves
$\gamma_k$, $1 \leq k \leq n$, which may or may not have intersections.
Then
$$
\int_\gamma f = \sum_{k=1}^n \int_{\gamma_k} f, \quad
W(\gamma, \alpha) = \sum_{k=1}^n W(\gamma_k, \alpha)
$$
where
$$
\alpha \in \mathbb{C} \backslash \bigcup_{k=1}^n \gamma_k
       =   \mathbb{C} \backslash \gamma.
$$
\end{defn}

If we cross the curve from the left of the curve to the right
of the curve, the winding number decreases by one. If we cross
from the right to the left, the winding number increases.

\begin{theorem}[General Cauchy's Theorem]
Let $f$ be holomorphic on a domain $U$. Let $\gamma$ be a
contour in $U$ such that $W(\gamma, \alpha) = 0$ for all
$\alpha \in \mathbb{C} \backslash U$. Then $\int_\gamma f = 0$.
\end{theorem}

\begin{xmpl}
Let $V = \{ \alpha : W(\gamma, \alpha) = 0 \}$. Then the theorem
condition means
$$
\mathbb{C} \backslash U \subset V
\iff
\mathbb{C} \backslash V \subset U.
$$
\end{xmpl}

\begin{theorem}[Cauchy's Formula]
Let $f, U, \gamma$ as before. Let $z_0 \in U \backslash \gamma$. Then
$$
  \frac{1}{2 \pi i}
  \int_\gamma
    \frac{f(z)}
         {z - z_0}
    \dif z
= W(\gamma, z_0) f(z_0)
$$
\end{theorem}
