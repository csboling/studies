\section{Curves}

\begin{defn}[Curves]
  A \emph{curve} is a continuous function
  $\gamma : [a, b] \to \mathbb{C}$ with $a < b \in \mathbb{R}$,
  with \emph{beginning point} $\gamma(a)$, \emph{endpoint} $\gamma(b)$, and
  \emph{image set} $\{ \gamma(t) : a \leq t \leq b \}$. Sometimes we use
  $\gamma$ to denote its image set, which gives meaning to the notion that
  $\gamma \subset D$. The \emph{reverse} of $\gamma$ is a function
  $$
  \gamma^{-}(t) = \gamma(a + b - t), \quad a \leq t \leq b,
  $$
  which has the same image set as $\gamma$ and the opposite endpoints,
  i.e. $\gamma^{-}(a) = \gamma(b)$.

  A curve is called \emph{closed} if $\gamma(a) = \gamma(b)$ We say that
  $\gamma$ is $C^1$ if its real part $\mathrm{Re}~\gamma$ and imaginary part
  $\mathrm{Im}~\gamma$ (real functions) both have continuous derivatives.
  Thus
  $$
  \gamma^\prime(t) = (\mathrm{Re}~\gamma)^\prime(t)
                  + i (\mathrm{Im}~\gamma)^\prime(t).
  $$
  The curve $\gamma$ is called \emph{piecewise $C^1$} if
  there exists a partition
  $$
  a = x_1 < x_2 < \cdots < x_n = b
  $$
  such that $\left.\gamma\right|_{[x_{k-1}, x_k]}$ is $C^1$
  for every $k$.
\end{defn}

Curves will always be assumed to be piecewise $C^1$.

\begin{defn}[Length of a curve]
  We define
  $$
  L(\gamma) = \int_a^b |\gamma^\prime(t)| \dif t
  $$
  for $C^1$ curves and
  $$
  L(\gamma) = \sum_{k=1}^n \int_{x_{k-1}}^{x_k} |\gamma^\prime(t)| \dif t
  $$
  for piecewise $C^1$ curves.
\end{defn}

We observe that
\begin{align*}
   L(\gamma^{-})
&= \int_a^b |-\gamma^\prime(t)| \dif t \\
&= \int_a^b |\gamma^\prime(a + b - t)| \dif t \\
&= -\int_b^a |\gamma^{\prime}(s)| \dif s \\
&= L(\gamma).
\end{align*}

\begin{xmpl}
  \begin{enumerate}
    \item{
      Let $z_0, w_0 \in \mathbb{C}$ and
      $$
      \gamma(t) = (1-t)z_0 + t w_0.
      $$
      Then $\gamma(0) = z_0$ and $\gamma(1) = w_0$,
      and the image set of $\gamma$ is the line segment
      connecting $z_0$ and $w_0$, and
      since $\gamma^\prime(t) = w_0 - z_0$ we have
      $L(\gamma) = |w_0 - z_0|$. We denote this curve by
      $[z_0, w_0]$.
    }
    \item{
      For $z_0 \in \mathbb{C}$ and $r > 0$, define
      $\gamma(t) = z_0 + r e^{it}$, $0 \leq t \leq 2\pi$.
      We see that $\gamma(0) = z_0 + r = \gamma(2\pi)$,
      so this is a closed curve with a circle of radius
      $r$ centered on $z_0$ as its image set.
      We have $\gamma^\prime(t) = i r e^{it}$, so
      $$
      L(\gamma) = \int_0^{2\pi} r dt = 2 \pi r.
      $$
    }
    \item{
      Suppose two curves $\gamma : [a, b] \to \mathbb{C}$
      and $\eta: [c, d] \to \mathbb{C}$ satisfy
      $\gamma(b) = \eta(c)$. Define the curve
      $$
      \gamma \oplus \eta : [a + c, b + d] \to \mathbb{C}
      $$
      by
      $$
        (\gamma \oplus \eta)(t)
      = \left\{\begin{array}{l l}
          \gamma(t - c), & \quad a + c \leq t \leq b + c \\
          \gamma(t - b), & \quad b + c \leq t \leq b + d
        \end{array}\right.
      $$
      Then $\gamma \oplus \eta$ is also piecewise $C^1$,
      has beginning point $\gamma(a)$ and endpoint $\eta(d)$,
      has the union of the image sets of these curves as its
      image set, and has the sum of the lengths of these curves
      as its length. We extend this notion to the curve
      $\oplus_{k=1}^n \gamma_k$ for $\gamma_k$ with suitable
      endpoints.
    }
    \item{
      If $\gamma$, $\eta$ have the same endpoints, then
      $\gamma \oplus \eta^{-}$ is closed.

      If $z_0, \dots, z_n \in \mathbb{C}$ then
      $\oplus_{k=0}^{n-1}[z_k, z_{k+1}]$ is called
      a \emph{polygonal curve}. and is closed if $z_0 = z_n$.
    }
  \end{enumerate}
\end{xmpl}

Let $\gamma : [a, b] \to \mathbb{C}$ and
$\phi : [c, d] \to [a, b]$ be $C^1$ such that
$\phi^\prime > 0$ and $\phi(c) = d$, $\phi(d) = b$.
Then $\gamma \circ \phi$ is a curve defined on
$[c, d]$, and the endpoints and image of
$\gamma \circ \phi$ agree with those of $\gamma$
since $\phi$ is onto. We see
\begin{align*}
   L(\gamma \circ \phi)
&= \int_c^d |(\gamma \circ \phi)^\prime(t)| \dif t \\
&= \int_c^d |\gamma^{\prime}(\phi(t)) \cdot \phi^\prime(t)| \dif t \\
&= \int_c^d |\gamma^\prime(\phi(t))| \phi^\prime(t) \dif t \\
&= \int_a^b |\gamma^\prime(s)| \dif s \\
&= L(\gamma)
\end{align*}
where $s = \phi(t)$. We call $\gamma \circ \phi$ a reparametrization
of $\gamma$. There are multiple definitions of $\gamma \oplus \eta$
but they are reparametrizations of each other.

Suppose $f$ is holomorphic on an open set $U$ and $\gamma \subset U$.
Then $f \circ \gamma$ is also a curve with
$$
(f \circ \gamma)^(t) = f^\prime(\gamma(t)) \cdot \gamma^\prime(t).
$$

Recall that $D$ is a \emph{domain} if $D$ is open, nonempty, and
any two points in $D$ can be connected by a continuous
curve in $D$. It is then possible to connect these curves by
a piecewise $C^1$ curve. The image of this curve is a compact
set, and then a polygonal curve can be used to approximate this
continuous curve with a piecewise $C^1$ curve.
