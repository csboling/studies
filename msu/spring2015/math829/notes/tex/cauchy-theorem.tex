\section{Curves}

\begin{defn}[Curves]
  A \emph{curve} is a continuous function
  $\gamma : [a, b] \to \mathbb{C}$ with $a < b \in \mathbb{R}$,
  with \emph{beginning point} $\gamma(a)$, \emph{endpoint} $\gamma(b)$, and
  \emph{image set} $\{ \gamma(t) : a \leq t \leq b \}$. Sometimes we use
  $\gamma$ to denote its image set, which gives meaning to the notion that
  $\gamma \subset D$. The \emph{reverse} of $\gamma$ is a function
  $$
  \gamma^{-}(t) = \gamma(a + b - t), \quad a \leq t \leq b,
  $$
  which has the same image set as $\gamma$ and the opposite endpoints,
  i.e. $\gamma^{-}(a) = \gamma(b)$.

  A curve is called \emph{closed} if $\gamma(a) = \gamma(b)$ We say that
  $\gamma$ is $C^1$ if its real part $\mathrm{Re}~\gamma$ and imaginary part
  $\mathrm{Im}~\gamma$ (real functions) both have continuous derivatives.
  Thus
  $$
  \gamma^\prime(t) = (\mathrm{Re}~\gamma)^\prime(t)
                  + i (\mathrm{Im}~\gamma)^\prime(t).
  $$
  The curve $\gamma$ is called \emph{piecewise $C^1$} if
  there exists a partition
  $$
  a = x_1 < x_2 < \cdots < x_n = b
  $$
  such that $\left.\gamma\right|_{[x_{k-1}, x_k]}$ is $C^1$
  for every $k$.
\end{defn}

Curves will always be assumed to be piecewise $C^1$.

\begin{defn}[Length of a curve]
  We define
  $$
  L(\gamma) = \int_a^b |\gamma^\prime(t)| \dif t
  $$
  for $C^1$ curves and
  $$
  L(\gamma) = \sum_{k=1}^n \int_{x_{k-1}}^{x_k} |\gamma^\prime(t)| \dif t
  $$
  for piecewise $C^1$ curves.
\end{defn}

We observe that
\begin{align*}
   L(\gamma^{-})
&= \int_a^b |-\gamma^\prime(t)| \dif t \\
&= \int_a^b |\gamma^\prime(a + b - t)| \dif t \\
&= -\int_b^a |\gamma^{\prime}(s)| \dif s \\
&= L(\gamma).
\end{align*}

\begin{xmpl}
  \begin{enumerate}
    \item{
      Let $z_0, w_0 \in \mathbb{C}$ and
      $$
      \gamma(t) = (1-t)z_0 + t w_0.
      $$
      Then $\gamma(0) = z_0$ and $\gamma(1) = w_0$,
      and the image set of $\gamma$ is the line segment
      connecting $z_0$ and $w_0$, and
      since $\gamma^\prime(t) = w_0 - z_0$ we have
      $L(\gamma) = |w_0 - z_0|$. We denote this curve by
      $[z_0, w_0]$.
    }
    \item{
      For $z_0 \in \mathbb{C}$ and $r > 0$, define
      $\gamma(t) = z_0 + r e^{it}$, $0 \leq t \leq 2\pi$.
      We see that $\gamma(0) = z_0 + r = \gamma(2\pi)$,
      so this is a closed curve with a circle of radius
      $r$ centered on $z_0$ as its image set.
      We have $\gamma^\prime(t) = i r e^{it}$, so
      $$
      L(\gamma) = \int_0^{2\pi} r dt = 2 \pi r.
      $$
    }
    \item{
      Suppose two curves $\gamma : [a, b] \to \mathbb{C}$
      and $\eta: [c, d] \to \mathbb{C}$ satisfy
      $\gamma(b) = \eta(c)$. Define the curve
      $$
      \gamma \oplus \eta : [a + c, b + d] \to \mathbb{C}
      $$
      by
      $$
        (\gamma \oplus \eta)(t)
      = \left\{\begin{array}{l l}
          \gamma(t - c), & \quad a + c \leq t \leq b + c \\
          \gamma(t - b), & \quad b + c \leq t \leq b + d
        \end{array}\right.
      $$
      Then $\gamma \oplus \eta$ is also piecewise $C^1$,
      has beginning point $\gamma(a)$ and endpoint $\eta(d)$,
      has the union of the image sets of these curves as its
      image set, and has the sum of the lengths of these curves
      as its length. We extend this notion to the curve
      $\oplus_{k=1}^n \gamma_k$ for $\gamma_k$ with suitable
      endpoints.
    }
    \item{
      If $\gamma$, $\eta$ have the same endpoints, then
      $\gamma \oplus \eta^{-}$ is closed.

      If $z_0, \dots, z_n \in \mathbb{C}$ then
      $\oplus_{k=0}^{n-1}[z_k, z_{k+1}]$ is called
      a \emph{polygonal curve}. and is closed if $z_0 = z_n$.
    }
  \end{enumerate}
\end{xmpl}

Let $\gamma : [a, b] \to \mathbb{C}$ and
$\phi : [c, d] \to [a, b]$ be $C^1$ such that
$\phi^\prime > 0$ and $\phi(c) = d$, $\phi(d) = b$.
Then $\gamma \circ \phi$ is a curve defined on
$[c, d]$, and the endpoints and image of
$\gamma \circ \phi$ agree with those of $\gamma$
since $\phi$ is onto. We see
\begin{align*}
   L(\gamma \circ \phi)
&= \int_c^d |(\gamma \circ \phi)^\prime(t)| \dif t \\
&= \int_c^d |\gamma^{\prime}(\phi(t)) \cdot \phi^\prime(t)| \dif t \\
&= \int_c^d |\gamma^\prime(\phi(t))| \phi^\prime(t) \dif t \\
&= \int_a^b |\gamma^\prime(s)| \dif s \\
&= L(\gamma)
\end{align*}
where $s = \phi(t)$. We call $\gamma \circ \phi$ a reparametrization
of $\gamma$. There are multiple definitions of $\gamma \oplus \eta$
but they are reparametrizations of each other.

Suppose $f$ is holomorphic on an open set $U$ and $\gamma \subset U$.
Then $f \circ \gamma$ is also a curve with
$$
(f \circ \gamma)^(t) = f^\prime(\gamma(t)) \cdot \gamma^\prime(t).
$$

Recall that $D$ is a \emph{domain} if $D$ is open, nonempty, and
any two points in $D$ can be connected by a continuous
curve in $D$. It is then possible to connect these curves by
a piecewise $C^1$ curve. The image of this curve is a compact
set, and then a polygonal curve can be used to approximate this
continuous curve with a piecewise $C^1$ curve.

\begin{theorem}
  Suppose $f$ is holomorphic on a domain $U$ such that
  $f^\prime = 0$ on $U$. Then $f$ is constant.
\end{theorem}

\begin{proof}
It suffices to show that $\forall z_0, w_0 \in U$,
$f(z_0) = f(w_0)$. Since $U$ is connected, there exists a
$C^1$ curve $\gamma: [a, b] \to U$ such that
$\gamma(a) = z_0$ and $\gamma(b) = w_0$. Then
$$
  (f \circ \gamma)^\prime(t)
= f^\prime(\gamma(t)) \gamma^{\prime}(t)
= 0
$$
so that
$$
f(w_0) = (f \circ \gamma)(b) = (f \circ \gamma)(a) = f(z_0).
$$
This implies that if $f$ on a domain $U$ has primitives
($g$ such that $g^\prime = f$) then such a primitive is unique
up to an additive constant, since if $g_1, g_2$ are primitives
then $(g_1 - g_2)^\prime = 0$ so that $g_1 - g_2$ is constant.
\end{proof}

\section{Integrals over Curves}
\begin{defn}[Integration of a curve]
Let $F : [a, b] \to \mathbb{C}$ be continuous. Write
$$
F(t) = u(t) + i v(t)
$$
where $u, v : [a, b] \to \mathbb{R}$. Then
$$
  \int_{a}^{b} F(t) \dif t \triangleq
  \int_{a}^{b} u(t) \dif t
+ i \int_{a}^{b} v(t) \dif t.
$$
\end{defn}

\begin{obsv}
  \begin{enumerate}
    \item{
      $$
        \int_a^b (F(t) + G(t)) \dif t
      = \int_a^b F(t) \dif t + \int_{a^b} g(t) \dif t.
      $$
    }
    \item{
      $$
        \int_a^b C F(t) \dif t
      = C \int_a^b F(t) \dif t
      $$
      for any $C \in \mathbb{C}$.
    }
    \item{
      $$
           \left|\int_a^b F(t) \dif t\right|
      \leq \int_a^b |F(t)| \dif t.
      $$
      Let $L = \int_a^b F(t) \dif t$. Then there is some
      $C \in \mathbb{C}$ with $|C| = 1$ such that
      $CL = |L|$. But then
      \begin{align*}
          \left|\int_a^b F(t) \dif t \right|
      &= |L| = CL \\
      &= C \int_a^b F(t) \dif t \\
      &= \int_a^b C F(t) \dif t \\
      &= \int_a^b \mathrm{Re}(C F(t)) \dif t \\
      &\leq \int_a^b |C F(t)| \dif t \\
      &= \int_a^b |F(t)| \dif t.
      \end{align*}
    }
  \end{enumerate}
\end{obsv}

\begin{defn}[Integration of functions over curves]
Let $f$ be continuous on an open set $U$ and let
$\gamma$ be a curve in $U$. The integral of
$f$ over $\gamma$ is defined to be
$$
\int_\gamma f
=
\int_\gamma f(z) \dif z
\triangleq
\sum_{k=1}^n
\int_{x_{k-1}}^{x_k} f(\gamma(t)) \gamma^\prime(t) \dif t
$$
where $\{x_k\}_{k=0}^n$ is a partition of $[a,b]$
making $\gamma$ piecewise continuous.
\end{defn}

\begin{xmpl}
  \begin{enumerate}
    \item{
      Consider the integral over the circle
      $$
      \int_{\{|z| = r\}} \bar{z} \dif z
      $$
      where $\gamma(t) = re^{it}$, so
      $\gamma^\prime(t) = i r e^{it}$
      and $\bar{\gamma(t)} = re^{-it}$. Then
      \begin{align*}
         \int_\gamma \bar{z} \dif z
      &= \int_0^{2\pi}
          \overline{\gamma(t)} \cdot \gamma^\prime(t) \dif t
       = \int_0^{2\pi} r^2 i \dif t \\
      &= 2 \pi r^2 i.
      \end{align*}
    }
    \item{
      Consider the line segment $[1, 1+i]$ given by
      $\gamma(t) = (1-t)1 + t(1+i) = 1 + it$
      so that $\gamma^\prime(t) = i$.
      Then
      \begin{align*}
         \int_\gamma \frac{\dif z}{z}
      &= \int_0^1 \frac{\gamma^\prime(t)}{\gamma(t)} \dif t
       = \int_0^1 \frac{i}{1 + it} \dif t \\
      &= \int_0^1 \frac{i(1 - it)}{(1+it)(1-it)} \dif t \\
      &= \int_0^1 \frac{t}{1 + t^2} \dif t
       + i \int_0^1 \frac{1}{1 + t^2} \dif t \\
      &= \left.\frac{1}{2} \log (1 + t^2)\right|_0^1
       + i\left.\arctan(t)\right|_0^1
       = \frac{1}{2} \log 2 + i \frac{\pi}{2}.
      \end{align*}
    }
  \end{enumerate}
\end{xmpl}

\begin{obsv}
  \begin{enumerate}
    \item{
      The integral of a curve does not change if the curve is
      reparameterized. Let $\gamma : [a, b] \to \mathbb{C}$ and
      $\phi : [c, d] \to [a, b]$. Then
      \begin{align*}
         \int_{\gamma \circ \phi} f
      &= \int_c^d
           f((\gamma \circ \phi)(t)) \cdot
           (\gamma \circ \phi)^\prime(t)
           \dif t \\
      &= \int_c^d
           f(\gamma(\phi(t)))
           \gamma^\prime(\phi(t))
           \phi^\prime(t)
           \dif t \\
      &= \int_a^b
           f(\gamma(s)) \gamma^\prime(s) \dif s \\
      &= \int_\gamma f
      \end{align*}
      under the change of variable $s = \phi(t)$.
    }
    \item{
      $$
      \int_{\gamma^{-}} f = -\int_{\gamma} f.
      $$
    }
    \item{
      $$
        \int_{\oplus_{i=1}^n \gamma_i} f
      = \sum_{i=1}^n \int_{\gamma_i} f.
      $$
    }
  \end{enumerate}
\end{obsv}

\begin{theorem}[Fundamental theorem of calculus]
  Let $f$ be continuous on an open set $U$ and suppose that
  $g$ is a primitive of $f$ in $U$. Let $\gamma$ be a curve
  in $U$ that starts from $\alpha$ and ends at $\beta$.
  Then
  $$
  \int_\gamma f = g(\beta) - g(\alpha).
  $$
\end{theorem}

\begin{proof}
Suppose $\gamma : [a, b] \to U$. Then
\begin{align*}
   \int_\gamma f
&= \int_a^b f(\gamma(t)) \gamma^\prime(t) \dif t \\
&= \int_a^b g^\prime(\gamma(t)) \gamma^\prime(t) \dif t \\
&= \int_a^b (g \circ \gamma)^\prime \dif t \\
&= (g \circ \gamma)(b) - (g \circ \gamma)(a) \\
&= g(\beta) - g(\alpha)
\end{align*}
by the fundamental theorem of real calculus.

To make this proof more rigorous we must take care of the
partition points of a piecewise $C^1$ $\gamma$, but in general
this is correct. For example in the integral of $\frac{1}{z}$ we
considered earlier, primitives do not exist in the whole plane
and require a choice of branch cut.

If $\gamma$ is closed, then this means $\int_\gamma f = 0$.
\end{proof}

\begin{xmpl}
Let $\gamma(t) = e^{it}$ and $f(z) = z^n$, $n \in \mathbb{Z}$.
If $n \neq -1$, then $g(z) = \frac{z^{n+1}}{n+1}$ is a primitive
of $f$ in $U$ so
$$
\int_\gamma z^n \dif z = 0, \quad n \neq -1.
$$
If $n = -1$ then
\begin{align*}
   \int_\gamma z^{-1} \dif z
&= \int_0^{2\pi}
     \frac{\gamma^\prime(t)}{\gamma(t)} \dif t \\
&= \int_0^{2\pi} i \dif t \\
&= 2 \pi i,
\end{align*}
and it follows that $\frac{1}{z}$ has no primitive in
$\mathbb{C} \backslash \{ 0 \}$.
\end{xmpl}

\begin{theorem}
  Let $f$ be continuous on a domain $U$. The following are equivalent:
  \begin{enumerate}
    \item{
      $f$ has a primitive in $U$.
    }
    \item{
      If $\gamma_1, \gamma_2$ are curves in $U$ that have the
      same endpoints, then $\int_{\gamma_1} f = \int_{\gamma_2} f$.
    }
    \item{
      If $\gamma$ is a closed curve in $U$, then
      $\int_\gamma f = 0$.
    }
  \end{enumerate}
\end{theorem}

\begin{proof}
  \begin{itemize}
    \item{
      (i) implies (iii) from the previous theorem.
    }
    \item{
      (iii) implies (ii). Note that
      $\gamma_1 \oplus \gamma_2^{-}$ is a closed curve, so
      $$
      0
      =
      \int_{\gamma_1 \oplus \gamma_2^{-}} f
      =
      \int_{\gamma_1} f + \int_{\gamma_2^{-}} f
      =
      \int_{\gamma_1} f - \int_{\gamma_2} f.
      $$
    }
    \item{
      (ii) implies (i). Fix $z_0 \in U$. Since $U$ is path-connected,
      for every $z \in U$ there is a (non-unique) curve $\gamma_z$ in $U$ from
      $z_0$ to $z$. Define
      $$
      g(z) = \int_{\gamma_z} f,
      $$
      which is well-defined since $g(z)$ does not depend on the choice of curve,
      only the endpoints.

      Suppose $[z, w] subset U$. By definition,
      $$
      g(z) = \int_{\gamma_z} f, \quad
      g(w) = \int_{\gamma_w} f.
      $$
      Since $\gamma_z \oplus [z, w]$ has the same endpoints as $\gamma_w$,
      we have
      $$
      \int_{\gamma_w} f
      =
      \int_{\gamma_z \oplus [z, w]} f
      =
      \int_{\gamma_z} f + \int_{[z,w]} f,
      $$
      so $g(w) - g(z) = \int_{[z,w]} f$.

      \begin{lemma}
        Let $f, g$ be defined on an open set $U$ and suppose $f$ is continuous.
        Suppose whenever $[z, w] \subset U$ we have
        $g(w) - g(z) = \int_{[z, w]} f$. Then $g$ is a primitive of $f$ in $U$.
      \end{lemma}
      \begin{proof}
        Recall that the line segment $[z, w]$ is parameterized by
        $\gamma(t) = z + t(w-z)$. Then
        $$
        g(w) - g(z)
        =
        \int_{[z, w]} f
        =
        \int_0^1 f(z + t(w-z))(w-z) \dif t.
        $$
        If $w \neq z$ then
        $$
        \frac{g(w) - g(z)}{w - z}
        =
        \int_0^1 f(z + t(w - z)) \dif t.
        $$
        Since
        $$
        f(z) = \int_0^1 f(z) \dif t,
        $$
        we have
        $$
        \left|\frac{g(w) - g(z)}{w - z} - f(z)\right|
        \leq
        \int_0^1 |f(z + t(w-z)) - f(z)| \dif t.
        $$

        Now fix $z \in U$, $\varepsilon > 0$. Since $f$
        is continuous at $z$, there exists a $\delta$ such that
        $D(z, \delta) \subset U$ implies that $[z, w] \subset U$ and
        $|f(w) - f(z)| < \varepsilon$.
        If $|w - z| < \delta$ and $0 \leq t \leq 1$
        $$
        |z + t(w-z) - z + w| = t |w-z| < \delta
        $$
        so that
        $$
        f(z + t(w-z)) - f(z)| < \int_0^1 \varepsilon \dif t = \varepsilon.
        $$
        It follows that
        $$
        \left|\frac{g(w) - g(z)}{w - z} - f(z)\right| < \varepsilon
        $$
        so
        $$
        \lim_{w \to z} \frac{g(w) - g(z)}{w - z} = f(z),
        $$
        i.e. $g^\prime(z) = f(z)$ as desired.
      \end{proof}
    }
  \end{itemize}
\end{proof}

\begin{lemma}
$$
\left| \int_\gamma f \right| \leq \| f \|_\gamma L(\gamma).
$$
\end{lemma}
\begin{proof}
Let $\gamma : [a, b] \to \mathbb{C}$. Then
\begin{align*}
      \left| \int_\gamma f \right|
&=    \left| int_a^b f(\gamma(t)) gamma^\prime(t) \dif t \right| \\
&\leq \int_a^b |f(\gamma(t))| |\gamma^\prime(t)| \dif t \\
&\leq \int_a^b \| f \|_\gamma |\gamma^\prime(t)| \dif t \\
&=    \| f \|_\gamma L(\gamma).
\end{align*}
\end{proof}

Observe that we do not have
$$
\left| \int_\gamma f \right| \leq \int_\gamma |f|
$$
since the right side may not even be a real number.

\begin{theorem}
Let $(f_n)$ be a sequence of continuous functions on $U$
converging uniformly to $f$ on $U$. Let $\gamma$ be a curve in $U$.
Then
$$
\lim_{n \to \infty} \int_\gamma f_n = \int_\gamma f.
$$
If $\sum f_n$ is a series of continuous functions on $U$ that
uniformly converges to $f$ on $U$ then
$$
\int_\gamma \sum_n f_n = \sum_n \int_\gamma f_n.
$$
\end{theorem}
\begin{proof}
$f$ is continuous because it is the limit of continuous functions so
$\int_\gamma f$ is well defined. Now
\begin{align*}
      \left| \int_\gamma f_\gamma - \int_\gamma f \right|
&=    \left| \int_\gamma f_n - f \right| \\
&\leq \| f_n - f \|_\gamma L(\gamma) \\
&\leq \| f_n - f \|_U L(\gamma),
\end{align*}
and since $L(\gamma) \in \mathbb{R}$ and $\| f_n - f \|_U \to 0$, so
$\int_\gamma f_n \to \int_\gamma f$.

Let $s_n(z)$ be the partial sum sequence. Then $s_n \to \sum f_n$ uniformly,
so
$$
\int_\gamma \sum f_n = \lim \int_\gamma s_n.
$$
But $\int s_n$ is the partial sum for $\sum \int_\gamma f_n$.
\end{proof}

\begin{proof}[Derivatives of power series]
Let
$$
f(z) = \sum_{n=0}^\infty a_n z^n, \quad
g(z) = \sum_{n=0}^\infty n a_{n-1} z^{n-1}
$$
on $U = D(0, R)$. Let $f_n(z) = a_n z^n$ and
$g_n(z) = f_n^\prime(z) = n a_n z^{n-1}$. Let $z_0, w_0 \in U$.
We may pick $r \in (0, R)$ such that $z_0, w_0 \in D(0, r)$.
Since $f_n$ is a primitive of $g_n$, we have
$$
\int_{[z_0, w_0]} g = f_n(w_0) - f_n(z_0)
$$
so
$$
\int_{[z_0, w_0]} g = \int_{[z_0, w_0]} \sum g_n
=
\sum (f_n(w_0) - f_n(z_0))
=
f(w_0) - f(z_0).
$$
Thus $f$ is a primitive of $g$.
\end{proof}

\section{Goursat Theorem}
\begin{defn}[Diameter of a set]
Define
$$
\mathrm{diam}(S) \triangleq
\sup \{ |z - w| : z, w \in S \}, \quad
S \neq \varnothing.
$$
If $S$ is bounded then $\mathrm{diam}(S) < \infty$.
\end{defn}
\begin{defn}[Triangle integral]
If $\Delta$ is a triangle with vertices $A, B, C$
oriented in the counterclockwise direction, then we write
$$
\int_{\partial \Delta} = \int_{[A, B]} + \int_{[B, C]} + \int_{[C, A]}.
$$
\end{defn}

\begin{theorem}
Let $f$ be holomorphic on a closed triangle $\Delta$, meaning that
$f$ is holomorphic on an open set $U$ containing $\Delta$. Then
$$
\int_{\partial \Delta} f = 0.
$$
Here the condition means that there exists an open set
$U$ such that $\Delta \subset U$, and $f$ is holomorpphic on $U$.
\end{theorem}

\begin{proof}
Decompose $\Delta$ into 4 triangles $\Delta_j$ of similar shape using
the midpoints of its sides. Then
$$
C = \int_{\partial \Delta} f = \sum_{k=1}^4 \int_{\partial \Delta_k} f.
$$
Then there exists a $j_0 \in \{1,2,3,4\}$ such that
$$
\left|\int_{\partial \Delta_{j_0}}\right| \geq \frac{|C|}{4}.
$$
Let $\Delta^{(0)} = \Delta$, $\Delta^{(1)} = \Delta_{j_0}$. We may
further decompose $\Delta^{(1)}$ into four similar triangles using
their midpoints. Then one of these triangles, denoted $\Delta^{(2)}$,
satisfies
$$
     \left| \int_{\partial \Delta} f \right|
\geq \frac{1}{4}\left|\int_{\partial \Delta^{(2)}} f \right|
\geq \frac{|C|}{4^2}.
$$
Thus we get a sequence
$$
\Delta = \Delta^{(0)} \supset \Delta^{(1)} \supset \cdots
$$
of similar triangles, with the size of $\Delta^{(n)}$ equal to
$\frac{1}{2^n}$ that of $\Delta$, and
$$
\left| \int_{\partial \Delta^{(n)}} f \right| \geq \frac{|C|}{4^n}.
$$
Since each $\Delta^{(n)}$ is compact and nonempty, this sequence
has nonempty intersection, so pick $z_0 \in \cap \Delta^{(n)}$.
Since $z_0 \in \Delta$, $f$ is complex differentiable at $z_0$.

Define
$$
h(z)
=
\left\{\begin{array}{l l}
  \frac{f(z) - f(z_0)}{z - z_0} - f^\prime(z_0), &
  \quad z \in U \backslash \{ z_0 \} \\
  0, & \quad z = z_0
\end{array}\right..
$$
Note that $h$ is continuous on $U$ and
$$
f(z) = f(z_0) + f^\prime(z_0)(z-z_0) + h(z)(z - z_0).
$$

Let $P(z) = f(z_0) + f^\prime(z_0) (z - z_0)$. Then
$P$ is a polynomial of degree at most 1, and
$f(z) = P(z) + h(z)(z - z_0)$. Since $P$ is a polynomial
it has a primitive in $\mathbb{C}$, so
$$
\int_{\partial \Delta^{(n)}} P = 0, \forall n.
$$
Thus
$$
\int_{\partial \Delta^{(n)}} f
= \int_{\partial \Delta^{(n)}} h(z) (z - z_0) \dif z.
$$

Now
$$
  \mathrm{diam}(\Delta^{(n)})
= \frac{1}{2^n} \mathrm{diam}(\Delta)
< \infty
$$
and
$$
  L(\partial \Delta^{(n)})
= \frac{1}{2^n} L(\partial \Delta)
< \infty.
$$
Since $z_0 \in \Delta$,
$|z - z_0| \leq \mathrm{diam}(\Delta^{(n)})$
for any $z \in \partial \Delta^{(n)}$. Thus
\begin{align*}
\frac{|C|}{4^n}
&\leq \int_{\partial \Delta^{(n)}} h(z) (z-z_0) \dif z
&\leq \|h\|_{\partial \Delta^{(n)}}
      \mathrm{diam}(\Delta^{(n)})
      L(\partial\Delta^{(n)}) \\
&\leq \|h\|_{\partial \Delta^{(n)}}
      \frac{\mathrm{diam}(\Delta)}{2^n}
      \frac{L(\partial \Delta)}{2^n}
\end{align*}
so that
$$
|C| \leq \|h\|_{\partial \Delta^{(n)}}
         \mathrm{diam}(\Delta)
         L(\partial \Delta).
$$
Since $h$ is continuous at $z_0$ with $h(z_0) = 0$
and
$$
        \Delta^{(n)}
\subset D\left(z_0, \mathrm{diam}(\Delta^{(n)})\right)
=       D\left(z_0, \frac{\mathrm{diam}(\Delta)}{2^n}\right)
$$
we must have $\| h \|_{\Delta^{(k)}} \to 0$. Then $|C| = 0$ so $C = 0$.
\end{proof}

\begin{defn}[Convex and star domains]
Let $U$ be open.
We call $U$ a \emph{convex domain} if $\forall z_0, w_0 \in U$,
$[z_0, w_0] \subset U$. We call $U$ a \emph{star domain} if
$\exists z_0 \in U$ such that $\forall w_0 \in U$,
$[z_0, w_0] \subset U$ and we call $z_0$ its center.
\end{defn}

\begin{obsv}
  \begin{enumerate}
    \item{
      $D(z_0, r)$ is a star domain with $z_0$ as center. In fact,
      it is also a convex domain.
    }
    \item{
      Any convex domain is a star domain, with every point as a center.
    }
    \item{
      Let $U$ be a star domain with center $z_0$. Let $z_1, z_2 \in U$.
      Suppose $[z_1, z_2] \in U$ and $z_0, z_1, z_2$ are not on the same
      line. Then the triangle with vertices $z_i$ is contained in $U$.
    }
  \end{enumerate}
\end{obsv}

\begin{theorem}
If $f$ is holomorphic on a star domain $U$, then $f$ has a primitive on $U$.
\end{theorem}
\begin{proof}
Let $z_0$ be a center of $U$. Define $g$ on $U$ such that
$$
g(z) = \int_{[z_0, z]} f, z \in U
$$
which is well-defined since these line segments lie in $U$ for any
$z \in U$. Then for any $z_1, z_2 \in U$ we have
$$
\int_{[z_1, z_2]} f + \int_{[z_2, z_0]} f + \int_{[z_0, z_1]} f = 0.
$$

\begin{enumerate}
  \item{
    Suppose $z_0, z_1, z_2$ are colinear. Then without loss of generality
    $$
    \int_{[z_0, z_2]} f = \int_{[z_0, z_1]} f + \int_{[z_1, z_2]} f.
    $$
  }
  \item{
    Suppose $z_0, z_1, z_2$ are not colinear. Then
    $\Delta z_0, z_2, z_2 \subset U$ and we may apply the Goursat theorem.
  }
\end{enumerate}
It follows that if $[z_1, z_2] \subset U$ then
$$
g(z_2) - g(z_1) = \int_{[z_1, z_2]} f
$$
so $g^\prime = f$ on $U$.
\end{proof}

\begin{corol}
If $f$ is holomorphic on a star domain $U$, then for any closed curve
$\gamma \subset U$ we have $\int_\gamma f = 0$. If $f$ is holomorphic on
an open set $U$ and $D(z_0, r) \subset U$, then $\int_{D(z_0, r)} f = 0$.
\end{corol}

\section{Cauchy Theorem}

\begin{defn}[Simple and Jordan curves]
A continuous curve $\gamma: [a, b] \to \mathbb{C}$ is called
\emph{simple} if it is one-to-one. It is called a
\emph{simple closed curve} if the above holds except that
$\gamma(a) = \gamma(b)$. A simple closed curve is also called
a \emph{Jordan curve}.
\end{defn}

\begin{theorem}[Jordan Curve Theorem]
Let $\gamma$ be a Jordan curve. Then $\mathbb{C} \backslash \gamma$
is disjoint union of two domains, one bounded and one unbounded,
and $\gamma$ is the boundary of both domains. We use
$\mathrm{Int}(\gamma)$ to denote the bounded domain.
\end{theorem}

\begin{theorem}[Cauchy Theorem for Jordan Curves]

\end{theorem}
