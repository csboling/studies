\section{Complex Logarithm}

\begin{defn}[Branch]
  A \emph{branch} of $\log z$ in an open set
  $U \subset \mathbb{C} - \{ 0 \}$ is a continuous function
  $L(z)$ defined on $U$ such that $L(z) \in \log(z)$, i.e.
  $e^{L(z)} = z$ for any $z \in U$.
\end{defn}

Suppose that $L(z) = u(z) + i v(z)$ is a branch of $\log z$ in $U$.
Then $U(z) = \log |z|$ which is continuous and
$v(z) \in \arg z$. Showing that $L$ is continuous is therefore
equivalent to showing that $V$ is continuous.

Consider the principal argument $\mathrm{Arg}(z)$. If
$\mathrm{Arg}(z) \in [0, 2\pi)$, then $\mathrm{Arg}$ is
continuous on $\mathbb{C} - \{ x \in \mathbb{R} : x \geq 0 \}$.
Similarly if $\mathrm{Arg}(z) \in (-\pi, \pi]$ then $\mathrm{Arg}$
is continuous on $\mathbb{C} - \{ x \in \mathbb{R} : x \leq 0 \}$.
We may define branches of $\arg z$ for any $\theta_0$ \in \mathbb{R}$,
removing the appropriate half-line from the plane and choosing
$v(z)$ such that $\theta_0 < v(z) < \theta_0 + 2 \pi$.
A branch cut can also be made along any curve. It is impossible
to get a branch of $\arg z$ or $\log z$ in $\mathbb{C} - \{ 0 \}$.

Once a branch of $\log z$ is fixed, it may be denoted by $\log z$
and then becomes a single-valued function. We can also define the
complex power by $z^\alpha = e^{\alpha \log z}$, $\alpha \in \mathbb{C}$,
$z \in U$, $\log z$ a branch in $U$.
