\section{Number Systems}
This course concerns the set of complex numbers
$$
\mathbb{C} = \{ x + yi : x, y \in \mathbb{R} \}
$$
where $i^2 = -1$, i.e. $i \notin \mathbb{R}$. For some such
$z = x + yi$ we call $x$ the real part $\mathrm{Re}~z$ and
$y$ the imaginary part $\mathrm{Im}~z$. We see that
$$
           \mathbb{N}
\subsetneq \mathbb{Z}
\subsetneq \mathbb{Q}
\subsetneq \mathbb{R}
\subsetneq \mathbb{C}
$$
since we can identify $x \in \mathbb{R}$ with
$x + 0i \in \mathbb{C}$. We define operations on the complex numbers by
$$
(x_1 + y_1 i) + (x_2 + y_2 i) = (x_1 + x_2) + (y_1 + y_2) i
$$
and
$$
  (x_1 + y_1 i) \cdot (x_2 + y_2 i)
= (x_1 x_2 - y_1 y_2) + (x_1 y_2 + x_2 y_1)i.
$$
We can formally expand this product as
$$
x_1 x_2 + x_1 y_2 i + x_2 y_1 i + y_1 y_2 i^2
$$
to see that this coincides with the product of binomials.
We will observe that $\mathbb{C}$ is a field, and will see how
to find the multiplicative inverse $\mathbb{z}^{-1}$ later.

The \emph{complex conjugate} of a complex number $z = x + yi$ is given by
$\bar{Z} = x - yi$. We observe the following properties:
\begin{itemize}
  \item{
    $\bar{\bar{z}} = z$.
  }
  \item{
    $\bar{z} = z \iff z \in \mathbb{R}$.
  }
  \item{
    $z + \bar{z} = 2 \mathrm{Re}~z$.
  }
  \item{
    $z - \bar{z} = 2i \mathrm{Im}~z$.
  }
  \item{
    $\bar{z \pm w} = \bar{z} \pm \bar{w}$.
  }
  \item{
    $\bar{zw} = \bar{z}\bar{w}$.
  }
  \item{
    $z\bar{z} = x^2 + y^2 \geq 0$.
  }
\end{itemize}

For $z = x + yi \in \mathbb{C}$ we define the absolute value of $z$ by
$$
|z| = \sqrt{x^2 + y^2}
$$
so that $z \bar{z} = |z|^2$. This means
\begin{align*}
  |zw|^2 &= (zw)\bar{(zw)} = (zw)(\bar{z}\bar{w}) \\
         &= (z \bar{z})(w \bar{w}) = |z|^2|w|^2.
\end{align*}
We can also see that $|\bar{z}| = |z|$ and
$|\mathrm{Re}~z| \leq |z|$, $|\mathrm{Im}~z| \leq |z|$.

Now note that
$$
z \cdot \bar{z} = x^2 + y^2 > 0
$$
so
$$
z \frac{x - y i}{x^2 + y^2} = 1
$$
and then
$$
z^{-1} = \frac{x}{x^2 + y^2} - \frac{y}{x^2 + y^2} i.
$$

Define $z^0 = 1$ and $z^n = \prod_{i=1}^n z$ for any positive $n$.
Define $z^n = \prod_{i=1}^{-n} z^{-1}$ for negative $n$. We can also
check that $|z^n| = |z|^n$ and $|z^{-1}| = |z|^{-1}$.

\section{Geometry}
We can view every $z = x + yi \in \mathbb{C}$ as a point $(x,y)$ on the plane
$\mathbb{R}^2$. Under this interpretation the conjugate map is a reflection
about the real axis and the modulus $|z|$ is equal to the distance from such a
point to the origin. Addition follows the parallelogram rule, which gives the
triangle inequality $|z + w| \leq |z| + |w|$. This is because
\begin{align*}
      |z + w|^2
&=    z \bar{z} + z\bar{w} + \bar{(z\bar{w})} + |w|^2 \\
&=    |z|^2 + |w|^2 + 2 \mathrm{Re}(z\bar{w}) \\
&\leq |z|^2 + |w|^2 + 2|z||\bar{w}| \\
&=    (|z| + |w|)^2.
\end{align*}

\section{Polar Form}

The expression $z = x + yi$ is called the \emph{rectangular form}
of $z$. Suppose $x = r \cos \theta$, $y = r \sin \theta$. We see that
$x^2 + y^2 = r^2 (\cos^2 \theta + \sin^2 \theta)$ and so
$r = \sqrt{x^2 + y^2} = |z| \geq 0$, equalling zero if and only if $z = 0$.
Further we have
$$
\cos \theta = \frac{x}{r}, \quad
\sin \theta = \frac{y}{r}.
$$
The solution $\theta$ is therefore not unique. We call such a $\theta$ an
\emph{argument} of $z$ and write $\theta = \arg z$. We may understand
$\theta$ as an element in the quotient group $\mathbb{R} / 2 \pi \mathbb{Z}$.
The obvious choice of representative is called the \emph{principal argument}
$\mathrm{Arg}~z$, the unique $\theta \in \arg z$ that lies in $[0, 2\pi)$.
We might instead choose $\mathrm{Arg}~z$ to fall in $(-\pi, \pi]$.

We write $z = r e^{i\theta}$ to mean $z = r (cos \theta + i \sin \theta)$.

\begin{theorem}
  $$
  e^{i\theta_1} \cdot e^{i \theta_2} = e^{i(\theta_1 + \theta_2)}
  $$
\end{theorem}
\begin{proof}
  \begin{align*}
     (\cos \theta_1 + i \sin \theta_1)(\cos \theta_2 + i \sin \theta_2)
  &= (\cos \theta_1 \cos \theta_2 - \sin \theta_1 \sin \theta_2)
   + i(\cos \theta_1 \sin \theta_2 + \sin \theta_1 \cos \theta_2) \\
  &= \cos (\theta_1 + \theta_2) + i \sin (\theta_1 + \theta_2)
  \end{align*}
\end{proof}
It follows that $\arg z_1 z_2 = \arg z_1 + \arg z_2$,
$\arg z^{-1} = -\arg z$, and $\arg z^n = n \arg z$. Then
for $z = r e^{i\theta}$, $z^n = r^n e^{i n \theta}$.

\subsection{Power Roots}
Given $z \in \mathbb{C}$ and $n \in \mathbb{Z}$, we can find $w$
such that $w^n = z$. If $z = 0$ then $w = 0$, so suppose $z \neq 0$.
Write $z = |z| e^{i \theta}$. Take $w = r e^{i \phi}$. To have
$w^n = z$ we must have have $|w^n| = |z|$, so
$|w^n| = |w|^n = |z|$ and thus $|w| = |z|^{\frac{1}{n}}$. Furthermore
we must have $e^{i n \phi} = e^{i \theta}$. Then $\phi = \frac{\theta}{n}$
gives one solution, i.e. $w = |z|^{\frac{1}{n}} e^{i \frac{\theta}{n}}$
satisfies the desired property. To find all roots we need
$n \phi = \theta + 2 k \pi$ for some $k \in \mathbb{Z}$. This gives
$\phi_k = \frac{\theta}{n} + \frac{k}{n} 2\pi$. Therefore
$$
w_k = |z|^{\frac{1}{n}} e^{i\left(\frac{\theta}{n} + \frac{k}{n} 2\pi\right)}
$$
Observing that $\phi_{k + n} = \phi_k + 2\pi$, $w_k = w_{k+n}$, so there
are exactly $n$ distinct $w$, called the $n$th roots of $z$. Note that
if $n \geq 3$ then these roots are vertices of a regular $n$-gon.

\section{Complex Functions}

\subsection{Complex Exponential}
For $z = x + yi = r e^{i\theta}$, we define the \emph{exponential}
$\mathbb{C} \to \mathbb{C}$ by
$$
\exp(z) = e^x e^{iy} = e^x \cos y + i e^x \sin y.
$$
If $z \in \mathbb{R}$, this agrees with the real exponential function.
This function has the following properties.
\begin{enumerate}
  \item{
    $e^{z_1} e^{z_2} = e^{z_1 + z_2}$, since
    \begin{align*}
       e^{x_1 + i y_1} e^{x_2 + i y_2}
    &= e^{x_1} e^{x_2} e^{i y_1} e^{i y_2} \\
    &= e^{x_1 + x_2} e^{i(y_1 + y_2)}.
    \end{align*}
  }
  \item{
    $|e^z| = |e^x||e^{iy}| = e^x > 0$ so
    $|e^z| = e^{\mathrm{Re}~z}$.
  }
  \item{
    $\arg e^{z} = \mathrm{Im}~z$.
  }
  \item{
    $\forall z \in \mathbb{C}$ with
    $z \neq 0$, $\exists w \in \mathbb{C}$ such that
    $e^w = z$. Indeed for $z = r e^{i \theta}$,
    this is satisfied by $w = \log r + i \theta$.
  }
  \item{
    The exponential function has the period $2 \pi i$,
    i.e. $e^{z + 2 n \pi i} = e^z$, $\forall n \in \mathbb{Z}$.
    This means that the $w$ above satisfying $e^w = z$ is not unique.
  }
\end{enumerate}

\subsection{Complex Logarithm}
We define the \emph{logarithm}
$\log : \mathbb{C} \to \mathbb{C} + i \mathbb{R} / 2 \pi \mathbb{Z}$
by
$$
\log z = \{ w \in \mathbb{C} : e^w = z \}
$$
and note that $\log z = \log |z| + i \arg z$. We define the
\emph{principal logarithm}
by $\mathrm{Log}~z = \log |z| + i \mathrm{Arg}~z$.

For example,
$$
  \mathrm{Log}(1 - i)
= \log(\sqrt{2}) + i \mathrm{Arg}(1 - i)
= \log(\sqrt{2}) \pm i \frac{1}{4} \pi.
$$
depending on the branch cut chosen for the principal argument.

\section{Complex Valued Functions}
Let $S \subset \mathbb{C}$ and $f : S \to \mathbb{C}$. We can write
such a function in rectangular form as
$$
f(x + iy) = u(x, y) + i v(x, y)
$$
where $u, v : S \times T \to \mathbb{R}$ for some
$S, T \subset \mathbb{R}$. For example,
\begin{enumerate}
  \item{
    $f(z) = \mathrm{Im}~z$ has the form $u(x,y) = y$, $v(x,y) = 0$.
  }
  \item{
    $f(z) = |z|$ is given by $u = \sqrt{x^2 + y^2}$, $v = 0$.
  }
  \item{
    $f(z) = z^2$ is given by $u = x^2 - y^2$, $v = 2xy$.
  }
  \item{
    $f(z) = e^z$ is given by $u = e^x \cos y$, $v = e^x \sin y$.
  }
  \item{
    $f(z) = z^{-1}$ is given by $u = \frac{x}{x^2 + y^2}$, $v = -\frac{y}{x^2 + y^2}$.
  }
  \item{
    $f(z) = \mathrm{Log}~z$ is given by $u = \log \sqrt{x^2 + y^2}$,
    $v = \mathrm{Arg}~z$ or
    $v = \mathrm{arccot}\left(\frac{x}{y}\right)$ for $y > 0$.
  }
\end{enumerate}
