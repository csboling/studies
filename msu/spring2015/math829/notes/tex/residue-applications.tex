\section{Applications}

\subsection{Rouche's Theorem}

Suppose $\ord_{z_0} f = m$. Then $f(z) = (z - z_0)^m g(z)$
where $g$ is holomorphic at $z_0$ and $g(z_0) \neq 0$. Then
$$
  f^\prime(z)
= m (z - z_0)^{m-1} g(z)
+ (z - z_0)^m g^\prime(z)
$$
so
$$
  \frac{f^\prime(z)}{f(z)}
= \frac{m}{z - z_0}
+ \frac{g^\prime(z)}{g(z)}.
$$
But $g$ and thus $g^\prime$ are holomorphic at $z_0$ and
$g(z_0) \neq 0$, so $\frac{g^\prime(z)}{g(z)}$ has no
negative terms in the expansion about $z_0$, so
$$
  \mathrm{Res}_{z_0} \frac{f^\prime}{f}
= m
= \ord_{z_0} f.
$$

\begin{theorem}
Let $f$ be meromorphic on $U$ and let $\gamma$ be
a positively oriented Jordan curve in $U$ such that
$\gamma$ does not pass through any zeros or poles of $f$
and $\mathrm{Int}(\gamma) \subset U$. Then
$$
  \int_\gamma
    \frac{f^\prime}{f}
= 2 \pi i
  \sum_{j=1}^n
    \ord_{z_i} f
$$
where $z_i$ are the zeros and poles of $f$ that lie inside $\gamma$.
\end{theorem}
\begin{proof}
From the previous observation, a singularity of
$\frac{f^\prime}{f}$ is either a zero or a pole of $f$, so
$\mathrm{Res}_{z_i} \frac{f^\prime}{f} = \ord_{z_0} f$ from the
previous observation, and so
$$
  \int_\gamma \frac{f^\prime}{f}
= 2 \pi i
  \sum_{j=1}^n
    \mathrm{Res}_{z_i} \frac{f^\prime}{f}
= 2 \pi i
  \sum_{j=1}^n
    \ord_{z_i} f.
$$
\end{proof}

If $z_0$ is a zero (pole) of $f$ of order $m$, we say that
there exist $m$ zeros (poles) of $f$ at $z_0$. Using this terminology,
we say that
$$
\int_\gamma \frac{f^\prime}{f} = 2 \pi i (Z - P),
$$
where $Z$ is the number of zeros inside $\gamma$ counting
multiplicities and $P$ is the number of poles inside $\gamma$
counting multiplicities.

\begin{theorem}
Let $J$ be a Jordan curve and $f$, $g$ analytic on $J \cup
\mathrm{Int}(J)$. Suppose $|f(z) - g(z)| < |f(z)|$ for all
$z \in J$. Then $f, g$ have the same number of zeros inside
$J$, counting multiplicities.
\end{theorem}
