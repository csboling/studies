\section{Applications}

\subsection{Rouche's Theorem}

Suppose $\ord_{z_0} f = m$. Then $f(z) = (z - z_0)^m g(z)$
where $g$ is holomorphic at $z_0$ and $g(z_0) \neq 0$. Then
$$
  f^\prime(z)
= m (z - z_0)^{m-1} g(z)
+ (z - z_0)^m g^\prime(z)
$$
so
$$
  \frac{f^\prime(z)}{f(z)}
= \frac{m}{z - z_0}
+ \frac{g^\prime(z)}{g(z)}.
$$
But $g$ and thus $g^\prime$ are holomorphic at $z_0$ and
$g(z_0) \neq 0$, so $\frac{g^\prime(z)}{g(z)}$ has no
negative terms in the expansion about $z_0$, so
$$
  \mathrm{Res}_{z_0} \frac{f^\prime}{f}
= m
= \ord_{z_0} f.
$$

\begin{theorem}
Let $f$ be meromorphic on $U$ and let $\gamma$ be
a positively oriented Jordan curve in $U$ such that
$\gamma$ does not pass through any zeros or poles of $f$
and $\mathrm{Int}(\gamma) \subset U$. Then
$$
  \int_\gamma
    \frac{f^\prime}{f}
= 2 \pi i
  \sum_{j=1}^n
    \ord_{z_i} f
$$
where $z_i$ are the zeros and poles of $f$ that lie inside $\gamma$.
\end{theorem}
\begin{proof}
From the previous observation, a singularity of
$\frac{f^\prime}{f}$ is either a zero or a pole of $f$, so
$\mathrm{Res}_{z_i} \frac{f^\prime}{f} = \ord_{z_0} f$ from the
previous observation, and so
$$
  \int_\gamma \frac{f^\prime}{f}
= 2 \pi i
  \sum_{j=1}^n
    \mathrm{Res}_{z_i} \frac{f^\prime}{f}
= 2 \pi i
  \sum_{j=1}^n
    \ord_{z_i} f.
$$
\end{proof}

If $z_0$ is a zero (pole) of $f$ of order $m$, we say that
there exist $m$ zeros (poles) of $f$ at $z_0$. Using this terminology,
we say that
$$
\int_\gamma \frac{f^\prime}{f} = 2 \pi i (Z - P),
$$
where $Z$ is the number of zeros inside $\gamma$ counting
multiplicities and $P$ is the number of poles inside $\gamma$
counting multiplicities.

\begin{theorem}
Let $J$ be a Jordan curve and $f$, $g$ analytic on $J \cup
\mathrm{Int}(J)$. Suppose $|f(z) - g(z)| < |f(z)|$ for all
$z \in J$. Then $f, g$ have the same number of zeros inside
$J$, counting multiplicities inside $J$.
\end{theorem}

\begin{proof}
It suffices to show that
$\int_J \frac{g^\prime}{g} = \int_J \frac{f^\prime}{f}$.
Let $h = g - f$ and $g_t = f + th$. Then
$g_0 = f$ and $g_1 = g$. Since $|h| < |f|$ on $J$,
we see that $\forall t \in [0, 1]$, $g_t \neq 0$
on $J$.

Let
$$
  m(t)
= \frac{1}{2 \pi i}
  \int_J
    \frac{g_t^\prime}{g_t}
= \frac{1}{2 \pi i}
  \int_J
    \frac{f^\prime + t h^\prime}
         {f + t h},
$$
so that $m(t)$ is the number of zeros
of $g_t$ counting multiplicity and in particular
$m(0)$ counts the zeros of $f$, $m(1)$ counts the
zeros of $g$ so that $m(t) \in \mathbb{Z}$ for
$t \in [0, 1]$. We may show that $m$ is
continuous on $[0, 1]$, so it must be constant.
Then $m$ is constant on $[0, 1]$, so
$m(1) = m(0)$.
\end{proof}

\begin{xmpl}
Let $P(z) = z^8 - 5z^3 + z - 2$.
\begin{itemize}
  \item{
    Find the zeros
    counting multiplicities of $P$ inside $\{|z| < 1\}$.
    Compare $P$ with $f(z) = -5z^3$. On $\{|z| = 1\}$,
    \begin{align*}
       |P(z) - f(z)|
    &= |z^8 + z - 2|
    \leq |z|^8 + |z| + |-2| = 4,
    \end{align*}
    and $|f(z)| = |-5z^3| = 5 > |P(z) - f(z)|$. Then
    from Rouche's theorem, $P$ has the same number of
    zeros inside the disk $\{ |z| < 1 \}$ as $f$, which
    is 3.
  }
  \item{
    Let $f(z) = z^8$. On $\{|z| = 2\}$,
    $$
      |P(z) - f(z)|
    = |-5z^3 + z - 2|
    \leq 5 \cdot 8 + 2 + 2 = 44 < 256
    $$
    so $P$ has the same number of zeros as $f$ inside
    $\{ |z| = 2 \}$, namely 8.
  }
\end{itemize}
\end{xmpl}

There is a method to determine whether a polynomial has
zeros with multiplicity greater than 1. If $z_0$ is a zero
of $P$ of order $m \geq 2$, then $z_0$ is a zero of $P^\prime$
of order $m - 1$, so $z_0$ is a zero of $\mathrm{gcd}(P, P^\prime)$.
To find the greatest common denominator of two polynomials
we can use the same division algorithm as for integers.

$P = x^3 + x^2 + x + 1$, $P^\prime = 3x^2 + 2x + 1$
Then
$$
  \frac{x^3 + x^2 + x + 1}{3x^2 + 2x + 1}
= \frac{x}{3} + \frac{1}{9}
$$
with remainder $\frac{4}{9} x + \frac{8}{9}$ so that
$$
  \frac{3x^2 + 2x + 1}{\frac{4}{9} x + \frac{8}{9}}
= \frac{27}{4} x - 9
$$
with constant remainder $9$, so these polynomials are
coprime.

Take $P = x^4 - 2x^2 + 1$ so that $P^\prime = 4x^3 - 4x$
so
$$
  \frac{x^3 - 2x^2 + 1}{4x^3 - 4x}
= \frac{x}{4}
$$
with remainder $-x^2 + 1$ so that
$$
  \frac{4x^3 - 4x}{-x^2 + 1}
= -
$$
so that $(x^4 - 2x^2 + 1, 4x^3 - 4x) = -x^2 + 1$, which has
zeros at $\pm 1$, so these are the multiple zeros of $P$.

\begin{proof}[Fundamental Theorem of Algebra]
Suppose $P(z) = a_n z^n + a_{n-1} z^{n-1} + \cdots + a_0$, and
$a_n \neq 0$. Compare $P$ with $f(z) = a_n z^n$. Observe that
$$
  \frac{P(z) - f(z)}{f(z)}
= \frac{a_{n-1}}{a_n}\frac{1}{z}
+ \frac{a_{n-2}}{a_n}\frac{1}{z^2}
+ \cdots
+ \frac{a_0}{a_n} \frac{1}{z^n}.
$$
This goes to zero as $|z| \to \infty$, and in particular
$\exists M > 0$ such that
$$
  \left|\frac{P(z) - f(z)}{f(z)}\right|
< 1 \quad |z| < R, \forall R > M
$$
so $|P(z) - f(z)| < |f(z)|$ on $\{|z| = R\}$. Then
$P$ has the same number of zeros as $f$, which is $n$.
Since this holds for any $R > M$, $P$ has exactly $n$
zeros counting multiplicity in $\mathbb{C}$.
\end{proof}

\begin{theorem}[Open Mapping Theorem]
Let $f$ be analytic on an open set $U$ such that $f$
is not constant on any disc. Then $f(U)$ is open.
\end{theorem}

\begin{proof}
Let $z_0 \in U$ and $w_0 = f(z_0) \in f(U)$, so
since $f$ is not constant we have a well-defined
$m = \ord_{z_0} (f - w_0) \geq 1$. Then
$$
f(z) = w_0 + \sum_{n=m}^\infty a_n (z - z_0)^n, \quad |z - z_0| < r
$$
and let
$$
g(z) = w_0 + a_m (z - z_0)^m, \quad
h(z) = \frac{f(z) - g(z)}{(z - z_0)^m}
     = \sum_{n=m+1}^\infty a_n (z - z_0)^{n - m}.
$$
Let $r > 0$ such that
$\ h \|_{|z - z_0| = r} < \frac{|a_m|}{2}$ and
$\varepsilon = \frac{|a_m|}{2} r^m$.

Let $w \in D(w_0, \varepsilon)$, and define
$f_w = f - w$, $g_w = g - w$. On $\{|z - z_0| = r\}$,
\begin{align*}
      |g_w(z)|
&=    |a_m (z - z_0)^m + w_0 - w|
 \geq |a_m| r^n - |w - w_0| \\
&>    |a_m|r^m -\varepsilon
 =    \frac{|a_m|}{2} r^m \\
&>    r^m \| h \|_{|z - z_0| = r}
 \geq |(z - z_0)^m h(z)| \\
&=    |f(z) - g(z)|
 =    |f_w(z) - g_w(z)|.
\end{align*}
Then from Rouche's theorem, $f_w$ has the same number of zeros as
$g_w$ in $D(z_0, r)$, which is $m$. Therefore
$\exists z \in D(z_0, r)$ such that $f(z) = w$. Then
$w \in f(D(z_0, r)) \subset f(U)$. Since $w$ was chosen arbitrarily
in $D(w_0, \varepsilon)$, we get $D(w_0, \varepsilon) \subset f(U)$.
\end{proof}

\begin{proof}
Let $w_0 \in f(U)$. We will show $\exists \varepsilon > 0$
such that $D(w_0, \varepsilon) \subset f(U)$. Suppose
$z_0 \in U$, $f(z_0) = w_0$ and
$$
  f(z)
= \sum_{n=0}^\infty a_n (z - z_0)^n
$$ near $z_0$.
Then $a_0 = w_0$, so $f - w_0 = 0$ at $z_0$.
Let $m = \ord_{z_0} (f - w_0) \in mathbb{N}$, so that
$a_m \neq 0$ and
$$
f(z) = w_0 + \sum_{n=m}^\infty a_n (z - z_0)^n
$$
near $z_0$. Let $g(z) = w_0 + a_m (z - z_0)^m$ and
$$
  h(z)
= \frac{f(z) - g(z)}{(z - z_0)^m}
= \sum_{n=m+1}^\infty a_n (z - z_0)^{n-m}.
$$
Therefore $h$ is analytic at $z_0$ and $h(z_0) = 0$.

Pick $r > 0$ such that
$\bar{D}(z_0, r) \subset U$ and
$\| h \|_{|z - z_0| = r} < \frac{|a_m|}{2}$. Let
$\varepsilon = \frac{|a_m|}{2} r^m > 0$, and fix
$w \in D(w_0, \varepsilon)$. Let $f_w(z) = f(z) - w$
and $g_w(z) = g(z) - w$. We will show that
$w \in f(D(z_0, r))$, which is equivalent to showing
that $f_w$ has a zero in $D(z_0, r)$

Compare $f_w$ with $g_w$. First, $g_w$ has a zero in
$D(z_0, r)$ since
$$
     g_w(z) = 0
\iff w_0 - w + a_n(z - z_0)^m = 0
\iff (z - z_0)^m = \frac{w - w_0}{a_m}.
$$
Therefore $g_w$ has $m$ roots, where
$$
  |z - z_0|
= \left(\frac{|w - w_0|}{|a_m|}\right)^{1 / m}
< \left(\frac{\varepsilon}{|a_m|}\right)^{1 / m}
= \frac{r}{2^{1 / m}}
< r.
$$
We wish to show that $|f_w - g_w| < |g_w|$.
\end{proof}


\begin{remark}
If $U$ is a domain, the assumption that $f$ is not constant
on any disk is equivalent to the assumption that $f$ is not
constant. In this case, $f(U)$ is either a single point or
an open set.
\end{remark}

\begin{defn}[Analytic isomorphisms and automorphisms]
An analytic function defined on an open set $U$ is called
an \emph{analytic isomorphism}
(or \emph{biholomorphism}) if there is an analytic
function $g$ define on $f(U)$ such that $g(f(z)) = z$
for every $z \in U$. If $f(U) = U$ then we say $f$ is
an analytic automorphism of $U$. We say that $f$ is a local
analytic isomorphism at $z_0$ if there is an open set $U \ni z_0$
such that $f$ is an analytic isomorphism on $U$.

More concisely, an analytic isomorphism is an isomorphism in the
category with open subsets of $\mathbb{C}$ as objects and
holomorphisms as morphisms.
\end{defn}

\begin{theorem}
Let $f$ be holomorphic and injective on an open set $U$. Then
$f$ is a biholomorphism.
\end{theorem}

\begin{proof}
Define $g = f^{-1}$ on $f(U)$. Such $g$ is well-defined since
$f$ is injective. It remains to show that $g$ is holomorphic.

From the open mapping theorem, for any open set $O \subset U$,
$g^{-1}(0) = f(0)$ is open. Thus $g$ is continuous.

Let $Z$ denote the set of zeros of $f^\prime$. Then $Z$ has no
accumulation point in $U$. Let $U^\prime = U \backslash Z$, and
fix $w_0 \in f(U^\prime) = f(U) \backslash f(Z)$. Let
$z_0 = g^{-1}(w_0) \in U^\prime$, $f(z_0) = w_0$.
Consider
$$
  \frac{g(w) - g(w_0)}{w - w_0}
= \frac{g(w) - g(w_0)}{f(g(w)) - f(g(w_0))}.
$$
If $w \to w_0$, then $g(w) \to g(w_0)$. Since
$f^\prime(g(w_0)) = f^\prime(z_0) \neq 0$, we have
$$
  \lim_{w \to w_0}  \frac{g(w) - g(w_0)}{w - w_0}
= \frac{1}{f^\prime(z_0)}
$$
exists. Thus $g$ is holomorphic on $U^\prime$.

Since $g$ is continuous on $U$, every $z \in Z$ is
a removable singularity. After removing these singularities
we have that $g$ is holomorphic on $U$.

Since $z = g(f(z))$, $1 = g^\prime(f(z)) f^\prime(z)$, so
$f^\prime(z) \neq 0$ for any $z$, and therefore in fact
$Z = \varnothing$. However, $f^\prime(z) \neq 0$ for all $z$
does not imply that $f$ is an analytic isomorphism, for instance
$e^z$.
\end{proof}

\begin{theorem}[Inverse Mapping Theorem]
If $f$ is holomorphic at $z_0$ and $f^\prime(z_0) \neq 0$, then
$f$ is a local analytic isomorphism at $z_0$.
\end{theorem}

\begin{proof}
Let $w_0 = f(z_0)$. Recall $m = \ord_{z_0} (f - w_0)$. Since
$f^\prime(z_0) \neq 0$, $m = 1$. We found $r, e > 0$ such that
$\forall D(w_0, \varepsilon$, $f_w = f - w$ has $m$ zeros in
$D(z_0, r)$. Since $m = 1$, there is a unique $z \in D(z_0, r)$
such that $f(z) = w$. Let
$U = D(z_0, r) \cap f^{-1}(D(w_0, \varepsilon))$.
Then $z_0 \in U$ and $f$ is injective on $U$.
\end{proof}
