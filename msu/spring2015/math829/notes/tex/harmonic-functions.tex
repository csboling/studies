\section{Harmonic Functions}

\begin{defn}[Harmonic functions, Laplace operator]
Let $n \in \mathbb{N}$, $U \subset \mathbb{R}^n$ be open.
Let $f : U \to \mathbb{C}$. If $f$ is $C^2$ on $U$ and
satisfies
$$
           \Delta f(x)
\triangleq \sum_{k=1}^n \frac{\partial^2 f}{\partial x_k^2}(x) = 0
$$
for all $x \in U$, then we say that $f$ is a \emph{harmonic function}.
The symbol $\Delta$ is called the \emph{Laplace operator}.
\end{defn}

We focus on $n = 2$ and identify $\mathbb{R}^2$ with $\mathbb{C}$.
Here the Laplace operator is
$$
\Delta f(z)
\triangleq
  \frac{\partial^2 f}{\partial x^2}(z)
+ \frac{\partial^2 f}{\partial y^2}(z).
$$
Note that $f$ is harmonic iff. $\mathrm{Re} f$ and
$\mathrm{Im} f$ are harmonic.

\begin{theorem}
If $f$ is holomorphic on $U$ then it is harmonic on $U$.
\end{theorem}
\begin{proof}
Write $f = u + iv$. We know that $f$ is infinitely many times
complex differentiable and thus $u, v \in C^\infty \subset C^2$.
Furthermore by the Cauchy-Riemann equations
$$
u_x = v_y, \quad u_y = -v_x
$$
so that
$$
\Delta u = u_{xx} + u_{yy} = (v_y)_x + (-v_x)_y = 0
$$
and
$$
\Delta v = v_{xx} + v_{yy} = (-u_y)_x + (u_x)_y = 0.
$$
\end{proof}

From now on we assume that a harmonic function is
always real-valued.

\begin{lemma}
Let $u$ be a real-valued $C^2$ function on an open set
$U \subset \mathbb{C}$. Then $u$ is harmonic on $U$ iff.
$u_x - i u_y$ is holomorphic on $U$.
\end{lemma}

\begin{proof}
\begin{itemize}
  \item[($\implies$)]{
    If $u \in C^2$ then $u_x, u_y \in C^1$. We check the
    Cauchy-Riemann equations for $(u_x, -u_y)$.
    $(u_x)_x = (-u_y)_y$ is equivalent to $\Delta u = 0$,
    and $(u_x)_y = -(-u_y)_x$ is equivalent to $u_{xy} = u_{yx}$.
  }
  \item[($\impliedby$)]{
    In this case $(u_x, -u_y)$ satisfies the Cauchy-Riemann equations
    so $(u_x)_x = (-u_y)_y$ and thus $\Delta u = 0$.
  }
\end{itemize}
\end{proof}

\begin{defn}
Let $u$ be harmonic on a domain $U$. If a real-valued function $v$
satisfies that $u + iv$ is holomorphic on $U$, then we say that
$v$ is a harmonic conjugate of $u$ in $U$.
\end{defn}

\begin{remark}
  \begin{enumerate}
    \item{
      The $v$ must also be harmonic, because it is the imaginary part
      of a holomorphic (and thus harmonic) function.
    }
    \item{
      A harmonic conjugate, if it exists, is unique up to an additive
      constant. If $v$ is a harmonic conjugate of $u$ in $U$, then so
      is $v + C$ for any real $C \in \mathbb{R}$. On the other hand,
      if $v$ and $\tilde{v}$ are both harmonic conjugates of $u$, then
      $v_x = \tilde{v}_x$ and $v_y = \tilde{v}_y$. Since $U$ is connected,
      this means $\tilde{v} - v$ is constant.
    }
    \item{
      If $v$ is a harmonic conjugate of $u$ then $-u$ is a harmonic
      conjugate of $v$. Note that $-i(u + iv) = v - iu$.
    }
  \end{enumerate}
\end{remark}

\begin{theorem}
Let $u$ be harmonic on a simply connected domain. Then there is a harmonic
conjugate of $u$ in $U$.
\end{theorem}

\begin{proof}
Let $f = u_x - i u_y$. Then $f$ is holomorphic on $U$. Since $U$ is simply
connected, there exists a primitive $F$ of $f$ in $U$. Write
$F = \tilde{u} + i \tilde{v}$. Then
\begin{align*}
   u_x - i u_y
&= f = F^\prime
 = \tilde{u}_x + i \tilde{v}_x \\
&= \tilde{u}_y - i \tilde{v}_y,
\end{align*}
so $u_x = \tilde{u}_x$ and
$u_y = \tilde{u}_y$. Since $U$ is connected,
this means $\tilde{u} - u \equiv C$. Then
$$
  F - C
= \tilde{u} - C + i \tilde{v}
= u + i \tilde{v}
$$
is holomorphic, so $\tilde{v}$ is a harmonic
conjugate of $u$ in $U$.
\end{proof}

\begin{remark}
\begin{enumerate}
  \item{
    The assumption that $U$ is simply connected cannot be removed.
    Let $D = C \backslash \{ 0 \}$ and let
    $$
    u(z) = \frac{1}{2} \log |z| = \frac{1}{2} \log (x^2 + y^2).
    $$
    Then
    $$
    u_x = \frac{x}{x^2 + y^2}, \quad
    u_y = \frac{y}{x^2 + y^2}
    $$
    so
    $$
    u_{xx} = \frac{1}{x^2 + y^2} - \frac{2 x^2}{(x^2 + y^2)^2}, \quad
    u_{yy} = \frac{1}{x^2 + y^2} - \frac{2 y^2}{(x^2 + y^2)^2},
    $$
    and thus $u_{xx} + u_{yy} = 0$, i.e. $u$ is harmonic on $D$.

    Suppose there exists a harmonic conjugate $v$ of this $u$.
    Then $f = u + iv$ is holomorphic on $\mathbb{C} \backslash \{ 0 \}$,
    so
    $$
      f^\prime
    = u_x - i u_y
    = \frac{x}{x^2 + y^2} - \frac{iy}{x^2 + y^2}
    = \frac{1}{z}
    $$
    which means $f$ is a primitive of $\frac{1}{z}$ in
    $\mathbb{C} \backslash \{ 0 \}$, a contradiction.
  }
  \item{
    A harmonic conjugate always exists locally, and on a small disk
    a harmonic function $h$ is the real part of a holomorphic function $f$
    We know that $f$ is infinitely complex differentiably, so we coonclude that
    $h$ is infinitely differentiable in that disk. Therefore a harmonic
    function must be $C^\infty$.
  }
\end{enumerate}
\end{remark}

We can find the harmonic conjugate of a harmonic function in a simply
connected domain as follows.
Let
$$
u(x, y) = x^2 + 2xy - y^2.
$$
Then $u_x = 2x + 2y$, $u_y = 2x - 2y$,
$u_{xx} = 2$, $u_{yy} = -2$, so $\Delta u = 0$.

We want a harmonic conjugate $v$ such that $u_x = v_y$,
$u_y = -v_x$. We have $v_y = 2x + 2y$, so
$v = 2xy + y^2 + h(x)$. Then
$v_x = 2y + h^\prime(x)$, so since $v_x = 2y - 2x$ we have
$h^\prime(x) = -2x$. Then $h(x) = -x^2 + C$. Therefore
$$
v(x, y) = 2xy + y^2 - x^2 + C
$$
is a harmonic conjugate of $u$ for any $C$.

\begin{theorem}[Mean Value Theorem for harmonic functions]
Let $u$ be harmonic on $D(z_0, R)$. Then for any $r \in (0, R)$,
\begin{align*}
   u(z_0)
&= \frac{1}{2 \pi}
   \int_0^{2 \pi}
     u(z_0 + r e^{i \theta})
     \dif \theta \\
&= \frac{1}{\pi r^2}
   \int_{|z - z_0| \leq r}
     u(z)
     \dif x \dif y.
\end{align*}
\end{theorem}

\begin{proof}
Since $D(z_0, R)$ is simply connected, there exists $f$ that is
holomorphic on $D(z_0, R)$ such that $u = \mathrm{Re} f$. From the
MVT for holomorphic functions, the above formulas hold with
$u$ replaced by $f$.
\end{proof}

\begin{corol}
If $u$ attains its maximum at $z_0$, then $u$ is constant.
\end{corol}
\begin{proof}
Let $f$ be holomorphic on $D(z_0, R)$ such that $u = \mathrm{Re} f$.
Then $\exp \circ f$ is holomorphic on $D(z_0, R)$ and $|e^f| = e^u$.
$u$ attains its maximum at $z_0$ if and only if $|e^f|$ attains its
maximum at $z_0$, in which case $e^f$ is constant, so
$u = \log |e^f|$ is constant.
\end{proof}

\begin{theorem}[Maximum Principle for harmonic functions]
Let $u$ be harmonic on a domain $D$.
\begin{enumerate}[(i)]
  \item{
    Suppose that $u$ has a local maximum at $z_0 \in D$.
    Then $u$ is constant.
  }
  \item{
    If $D$ is bounded and $u$ is continuous on $\bar{D}$, then
    $\exists z_0 \in \partial D$ such that
    $u(z_0) = \max \{ u(z) : z \in \bar{D} \}$.
  }
  \item{
    The above statements also hold with $\max$ replaced by
    $\min$.
  }
\end{enumerate}
\end{theorem}
\begin{proof}
  \begin{enumerate}[(i)]
    \item{
      From the corollary above, there exists an $r_0 > 0$
      such that $u$ is constant on $D(z_0, r_0$. Let
      $w \in D$. Since $D$ is connected, we may find disks
      $D_k = D(z_k, r_k)$, $0 \leq k \leq n$ such that
      $D_{k-1} \cap D_k \neq \varnothing$, and $w \in D_n$.
      Since each $D_k$ is simply connected, we can find $f_k$
      holomorphic on $D_k$ such that $u = \mathrm{Re} f_k$ on $D_k$.
      Since $u$ is constant on $D_0$, the Cauchy-Riemann equations
      require that $\mathrm{Im} f_0$ is constant also, so $f_0$ is
      constant. Furthermore $u$ is constant on $D_0 \cap D_1$,
      so $f_1$ is constant on $D_0 \cap D_1$, and from the Uniqueness
      Theorem this means $f_1$ is constant on $D_1$, etc.
      Therefore $u$ is constant on $D_k$. Since $w \in D_n$, we have
      $u(w) = u(z_0)$. Since $w$ was chosen arbitrarily, $u$ is constant.
    }
    \item{
      $\bar{D}$ is compact and $u$ is continuous on $\bar{D}$, so
      $u$ attains a maximum $u(z_0)$ for some $z_0 \in \bar{D}$.
      If $z_0 \in \partial D$, we are done. If $z_0 \in D$, then $u$ is
      constant by (i) and we may take any $z_0 \in \partial D$.
    }
    \item{
      This follows by the same argument and the observation that $-u$ is
      also harmonic.
    }
  \end{enumerate}
\end{proof}

\begin{corol}
Suppose $u$, $v$ are harmonic on a bounded domain $D$ and are
continuous on $\bar{D}$. Suppose $u = v$ on $\partial D$. Then
$u = v$ on $\bar{D}$.
\end{corol}

\begin{proof}
Let $h = u - v$. Then $h$ is harmonic on $D$, continuous on
$\bar{D}$, and $h \equiv 0$ on $\partial D$. Since $h$
attains its maximum and minimum both on $\partial D$,
$h$ has to be constantly zero.
\end{proof}

\begin{defn}[Wirtinger derivatives]
We define the Wirtinger differential operators
$$
  \frac{\partial}{dz}
= \frac{1}{2}
  \left(
      \frac{\partial}{\partial x}
  - i \frac{\partial}{\partial y}
  \right), \quad
  \frac{\partial}{d\bar{z}}
= \frac{1}{2}
  \left(
      \frac{\partial}{\partial x}
  + i \frac{\partial}{\partial y}
  \right).
$$
For example, for a function $f = u + iv$
not necessarily holomorphic but with $u, v \in C^1$,
$$
  \frac{\partial f}{\partial z}
= \frac{u_x + v_y}{2} + i \frac{v_x - u_y}{2}
$$
and
$$
  \frac{\partial f}{\partial \bar{z}}
= \frac{u_x - v_y}{2} + i \frac{u_y + v_x}{2}.
$$

Note that
$\frac{\partial f}{\partial \bar{z}} \equiv 0$ is equivalent
to the Cauchy-Riemann equations. If $f$ is holomorphic, then
$f_z = f^\prime$. Observe also that
\begin{align*}
   \frac{\partial}{dz} \frac{\partial}{\partial \bar{z}}
&= \frac{1}{4}
   \left(
     \frac{\partial}{\partial x} - i \frac{\partial}{\partial y}
   \right)
   \left(
     \frac{\partial}{\partial x} + i \frac{\partial}{\partial y}
   \right) \\
&= \frac{\Delta}{4}
 = \frac{\partial}{\partial \bar{z}} \frac{\partial}{\partial z}.
\end{align*}
If $u$ is harmonic then $(u_z)_{\bar{z}} = 0$, and
$u_z$ is holomorphic.
\end{defn}

\begin{remark}
The smoothness, MVT, MP also hold for harmonic function in
$\mathbb{R}^n$, $n \geq 3$.
\end{remark}
