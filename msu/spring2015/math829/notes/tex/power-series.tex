\begin{defn}[Power Series]
Let $(a_n)_{n=1}^\infty$ be a sequence of complex numbers.
Then a \emph{power series} is a limit of the form
$$
           \sum_{n=0}^\infty a_n z^n
\triangleq \lim_{n \to \infty} \sum_{k=0}^n a_k z^k.
$$
\end{defn}

For what $z$ does series converge or diverge?
If $f(z) = \sum_{n=0}^\infty a_n z^n$, defined only
for $z$ such that the series converges, then what
are the properties of $f$?

\begin{theorem}
  Let $R = \frac{1}{\limsup |a_n|^{\frac{1}{n}}}$.
  Then the series converges if $|z| < R$ and diverges
  if $|z| > R$. We assume $1 / 0 = \infty$ and
  $1 / \infty = 0$. If $R = 0$, the series converges
  only at $z = 0$. If $R = \infty$, the series converges
  for all $z$. If $0 < R < \infty$, then for $|z| = R$
  the series may or may not converge. We define this
  $R$ as the \emph{radius of convergence}.
\end{theorem}

\begin{prop}[Root and ratio tests]
  If $\lim |a_n|^{\frac{1}{n}}$ exists, then
  $\limsup |a_n|^{\frac{1}{n}} = \lim |a_n|^{\frac{1}{n}}$.

  If $\lim \frac{|a_n|}{|a_{n+1}|}$ exists, then
  $R = \lim \frac{|a_n|}{|a_{n+1}|}$.
\end{prop}

\begin{xmpl}
  \begin{enumerate}
    \item{
      $\sum_{n=0}^\infty z^n$, i.e. $a_n = 1$ $\forall n$.
      Then $R = \frac{1}{1} = 1$.
    }
    \item{
      $\sum_{n=0}^\infty n! z^n$ has $\frac{|a_n|}{|a_{n+1}|} = \frac{1}{n+1} \to 0$.
      Therefore $R = 0$.
    }
    \item{
      $\sum_{n=0}^\infty \frac{z^n}{n!}$ has $\frac{|a_n|}{|a_{n+1}|} = n + 1 \to \infty$,
      so $R = \infty$.
    }
    \item{
      Take $a_n = \frac{n!}{n^n}$. Then
      \begin{align*}
           \frac{|a_n|}{a_{n+1}|}
        &= \frac{\frac{n!}{n^n}}{\frac{(n+1)!}{(n+1)^{n+1}}}
         = \frac{n!}{(n+1)!}\frac{(n+1)^{n+1}}{n^n} \\
        &= \frac{(n+1)^n}{n^n}
         = (1 + \frac{1}{n})^n \to e.
      \end{align*}
    }
    \item{
      For $\alpha \in \mathbb{C}$, define
      ${\alpha \choose 0} = 1$,
      $$
        {\alpha \choose n}
      = \frac{\alpha(\alpha - 1) \cdots (\alpha - n + 1}
             {n!}.
      $$
      In this case the ratio test does not apply for
      $\alpha$ a nonnegative integer, since this yields the
      indeterminate form $\frac{0}{0}$. Otherwise we have
      $$
          \left|\frac{a_n}{a_{n+1}}\right|
      =   \left|\frac{n+1}{\alpha - n}\right|
      \to 1.
      $$
      But if $\alpha$ is a nonnegative integer then
      $a_n = 0$ for $n > \alpha + 1$, so this power series
      is in fact a polynomial and thus $R = \infty$.

      We will see later that this series converges to
      $(1 + z)^\alpha$ in $D(0, 1)$ if the branch of
      the $\log(1 + z)$ in $D(0, 1)$ is chosen such that
      $\log 1 = 0$.
    }
  \end{enumerate}
\end{xmpl}

\begin{theorem}
  Let $f(z) = \sum_{n=0}^\infty a_n z^n$ for $|z| < R$.
  Then $f$ is holomorphic on $D(0, R)$ and
  $$
  f^\prime(z) = \sum_{n=1}^\infty n a_n z^{n-1}
  $$
  on $D(0, R)$, and this series also has radius of convergence $R$.
  This is also a holomorphic power series, and inductively
  $f \in C^\infty(D(0, R), \mathbb{C})$.
\end{theorem}

\begin{xmpl}
  \begin{enumerate}
    \item{
      Consider $\sum_{n=0}^\infty \frac{z^n}{n!}$ and recall that $R = \infty$.
      Let $f(z) = \sum_{n=0}^\infty \frac{z^n}{n!}$. Then $f$ is entire and
      \begin{align*}
         f^\prime(z)
      &= \sum_{n=1}^\infty n \frac{z^{n-1}}{n!}
       = \sum_{n=1}^\infty \frac{z^{n-1}}{(n-1)!} \\
      &= \sum_{m}^\infty \frac{z^m}{m!} = f,
      \end{align*}
      so $f(z) = C e^z$. Since $f(0) = 1$ and $e^0 = 1$,
      we conclude that $f(z) = e^z$.

      Given $r \in \mathbb{C}$ we also have
      $$
      e^{rz} = \sum_{n=0}^\infty \frac{r^n}{n!}z^n.
      $$
      In particular
      $$
      e^{iz} = \sum_{n=0}^\infty \frac{i^n}{n!} z^n
      $$
      and
      $$
      e^{-iz} = \sum_{n=0}^\infty \frac{(-i)^n}{n!} z^n
      $$
      so that
      $$
        \cos z
      = \frac{e^{iz} + e^{-iz}}{2}
      = \frac{1}{2} \sum_{n=0}^\infty \frac{i^n + (-i)^n}{n!} z^n.
      $$
      If $n$ is odd then $i^n + (-i)^n = i^n + (-1)^n i^n = 0$.
      If $n$ is even then
      $i^n + (-i)^n = i^{2k} + (-i)^{2k} = (-1)^k + (-1)^k = 2(-1)^k$.
      Therefore
      $$
      \cos z = \sum_{k=0}^\infty \frac{(-1)^k}{(2k)!} z^{2k}.
      $$
      Similarly
      $$
      \sin z = \sum_{k=0}^\infty \frac{(-1)^k}{(2k+1)!} z^{2k+1}.
      $$
    }
  \end{enumerate}
\end{xmpl}

\begin{defn}[Partial sum, absolute convergence]
  The \emph{partial sum} of a series $\sum_{n=1}^\infty z_n$
  is given by
  $s_n = \sum_{k=1}^n z_k = z_1 + \cdots + z_n$.
  We say that the series converges/diverges if the sequence
  $(s_n)$ converges/diverges. Scalar multiples and sums of
  convergent series also converge. If $\sum a_n$ converges,
  we must have $a_n \to 0$, since $s_n$ and $s_{n-1}$ have
  the same limit, so $a_n = s_n - s_{n-1} \to 0$. We say
  that $\sum a_n$ \emph{converges absolutely} if
  $\sum |a_n|$ converges.
\end{defn}

Let $t_n = \sum_{k=1}^n |a_n|$ and $s_n = \sum_{k=1}^n a_k$.
We show that $(t_n)$ is Cauchy implies that $(s_n)$ is Cauchy.
In fact, $s_n - s_m = \sum_{k=m+1}^n a_k$ and
$t_n - t_m = \sum_{k=m+1}^n |a_k|$, so
$$
|s_n - s_m| \leq \sum_{k=m+1}^n |a_k| = t_n - t_m,
$$
from which the assertion follows.

Let $\sum a_n$ be a series of complex numbers and
$\sum c_n$ be a series of nonnegative real numbers.
If $|a_n| \leq c_n$ $\forall n$
and $\sum c_n$ converges, then $\sum a_n$
converges absolutely.

\subsection{Sequences and Series of Functions}
\begin{defn}[Pointwise Convergence]
Let $S \subset \mathbb{C}$ and $f_n : S \to \mathbb{C}$.
Then $(f_n)$ \emph{converges pointwise} on $S$
if $\forall z \in S$, $(f_n(z))$ converges.
\end{defn}

\begin{defn}[Supremum norm]
  If $f : S \to \mathbb{C}$, then
  $$
  \| f \|_S \triangleq \sup_{z \in S} |f(z)|.
  $$
  It can be confirmed that this is a norm on the
  space of functions $S \to \mathbb{C}$.
\end{defn}

\begin{defn}[Uniform Convergence]
  The sequence $(f_n)$ \emph{converges uniformly} to $f$
  on $S$ if $\lim_{n \to \infty} \|f_n - f\|_S = 0$.
  Equivalently, $\forall \varepsilon > 0$,
  $\exists N \in \mathbb{N}$ such that
  $\forall x \in S$ and $\forall n > N$,
  $|f_n(z) - f(z)| < \varepsilon$. Here $N$ is
  independent of $z$. Uniform convergence implies
  pointwise convergence.
\end{defn}

\begin{theorem}
  Let $(f_n)$ be a sequence of continuous functions which converges
  uniformly to $f$ on $S$. Then $f$ is continuous on $S$.
\end{theorem}

\begin{proof}
  Let $z_0 \in S$, $\varepsilon > 0$. We can find $N$ such that
  $\|f_N - f\| < \frac{\varepsilon}{3}$. We can then find
  $\delta > 0$ such that
  $|z - z_0| < \delta \implies |f_N(z) - f_N(z_0)| < \frac{\varepsilon}{3}$
  since $f_N$ is continuous. But then
  \begin{align*}
         |f(z) - f(z_0)|
  & \leq |f(z) - f_N(z)| + |f_N(z) - f_N(z_0)| + |f_N(z_0) - f(z_0)| \\
  & < \varepsilon.
  \end{align*}
\end{proof}

\begin{defn}[Uniformly Cauchy]
  A sequence $(f_n)$ is \emph{uniformly Cauchy} on $S$
  if $\forall \varepsilon > 0$, $\exists N$ such that $\forall n, m > N$,
  $\| f_n - f_m \|_S < \varepsilon$.
\end{defn}

\begin{theorem}
  $(f_n)$ is uniformly Cauchy on $S$ if and only if
  $f_n \to f$ uniformly on $S$ for some $f : S \to \mathbb{C}$.
\end{theorem}

\begin{proof}
  \begin{itemize}
    \item[($\impliedby$)]{
      This follows from the fact that the supremum norm
      obeys the triangle inequality.
    }
    \item[($\implies$)]{
      For any $z \in S$,
      $|f_n(z) - f_m(z)| \leq \| f_n - f_m \|_S$.
      From the definition of a uniform Cauchy sequence, we see that
      $(f_n(z))$ is a Cauchy sequence of complex numbers for any $z$.
      Let $f(z)$ be the limit of this convergent sequence. Then
      $f : S \to \mathbb{C}$ is such that $f_n \to f$ pointwise.

      Now fix $\varepsilon > 0$. Then there is an $N$ such that
      $n, m > N \implies \| f_n - f_m \|_S < \frac{\varepsilon}{2}$,
      so $|f_n(z) - f_m(z)| < \frac{\varepsilon}{2}$.
      Fix $n > N$ and let $z \in S$. Let $m \to \infty$. Then
      $|f_n(z) - f(z)| \leq \frac{\varepsilon}{2}$ so
      $\| f_n - f \|_S \leq \frac{\varepsilon}{2}$.
    }
  \end{itemize}
\end{proof}

Consider a series of functions
$$
\sum_{n=1}^\infty f_n(z)
$$
with partial sum sequence $s_n(z) = \sum_{k=1}^n f_k(z)$.
We say that $\sum f_n$ converges pointwise (uniformly) if $(s_n)$
converges pointwise (uniformly). If every $f_n$ is continuous and
$\sum f_n$ converges uniformly, then every $s_n$ is continuous
so $f(z) = \sum f_n(z)$ is continuous.

\begin{theorem}
  Let $\sum c_n$ be a series of nonnegative real numbers.
  Assume that $\sum c_n$ converges. Let $f_n : S \to C$
  and suppose $\| f_n \|_S \leq c_n$ for all $n$. Then
  $\sum f_n$ converges uniformly and absolutely on $S$.
\end{theorem}

\begin{proof}
  Note that $|f_n(z)| \leq \|f_n\|_S \leq c_n$, so
  $\sum f_n$ converges absolutely. Note that the
  partial sum sequence is uniformly Cauchy on $S$.
  Indeed, if $n > m$ then
  $$
       \| s_n - s_m \|
  =    \left\| \sum_{k=m+1}^n f_k \right\|
  \leq \sum_{k=m+1}^n \| f_k \| \leq \sum_{k=n+1}^n c_k
  $$
  and since the partial sum sequence for $\sum c_k$ is Cauchy
  it follows that $(s_n)$ is uniformly Cauchy.
\end{proof}

Recall that
$$
\limsup x_n = \lim U_n = \inf \{ U_n : n \geq 1 \}
$$
where $U_n = \sup \{ x_k : k \geq n \}$.
Note that if $L > \limsup x_n$, then $L$ is not
a lower bound of $\{ U_n \}$, i.e.
$\exists n_0$ such that $L > U_{n_0}$.
Similarly if $L < \limsup x_n$, then
$L$ is not an upper bound of
$\{ x_k : k \geq n \}$, i.e.
$\forall n \in \mathbb{N}$,
$\exists k \geq n$ such that $x_k > L$.

\begin{theorem}
  Consider the power series $\sum a_n z^n$.
  Let
  $$
  R = \frac{1}{\limsup |a_n|^{1 / n}}.
  $$
  Then the series diverges if  $|z| > R$, converges absolutely if
  $|z| < R$, and if $R > r > 0$ then the series converges
  uniformly on $\bar{D}(0, r)$.
\end{theorem}

\begin{proof}
  \begin{itemize}
    \item{
      Suppose $|z| > R$. Then
      $\frac{1}{|z|} < \frac{1}{R} = \limsup |a_n|^{1 / n}$.
      Then for any $n$ there exists a $k \geq n$ such that
      $|a_k|^{1 / k} > \frac{1}{|z|}$, so
      $|a_k z^k| = (|a_k|^{1 / k} |z|)^k > 1$. Therefore
      $a_n z^n {\not \to} 0$, so the sum diverges.
    }
    \item{
      In the case where $R = 0$ there is nothing to prove.
      Assume $R > 0$ and $r \in (0 , R)$. Choose
      $L \in (r, R)$. Since $L < R$,
      $\frac{1}{L} > \frac{1}{R} = \limsup |a_n|^{1 / n}$
      so that $|a_k|^{1 / k} < \frac{1}{L}$ whenever
      $k \geq n_0$, for some $n_0$. Then $|a_k L^k| < 1$,
      so the sequence $(a_n L^n)$ is bounded.

      Now let $x = \frac{r}{L} \in (0, 1)$. Then when $|z| \leq r$,
      $$
      |a_n z^n| \leq |a_n| r^n = |a_n L^n| x^n \leq C x^n,
      $$
      and $\sum_{n=0}^\infty C x^n$ converges since $|x| < 1$. By the
      comparison test, the series converges on $\bar{D}(0, r)$.

      Since the series converges absolutely on this set for every
      $r \in (0, R)$, it converges on $D(0, R)$.
    }
  \end{itemize}
\end{proof}

\begin{corol}
  Suppose $R > 0$, and let $f(z) = \sum_{n=0}^\infty a_n z^n$.
  $f$ is continuous on $\bar{D}(0, r)$ for all $r \in (0, R)$.
\end{corol}

\subsection{Differentiation of Power Series}

\begin{lemma}
  The series $\sum_{n=1}^\infty a_n z^{n-1}$ also has radius $R$.
\end{lemma}
\begin{proof}
  $\sum n a_n z^{n-1}$ has the same radius as $\sum n a_n z^n$ since
  $z \sum n a_n z^{n-1} = \sum n a_n z^n$. To find the radius of
  $\sum n a_n z^n$, we have
  $$
  |n a_n|^{1 / n} = n^{1 / n} |a_n|^{1 / n} \to \frac{1}{R}
  $$
  since $n^{1 / n} \to 1$.
\end{proof}

\begin{theorem}
  The function $f(z) = \sum_{n=0}^\infty a_n z^n$ is
  holomorphic on its region of convergence, and its derivative
  is $f^\prime(z) = \sum_{n=0}^\infty n a_n z^{n-1}$.
\end{theorem}

\begin{proof}
  Let
  $$
  f(z) = \sum_{n=0}^\infty a_n z^n, \quad
  g(z) = \sum_{n=0}^\infty n a_n z^{n-1}.
  $$
  Fix $w \in D(0, R)$. Then
  \begin{align*}
     \frac{f(z) - f(w)}{z - w} - g(w)
   &= \frac{1}{z - w} \sum_{n=0}^\infty a_n (z^n - w^n)
    - \sum_{n=0}^\infty n a_n w^{n-1} \\
   &= \sum_{n=1}^\infty a_n \left(\frac{z^n - w^n}{z - w} - nw^{n-1}\right) \\
   &= \sum_{n=2}^\infty a_n
        ( z^{n-1} + z^{n-2} w + z^{n-3} w^2 + \cdots + w^{n-1} - n w^{n-1} \\
   &= \sum_{n=2}^\infty a_n \sum_{k=1}^{n-1} (z^k w^{n-1-k} - w^{n-1}) \\
   &= \sum_{n=2}^\infty a_n \sum_{k=1}^{n-1} w^{n-1-k}(z^k - w^k) \\
   &= (z - w) \sum_{n=2}^\infty a_n
                \sum_{k=1}^{n-1} w^{n-1-k}
                  \sum_{j=0}^{k-1} x^j w^{k-1-j} \\
   &= (z - w) \sum_{n=2}^\infty a_n
                \sum_{k=1}^{n-1}
                  \sum_{j=0}^{k-1} z^j w^{n - 2 - j}.
  \end{align*}
  Choose $L \in (|w|, R)$ and let $\varepsilon = L - |w|$.
  If $|z - w| < \varepsilon$ then $|z| < L$ and $|w| < L$ so
  \begin{align*}
        \left\| \frac{f(z) - f(w)}{z - w} - g(w) \right|
  &\leq |z - w| \sum_{n=2}^\infty |a_n \sum_{k=1}^{n-1} \sum_{j=0}^{k-1} L^{n-2} \\
  &=    |z - w| \sum_{n=2}^\infty |a_n| \frac{n(n-1)}{2} L^{n-2}
   =    |z - w| U.
  \end{align*}
  Since $L < R$ we know that this converges and so $U < \infty$, so
  the limit of the expression above is 0 and thus $f^\prime(w) = g(w)$.
\end{proof}

\begin{remark}
  \begin{itemize}
    \item{
      Observe that the \emph{primitive of $f$} given by
      $$
      F(z) = C + \sum_{n=0}^\infty \frac{a_n}{n + 1} z^{n+1}
      $$
      gives $F^\prime(z) = f(z)$.
    }
    \item{
      The $k$ derivative is given by
      $$
      f^{(k)}(z) = \sum_{n=k}^\infty a_n z^{n-k} \prod_{j=0}^{k-1} (n - j),
      $$
      and when $k = n$ the product term is $n!$. Therefore
      $f^{(k)}(0) = k! a_k$, so we may recover the coefficients $a_k$ from
      $$
      a_k = \frac{f^{(k)}(0)}{k!}
      $$
      so that
    }
    \item{
      We may also consider the power series centered at $z_0$
      $$
      \sum_{n=0}^\infty (z - z_0)^n,
      $$
      and here the series converges if $|z - z_0| < \frac{1}{\limsup |a_n|^{1 / n}}$.
    }
  \end{itemize}
\end{remark}
