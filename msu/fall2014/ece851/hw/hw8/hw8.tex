\documentclass{article}

\usepackage{amsmath}
\usepackage{amsfonts}

\usepackage{commath}
\usepackage{enumerate}

\usepackage{graphicx}

\title{ECE/ME 851 Homework \#8}
\date{November 12, 2014}
\author{Sam Boling \\ PID A48788119}

\begin{document}

\maketitle

\section*{Problem \#1}
The system is given by
$$
A =
\left[\begin{array}{r r r}
  0 & 1 &  0 \\
  0 & 0 &  1 \\
  1 & 0 & -1
\end{array}\right], \quad
B =
\left[\begin{array}{c}
  0  \\
  0  \\
  1
\end{array}\right], \quad
\Gamma =
\left[\begin{array}{c}
  0.1 \\
    0 \\
    0
\end{array}\right], \quad
C =
\left[\begin{array}{c c c}
  1 & 2 &  0
\end{array}\right], \quad
V = 0.1.
$$
The optimal gains can be found by solving the dual algebraic Riccati
equation
$$
P_e A^T + A P_e - P_e C^\top V^{-1} C P_e + \Gamma W \Gamma^\top = 0
$$
for $P_e$, where $W$ was not given and so will be taken to be 1. This
solution gives
% To find the dual system, let
% $$
% \tilde{R} = V, \quad
% \tilde{M} = \Gamma^\top, \quad
% \tilde{A} = A^\top, \quad
% \tilde{B} = C^\top.
% $$
% The optimal gains can then be found by solving the Riccati equation
% $$
%   \tilde{P} \tilde{A} + \tilde{A}^\top P
% + \tilde{M}^\top \tilde{M}
% - \tilde{P} \tilde{B} \tilde{R}^{-1} \tilde{B}^\top P = 0
% $$
% for $P$, which yields
$$
P_e =
\left[\begin{array}{c c c}
  0.0179 & 0.0164 & 0.0078 \\
  0.0164 & 0.0204 & 0.0111 \\
  0.0078 & 0.0111 & 0.0066
\end{array}\right]
$$
so that the controller gains are
$$
  K
= P_e C^\top V^{-1}
=
\left[\begin{array}{c}
  0.5059 \\
  0.5721 \\
  0.2996
\end{array}\right].
$$
In the case where $V = 1$ instead, we have
$$
P_e =
\left[\begin{array}{c c c}
  0.1203 & 0.1517 & 0.0847 \\
  0.1517 & 0.1996 & 0.1131 \\
  0.0847 & 0.1131 & 0.0647
\end{array}\right], \quad
K =
\left[\begin{array}{c}
  0.4236 \\
  0.5508 \\
  0.3109
\end{array}\right].
$$

\section*{Problem \#2}
Assigning states
$$
\left[\begin{array}{c}
  z_1 \\
  z_2 \\
  z_3 \\
  z_4
\end{array}\right]
=
\left[\begin{array}{c}
        x \\
  \dot{x} \\
        y \\
  \dot{y}
\end{array}\right],
$$
we have the system
$$
\left[\begin{array}{c}
  \dot{z_1} \\
  \dot{z_2} \\
  \dot{z_3} \\
  \dot{z_4}
\end{array}\right]
=
\left[\begin{array}{r r r r}
           0 &        1 &          0 &       0 \\
  9 \omega^2 &        0 &          0 & 2 \omega \\
           0 &        0 &          1 &       0  \\
           0 & -2\omega & -4\omega^2 &       0
\end{array}\right]
\left[\begin{array}{c}
  z_1 \\
  z_2 \\
  z_3 \\
  z_4
\end{array}\right]
+
\left[\begin{array}{r}
  0 \\
  0 \\
  0 \\
  1
\end{array}\right].
$$

\begin{enumerate}[(a)]
\item{
The eigenvalues of the matrix $A$ are then located at
$$
\lambda_1 = 0, \quad
\lambda_2 = 1, \quad
\lambda_3 =  \sqrt{5} \omega, \quad
\lambda_4 = -\sqrt{5} \omega,
$$
and since three of these have non-negative real part, the system is
unstable at the origin.
}
\item{
The controllability matrix is given by
$$
\mathcal{C} =
\left[\begin{array}{r r r r}
  0 &        0 &    2 \omega &           0 \\
  0 & 2 \omega &           0 & 10 \omega^3 \\
  0 &        0 &           0 &           0 \\
  1 &        0 & -4 \omega^2 &           0
\end{array}\right],
$$
so we take
$$
Q =
\left[\begin{array}{r r r r}
  0 &        0 &    2 \omega & 0 \\
  0 & 2 \omega &           0 & 0 \\
  0 &        0 &           0 & 1 \\
  1 &        0 & -4 \omega^2 & 0
\end{array}\right], \quad
$$
and then
$$
\hat{A} = Q^{-1} A Q =
\left[\begin{array}{r r r r}
  0 & 0 &           0 & -4 \omega^2 \\
  1 & 0 &  5 \omega^2 & 0 \\
  1 & 0 & -1
\end{array}\right],
$$
indicating that $\lambda_1, \lambda_2, \lambda_3$ are
controllable. Therefore the system is stabilizable, since the only
uncontrollable eigenvalue has negative real part.
}
\item{
The similarity transformation $Q$ into uncontrollable form yields
$$
\hat{\mathcal{C}} = Q^{-1} \mathcal{C} =
\left[\begin{array}{r r r r}
  1 & 0 & 0 &          0 \\
  0 & 1 & 0 & 5 \omega^2 \\
  0 & 0 & 1 &          0 \\
  0 & 0 & 0 &          0
\end{array}\right],
$$
so all values of $z_1 = x$ and $z_2 = \dot{x}$ are in the range space
of $\hat{\mathcal{C}}$, which is the controllable subspace of the
system. Therefore the position and velocity of $x$ can be controlled,
and so the system can be steered to any value of $x$ under state feedback.
}
\item{
$z_3 = y$ is in the controllable subspace of the system but
$z_4 = \dot{y}$ is not. Therefore it is possible to bring $y$ to a
  desired position, but the velocity $\dot{y}$ cannot be controlled,
  so the system may drift rapidly away from this position.
}
\end{enumerate}

\section*{Problem \#3}
\begin{enumerate}[(a)]
\item{
The controllability matrix
$$
\mathcal{C} =
\left[\begin{array}{r r r}
  1 & 4 &  0 \\
  1 & 0 &  8 \\
 -1 & 4 &  0
\end{array}\right]
$$
has full rank, so with state feedback control all eigenvalues can be
positioned arbitrarily. Therefore the closed-loop eigenvalues can be
made to have real part less than -3.
}
\item{
% We can construct a similarity transformation
% $$
% P =
% \left[\begin{array}{c}
%   C \\ \hat{C}
% \end{array}\right] =
% \left[\begin{array}{r r r}
%   2 & 1 & -1 \\
%   1 & 0 &  0 \\
%   0 & 1 &  0
% \end{array}\right]
% $$
% to arrive at the transformed system
% $$
% \bar{A} =
% \left[\begin{array}{r r r}
%   0 &  2 & 2 \\
%   1 & -1 & 1 \\
%  -1 &  3 & 1
% \end{array}\right], \quad
% \bar{B} =
% \left[\begin{array}{r}
%   4 \\ 1 \\ 1
% \end{array}\right], \quad
% \bar{C} =
% \left[\begin{array}{r r r}
%   1 & 0 & 0
% \end{array}\right].
% $$
% We observe that the observability matrix for the pair $(A, \bar{C})$
% has full rank, so that it is possible to construct a reduced-order
% estimator
% We take $A_{11}$ to be $n-p \times n-p$ so that
% $$
% A_{11} =
% \left[\begin{array}{r r}
%   0 &  2 \\
%   1 & -1
% \end{array}\right], \quad
% A_{12} =
% \left[\begin{array}{r}
%   2 \\ 1
% \end{array}\right], \quad
% A_{21} =
% \left[\begin{array}{r r}
%   -1 & 3
% \end{array}\right], \quad
% A_{22} =
% \left[\begin{array}{r}
%   1
% \end{array}\right],
% $$
% and then we have the estimator
% $$
% $$
The system has observability matrix
$$
\mathcal{O} =
\left[\begin{array}{r r r}
  2 & 1 & -1 \\
  2 & 2 &  0 \\
  4 & 4 &  0
\end{array}\right],
$$
which has a null space spanned by
$
\left[\begin{array}{r r r}
  -1 & 1 & -1
\end{array}\right]^\top
$
and therefore a transformation to unobservable form given by
$$
\left[\begin{array}{r r r}
  1 & 0 & -1 \\
  0 & 1 &  1 \\
  0 & 0 & -1
\end{array}\right]
$$
so that
$$
\hat{\mathcal{O}} =
\left[\begin{array}{r r r}
  2 & 1 & 0 \\
  2 & 2 & 0 \\
  4 & 4 & 0
\end{array}\right]
$$
which means the eigenvalues 2 and 0 are observable. The system is
therefore controllable and detectable, so it is possible to stabilize
it with output feedback.
}
\end{enumerate}

\end{document}
