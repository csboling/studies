\documentclass{article}

\usepackage{amsmath}
\usepackage{amsfonts}
\usepackage{enumerate}

\title{ECE/ME 851 Homework \#1}
\date{September 12, 2014}
\author{Sam Boling \\ PID A48788119}

\begin{document}

\maketitle

\section*{Problem \#1}

Writing the node-voltage equation at the positive terminal of the
nonlinear resistor gives
\begin{align*}
i_R(t) + C\dot{v_R}(t) + \frac{v_R(t)}{R} - v_i(t) & = 0, \\
\end{align*}
where $R = C = 1$, so
\begin{align*}
1.5[v_R(t)]^3 + \dot{v_R}(t) + v_R(t) & = v_i(t)
\end{align*}
or
\begin{align*}
\dot{v_R}(t) & = -1.5 [v_R(t)]^3 - v_R(t) + v_i(t) = f(t, v_R, v_i).
\end{align*}

Letting $f(t, v_R, v_i) = 0$ for the nominal value $v_i = 14$ gives
$$
1.5[v_R(t)]^3 + v_R(t) = 14,
$$
and thus equilibrium solutions 
$$\tilde{v_R} = 2, -1 \pm i \sqrt{\frac{11}{3}}$$
The only real-valued equilibrium point is then at $\tilde{v_R} = 2$.

Taking the Jacobian of the right-hand side $f(t, v_R, v_i)$ gives
\begin{align*}
\left[\begin{array}{c c}
  \left.\frac{\partial f}{\partial v_R}\right|_{v_R = \tilde{v_R}} 
& \left.\frac{\partial f}{\partial v_i}\right|_{v_i = \tilde{v_i}}
\end{array}\right]
&=
\left[\begin{array}{c c}
  \left[-4.5 [v_R(t)]^2 - 1\right]_{v_R = \tilde{v_R}} 
& \left[1\right]_{v_i = \tilde{v_i}}
\end{array}\right]. \\
\end{align*}
To linearize the system, we write
\begin{align*}
\dot{v_R}(t) 
  &=
     \dot{\tilde{v_R}}
   + \dot{v_{R,\delta}} \\
  &= f(t, 
       \tilde{v_R} + v_{R,\delta},
       \tilde{v_i} + v_{i,\delta}) \\
  &=
     \dot{\tilde{v_{R}}}
   + \left.\frac{\partial f}{\partial v_R}\right|_{v_R = \tilde{v_R}}
       (v_R - \tilde{v_R})
   + \left.\frac{\partial f}{\partial v_i}\right|_{v_i = \tilde{v_i}}
       (v_i - \tilde{v_i})
   + O(v_{R,\delta}^2) + O(v_{R,i}^2) \\
  &\approx
    \dot{\tilde{v_{R}}}
     + \left.\frac{\partial f}{\partial v_R}\right|_{v_R = \tilde{v_R}}
         (v_R - \tilde{v_R})
     + \left.\frac{\partial f}{\partial v_i}\right|_{v_i = \tilde{v_i}}
         (v_i - \tilde{v_i})
\end{align*}
so that
\begin{align*}
\dot{v_{R,\delta}} 
  &\approx
     \left.\frac{\partial f}{\partial v_R}\right|_{v_R = \tilde{v_R}}
       (v_{R,\delta})
   + \left.\frac{\partial f}{\partial v_i}\right|_{v_i = \tilde{v_i}}
       v_{i,\delta} \\
  &= (-4.5 \tilde{v_R}^2 - 1)v_{R,\delta}
   + v_{i,\delta} \\
  &= (-4.5 \cdot 4 - 1)v_{R,\delta} + v_{i,\delta} \\
  &= -19 v_{R, \delta} + v_{i, \delta}.
\end{align*}
%\begin{align*}
%v_{R,\delta}(0) &= v_R(0) - \tilde{v_R}(0)
%\end{align*}

\pagebreak

\section*{Problem \#2}
We have
\begin{align*}
  M \ddot{y}_1 &= -K_1 y_1 +K (y_2 - y_1) +B (\dot{y}_2 - \dot{y}_1) + f_1 , \\
  M \ddot{y}_2 &= -K (y_2 - y_1) -B (\dot{y}_2 - \dot{y}_1) -K_1 y_2 - f_2
\end{align*}
so assigning
\begin{align*}
x_1 &= y_1 , \\
x_2 &= y_2 , \\
x_3 &= \dot{y}_1 , \\
x_4 &= \dot{y}_2 , \\
u_1 &= f_1 , \\
u_2 &= f_2 ,
\end{align*}
we see that
\begin{align*}
\dot{x}_1 &= \dot{y}_1  = x_3 , \\
\dot{x}_2 &= \dot{y}_2  = x_4 , \\
\dot{x}_3 &= \ddot{y}_1 = -\frac{1}{M_1}\left( K_1 + K \right) x_1
                        +  \frac{K}{M_1} x_2
                        -  \frac{B}{M_1} x_3
                        +  \frac{B}{M_1} x_4
                        +  \frac{1}{M_1} u_1, \\
\dot{x}_4 &= \ddot{y}_2 = \frac{K}{M_2} x_1
                        - \frac{1}{M_2} \left( K + K_1 \right) x_2
                        + \frac{B}{M_2} x_3
                        - \frac{B}{M_2} x_4 
                        - u_2.
\end{align*}

Then the state equation $x = Ax + Cu$ can be written as

\begin{align*}
\left[\begin{array}{c}
x_1 \\ x_2 \\ x_3 \\ x_4
\end{array}\right] &=
\left[\begin{array}{c c c c}
0 & 0 & 1 & 0 \\
0 & 0 & 0 & 1 \\
-\frac{K_1 + K}{M_1}
  & \frac{K}{M_1}
  & -\frac{B}{M_1}
  & \frac{B}{M_1} \\
\frac{K}{M_2} 
\  & -\frac{K + K_1}{M_2} 
  & \frac{B}{M_2}
  & -\frac{B}{M_2} 
\end{array}\right]
\left[\begin{array}{c}
x_1 \\ x_2 \\ x_3 \\ x_4
\end{array}\right]
+ \left[\begin{array}{c c}
\frac{1}{M_1} & 0 \\ 0 & -1 \\ 0 & 0 \\ 0 & 0
\end{array}\right]
\left[\begin{array}{c}
u_1 \\ u_2
\end{array}\right].
\end{align*}

Taking the output to by $z = y_1 - y_2 = x_1 - x_2$, the output equation is then
\begin{align*}
\left[\begin{array}{c}
z
\end{array}\right]
&=
\left[\begin{array}{c c c c}
-1 & 0 & 1 & 0
\end{array}\right]
\left[\begin{array}{c}
x_1 \\ x_2 \\ x_3 \\ x_4
\end{array}\right]
+
\left[\begin{array}{c c}
0 & 0
\end{array}\right]
\left[\begin{array}{c}
u_1 \\ u_2
\end{array}\right].
\end{align*}

\section*{Problem \#3}

\begin{enumerate}[(a)]
  \item{The circuit is governed by the differential equation
        \begin{align*}
          v(t) & = L[s(t)]\dot{i}(t) + Ri(t) \\
               & = \frac{L}{s(t)} \dot{i}(t) + Ri(t) & \implies \\
          \frac{1}{L}s(t) v(t) 
             & = \dot{i}(t) + \frac{R}{L} s(t) i(t) & \implies \\
          \dot{i}(t) & = s(t) \left(\frac{1}{L} v(t) - \frac{R}{L} i(t)\right),
        \end{align*}
        while the free-body diagram on the steel ball yields the equation
        \begin{align*}
          \ddot{s}(t) = \ddot{x_2}(t) 
            & = \frac{1}{M}\left(-Mg + F\right) \\
            & = \frac{K}{M} \left(\frac{i(t)}{s(t)}\right)^2  - g.
        \end{align*}

        This can be rewritten as a first-order system by setting
        $$
              x_1(t) = i(t), 
        \quad x_2(t) = s(t),
        \quad x_3(t) = \dot{s}(t)
        \quad u(t)   = v(t)
        $$
        so that
        \begin{align*}
          \dot{x_1}(t) 
          &= -\frac{R}{L} x_1(t) x_2(t) + \frac{1}{L} x_2(t) u(t) 
           = f_1(t, x_1, x_2, x_3, u)\\
          \dot{x_2}(t) =
          &= x_3 = f_2(t, x_1, x_2, x_3, u)\\
          \dot{x_3}(t)
          &= \frac{K}{M}\left(\frac{x_1}{x_2}\right)^2 - g
           = f_3(t, x_1, x_2, x_3, u).
        \end{align*}
       }
  \item{ If $v(t) = v_e$ and 
         $\dot{x_1}(t) = \dot{x_2}(t) =  \dot{x_3}(t) = 0, \forall t$, 
         then we have the equations
         \begin{align*}
         R \tilde{x_1} \tilde{x_2} &= \tilde{x_2} v_{eq},\\
         \frac{K}{M}\left(\frac{\tilde{x_1}}{\tilde{x_2}}\right)^2 = g
         \end{align*}
         which gives
         \begin{align*}
         \tilde{x_1} &= \frac{v_{eq}}{R}, \\
         \tilde{x_2} &= \sqrt{\frac{K}{gM}} \tilde{x_1}
                      = \sqrt{\frac{K}{gM}}\frac{v_{eq}}{R}.
         \end{align*}
       }
       \item{ Computing the Jacobian of the system of equations given
              in (a) at the equilibrium point found in (b) gives
           \begin{align*}
              \mathbf{A}
              &= \left[\begin{array}{c c c}
                 \frac{\partial f_1}{\partial x_1}
              &  \frac{\partial f_1}{\partial x_2}
              &  \frac{\partial f_1}{\partial x_3} \\
                 \frac{\partial f_2}{\partial x_1}
              &  \frac{\partial f_2}{\partial x_2}
              &  \frac{\partial f_2}{\partial x_3} \\
                 \frac{\partial f_3}{\partial x_1}
              &  \frac{\partial f_3}{\partial x_2}
              &  \frac{\partial f_3}{\partial x_3}
                 \end{array}\right]_{\substack{x_1 = \tilde{x_1} \\
                                               x_2 = \tilde{x_2} \\
                                               x_3 = \tilde{x_3} \\
                                               u   = \tilde{u}}} \\
              &= \left[\begin{array}{c c c}
                   -\frac{R}{L} x_2 
                 & -\frac{R}{L} x_1 + \frac{1}{L} u
                 & 0 \\
                   0 & 0 & 1 \\
                   \frac{2K}{M}\frac{x_1}{x_2^2}
                 & -\frac{2K}{M}\frac{x_1^2}{x_2^3}
                 & 0
                 \end{array}\right]_{\substack{x_1 = \tilde{x_1} \\
                                               x_2 = \tilde{x_2} \\
                                               x_3 = \tilde{x_3} \\
                                               u   = \tilde{u}}} \\
                 &= \left[\begin{array}{c c c}
                      -\frac{v_{eq}}{L}\sqrt{\frac{K}{gM}}
                    & -\frac{v_{eq}}{L} + \frac{v_{eq}}{L}
                    & 0 \\
                    0 & 0 & 1 \\
                    \frac{2K}{M}\frac{v_{eq}}{R}\frac{gM}{K}
                    \left(\frac{R}{v_{eq}}\right)^2
                  & -\frac{2K}{M}\left(\frac{v_{eq}}{R}\right)^2
                     \left(\frac{k}{gM}\right)^{-3/2}
                     \left(\frac{R}{v_{eq}}\right)^3
                  & 0
                 \end{array}\right] \\
               &= \left[\begin{array}{c c c}
                   -\frac{v_{eq}}{L}\sqrt{\frac{K}{M}} & 0 & 0 \\
                   0 & 0 & 1 \\
                   \frac{2 R g}{v_{eq}} 
                & -\frac{2 R g^{3/2}}{v_{eq}}\left(\frac{K}{M}\right)^{-1/2}
                & 0
                  \end{array}\right], \\
              \mathbf{B} &= \left[\begin{array}{c}
                            \frac{\partial f_1}{\partial u} \\
                            \frac{\partial f_2}{\partial u} \\
                            \frac{\partial f_3}{\partial u}
                            \end{array}\right]_{\substack{x_1 = \tilde{x_1} \\
                                               x_2 = \tilde{x_2} \\
                                               x_3 = \tilde{x_3} \\
                                               u   = \tilde{u}}} 
                         = \left[\begin{array}{c}
                            \frac{1}{L} x_2 \\ 0 \\ 0
                            \end{array}\right]_{\substack{x_1 = \tilde{x_1} \\
                                               x_2 = \tilde{x_2} \\
                                               x_3 = \tilde{x_3} \\
                                               u   = \tilde{u}}} 
                         = \left[\begin{array}{c}
                            \frac{v_{eq}}{RL} \sqrt{\frac{K}{gM}} \\
                            0 \\ 0
                            \end{array}\right].
              \end{align*}
The linearized state space equation can then be written as
\begin{align*}
\dot{\mathbf{x}} &= \mathbf{A}\mathbf{x} + \mathbf{B}u \\
\end{align*}
or
\begin{align*}
\left[\begin{array}{c}
\dot{x_1} \\ \dot{x_2} \\ \dot{x_3}
\end{array}\right] &=
\left[\begin{array}{c c c}
                   -\frac{v_{eq}}{L}\sqrt{\frac{K}{M}} & 0 & 0 \\
                   0 & 0 & 1 \\
                   \frac{2 R g}{v_{eq}} 
                & -\frac{2 R g^{3/2}}{v_{eq}}\left(\frac{K}{M}\right)^{-1/2}
                & 0
                  \end{array}\right]
\left[\begin{array}{c}
x_1 \\ x_2 \\ x_3
\end{array}\right]
+
\left[\begin{array}{c}
      \frac{v_{eq}}{RL} \sqrt{\frac{K}{gM}} \\ 0 \\ 0
\end{array}\right] u.
\end{align*}
}
\end{enumerate}

\end{document}
