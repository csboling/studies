\section{Abelian Groups}

\subsection{Direct Product}
\begin{defn}[Direct Product]
Let $I$ be a set, and $\forall i \in I$ let $G_i$ be a group. Then we
have the \emph{direct product} $\prod_{i \in I} G_i$ as a set is the
set of all families $(g_i)_{i \in I}$ with $g_i \in G_i$. The group
operation is given by
$$
(g_i)_{i \in I} \cdot (g_i^\prime)_{i \in I}
= (g_i \cdot g_i^\prime)_{i \in I}.
$$
\end{defn}
This group is equipped with homomorphisms
$$
\mathrm{pr}_j : \prod_{i \in I} G_i \to G_j,
$$
the \emph{projection on the $j$ component} given by
$$
(g_i)_{i \in I} \mapsto g_j.
$$

\begin{prop}
$\prod_{i \in I} G_i$ satisfies the following \emph{universal
  property}: given a group $H$ and a family of group homomorphisms
$\varphi_i : H \to G_i$, $\forall i \in I$, there is a unique
homomorphism $\varphi : H \to \prod_{i \in I} G_i$ such that
$\varphi_i = \mathrm{pr}_i \circ \varphi$. This may be written

$$
\begindc{\commdiag}[500]
\obj(0,1)[h]{$H$}
\obj(2,1)[p]{$\prod_{i \in I} G_i$}
\obj(2,0)[g]{$G_i$}
\mor{h}{p}{$\varphi$}[\atright, \dashArrow]
\mor{p}{g}{$\mathrm{pr}_i$}
\mor{h}{g}{$\varphi_i$}
\enddc
$$
\end{prop}
\begin{proof}
Define
$$
\varphi(h) = (\varphi_i(h))_{i \in I} \in \prod_{i \in I} G_i.
$$
\end{proof}

For abelian groups recall that we denote the operation by $+$ and the
identity by $0_i$.

\begin{defn}[Direct Sum]
Let $G_i$ be a family of abelian groups. The \emph{direct sum}
$\oplus_{i \in I} G_i$ is the subgroup of the direct product consisting
of families $(g_i)_{i \in I}$ such that for all but a finite number of
indices $i \in I$, $g_i = 0_i$.
\end{defn}

\begin{xmpl}
Let $I = \{ 1, 2, \dots \}$. Let $G_i = \mathrm{Z}$ for all $i$. Then
we have
$$
\prod_{i \in \mathbb{Z_+}} \mathbb{Z}
  = \{(n_1, n_2, \dots) \mid n_i \in \mathbb{Z}\}
$$
while
$$
\bigoplus_{i \in \mathbb{Z_+}} \mathbb{Z}
  = \{(n_1, n_2, \dots, n_j, 0, 0, \dots ) \mid n_i \in \mathbb{Z}\}.
$$
\end{xmpl}

The direct sum $\oplus_{i \in I} G_i$ is equipped with homomorphisms
$\mathrm{inj}_i : G_i \to \oplus_{i \in I} G_i$ given by
$\mathrm{inj}_i(g_i) = (\dots, 0, g_i, 0, \dots)$.

\begin{prop}[Universal property of the direct sum]
Suppose $G$ is an abelian group and
$f_i : G_i \to G$ is a group homomorphism for each $i \in I$. Then
there is a group homomorphism $f : \oplus_{i \in I} G_i \to G$ such
that $\forall i \in I$,
$$
f( (\dots, 0, g_i, 0, \dots) ) = f_i(g_i).
$$
i.e. $f \circ \mathrm{inj}_i = f_i$. $f$ is uniquely determined by
this property.
\end{prop}

This property can also be expressed by the diagram
$$
\begindc{\commdiag}[500]
\obj(0,1)[h]{$H$}
\obj(2,1)[p]{$\bigoplus_{i \in I} G_i$}
\obj(2,0)[g]{$G_i$}
\mor{p}{h}{$\varphi$}[\atright, \dashArrow]
\mor{g}{p}{$\mathrm{inj}_i$}
\mor{g}{h}{$\varphi_i$}
\enddc
$$

\begin{xmpl}
Given $f_1 : G_1 \to G$ and $f_2 : G_2 \to G$ we have
$f: G_1 \oplus G_2 \to G$ given by
$f((g_1, g_2)) = f_1(g_1) + f_2(g_2)$. This is only a group
homomorphism for abelian $G$ since we require
\begin{align*}
  f((g_1, g_2)(h_1, h_2))
&= f((g_1 h_1, g_2 h_2))
 = f_1(g_1 h_1) + f_2(g_2 h_2) \\
&= f_1(g_1) + f_1(h_1) + f_2(g_2) + f_2(h_2) \\
&= f_1(g_1) + f_2(g_2) + f_1(h_1) + f_2(h_2) \\
&= f((g_1, g_2)) + f((h_1, h_2)).
\end{align*}
More generally
$$
f((g_i)_{i \in I}) = \sum_{i \in I} f_i(g_i).
$$
The reason this is well-defined is since $g_i = 0_i$ and therefore
$f_i(g_i) = 0$ for all but a finite number of $g_i$. This cannot be
defined on the direct product unless $I$ is a finite set, in which
case there is no distinction between the direct sum and the direct
product. The reason this is unique is that
$$
(g_1, g_2) = (g_1, 0) + (0, g_2)
$$
so that
$$
f((g_1, g_2)) = f((g_1, 0)) + f((0, g_2)) = f_1(g_1) + f_2(g_2).
$$
\end{xmpl}

\subsection{Free abelian groups}
Consider the direct sum
$$
\bigoplus_{i \in I} \mathbb{Z} \triangleq \mathbb{Z}[I]
$$
where $I$ is any set.

\begin{defn}[Free Abelian Group]
An abelian group $A$ is called \emph{free} if there is a set $I$ such
that
$$
A \simeq \mathbb{Z}[I].
$$
\end{defn}

\begin{defn}[Equivalent]
An abelian group $A$ is free if there is a subset (called a \emph{basis})
$\{ e_i \}_{i \in I}$ of $A$ such that every element $a \in A$ can be
written \emph{uniquely} as a linear combination
$$
a = \sum_{i \in I} n_i e_i,
$$
where $n_i \in \mathbb{Z}$ and all but a finite number of $n_i$ are 0.
Here $n_i e_i$ denotes $n_i-fold$ group
operations of $e_i$ on itself.

This is equivalent to the above definition. Observe that
$\mathbb{Z}[I]$ has a ``canonical'' basis given by
$e_i = \mathrm{inj}_i(1)$ and that giving a basis of $A$ is now
equivalent to giving an isomorphism
$\varphi : \mathbb{Z}[I] \xrightarrow{\sim} A$ given by a basis
$$
\{\varphi(e_i)\}_{i \in I}.
$$
That is, given a basis $\{ a_i \}_{i \in I}$ a basis of $A$, define
$\varphi : \mathbb{Z}[I] \to A$ by
$$
\varphi((n_i)) = \sum_{i \in I} n_i a_i.
$$
\end{defn}

\begin{prop}[Universal Property of Free Abelian Groups]
Let $I$ be a set and let
$\mathbb{Z}[I] = \oplus_{i \in I} \mathbb{Z}$.
Suppose we are given an abelian group $A$ and a set map
$f : I \to A$. There is a unique group homomorphism
$\tilde{f} : \mathbb{Z}[I] \to A$ such that
$\tilde{f}(e_i) = f(i)$, $\forall i$.
\end{prop}

\begin{proof}
Define
$$
\tilde{f}\left(\sum_{i \in I} n_i e_i\right)
= \sum_{i \in I} n_i f(i).
$$
\end{proof}

First we require the following:
\begin{lemma}
Let $f : A \to A^\prime$ be a group homomorphism of abelian
groups. Suppose $f$ is surjective and $A^\prime$ is free. Then there
is a group homomorphism $g : A^\prime \to A$ such that
$f \circ g = \mathrm{id}_{A^\prime}$.
\end{lemma}
\begin{proof}
Since $A^\prime$ is free it has a basis $\{ a_i^\prime \}_{i \in
  I}$. Since $f$ is surjective, we have $a_i^\prime = f(a_i)$ for some
$a_i \in A$. Consider $g : A^\prime \to A$ given by
$$
g\left(\sum_{i \in I} n_i a_i^\prime\right) = \sum_{i \in I} n_i a_i.
$$
Then
$$
(f \circ g)\left(\sum_{i \in I} n_i a_i^\prime\right)
= f\left(\sum_{i \in I} n_i a_i\right)
= \sum_{i \in I} n_i f(a_i)
= \sum_{i \in I} n_i a_i^\prime.
$$
since $f$ is a homomorphism. Check that $g$ is a homomorphism also.

We can think of this as analogous to producing a map
$\mathbb{R} \to \mathbb{R}^2$ from a surjective map in the opposite
direction -- the map we produce corresponds to projection of a line
$L \in \mathbb{R}^2$ onto a non-parallel line.
\end{proof}

\begin{lemma}
Let $A$ be an abelian group with two subgroups $B$ and $C$
such that $B \cap C = \{ 0 \}$ and $A = B + C$, i.e.
$\forall a \in A$ there exist $b \in B$ and $c \in C$ such that
$a = b + c$. Then the map $B \times C \to A$ given by
$(b, c) \mapsto b + c$ is a group isomorphism, so
$A \simeq B \times C = B \oplus C$.
\end{lemma}

\begin{proof}
Since $A$ is abelian, $(b, c) \mapsto b + c$ is a group
homomorphism. The condition $A = B + C$ implies it is surjective and
the condition $B \cap C = \{ 0 \}$ implies it is injective, since
the kernel is
\begin{align*}
 & \{ (b, c)  \mid b + c = 0 \} \\
=& \{ (b, -b) \mid b \in B, b \in C \}
=& \{ (0, 0) \}
\end{align*}
since $B \cap C = \{ 0 \}$. In this situation we say $A$ is the
\emph{direct sum} of its subgroups and write $A = B \oplus C$
(as abuse of notation, since this is in fact an isomorphism).
\end{proof}

More generally we have the following:
\begin{lemma}
Let $A$ be an abelian group and let $B_i$, $i \in I$ be a family of
subgroups of $A$ such that every element of $a \in A$ can be written
as a sum
$$
a = \sum_{i \in I} b_i
$$
in a unique way, with a finite number of $b_i$ nonzero.
(This condition implies that for a given $B_i$ its intersection with
 the sum of all other $B_j$ is trivial).
Then the map
$$
\bigoplus_{i \in I} B_i \to A
$$
given by
$$
(b_i)_{i \in I} \mapsto \sum_{i \in I} b_i
$$
is a group isomorphism, and in this case we write
$$
A = \bigoplus_{i \in I} B_i.
$$
\end{lemma}

\begin{lemma}
Consider our lemma about a surjective map $f$ onto a free abelian
codomain. Then we have a right inverse $g$ of $f$. we can write
$$
A = \ker f \oplus C,
$$
where $C \simeq A^\prime$ so that $A \simeq B \oplus A^\prime$.
\end{lemma}

\begin{proof}
Set $C = \mathrm{Im}(g) \subset A$. We observe that
\begin{enumerate}
  \item{$\mathrm{Im}(g) \cap \ker(f) = \{ 0 \}$.
        Let $x = g(y)$ and $f(x) = 0$ for some $x$. Then
        $f(g(y)) = 0$, so since $f \circ g = \mathrm{id}$ this means
        $y = 0$, and therefore $x = 0$.
       }
  \item{$A = \mathrm{Im}(g) + \ker(f)$. Write
        $$
        x = (x - g(f(x)) + g(f(x)).
        $$
        Then $g(f(x)) \in \mathrm{Im}(g)$ and
        $$
        f(x - g(f(x))) = f(x) - f(g(f(x))) = f(x) - f(x) = 0
        $$
        so $x - g(f(x)) \in \ker(f)$.
       }
  \item{Take $C = \mathrm{Im}(g)$. The restriction $f|_c : C \to
        A^\prime$ is surjective since $\forall y \in A^\prime$,
        $g(y) \in C$ and $f(g(y)) = y$, and injective since
        $\ker (f|_C) = \ker(f) \cap C = \{ 0 \}$ as previously shown.
       }
\end{enumerate}
Therefore $A = \ker(f) \oplus \mathrm{Im}(g)$ and $\mathrm{Im}(g)
\simeq A^\prime$.
\end{proof}

\begin{theorem}
Suppose $A$ is a free abelian group. Let $B$ be a subgroup of $A$.
Then $B$ is also free.
\end{theorem}

\begin{proof}
The general case is similar to the finite argument.
For a more particular case, assume $A$ has a finite basis
$\{e_i\}_{i = 1}^n$.
This means $A \simeq \bigoplus_{i = 1}^n \mathbb{Z} = \mathbb{Z}^n$.

We have $B \subset \mathbb{Z}^n$. Prove $B$ is free by induction on
$n$.

\begin{enumerate}
  \item{
    Take $n = 1$. Then $B \subset \mathbb{Z}$, so $B$ is generated
    by the smallest positive integer $b_0 \in B$, so
    $B = \langle b_0 \rangle = \mathbb{Z} \cdot b_0$. Then
    $B \simeq \mathbb{Z}$ and so $B$ is free with $b_0$ as a
    basis, since we can write $b = q b_0 + r$, so
    $r = b - q b_0$, so $r \in B$ since $B$ is a subgroup. But
    $0 \leq r < b_0$, and $b_0$ is the smallest nonzero element in
    $B$ by assumption, so $r = 0$. Therefore $b = q b_0$ for some $q$.
  }
  \item{
    We next have
    $$
    B \subset \mathbb{Z}^n = \mathbb{Z} \cdot e_1
                           + \cdots
                           + \mathbb{Z} \cdot e_n.
    $$
    Consider the projection $\mathrm{pr}_1$ of $\mathbb{Z}^n$ to the
    first element and consider
    $f = \mathrm{pr}_1|_B : B \to \mathbb{Z}$. There are two
    possibilities:
    \begin{itemize}
      \item{
        $f = 0$. Then
        $$
        B \subset \mathbb{Z} \cdot e_2 + \cdots + \mathbb{Z} \cdot e_n
          =       \mathbb{Z}^{n-1}
        $$
        so $B$ is free by induction.
      }
      \item{
        Take $\{ 0 \} \neq \mathrm{Im}(f) \subset \mathbb{Z}$. Then as
        in the $n = 1$ case, $\mathrm{Im}(f) = \mathbb{Z} \cdot b_0$
        is free. Then $f : B \to \mathrm{Im}(f)$ is a surjective map
        onto a free group, so
        $B = \ker(f) \oplus C$ where $C \simeq \mathrm{Im}(f) \simeq
        \mathbb{Z}$, and then
        $B \simeq \ker(f) \oplus \mathbb{Z}$.

        Look at $\ker(f) \subset B$. Take $b \in \ker (f)$. Then
        $\mathrm{pr}_1|_B(b) = 0$ so $b \in \mathbb{Z}^{n-1}$, so by
        the induction hypothesis $\ker(f)$ is free. Therefore
        $B \simeq \ker(f) \oplus \mathbb{Z}$ is free.
      }
    \end{itemize}
  }
\end{enumerate}
This means also that $B$ has a basis with $\leq n$ elements.
\end{proof}
