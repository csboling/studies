\section{Permutation (Symmetric) Groups}
Recall that
$$
  S_n
= \mathrm{Perm}(\{1, 2, \dots, n\})
= \{ f : \{ 1, \dots, n \} \to \{ 1, \dots, n \} \}
$$
with the group operation given by composition of maps.

We denote elements of this as, for example
$$
  (1432)(57)
\triangleq
  (1 \to 4 \to 3 \to 2 \to 1) \circ (5 \to 7 \to 7),
$$
where all unlisted elements are kept fixed.

\begin{defn}
An element of the form $(a_1 a_2 \cdots a_k)$ with
$a_i \in \{ 1, \dots, n \}$, $a_i$ distinct is called a
\emph{$k$-cycle}.
For instance $(1432)$ is a 4-cycle. We will call a 2-cycle
a \emph{transposition}.
\end{defn}

Since we have composition of maps as the operation, the calculation of
the product is done right-to-left:
$$
(1243)(412) = (1423).
$$

\begin{remark}
Every $k$-cycle is an element of order $k$:
$$
(a_1 \cdots a_k)^k = 1.
$$
\end{remark}

\begin{prop}
Every permutation is a product of transpositions.
\end{prop}
\begin{proof}
Consider $S_n$ and perform induction on $n$. The proposition is
vacuous for $n = 1$ and $n = 2$, so assume it for $S_{n-1}$.

Let $\sigma \in S_n$. We wish to find some $k \in \{1, \dots, n\}$.
Take $\sigma(k) = n$ or $k = \sigma^{-1}(n)$, since $\sigma$ is
injective. If $k = n$, then $\sigma(n) = n$, so it only permutes the
first $n-1$ and then
$\sigma \in S_{n-1}$. If $k < n$, let $\tau = (kn)$ and take
$$
\sigma^\prime = \sigma \tau.
$$
Then
$$
  \sigma^\prime(n)
= \sigma (\tau(n))
= \sigma(k)
= n
$$
so
$\sigma^\prime$ only permutes the first $n-1$ elements and thus
$\sigma^\prime \in S_{n-1}$. Then from the inductive hypothesis,
$\sigma^\prime = \tau_1 \cdots t_m$,
so $\sigma \tau = \tau_1 \cdots t_m$,
so
$$
\sigma
= \tau_1 \cdots t_m \tau^{-1}
= \tau_1 \cdots t_m \tau.
$$
\end{proof}

\subsection*{Even and Odd Permutations}
To every permutation $\sigma \in S_n$, associate a $n \times n$ matrix
with $w(\sigma)_{ij} = \delta_i^{\sigma(j)}$.

\begin{xmpl}
$$
w((12)) = \left(\begin{array}{c c}
            0 & 1 \\ 1 & 0
          \end{array}\right), \quad
w((123)) = \left(\begin{array}{c c c}
             0 & 0 & 1 \\ 1 & 0 & 0 \\ 0 & 1 & 0
           \end{array}\right).
$$
Check that $w : S_n \to \mathrm{GL}_n(\mathbb{R})$ is a group
homomorphism. Consider the determinant of $w(\sigma)$,
$$
\xi(\sigma) \triangleq \det(w(\sigma)) \in \{1, -1\}.
$$
Then there is a group homomorphism $\xi : S_n \to \{ \pm 1 \}$ called
the \emph{sign of the permutation}. The sign is even when $\xi(\sigma)
= 1$ (in which case it is a product of an even number of permutations)
and odd when $\xi(\sigma) = -1$. For a $k$-cycle, the sign is
given by $(-1)^{k-1}$ and in particular a 2-cycle (transposition) has
sign $-1$.

Note the circular definition, since the determinant is often defined
in terms of the sign of a permutation.
\end{xmpl}

\begin{remark}[Decomposition into Cycles]
Each $\sigma \in S_n$ can be written essentially uniquely as a product
of disjoint cycles. This is done by decomposing the permutation into
orbits. For example,
$$
(1 ~ \sigma(1) ~ \sigma^2(1) ~ \cdots ~ \sigma^{k-1}(1))
$$
is the orbit of 1 under $\sigma$, since $\sigma^k(1) = 1$ for some $k$.
Choosing the smallest number $m$ not found in this list we then compute
$$
(m ~ \sigma(m) ~ \cdots ~ \sigma^{l-1}(m)),
$$
et cetera. By ``essentially'' we mean up to permutation of factors in
this product, because disjoint cycles commute.
\end{remark}

\begin{theorem}[Conjugation Formula]
Let $\sigma \in S_n$ and $(a_1 ~ \cdots ~ a_k)$ be a $k$-cycle.
The \emph{conjugate of the $k$-cycle by $\sigma$} is given by
$$
\sigma(a_1 ~ \cdots ~ a_k)\sigma^{-1}
= (\sigma(a_1) ~ \cdots ~ \sigma(a_k)).
$$
For example
$$
(1 ~ 3 ~ 4) (2 ~ 3 ~ 1 ~ 4 ~ 5) (1 ~ 3 ~ 4)^{-1}
= (2 ~ 4 ~ 3 ~ 1 ~ 5).
$$
\end{theorem}

\begin{corol}
\begin{enumerate}
  \item{
    All $k$-cycles for a fixed $k$ are conjugate elements in $S_n$.
  }
  \item{
    The conjugacy classes in $S_n$ are in 1-1 correspondence to
    ``partitions of the number $n$''. By this we mean for instance in
    $S_5$, the conjugacy classes have representatives
    $$
    \mathrm{id} = (1), \quad
    (1 ~ 2),           \quad
    (1 ~ 2 ~ 3),       \quad
    \cdots
    $$
    and
    $$
    (1 ~ 2) (3 ~ 4), \quad
    (1 ~ 2 ~ 3)(4 ~ 5)
    $$
    since for instance
    $$
    \sigma (1~2) (3~4) \sigma^{-1}
  = \sigma (1~2) \sigma^{-1} \sigma (34) \sigma^{-1}
  = (\sigma(1) ~ \sigma(2))(\sigma(3) ~ \sigma(4)).
    $$
    Note that including unwritten elements e.g. $(3)$ we have a
    correspondence between these and the partitions of $n$:
    $$
    1 + 1 + 1 + 1 + 1
  = 2 + 1 + 1 + 1
  = 3 + 1 + 1
  = 4 + 1
  = 5
  = 2 + 2 + 1
  = 3 + 2.
    $$
    $$
    (14325)(12)(14325)^{-1}
    $$

  }
\end{enumerate}
\end{corol}

\begin{prop}
If $n \geq 5$, $S_n$ is not solvable. In the context of Galois theory,
this can be used to show that polynomials of order 5 or greater have
no closed form solution.
\end{prop}

\begin{proof}
We claim that given $N < H < S_n$, $N \triangleleft
H$ such that $H / N$ is abelian,
if $H$ contains all 3-cycles, then so does $N$.

Take $i, j, k, r, s \in \{1, \dots, n\}$ all distinct. This is
possible since $n \geq 5$.

Consider the 3-cycles $\sigma = (i~j~k)$, $\tau = (k~r~s)$. The
commutator is given by
$$
  \sigma \tau \sigma^{-1} \tau^{-1}
= (i~r~s)(r~k~s) = (r~k~i).
$$
Observe that if $\sigma, \tau \in H$ then
$\sigma \tau \sigma^{-1} \tau^{-1} \in N$ since $H / N$ is abelian and
so $\sigma \tau \sigma^{-1} \tau^{-1} = 1 mod N$.

Start with any 3-cycle $(r~k~i)$. Find $j$, $s$ disjoint from
$\{r, k, i\}$ and observe that since $H$ contains all 3-cycles,
$(i~j~k) = \sigma$, $(k~r~s) = \tau$ are in $H$. Then
$$
\sigma \tau \sigma^{-1} \tau^{-1} = (r~k~i) \in N.
$$

Suppose $S_n$ is solvable. Then we have a tower
$$
S_n =                G_0
      \triangleright G_1
      \triangleright \cdots
      \triangleright G_m = \{1\}
$$
with $G_i / G_{i+1}$ abelian. Apply the claim inductively to $H =
G_i$, $N = G_{i+1} \triangleleft G_i = H$. For $i = 0$,
$G_0 = S_n$ which contains all 3-cycles. By induction $G_i$ contains
all 3-cycles. But this implies $G_m = \{ 1 \}$ contains all 3-cycles,
which is impossible. Therefore $S_n$ is not solvable.
\end{proof}

\begin{defn}[]
The set of all even permutations in $S_n$ is a subgroup called the
alternating group $A_n = \ker (\xi : S_n \to \{ \pm 1 \})$. In fact
$A_n \triangleleft S_n$ of index 2.
\end{defn}

\begin{theorem}
For all $n$, $n \neq 4$, $A_n$ is a simple group.
\end{theorem}
\begin{proof}
$A_2 = \{1\}$, $A_3 = \{1, \sigma, \sigma^2 \simeq \mathbb{Z}_3$ where
$\sigma$ is the 3-cycle $(1~2~3)$.

First, some properties of $A_n$ for $n \geq 5$.
\begin{enumerate}
  \item{
    $A_n$ is generated by 3-cycles.
    Recall that $A_n$ consists of products of an even number of
    transpositions, i.e. $\sigma = (i~j)(r~s)$. In the case when
    $\{i,j\}$ is disjoint from $\{r,s\}$ then
    $$
    (i~j)(r~s) = (i~j~r)(j~r~s).
    $$
    If they are the same, then $(i~j)(r~s) = 1$.
    Otherwise assume for instance $j = r$. Then
    $$
    (i~j)(j~s) = (i~j~s).
    $$
  }
  \item{
  }
\end{enumerate}
\end{proof}

\begin{theorem}
$\forall n \neq 4$, the alternating group (the subgroup of $S_n$ of even
permutations) is simple.
\end{theorem}
\begin{proof}
We know that $A_n$ is generated by 3-cycles and that all 3-cycles are
conjugate to each other in $A_n$. Let $(ijk)$, $(i^\prime j^\prime
k^\prime)$ be two 3-cycles. Then from the conjugation formula
$$
\exists \gamma \in S_n . \gamma(ijk)\gamma^{-1} = (i^\prime j^\prime k^\prime),
$$
so it is enough to find $\gamma$ with $\gamma(i) = i^\prime$,
$\gamma(j) = j^\prime$, $\gamma(k) = k^\prime$.

If $\varepsilon(\gamma) =  1$, then $\gamma \in A_n$ as desired.
If $\varepsilon(\gamma) = -1$, then since $n \geq 5$ we can choose $r,
s$ disjoint from $i,j,k$ and consider $\tau = (rs)$. Now set
$\gamma^\prime = \gamma \tau$, so
$$
  \gamma^\prime (ijk) (\gamma^\prime)^{-1}
= \gamma \tau (ijk) \tau^{-1} \gamma^{-1}
= \gamma (ijk) \gamma^{-1}.
$$
But
$$
  \varepsilon(\gamma^\prime)
= \varepsilon(\gamma)\varepsilon(\tau) = (-1)(-1) = 1
$$
as desired.

Let $N$ be a normal subgroup of $A_n$ and assume $N \neq \{ 1 \}$. We
will show that $N = A_n$. It is enough to show that $N$ contains one
3-cycle.
Then since $N$ is normal, $N$ contains all 3-cycles since it contains
all its conjugates. Since by (1), 3-cycles generate $A_n$, so we
conclude that $N = A_n$. We want to find a 3-cycle in $N$. Pick
$\sigma \in N$, $\sigma \neq $, with the maximum number of ``fixed
points'' $i \in \{ 1, \dots, n \}$ such that $\sigma(i) = i$. We claim
that $\sigma$ has to be a 3-cycle.

Write $\sigma$ as a product of disjoint cycles. There are two
possibilities: that the decomposition of $\sigma$ is a product of
transpositions, in which case since $\sigma \in A_n$ and so $\sigma$
is even, the decomposition will be
$$
\sigma = \cdots (ij)(rs) \cdots
$$
for $i,j,r,s$ disjoint.  Since $n \geq 5$, there exists a $k$ disjoint
from $\{i,j,r,s\}$. Consider $\tau = (rsk)$ and the commutator
$$
\sigma^\prime = \sigma \tau \sigma^{-1} \tau^{-1}.
$$
Observe
\begin{enumerate}
  \item{
    $\sigma^\prime \in N.$. Indeed
    $$
    \sigma^\prime = \sigma (\tau \sigma^{-1} \tau^{-1}) \in N
    $$
    since $\sigma \in N$ and $\tau \sigma^{-1} \tau^{-1} \in N$
    because $\sigma \in N$ and $N$ is normal.
  }
  \item{
    $\sigma^\prime(i) = i$, $\sigma^\prime(j) = j$. We have
    \begin{align*}
      \sigma^\prime
        &= \sigma \tau \sigma^{-1} \tau^{-1} \\
        &= (\cdots (ij)(rs) \cdots)(rsk)(\cdots (ij)(rs) \cdots)^{-1}(rks).
    \end{align*}
    so $\sigma^\prime(i) = i$, $\sigma^\prime(j) = j$.
  }
  \item{
    $\sigma^\prime$ has at least one more fixed point than $\sigma$,
    since now $i, j$ are fixed points and we may have lost fixed point
    $k$. This contradicts our definition of $\sigma$, since $\sigma$
    is chosen to have the maximum number of fixed points.
  }
\end{enumerate}

The second case is that there is at least one cycle of length $\geq 3$
in the disjoint cycle decomposition of $\sigma$. We claim that
$\sigma$ has to be a 3-cycle. If not, then $\sigma$ must move at
least five elements $i,j,k,r,s$ since a 4-cycle is not even.
Consider $\tau = (krs)$ and $\sigma^\prime = \sigma \tau \sigma^{-1}
\tau^{-1} \in N$. Observe that $\sigma^\prime(i) = i$, so
$\sigma^\prime$ fixes more points than $\sigma$ (contradiction). This
is because
$$
(\cdots(ijk)\cdots)(rsk)(\cdots(ijk)\cdots)^{-1}(rsk)^{-1}
$$
so $j$ is fixed.

\end{proof}
