\begin{defn}[Module]
Let $R$ be a ring with unity (not always commutative).
A (left) \emph{$R$-module} is an abelian group $M$
(with operation written additively) together with
an operation $\cdot : R \times M \to M$ called
scalar multiplication written
$(r, m) \mapsto r \cdot m$ and such that
\begin{enumerate}
  \item{
    $(r + r^\prime) \cdot m =
     r \cdot m + r^\prime \cdot m,
    $
  }
  \item{
    $r \cdot (m + m^\prime) = r m + r m^\prime$,
  }
  \item{
    $r \cdot (r^\prime \cdot m) = (r r^\prime) \cdot m$,
  }
  \item{
    $1 \cdot m = m$
  }
\end{enumerate}
for any $r, r^\prime \in R$, $m, m^\prime \in M$.
We can also regard scalar multiplication as a map
$R \to \mathrm{End}(M)$.
\end{defn}

\begin{xmpl}
  \begin{enumerate}
    \item{
      If $R$ is a field $K$, then an $R$-module is simply a $K$-vector
      space.
    }
    \item{
      If $R = \mathbb{Z}$ then a $\mathbb{Z}$-module is exactly an
      abelian group since
      $$
        n \cdot m
      = (1 + \cdots + 1) \cdot m
      = 1 \cdot m + \cdots + 1 \cdot m
      = m + \cdots + m.
      $$
    }
    \item{
      If $M = R$ itself, then $R \times R \to R$ is ring
      multiplication.
    }
    \item{
      If $M = R \oplus R \oplus \cdots \oplus R = R^n$ then scalar
      multiplication is given by
      $$
      (r, (r_1, \dots, r_n)) \mapsto (rr_1, \dots, rr_n).
      $$
      This is called a \emph{free (left) $R$-module of rank $n$}.
    }
  \end{enumerate}
\end{xmpl}

There is a similar notion of a \emph{right $R$-module} $N$ which is an
abelian group with scalar multiplication $N \times R \to N$ given by
$(n, r) \mapsto n \cdot r$. If $R$ is commutative then to each
left $R$-module $M$ we can associate a right $R$-module $N$
by defining $m \cdot r \triangleq r \cdot m$. Axioms 1, 2, and 4 are
trivially satisfied and
$$
  (m \cdot r) \cdot r^\prime
= m \cdot (r r^\prime)
= (r r^\prime) \cdot m
= (r^\prime r) \cdot m
= r^\prime \cdot (r \cdot m).
$$

\begin{defn}[Bimodule]
An \emph{$R$-bimodule} $M$ is an abelian group with both left and
right scalar $R$-multiplication which make $M$ a left $R$-module and a
right $R$-module.
\end{defn}

\begin{defn}[Module homomorphism]
Let $M$, $N$ be left $R$-modules. A map
$f : M \to N$ is an \emph{$R$-module homomorphism} or
\emph{$R$-linear map} if
$$
f(m + m^\prime) = f(m) + f(m^\prime), \quad
f(r \cdot m) = r \cdot f(m).
$$
\end{defn}

\begin{defn}[Submodule]
A subset $N \subset M$ of a left $R$-module is a \emph{submodule} if
it is a subgroup of $M$ and is closed under scalar multiplication.
\end{defn}

\begin{xmpl}
  \begin{enumerate}
    \item{
      For $M = R$, a submodule of $R$ is a left ideal of $R$.
    }
    \item{
      In general,
      let $I$ be an ideal of $R$ and $M$ a left $R$-module. Then
      $$
      IM \triangleq
      \{
        a_1 m_1 + \cdots + a_r m_r
      \mid
        r \geq  1, m_i \in M, a_i \in I
      \}
      $$
      is a left submodule of $M$. The sum is necessary since a submodule
      must be closed under addition.
    }
    \item{
      If $f : M \to N$ is an $R$-module homomorphism, then $\ker(f)$ and
      $\mathrm{Im}(f)$ are submodules of $M$ and $N$ respectively.
      Let $m \in \ker(f)$. Then $r \cdot m \in \ker(f)$ since
      $f(r m) = r f(m) = r0 = 0$ (since $r0 = r(0 + 0) = r0 + r0$.)
    }
  \end{enumerate}
\end{xmpl}

\begin{defn}[Quotient]
Let $M$ be a left $R$-module and $N \subset M$ a submodule. The
quotient group $M / N$ has a natural left $R$-module structure given
by $R \times M / N \to M / N$ that maps
$$
(r, m + N) \mapsto rm + N.
$$
Indeed, this is well-defined since
$$
m^\prime + N = m + N \implies
m^\prime = m + n, n \in N \implies
r \cdot m^\prime = r m + r n
$$
where $r n \in N$ since $N$ is a submodule, so
$rm^\prime + N = rm + N$.
\end{defn}

\begin{theorem}
  \begin{enumerate}
    \item{
      If $f : M \to N$ is a surjective $R$-module homomorphism, then
      $f$ induces an $R$-module isomorphism
      $$
      M / \ker(f) \simeq N.
      $$
    }
    \item{
      If $N, N^\prime$ are submodules of $M$, then there is a module
      homomorphism
      $$
      N + N^\prime / N^\prime \simeq N / N \cap N^\prime.
      $$
      Here both $N + N^\prime$ and $N \cap N^\prime$ are submodules of $M$.
    }
    \item{
      If $M^{\prime\prime} \subset M^\prime \subset M$ is a chain of
      submodules of $M$, then there is an isomorphism
      $$
             \frac{M}{M^{\prime\prime}}
      \simeq \frac{M / M^{\prime\prime}}{M^\prime / M^{\prime\prime}}.
      $$
    }
  \end{enumerate}
\end{theorem}
