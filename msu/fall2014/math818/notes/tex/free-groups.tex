\section{Free Groups}

\begin{xmpl}
Let $S = \{ a, b \}$. Consider the set $F(S) = F(a, b)$ of all
expressions of the form
$$
x_1 x_2 \cdots x_n
$$
with $x_i = a^{r_i}$ or $x_i = b^{s_i}$ with
$r_i, s_i \in \mathbb{Z} - \{ 0 \}$ and the following property:
if $x_i$ is a power of $a$, then $x_{i+1}$ is a power of $b$ and
vice-versa. This set also includes the ``empty product'' for $n = 0$,
denoted by $1$.

Define a multiplication operation as with the free monoid. This
operation defines a group, with the empty product 1 as the
identity. This is called the free group on $\{a, b\}$. The inverse is
for instance
$$
(a b a^2 a^{-3})^{-1} = b^3 a^{-2} b^{-1} a^{-1}.
$$

This group satisfies the following universal property.
\begin{prop}[Universal Property of Free Groups]
Let $G$ be any group and take $g, h \in G$. Then there is a unique
group homomorphism $\varphi : F(a, b) \to G$ with $\varphi(a) = g$,
$\varphi(b) = h$.
\end{prop}
\begin{proof}
Define for instance
$$
\varphi\left(a b^2 a^{-1} b^3 a\right) = g h^2 g^{-1} h^3 g.
$$
\end{proof}

$G$ is generated by the elements $g, h$ if and only if
$\varphi : F(a,b) \to G$ is surjective.
\end{xmpl}

\begin{defn}[Group Presentation]
Suppose $G$ is generated by $g$ and $h$. Then
$\varphi$ is surjective and $G \simeq F(a,b) / \ker \varphi$.

Suppose we can find elements $r_1, r_2, \dots, r_n \in F(a,b)$ such
that $r_i \in \ker \varphi$ and $\ker \varphi$ is the smallest normal
subgroup of $F(a,b)$ that contains $r_i$. This is something like
saying $\ker \varphi$ is normally generated by $r_i$. Then we say $G$
is given by generators $g$, $h$ and \emph{relations} $r_i$ and we
write
$$
G = \langle g, h \mid r_1, r_2, \dots, r_n \rangle.
$$
This is called a \emph{presentation} of $G$.
\end{defn}

\begin{xmpl}
Consider $S_3$ and let $\sigma = (123)$ and $\tau = (12)$. We know
$$
S_3 = \{ 1, \sigma, \sigma^2, \tau, \sigma \tau, \sigma^2 \tau \}.
$$
Therefore $S_3$ is generated by $\sigma, \tau$. Then we have a map
$\varphi : F(a, b) \to S_3$ which is surjective, where
$\varphi(a) = \sigma$, $\varphi(b) = \tau$. Notice for instance
$$
\sigma^3 = 1, \quad
\tau^2 = 1, \quad
\tau \sigma = \sigma^2 \tau  \iff \tau \sigma \tau^{-1} \sigma^{-2} =
1 (?).
$$
These things particular to the group are the relations.

Then take $N$ to be the smallest normal subgroup of $F(a,b)$ that
contains
$$
r_1 = a^3, r_2 = b^2, r_3 = b a b^{-1} a^{-2}.
$$
We can take $N$ to be the set generated by all conjugates of $r_i$.

Consider $F(a,b) / N = Q$. Since $\ker \varphi \triangleleft G$ and
$r_i \in \ker \varphi$, $N \subset \ker \varphi$, so $\varphi$ factors
as
$$
F(a, b) \to F(a, b) / N \to S_3
$$
where we call the second map $\bar{\varphi}$. Then the proposition
that $N = \ker \varphi$ is equivalent to showing that $\bar{\varphi}$
is an isomorphism.

$\bar{\varphi}$ is surjective since $\varphi$ is. Let
$Q = F(a,b) / N$ and $\bar{a}$, $\bar{b}$ be images of $a, b$ in $Q$
under canonical projection. Consider
$$
\prod_{i=1}^n x_i \in F(a,b)
$$
where either $x_i = a^{m_i}$ or $x_i = b^{n_i}$. The image in $Q$ of
this element is then
$$
\prod_{i=1}^n \bar{x}_i
$$
where $\bar{x}_i = \bar{a}^{m_i}$ or $\bar{b}^{s_i}$. We can assume
$m_i = 0, 1, 2$ and $n_i = 0, 1$ since $\bar{a}^3 = 1$, $\bar{b}^2 =
1$. We also have $\bar{b} \bar{a} = \bar{a}^2 \bar{b}$, which lets us
assume that $\prod \bar{x}_i$ is of the form $\bar{a}^m \bar{b}^n$ for
some $m = 0, 1, 2$ and $n = 0, 1$.

This implies that $Q$ has at most six elements. But we have the
surjective $\bar{\varphi} : Q \to S_3$, and $|S_3| =
6$, which proves that $|Q| = 6$ and $\bar{\varphi}$ is a
bijection. Therefore $\bar{\varphi}$ is an isomorphism and so
$N = \ker \varphi$, so we found a presentation of $S_3$:
$$
S_3 = \langle \sigma, \tau
        \mid \sigma^3, \tau^2, \tau \sigma \tau^{-1} \sigma^{-2}
      \rangle
$$
or
$$
S_3 = \langle \sigma, \tau
        \mid \sigma^3 = 1,
             \tau^2 = 1,
             \tau \sigma = \sigma^2 \tau
      \rangle.
$$
\end{xmpl}

\begin{defn}[Free Group (Generalization)]
Take any set $S$ and the
free group generated by $S$ denoted by
$$
F(S) = \left\{ \prod_{i = 1}^n x_i
         \mid x_i = s_i^{r_i}, s_i \in S,  r_i \in \mathbb{Z}
       \right\}.
$$
This has the universal property that given any group $G$ with a set
map from $S \to G$, there is a unique group homomorphism
$\varphi : F(S) \to G$ such that $\varphi(s) = f(s)$.

We have that $G$ is generated by $\mathrm{Im}(f) \subset G$
iff. $\varphi$ is surjective. Assuming $\varphi$ is surjective and
supposing there exists an $R \subset F(S)$ such that $R \subset
\ker(\varphi)$ and $\ker \varphi$ is the smallest normal subgroup of
$F(S)$ containing $R$, we say that $G$ is generated by $S$ with
relations $R$. Then we write $G = \langle S \mid R \rangle$ for the
presentation of $G$.
\end{defn}

\begin{xmpl}[Presentation of $S_n$]
Set $\sigma_i = (i \to i + 1)$, $i = 1, \dots, n - 1$. Check that
$\sigma_i^2 = 1$, $\sigma_j \sigma_i = \sigma_i \sigma_j$ if
$j \neq i \pm 1$, and
$\sigma_i \sigma_{i+1} \sigma{i} = \sigma_{i+1} \sigma_i
\sigma_{i+1}$.

\begin{theorem}
$$S_n = \langle \sigma_1, \dots, \sigma_{n-1}
         \mid \sigma_i^2 = 1, \forall i,
              \sigma_j \sigma_i = \sigma_i \sigma_j, j \neq \pm 1,
              \sigma_i \sigma_{i+1} \sigma{i} = \sigma_{i+1} \sigma_i
              \sigma_{i+1}
       \rangle.
$$
\end{theorem}
This kind of structure allows us to manipulate the restrictions on the
groups. For instance by dropping the restriction $\sigma_i^2 = 1$ we
get the \emph{braid group} of entanglements of paths between $n$ objects.
\end{xmpl}
