\begin{defn}[Ring]
A \emph{ring} $A$ is a set with two binary operations $+$ (addition) and $\cdot$
(multiplication) such that
\begin{itemize}
  \item{
    $(A, +)$ is an abelian group.
  }
  \item{
    $(A, \cdot)$ is a monoid (associative).
  }
  \item{
    The distributive law
    $$
    x \cdot (y + z) = x \cdot y + x \cdot z, \quad
    (y + z) \cdot x = y \cdot x + z \cdot x.
    $$
  }
\end{itemize}

The ring is \emph{commutative} when $x \cdot y = y \cdot x$, $\forall x, y
\in A$.

The \emph{zero ring} is a one-element set $\{0\}$ with trivial
operations $0 + 0 = 0 = 0 \cdot 0$.

\end{defn}

\begin{xmpl}
\begin{enumerate}
  \item{
    $A = \mathbb{Z}$.
  }
  \item{
    $A = \mathbb{Z}_n$.
  }
  \item{
    $A = \mathbb{Z}[x]$, the polynomials in $x$ with coefficients in
    $\mathbb{Z}$, i.e.
    $$
    \{ f(x) = a_n x^n + \cdots + a_1 x + a_0 \mid a_i \in \mathbb{Z} \}
    $$
    with the usual addition and multiplication of polynomials.
  }
  \item{
    An example of a non-commutative ring: Say $R$ is a ring, $n \geq
    1$. The set of $n \times n$ square matrices
    $$
    M_{n \times n}(R)
      = \left\{ ( a_{ij} ) \mid a_{ij} \in R \right\}
    $$
    is a ring with addition
    $(a_{ij}) + (b_{ij}) = (a_{ij} + b_{ij})$
    and multiplication
    $(a_{ij}) \cdot (b_{ij}) = \left(\sum_{k=1}^n a_{ik}
      b_{kj}\right)$. This is not commutative for $n > 1$ and $R \neq
    \{ 0 \}$.
  }
\end{enumerate}
\end{xmpl}

\begin{obsv}
Let $A$ be a ring.
\begin{enumerate}
\item{
  $$
  0 \cdot x = x \cdot 0 = 0, \forall x \in A.
  $$
  \begin{proof}
  \begin{align*}
             & 0 + 0 = 0 \implies x \cdot (0 + 0) = x \cdot 0 \\
    \implies & x \cdot 0 + x \cdot 0 = x \cdot 0 \\
    \implies & x \cdot 0 = 0.
  \end{align*}
  \end{proof}
}
\item{
  $A = \{ 0 \}$ if and only if $0 = 1$. Suppose $0 = 1$. Then
  $\forall x$, $0 \cdot x = 0$, but also $0 \cdot x = x$ since $0 =
  1$.
}
\item{
  $$
  (-x) \cdot y = -(x \cdot y) = x \cdot (-y)
  $$
  since
  $$
  (x + (-x)) \cdot y = 0 \cdot y = 0
  $$
  so
  $$
  x \cdot y + (-x) \cdot y = 0
  $$
  and then
  $$
  (-x) \cdot y = -(xy).
  $$
  Similarly $(-x)(-y) = xy$.
}
\end{enumerate}
\end{obsv}

\begin{xmpl}
Let $G$ be a group. The \emph{integral group ring of $G$} is
$$
A = \mathbb{Z} [G]
  = \left\{ \left. \sum_{g \in G} a_g [g] \right|
            a_g \in \mathbb{Z}, \text{ all but finitely many zero}
    \right\}
$$
Elements written in brackets are inside the group $G$.
We have addition
$$
\sum_{g} a_g [g] + \sum_{g} b_g [g] = \sum_{g} (a_g + b_g)[g]
$$
and multiplication extends
$$
(1 \cdot [g]) \cdot (1 \cdot [h]) = 1 \cdot [g \cdot h].
$$
so that
$$
\left(\sum_g a_g [g]\right)
\cdot
\left(\sum_h a_h [h]\right)
=
\sum_{f \in g}
\left(\sum_{gh = f} a_g b_h \right)[f].
$$
This will fail to be a commutative ring for all nonabelian groups.
\end{xmpl}

\begin{defn}
Let $A$ be a ring. The \emph{units of $A$}, written $\mathcal{U}(A)$
or $A^\ast$, is the set
$$
\left\{ a \in A \mid \exists b, c \in A, a b = 1, c a = 1 \right\}.
$$

\begin{remark}
If $a b = 1$, $c a = 1$, then actually $b = c$ since
$$
(c a)b = c (a b) = c 1 = c.
$$
In general, it is possible that there exists a $c$ such that $c a = 1$
but there does not exist a $b$ with $a b = 1$.
\end{remark}
\end{defn}

\begin{prop}
For any ring $A$, the units $A^\ast$ form a group under multiplication.
\end{prop}
\begin{proof}
$1 \in A^\ast$ and $a \in A^\ast$ has an inverse $a^{-1}$ by
definition of $A^\ast$.
\end{proof}

Let $G$ be a finite group. Consider $A = \mathbb{Z}[G]$. Observe
that $[g]$ is a unit of $A$, since $[g] [g^{-1}] = [1]$, the unit
element in the ring. Certainly also $-[g] = (-1)[g]$ is also a unit
for all $g \in G$. Do we always have (for all $G$) that
$$
\mathbb{Z}[G]^\ast = \pm G = \{ \pm [g] \mid g \in G \}?
$$

\begin{defn}[Division ring, Field]
Let $A$ be a ring.
\begin{enumerate}
  \item{
    $A$ is a \emph{division ring} if $0 \neq 1$ and
    $A^\ast = A - \{ 0 \}$ so all non-zero elements are left and right invertible.
  }
  \item{
    $A$ is a \emph{field} if it is a commutative division ring.
  }
\end{enumerate}
\end{defn}

\begin{xmpl}
The Hamilton quaternions form a division ring that is not commutative.
$$
H = \mathbb{R} 1 + \mathbb{R} i + \mathbb{R} j + \mathbb{R} k
$$
with elementwise addition and multiplication determined by
$$
ij = k, \quad
jk = i, \quad
ki = j, \quad
ji = -k, \quad
kj = -i, \quad
ik = -j, \quad
i^2 = j^2 = k^2 = -1.
$$
We have
$$
(a + bi + cj + dk)^{-1} =
\frac{a - bi - cj - dk}{a^2 + b^2 + c^2 + d^2}.
$$
\end{xmpl}
