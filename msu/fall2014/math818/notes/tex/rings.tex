\begin{defn}[Ring]
A \emph{ring} $A$ is a set with two binary operations $+$ (addition) and $\cdot$
(multiplication) such that
\begin{itemize}
  \item{
    $(A, +)$ is an abelian group.
  }
  \item{
    $(A, \cdot)$ is a monoid (associative).
  }
  \item{
    The distributive law
    $$
    x \cdot (y + z) = x \cdot y + x \cdot z, \quad
    (y + z) \cdot x = y \cdot x + z \cdot x.
    $$
  }
\end{itemize}

The ring is \emph{commutative} when $x \cdot y = y \cdot x$, $\forall x, y
\in A$.

The \emph{zero ring} is a one-element set $\{0\}$ with trivial
operations $0 + 0 = 0 = 0 \cdot 0$.

\end{defn}

\begin{xmpl}
\begin{enumerate}
  \item{
    $A = \mathbb{Z}$.
  }
  \item{
    $A = \mathbb{Z}_n$.
  }
  \item{
    $A = \mathbb{Z}[x]$, the polynomials in $x$ with coefficients in
    $\mathbb{Z}$, i.e.
    $$
    \{ f(x) = a_n x^n + \cdots + a_1 x + a_0 \mid a_i \in \mathbb{Z} \}
    $$
    with the usual addition and multiplication of polynomials.
  }
  \item{
    An example of a non-commutative ring: Say $R$ is a ring, $n \geq
    1$. The set of $n \times n$ square matrices
    $$
    M_{n \times n}(R)
      = \left\{ ( a_{ij} ) \mid a_{ij} \in R \right\}
    $$
    is a ring with addition
    $(a_{ij}) + (b_{ij}) = (a_{ij} + b_{ij})$
    and multiplication
    $(a_{ij}) \cdot (b_{ij}) = \left(\sum_{k=1}^n a_{ik}
      b_{kj}\right)$. This is not commutative for $n > 1$ and $R \neq
    \{ 0 \}$.
  }
\end{enumerate}
\end{xmpl}

\begin{obsv}
Let $A$ be a ring.
\begin{enumerate}
\item{
  $$
  0 \cdot x = x \cdot 0 = 0, \forall x \in A.
  $$
  \begin{proof}
  \begin{align*}
             & 0 + 0 = 0 \implies x \cdot (0 + 0) = x \cdot 0 \\
    \implies & x \cdot 0 + x \cdot 0 = x \cdot 0 \\
    \implies & x \cdot 0 = 0.
  \end{align*}
  \end{proof}
}
\item{
  $A = \{ 0 \}$ if and only if $0 = 1$. Suppose $0 = 1$. Then
  $\forall x$, $0 \cdot x = 0$, but also $0 \cdot x = x$ since $0 =
  1$.
}
\item{
  $$
  (-x) \cdot y = -(x \cdot y) = x \cdot (-y)
  $$
  since
  $$
  (x + (-x)) \cdot y = 0 \cdot y = 0
  $$
  so
  $$
  x \cdot y + (-x) \cdot y = 0
  $$
  and then
  $$
  (-x) \cdot y = -(xy).
  $$
  Similarly $(-x)(-y) = xy$.
}
\end{enumerate}
\end{obsv}

\begin{xmpl}
Let $G$ be a group. The \emph{integral group ring of $G$} is
$$
A = \mathbb{Z} [G]
  = \left\{ \left. \sum_{g \in G} a_g [g] \right|
            a_g \in \mathbb{Z}, \text{ all but finitely many zero}
    \right\}
$$
Elements written in brackets are inside the group $G$.
We have addition
$$
\sum_{g} a_g [g] + \sum_{g} b_g [g] = \sum_{g} (a_g + b_g)[g]
$$
and multiplication extends
$$
(1 \cdot [g]) \cdot (1 \cdot [h]) = 1 \cdot [g \cdot h].
$$
so that
$$
\left(\sum_g a_g [g]\right)
\cdot
\left(\sum_h a_h [h]\right)
=
\sum_{f \in g}
\left(\sum_{gh = f} a_g b_h \right)[f].
$$
This will fail to be a commutative ring for all nonabelian groups.
\end{xmpl}

\begin{defn}
Let $A$ be a ring. The \emph{units of $A$}, written $\mathcal{U}(A)$
or $A^\ast$, is the set
$$
\left\{ a \in A \mid \exists b, c \in A, a b = 1, c a = 1 \right\}.
$$

\begin{remark}
If $a b = 1$, $c a = 1$, then actually $b = c$ since
$$
(c a)b = c (a b) = c 1 = c.
$$
In general, it is possible that there exists a $c$ such that $c a = 1$
but there does not exist a $b$ with $a b = 1$.
\end{remark}
\end{defn}

\begin{prop}
For any ring $A$, the units $A^\ast$ form a group under multiplication.
\end{prop}
\begin{proof}
$1 \in A^\ast$ and $a \in A^\ast$ has an inverse $a^{-1}$ by
definition of $A^\ast$.
\end{proof}

Let $G$ be a finite group. Consider $A = \mathbb{Z}[G]$. Observe
that $[g]$ is a unit of $A$, since $[g] [g^{-1}] = [1]$, the unit
element in the ring. Certainly also $-[g] = (-1)[g]$ is also a unit
for all $g \in G$. Do we always have (for all $G$) that
$$
\mathbb{Z}[G]^\ast = \pm G = \{ \pm [g] \mid g \in G \}?
$$

\begin{defn}[Division ring, Field]
Let $A$ be a ring.
\begin{enumerate}
  \item{
    $A$ is a \emph{division ring} if $0 \neq 1$ and
    $A^\ast = A - \{ 0 \}$ so all non-zero elements are left and right invertible.
  }
  \item{
    $A$ is a \emph{field} if it is a commutative division ring.
  }
\end{enumerate}
\end{defn}

\begin{xmpl}
The Hamilton quaternions form a division ring that is not commutative.
$$
H = \mathbb{R} 1 + \mathbb{R} i + \mathbb{R} j + \mathbb{R} k
$$
with elementwise addition and multiplication determined by
$$
ij = k, \quad
jk = i, \quad
ki = j, \quad
ji = -k, \quad
kj = -i, \quad
ik = -j, \quad
i^2 = j^2 = k^2 = -1.
$$
We have
$$
(a + bi + cj + dk)^{-1} =
\frac{a - bi - cj - dk}{a^2 + b^2 + c^2 + d^2}.
$$
\end{xmpl}

\begin{defn}[Ring Homomorphism]
Let $A$, $A^\prime$ be two rings. A map $f: A \to A^\prime$ is a
\emph{ring homomorphism} when
$$
f(a + b) = f(a) + f(b), \quad
f(a b) = f(a) f(b), \quad
f(1) = 1.
$$
\end{defn}

Therefore we have a \emph{category of rings} with objects unital rings and morphisms
ring homomorphisms.

\begin{defn}[Ideal]
Let $A$ be a ring. A subset $I \subset A$ is called a
\emph{left ideal of $A$} if it is a subggroup for $+$ such that
$\forall r \in A$, $\forall x \in I$, $a x \in I$. We can also write
this by $AI \subset I$. Similarly $I$ is a \emph{right ideal}
when $AI \subset I$, and a \emph{two-sided ideal} when both a
left and a right ideal.
\end{defn}

\begin{xmpl}
Let $A$ be a ring and $x \in A$. Then
$$
I = A x = \{ ax \mid a \in A \}
$$
is the left ideal generated by $x$. Ideals generated this way (by one
element) are called \emph{principal ideals}.

More generally, if $x_1, x_2, \dots, x_n \in A$, we can consider
$$
  I
= Ax_1 + Ax_2 + \cdots + Ax_n
= \{ a_1 x_1 + \cdots + a_n x_1 \mid a_i \in A \},
$$
the left ideal generated by $x_i$.

If $A$ is commutative, then left, right, and two-sided ideals are the
same. Then we denote by $(x_1, \cdots, x_n)$ the ideal generated by
$x_i$.
\end{xmpl}

\begin{prop}
Let $I$, $I^\prime$ be ideals (the statement applies to left,
right, two-sided ideals separately). Then
$$
I + I^\prime = \{
                 x + x^\prime
              \mid
                 x \in I, x^\prime \in I^\prime
              \}, \quad
I \cap I^\prime
$$
is also an (left, right, two-sided resp.) ideal.
\end{prop}

\begin{defn}[Quotient Ring]
Let $A$ be a ring and $I$ a two-sided ideal. The quotient ring $A / I$
is the quotient abelian group under $+$ with multiplication given by
$$
(x + I) \cdot (y + I) = x \cdot y + I.
$$
This is well-defined: Suppose $x + I = x^\prime + I$, $y + I =
y^\prime + I$. Then $x - x^\prime \in I$ and $y - y^\prime \in I$, and
\begin{align*}
   xy - x^\prime y^\prime
&= xy - x^\prime y + x^\prime y - x^\prime y^\prime
 = (x - x^\prime)y + x^\prime(y - y^\prime) \\
&\in I y + x^\prime I \subset I A + A I \subset I
\end{align*}
since $I$ is a two-sided ideal.
\end{defn}

\begin{prop}
There is the \emph{canonical ring homomorphism}
$\pi : A \to A / I$ given by $\pi(x) = x + I$.
In general, suppose $f : A \to A^\prime$ is a ring homomorphism. Then
$\ker(f)$ is a two-sided ideal of $A$ and $f$ factors as a composition
$$
A \to A / \ker(f) \to A^\prime.
$$
\end{prop}
\begin{proof}
Let $a \in A$, $x \in \ker(f)$. Then $f(x) = 0$, so
$f(ax) = f(a)f(x) = 0 = f(x)f(a) = f(xa)$, so the kernel is a
two-sided ideal.
\end{proof}

\section*{Commutative Rings}

\begin{defn}[Prime Ideals]
An ideal $I$ of $A$ is called \emph{prime} when $I \neq A$ and
$xy \in I$ implies that either $x \in I$ or $y \in I$, or equivalently
when $x \notin I$, $y \notin I \implies x y \notin I$.
\end{defn}

\begin{xmpl}
Let $A = \mathbb{Z}$. Then
$$
I = \mathbb{Z} p = (p) = \{ n p \mid n \in \mathbb{Z} \}
$$
is a prime ideal. If $xy \in (p)$ then $p$ divides $xy$, so
$p$ divides $x$ or $p$ divides $y$ and thus $x \in (p)$ or $y \in
(p)$.

Note that $I = \{ 0 \}$ is a prime ideal in $\mathbb{Z}$.
\end{xmpl}

\begin{defn}[Maximal Ideal]
An ideal $I$ of $A$ is called \emph{maximal} when
the only ideals that contain $I$ are $I$ and $A$, and
$I \neq A$. In other words, whenever $I \subset I^\prime \subset A$,
either $I^\prime = I$ or $I^\prime = A$.
\end{defn}

\begin{prop}
If an ideal is maximal, then it is also prime.
\end{prop}
\begin{proof}
Suppose $M$ is a
maximal ideal of $A$. We have $M \neq A$. Suppose $x, y \in A$ with
$xy \in M$. Assume $x \notin M$. Consider $J = Ax + M$. Then $J$ is an
ideal and $M \subset J$, but $M \neq J$ since $x \in J$ but $x \notin
M$. Therefore $J = A$ since $M$ is maximal. In particular, $1 \in J$,
so $1 = ax + b$ for some $a \in A$, $b \in M$. Then $y = axy +
by$. But $a x y \in M$ since $xy \in M$ by assumption, and $b \in M$
so $by \in M$. Therefore $y \in M$.
\end{proof}

\begin{defn}[Integral Domain]
Let $A$ be a commutative ring. $A$ is called an \emph{integral domain}
or \emph{domain} if $A \neq \{ 0 \}$ and $x y = 0$ implies $x = 0$ or
$y = 0$.
\end{defn}

\begin{prop}
Let $A$ be a ring, $I \subset A$ be an ideal. Then $I$ is a prime
ideal if and only if $A / I$ is a domain.
\end{prop}
\begin{proof}
Take $(x + I)(y + I) = 0 + I$. Then $xy + I = 0 + I$, so $xy \in I$.
Assuming $I$ is prime, $x \in I$ or $y \in I$, so $x + I = 0 + I$ or
$y + I = 0 + I$. Observe also $I \neq A$ is equivalent to
$A / I \neq \{ 0 \}$.
\end{proof}
