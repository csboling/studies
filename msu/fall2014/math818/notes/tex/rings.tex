\begin{defn}[Ring]
A \emph{ring} $A$ is a set with two binary operations $+$ (addition) and $\cdot$
(multiplication) such that
\begin{itemize}
  \item{
    $(A, +)$ is an abelian group.
  }
  \item{
    $(A, \cdot)$ is a monoid (associative).
  }
  \item{
    The distributive law
    $$
    x \cdot (y + z) = x \cdot y + x \cdot z, \quad
    (y + z) \cdot x = y \cdot x + z \cdot x.
    $$
  }
\end{itemize}

The ring is \emph{commutative} when $x \cdot y = y \cdot x$, $\forall x, y
\in A$.

The \emph{zero ring} is a one-element set $\{0\}$ with trivial
operations $0 + 0 = 0 = 0 \cdot 0$.

\end{defn}

\begin{xmpl}
\begin{enumerate}
  \item{
    $A = \mathbb{Z}$.
  }
  \item{
    $A = \mathbb{Z}_n$.
  }
  \item{
    $A = \mathbb{Z}[x]$, the polynomials in $x$ with coefficients in
    $\mathbb{Z}$, i.e.
    $$
    \{ f(x) = a_n x^n + \cdots + a_1 x + a_0 \mid a_i \in \mathbb{Z} \}
    $$
    with the usual addition and multiplication of polynomials.
  }
  \item{
    An example of a non-commutative ring: Say $R$ is a ring, $n \geq
    1$. The set of $n \times n$ square matrices
    $$
    M_{n \times n}(R)
      = \left\{ ( a_{ij} ) \mid a_{ij} \in R \right\}
    $$
    is a ring with addition
    $(a_{ij}) + (b_{ij}) = (a_{ij} + b_{ij})$
    and multiplication
    $(a_{ij}) \cdot (b_{ij}) = \left(\sum_{k=1}^n a_{ik}
      b_{kj}\right)$. This is not commutative for $n > 1$ and $R \neq
    \{ 0 \}$.
  }
\end{enumerate}
\end{xmpl}

\begin{obsv}
Let $A$ be a ring.
\begin{enumerate}
\item{
  $$
  0 \cdot x = x \cdot 0 = 0, \forall x \in A.
  $$
  \begin{proof}
  \begin{align*}
             & 0 + 0 = 0 \implies x \cdot (0 + 0) = x \cdot 0 \\
    \implies & x \cdot 0 + x \cdot 0 = x \cdot 0 \\
    \implies & x \cdot 0 = 0.
  \end{align*}
  \end{proof}
}
\item{
  $A = \{ 0 \}$ if and only if $0 = 1$. Suppose $0 = 1$. Then
  $\forall x$, $0 \cdot x = 0$, but also $0 \cdot x = x$ since $0 =
  1$.
}
\item{
  $$
  (-x) \cdot y = -(x \cdot y) = x \cdot (-y)
  $$
  since
  $$
  (x + (-x)) \cdot y = 0 \cdot y = 0
  $$
  so
  $$
  x \cdot y + (-x) \cdot y = 0
  $$
  and then
  $$
  (-x) \cdot y = -(xy).
  $$
  Similarly $(-x)(-y) = xy$.
}
\end{enumerate}
\end{obsv}

\begin{xmpl}
Let $G$ be a group. The \emph{integral group ring of $G$} is
$$
A = \mathbb{Z} [G]
  = \left\{ \left. \sum_{g \in G} a_g [g] \right|
            a_g \in \mathbb{Z}, \text{ all but finitely many zero}
    \right\}
$$
Elements written in brackets are inside the group $G$.
We have addition
$$
\sum_{g} a_g [g] + \sum_{g} b_g [g] = \sum_{g} (a_g + b_g)[g]
$$
and multiplication extends
$$
(1 \cdot [g]) \cdot (1 \cdot [h]) = 1 \cdot [g \cdot h].
$$
so that
$$
\left(\sum_g a_g [g]\right)
\cdot
\left(\sum_h a_h [h]\right)
=
\sum_{f \in G}
\left(\sum_{gh = f} a_g b_h \right)[f].
$$
This will fail to be a commutative ring for all nonabelian groups.
\end{xmpl}

\begin{defn}
Let $A$ be a ring. The \emph{units of $A$}, written $\mathcal{U}(A)$
or $A^\ast$, is the set
$$
\left\{ a \in A \mid \exists b, c \in A, a b = 1, c a = 1 \right\}.
$$

\begin{remark}
If $a b = 1$, $c a = 1$, then actually $b = c$ since
$$
(c a)b = c (a b) = c 1 = c.
$$
In general, it is possible that there exists a $c$ such that $c a = 1$
but there does not exist a $b$ with $a b = 1$.
\end{remark}
\end{defn}

\begin{prop}
For any ring $A$, the units $A^\ast$ form a group under multiplication.
\end{prop}
\begin{proof}
$1 \in A^\ast$ and $a \in A^\ast$ has an inverse $a^{-1}$ by
definition of $A^\ast$.
\end{proof}

Let $G$ be a finite group. Consider the free abelian group
$A = \mathbb{Z}[G]$. Observe that $[g]$ is a unit of $A$, since
$[g] [g^{-1}] = [1]$, the unit element in the ring. Certainly also
$-[g] = (-1)[g]$ is also a unit for all $g \in G$. Do we always
have (for all $G$) that
$$
\mathbb{Z}[G]^\ast = \pm G = \{ \pm [g] \mid g \in G \}?
$$
This is an open problem.

\begin{defn}[Division ring, Field]
Let $A$ be a ring.
\begin{enumerate}
  \item{
    $A$ is a \emph{division ring} if $0 \neq 1$ and
    $A^\ast = A - \{ 0 \}$ so all non-zero elements are left and right
    invertible under multiplication.
  }
  \item{
    $A$ is a \emph{field} if it is a commutative division ring.
  }
\end{enumerate}
\end{defn}

\begin{xmpl}
The Hamilton quaternions form a division ring that is not commutative.
$$
H = \mathbb{R} 1 + \mathbb{R} i + \mathbb{R} j + \mathbb{R} k
$$
with elementwise addition and multiplication determined by
$$
ij = k, \quad
jk = i, \quad
ki = j, \quad
ji = -k, \quad
kj = -i, \quad
ik = -j, \quad
i^2 = j^2 = k^2 = -1.
$$
We have
$$
(a + bi + cj + dk)^{-1} =
\frac{a - bi - cj - dk}{a^2 + b^2 + c^2 + d^2}.
$$
\end{xmpl}

\begin{defn}[Ring Homomorphism]
Let $A$, $A^\prime$ be two rings. A map $f: A \to A^\prime$ is a
\emph{ring homomorphism} when
$$
f(a + b) = f(a) + f(b), \quad
f(a b) = f(a) f(b), \quad
f(1) = 1.
$$
\end{defn}

Therefore we have a \emph{category of rings} with objects unital rings and morphisms
ring homomorphisms.

\begin{defn}[Ideal]
Let $A$ be a ring. A subset $I \subset A$ is called a
\emph{left ideal of $A$} if it is a subggroup for $+$ such that
$\forall r \in A$, $\forall x \in I$, $a x \in I$. We can also write
this by $AI \subset I$. Similarly $I$ is a \emph{right ideal}
when $AI \subset I$, and a \emph{two-sided ideal} when both a
left and a right ideal.
\end{defn}

\begin{xmpl}
Let $A$ be a ring and $x \in A$. Then
$$
I = A x = \{ ax \mid a \in A \}
$$
is the left ideal generated by $x$. Ideals generated this way (by one
element) are called \emph{principal ideals}.

More generally, if $x_1, x_2, \dots, x_n \in A$, we can consider
$$
  I
= Ax_1 + Ax_2 + \cdots + Ax_n
= \{ a_1 x_1 + \cdots + a_n x_1 \mid a_i \in A \},
$$
the left ideal generated by $x_i$.

If $A$ is commutative, then left, right, and two-sided ideals are the
same. Then we denote by $(x_1, \cdots, x_n)$ the ideal generated by
$x_i$.
\end{xmpl}

\begin{prop}
Let $I$, $I^\prime$ be ideals (the statement applies to left,
right, two-sided ideals separately). Then
$$
I + I^\prime = \{
                 x + x^\prime
              \mid
                 x \in I, x^\prime \in I^\prime
              \}, \quad
I \cap I^\prime
$$
is also an (left, right, two-sided resp.) ideal.
\end{prop}

\begin{defn}[Quotient Ring]
Let $A$ be a ring and $I$ a two-sided ideal. The quotient ring $A / I$
is the quotient abelian group under $+$ with multiplication given by
$$
(x + I) \cdot (y + I) = x \cdot y + I.
$$
This is well-defined: Suppose $x + I = x^\prime + I$, $y + I =
y^\prime + I$. Then $x - x^\prime \in I$ and $y - y^\prime \in I$, and
\begin{align*}
   xy - x^\prime y^\prime
&= xy - x^\prime y + x^\prime y - x^\prime y^\prime
 = (x - x^\prime)y + x^\prime(y - y^\prime) \\
&\in I y + x^\prime I \subset I A + A I \subset I
\end{align*}
since $I$ is a two-sided ideal.
\end{defn}

\begin{prop}
There is the \emph{canonical ring homomorphism}
$\pi : A \to A / I$ given by $\pi(x) = x + I$.
In general, suppose $f : A \to A^\prime$ is a ring homomorphism. Then
$\ker(f)$ is a two-sided ideal of $A$ and $f$ factors as a composition
$$
A \to A / \ker(f) \to A^\prime.
$$
\end{prop}
\begin{proof}
Let $a \in A$, $x \in \ker(f)$. Then $f(x) = 0$, so
$f(ax) = f(a)f(x) = 0 = f(x)f(a) = f(xa)$, so the kernel is a
two-sided ideal.
\end{proof}

\section*{Commutative Rings}

\begin{defn}[Prime Ideals]
An ideal $I$ of $A$ is called \emph{prime} when $I \neq A$ and
$xy \in I$ implies that either $x \in I$ or $y \in I$, or equivalently
when $x \notin I$, $y \notin I \implies x y \notin I$.
\end{defn}

\begin{xmpl}
Let $A = \mathbb{Z}$ and $p$ be prime. Then
$$
I = \mathbb{Z} p = (p) = \{ n p \mid n \in \mathbb{Z} \}
$$
is a prime ideal. If $xy \in (p)$ then $p$ divides $xy$, so
$p$ divides $x$ or $p$ divides $y$ and thus $x \in (p)$ or $y \in
(p)$.

Note that $I = \{ 0 \}$ is a prime ideal in $\mathbb{Z}$.
\end{xmpl}

\begin{defn}[Maximal Ideal]
An ideal $I$ of $A$ is called \emph{maximal} when
the only ideals that contain $I$ are $I$ and $A$, and
$I \neq A$. In other words, whenever $I \subset I^\prime \subset A$,
either $I^\prime = I$ or $I^\prime = A$.
\end{defn}

\begin{prop}
If an ideal is maximal, then it is also prime.
\end{prop}
\begin{proof}
Suppose $M$ is a
maximal ideal of $A$. We have $M \neq A$. Suppose $x, y \in A$ with
$xy \in M$. Assume $x \notin M$. Consider $J = Ax + M$. Then $J$ is an
ideal and $M \subset J$, but $M \neq J$ since $x \in J$ but $x \notin
M$. Therefore $J = A$ since $M$ is maximal. In particular, $1 \in J$,
so $1 = ax + b$ for some $a \in A$, $b \in M$. Then $y = axy +
by$. But $a x y \in M$ since $xy \in M$ by assumption, and $b \in M$
so $by \in M$. Therefore $y \in M$.
\end{proof}

\begin{defn}[Integral Domain]
Let $A$ be a commutative ring. $A$ is called an \emph{integral domain}
or \emph{domain} if $A \neq \{ 0 \}$ and $x y = 0$ implies $x = 0$ or
$y = 0$.
\end{defn}

\begin{prop}
Let $A$ be a ring, $I \subset A$ be an ideal. Then $I$ is a prime
ideal if and only if $A / I$ is a domain.
\end{prop}
\begin{proof}
Take $(x + I)(y + I) = 0 + I$. Then $xy + I = 0 + I$, so $xy \in I$.
Assuming $I$ is prime, $x \in I$ or $y \in I$, so $x + I = 0 + I$ or
$y + I = 0 + I$. Observe also $I \neq A$ is equivalent to
$A / I \neq \{ 0 \}$.
\end{proof}

\begin{prop}
$M$ is a maximal ideal of $A$ if and only if $A / M$ is a field.
\end{prop}

\begin{proof}
\begin{itemize}
  \item[$\implies$]
  {
    Let $M$ be a maximal ideal of $A$. Take $x + M \neq 0 + M$,
    i.e. $x \notin M$. Consider an ideal $I = A x + M$ of $A$. Observe
    that $M$ is a proper subset of $I$ since $x \notin M$. Then
    $I = Ax + M = A$. Then $1 = a x + m$ for some $m \in M$, and so
    $(a + M)(x + M) = 1 + M$, so $a + M$ is the inverse of $x + M$ in
    $A / M$. Therefore $A / M$ is a field.
  }
  \item[$\impliedby$]
  {
    Let $A / M$ be a field. Then $A / M$ is not the zero ring, so
    $M \neq A$. Take an ideal $I$ such that $M \subset I \subset
    A$. Consider $I / M \subset A / M$. Then $I / M$ is an ideal in
    the field. But the only ideals that can exist in this field
    are $I / M = \{ 0 \}$ or $I / M = A / M$. (Take $a \in A$,
    $a \neq 0$. Then $(a) = \{ ab \mid b \in A \}$ contains $a a^{-1}
    = 1$. But then $(a) = (1) = A$.) This completes the proof.
  }
\end{itemize}
\end{proof}

\begin{theorem}[?]
Let $A$ be a commutative ring. Every proper ideal in $A$ is contained
in a maximal ideal, where ``proper'' means the ideal is not the whole
ring $A$. We will see this is analogous to the axiom of choice.
\end{theorem}

\subsection{Zorn's Lemma}
First, let $S$ be a set. A partial order on $S$ is a relation $\leq$
with the properties:
\begin{enumerate}
\item{
  $x \leq x$, $\forall x \in S$.
}
\item{
  $x \leq y$ and $y \leq z$ implies $x \leq z$.
}
\item{
  $x \leq y$ and $y \leq x$ implies $x = y$.
}
\end{enumerate}

If for all $x, y \in S$, we have either $x \leq y$ or $y \leq x$, then
we say $\leq$ is a total order.

Suppose $(S, \leq)$ is a poset (``partially ordered set''). A
\emph{maximal element} in $(S, \leq)$ is any $m \in S$ such that
$\forall x \in S$, either $x \leq m$ or $x$ and $m$ are ``unrelated'',
i.e. there exists no $y \in S$ with $x \leq y$ and $y \neq m$.

Suppose $T$ is a subset of $S$. An \emph{upper bound for $T$ in $S$}
is an element $u \in S$ such that $\forall x \in T$, $x \leq u$.

\begin{defn}[Inductively ordered poset]
A poset $(S, \leq)$ is called \emph{inductively ordered} if
every totally ordered subset $T$ of $S$ has an upper bound in $S$.
\end{defn}

\begin{lemma}[Zorn's Lemma]
Every nonempty inductively ordered poset has at least one maximal
element. This is equivalent to the axiom of choice.
\end{lemma}

\begin{theorem}[?, redux]
Every proper ideal in $A$ is contained in a maximal ideal.
\end{theorem}

\begin{proof}
Let $S$ be the set of all proper ideals of $A$ that contain a given
$I_0$.
Then $S \neq \varnothing$ since it contains the zero ideal. $S$ has a
partial order given by $I \leq J$ when $I \subset J$. We check that
$S$ is inductively ordered.

Let $\{I_t\}$, $t \in T$ be a totally ordered subset of $S$. Then
$$
U = \bigcup_{t \in T} I_t
$$
is an ideal and $U \neq A$, so $U \in S$ and naturally $I_t \subset U$
for all $t$, so $U$ is an upper bound for $\{I_t\}$ in $S$. $U$ is an
ideal because for $x, y \in U$, there is a $t, t^\prime \in S$ such
that $x \in I_t$, $y \in I_{t^\prime}$. Then since $T$ is totally
ordered, either $I_t \subset I_{t^\prime}$ or $I_{t^\prime} \subset
I_t$. If$I_t \subset I_{t^\prime}$, $x \pm y \in I_{t^\prime} \subset U$, and
similarly $I_t \subset U$.

Why is $U \neq A$? This is equivalent to $1 \notin U$. But $1 \notin
I_t$, $\forall t \in T$, so $1 \notin \bigcup_{t \in T} I_t = U$.

Now applying Zorn's lemma gives that there exists a maximal element
$M$. This is a maximal ideal that contains the given ideal $I_0$.
\end{proof}


\begin{defn}[Radical]
The \emph{radical} of $A$, written $\mathrm{rad}(A)$, is
$$
\mathrm{rad}(A) = \{ x \in A \mid x^n = 0, n \geq 0 \}.
$$
\end{defn}

\begin{prop}
$\mathrm{rad}(A)$ is an ideal of $A$.
\end{prop}
\begin{proof}
\begin{enumerate}
  \item{
    Let $a \in A$, $x \in \mathrm{rad}(A)$. Then $x^n = 0$ so
    $$
    (ax)^n = a^n x^n = a^n 0 = 0
    $$
    since $A$ is commutative, so $ax \in \mathrm{rad}(A)$.
  }
  \item{
    Let $x, y \in \mathrm{rad}(A)$. Then $x^n = 0$, $y^m = 0$, and
    $$
    (x \pm y)^N = \sum_{i=0}^N c_i x^i y^{N-i}
    $$
    We want $i \geq n$ or $N - i \geq n + m$. It is enough to choose
    $N \geq n + m$, so $(x \pm y)^{n + m} = 0$ and then $x \pm y \in
    \mathrm{rad}(A)$, so $\mathrm{rad}(A)$ is a subgroup of
    $(A, +)$.
  }
\end{enumerate}
\end{proof}

\begin{theorem}
Let $A$ be a commutative ring. Then the radical of $A$ is the
intersection of all prime ideals of $A$.
\end{theorem}
\begin{proof}
Observe that if $x \in \mathrm{rad}(A)$ then $x$ belongs to every
prime ideal of $A$. Indeed, if $x^n = 0$, then $0 \in P$, so
$x^n = x x^{n-1} = 0$ is in the prime ideal. Then either $x \in P$ or
$x^{n-1} \in P$. An inductive argument of this type shows that $x \in
P$ for every prime ideal $x$.

Next show that $\bigcap_P P \subset \mathrm{rad}(A)$.
Suppose $a \notin \mathrm{rad}(A)$. Then we have
$$
S = \{1, a, a^2, \dots, a^n, \dots \}, \quad
0 \notin S.
$$
Let $\mathcal{A}$ be the set of all ideals $J$ of $A$ such that
$J \cap S = \varnothing$. Since $0 \notin S$, the zero ideal $(0)$
is in $\mathcal{A}$ and so $\mathcal{A}$ is not the empty set.
$\mathcal{A}$ is a poset with the partial order $\subset$. We claim
that this is an inductively ordered poset, i.e. each totally ordered
chain of ideals $\{J_i\}_{i \in I}$ has an upper bound. Notice that
$J = \bigcup_{i \in I} J_i$ is an upper bound in $\mathcal{A}$, and
that if $J_i \cap S = \varnothing, \forall i$ then $J \cap S =
\varnothing$, so $J \in \mathcal{A}$. Zorn's lemma applied to
$\mathcal{A}$ gives that there is a maximal element $Q$ in
$\mathcal{A}$. Then $Q$ is an ideal of $A$ such that
$Q \cap S = \varnothing$ and is maximal among all ideals with this
property.

We claim that $Q$ is a prime ideal, in which case $a \notin Q$, which
was to be shown. Take $x, y \in A$ such that $xy \in Q$. Suppose
$x \notin Q$ and $y \notin Q$. Consider $Q + Ax$, $Q + Ay$. Then these
are both ideals of $A$ and are strictly larger than $Q$, and by the
maximality of $Q$ among ideals such that $Q \cap S = \varnothing$,
this means $Q + Ax \cap S \neq \varnothing$ and
$Q + Ay \cap S \neq \varnothing$. Therefore we can write
$$
a^i = q + bx, \quad
a^j = q^\prime + b^\prime y
$$
and so
$$
  a^i a^j
= a^{i + j}
= q q^\prime + q b^\prime y + q^\prime b x + b b^\prime xy \in Q
$$
because $xy \in Q$. Therefore this product is in $Q \cap S$, but this
means $a^{i+j} \in Q \cap S$, which is a contradiction.
\end{proof}

\begin{defn}[Multiplicative Set]
Let $S \subset A$. We say that $S$ is multiplicative if
\begin{enumerate}
  \item{
    $1 \in S$,
  }
  \item{
    $a, b \in S \implies ab \in S$,
  }
  \item{
    $0 \notin S$.
  }
\end{enumerate}
\end{defn}

\begin{remark}
In the same argument above, $\mathcal{A}$ has maximal elements and
these maximal elements are prime ideals. We conclude that given any
multiplicative subset $S$ of $A$, there is a prime ideal $Q$ of $A$
such that $Q \cap S = \varnothing$.
\end{remark}

\subsection{Localization}
Let $A$ be a commutative ring, $S \subset A$ a multiplicative
subset. Construct a \emph{ring of fractions} $S^{-1} A$ by
$S^{-1} A = (A \times S) / \sim$, where
$(a, s) \sim (a^\prime, s^\prime)$ if there exists an
$s^{\prime\prime}$ such that
$s^{\prime\prime}(as^\prime - a^\prime s) = 0$. Think of the
equivalence class of $(a, s)$ as a fraction $\frac{a}{s}$, where
$\frac{a}{s} \sim \frac{a}{s^\prime}$ if
$s^{\prime\prime}(as^\prime - a^\prime s) = 0$.

Operations on this construction are given by
$$
  \frac{a}{s} + \frac{a^\prime}{s^\prime}
= \frac{a s^\prime + a^\prime s}{s s^\prime}, \quad
  \frac{a}{s} \frac{a^\prime}{s^\prime}
= \frac{a a^\prime}{s s^\prime}.
$$
We have to check that this is independent of representative of the
equivalence classes involved.

\begin{xmpl}
\begin{enumerate}
  \item{
    Let $A = \mathbb{Z}$, $S = \mathbb{Z} - \{ 0 \}$. Then
    $$
      S^{-1} A
    = \left\{
        \frac{a}{b} \mid a, b \in \mathbb{Z}, b \neq 0
      \right\}
    = \mathbb{Q}
    $$
    with the usual $+$.
  }
  \item{
    Let $A$ be an integral domain. $S = A - \{ 0 \}$ is multiplicative
    because $A$ is a domain. Then
    $$
      S^{-1} A
    = \left\{
        \frac{a}{b} \mid a, b \in \mathbb{A}, b \neq 0
      \right\}
    $$
    is a field because $\frac{a}{b} \neq \frac{0}{1} \iff a \neq 0$,
    and then $\frac{a}{b}$ has inverse $\frac{b}{a}$. This is called
    the \emph{fraction field} of $A$, denoted $\mathrm{Fr}(A)$. For a
    polynomial ring this produces the ring of rational functions.
  }
\end{enumerate}
\end{xmpl}

\begin{xmpl}[Left and Right Inverse]
Let $A = \prod_{i=0}^\infty \mathbb{Z}$. Consider the ring
$R = \mathrm{End}(A)$ of endomorphisms with addition as normal and
multiplication given by composition. Let
$L \in R$ be given by $L(x_0, x_1, \dots) = (0, x_0, x_1,
\dots)$. This map is injective but not surjective, so it has a left
inverse but not a right inverse. Consider also
$T \in R$ such that $T(y_0, y_1, \dots) = (y_1, y_2, \dots)$. Then
clearly $T \circ L = \mathrm{id}$, so $L$ has a left inverse. But
$L$ has no right inverse since there is no map $F$ such that
$L \circ F = \mathrm{id}$.
\end{xmpl}


\begin{defn}[Principal Ideal Domain (PID)]

Let $A$ be a nonzero commutative unital ring. Assume that $A$ is an
integral domain. $A$ is a \emph{principal ideal domain} if every ideal
is principal, i.e. generated by one element.
\end{defn}

\begin{xmpl}
Examples of PIDs include:
\begin{enumerate}
  \item{
    $\mathbb{Z}$. Take $I \neq (0)$. Take $a$ to be the smallest
    positive integer in $I$ (there is always such an element since $x
    \in I \implies -x \in I$. Use Euclidean division we show that
    $I = (a)$. Let $b \in I$. Then $b = aq + r$, where $0 \leq r <
    a$. Then $r$ is in the ideal. But $a$ is the smallest ideal in
    $I$, so $r = 0$. Then $b = aq \in (a)$.
  }
  \item{
    Let $A = F[x]$ be the polynomials with coefficients in a field
    $F$. Then $I = (f(x))$ using the same proof as before with
    polynomial division.
  }
\end{enumerate}
\end{xmpl}

\begin{obsv}
If $A$ is a PID then there exist greatest common divisors of any two
nonzero elements. Let $a, b$ be nonzero elements in $A$.
Here by $d \vert a$ we mean $\exists c \in A$ such that $a = dc$, or
rather $a \in (d)$. Then a greatest common divisor
$d = \mathrm{gcd}(a, b)$ is an element such that
if $d^\prime \vert a$ and $d^\prime \vert b$, then $d^\prime \vert d$.

Consider the ideal $I = (a, b)$. Since $A$ is a PID, $I = (c)$ for
some $c \in A$. Certainly $a \in (a,b)$, so $a \in (c)$, so
$c \vert a$. Similarly $c \vert b$. We also have
$$
(c) = (a, b) \implies c \in (a, b) \implies c = xa + yb
$$
for some $x, y \in A$.
Let $c^\prime \vert a$ and $c^\prime \vert b$. Then
$c^\prime \vert xa + yb = c$. We conclude that $c$ is a $\mathrm{gcd}$
of $A$.
\end{obsv}

\begin{defn}[Irreducible Elements]
Let $A$ be a domain. An element $a \in A$ such that $a \neq 0$,
$a \notin A^\ast$ is called \emph{irreducible} (or \emph{prime})
if $a = bc$ implies that one of $b$ or $c$ is a unit.
\end{defn}

\begin{xmpl}
In the integers, an irreducible element is a positive or negative
prime number.
\end{xmpl}

\begin{prop}
Let $A$ be a domain, $a \in A$ be nonzero. If $(a)$ is prime, then
$a$ is irreducible.
\end{prop}

\begin{proof}
Let $a = bc$. Then $bc \in (a)$. Since $(a)$ is
prime, then $b \in (a)$ or $c \in (a)$. Suppose $b \in (a)$. Then
$b = ax$, so $a = axc$ and then $a(1 - xc) = 0$. But $a \neq 0$, so
$xc = 1$ and then $c$ is a unit.

Furthermore if $(a)$ is prime then $(a) \neq A = (1)$. If $a$ were a
unit then we would have $1 = ay$, so then $1 \in (a)$, and then
$(a) = A$.
\end{proof}

\begin{remark}
The converse is not true in general -- there are rings with
irreducible elements that do not generate prime ideals. The rings for
which this converse is true are the unique factorization rings.
\end{remark}

\begin{defn}[Unique Factorization in Irreducibles]
Let $A$ be a domain. A nonzero element $a$ has a
\emph{unique factorization in irreducibles} if
we can write
$$
a = u p_1 p_2 \cdots p_n
$$
where $u \in A^\ast$ and $p_i$ are irreducible elements, and if
$a = u p_1 \cdots p_n = v q_1 \cdots q_m$ then $n = m$ and
up to permutation of indices, $q_i = u_i p_i$ with $u_i \in A^\ast$
for all $i$.
\end{defn}

\begin{defn}[Unique Factorization Domain]
An integral domain in which every nonzero element has a unique
factorization in irreducibles is called a
\emph{unique factorization domain} (UFD).
\end{defn}

\begin{theorem}
A PID is always a UFD.
\end{theorem}
\begin{proof}
Let $a \in A$, $a \neq 0$. If $a$ is irreducible, we are done. If not,
there exist $b$, $c$ such that $a = bc$, where $b, c \notin
A^\ast$. If $b, c$ are irreducible, we are done. If not, . . .

Why will this process end? In the integers the factors will get
smaller, but we are not guaranteed of this here.

Note that in general if $(a_1) = (a_2)$ then $a_2 = u a_1$, where
$u \in A^\ast$. This is because $a_2 = x a_1$, $a_1 = y a_2$ implies
that $a_2 = xya_2$ so $a_2(1 - xy) = 0$. Then $xy = 1$, so $x, y$ are
units. This means that $(a_1)$ and $(a_2)$ are strictly bigger.

When we assume that neither $b$ or $c$ is a unit, we get
$(a)$ is strictly smaller than $(b)$ and $(a)$ is strictly smaller
than $(c)$. If this procedure would go on forever, we would get
$(a) < (a_1) < (a_2) < \cdots < (a_n) < \cdots$. Consider
$$
\bigcup_{i=1}^\infty (a_i) = I
$$
where $I$ is an ideal of $A$. But $A$ is a PID, so $I = (z)$.
Then $z \in I$, so $\exists i$ such that $z \in (a_i)$.
Then $(z) \subset (a_i)$, so $\bigcup_j (a_j) \subset (a_i)$, so
$\forall j \geq 1$ we have $(a_i) = (a_{i+1})$, which is a
contradiction.

``This is more like a computer science proof, it's not really a good proof.''
\end{proof}

\begin{theorem}
If $A$ is a PID then $A$ is a UFD.
\end{theorem}

\begin{proof}
Consider the set $S$ of all ideals $(a)$, where $0 \neq a$ is \emph{not} a
product of units and irreducible elements. We want to show that $S =
\varnothing$.

Assume $S \neq \varnothing$. $S$ is a poset with usual order
$$
I = (a) \leq J = (b) \triangleq (a) \subset (b).
$$
Observe that $S$ is inductively ordered, i.e. every chain
$$
(a_1) \subset (a_2) \subset \cdots \subset (a_n) \subset \cdots
$$
has an upper bound given by $U = \bigcup_{n \geq 1} (a_n) = (a_{n_0})$ for
some $n_0$, since $U$ is principal. Therefore from Zorn's lemma $S$ has a maximal
element, say $(a)$, and by assumption $a$ is not irreducible since
$(a) \in S$. Then we can write $a = bc$ with $b, c$ not units. Then
$(a) \subsetneq (b)$ and $(a) \subsetneq (c)$, but by the maximality
of $S$ this means $(b), (c) \notin S$. But then $a = bc$ is a product
of units and irreducible elements, which is a contradiciton. Therefore
$S = \varnothing$, so all $a$ are such products.

It remains to show that the factorization is unique, i.e. if
$$
a = u p_1 p_2 \cdots p_n = v q_1 q_2 \cdots q_m
$$
with $u, v \in A^\ast$ and $p_i, q_j$ irreducible, then $n = m$ and up
to permutation $p_i = u_i q_i$ for some $u_i \in A^\ast$. This is done
similarly to the integers. To extend this proof we need to show the
following.

\begin{lemma}
If $p$ is irreducible in a PID $A$ and $p \vert ab$, then $p \vert a$ or $p
\vert b$.
\end{lemma}

Assume $p \not\vert a$. Then there is a GCD in $A$ given by
$g = \mathrm{gcd}(a, p)$ where $g \vert a$ and $g \vert p$. Then
$g \in A^\ast$, i.e. $(a, p) = (1)$. Then $1 = xa + yp$ for some $x, y
\in A$. Then $b = xab + ypb$, and since $p \vert a$ we have
$b = p(xz + yb)$ and so $p$ divides $b$.
\end{proof}

\section{Chains of Ideals}
Let $A$ be a commutative ring with 1.
\begin{defn}[Noetherian ring]
$A$ is called \emph{Noetherian} if every ideal in $A$ is finitely
generated, i.e. for every ideal $I \subset A$ there exist elements
$a_1, \dots, a_n \in I$ such that
$$
  I
= (a_1, \dots, a_n)
= \{ a_1 x_1 + \cdots + a_n x_n \mid x_i \in A \}.
$$

Rings that show up in algebra tend to be Noetherian, rings that show
up in analysis tend not to be.
\end{defn}

\begin{xmpl}
Every PID is Noetherian.
\end{xmpl}

\begin{prop}
$A$ is Noetherian if and only if one of the following equivalent
conditions is satisfied:
\begin{itemize}
  \item{
    Every chain of ideals
    $$
    I_1 \subseteq I_2 \subseteq \cdots \subseteq I_n \subseteq \cdots
    $$
    eventually stabilizes, i.e. $\exists n_0$ such that
    $\forall n \geq n_0$, $I_n = I_{n+1}$.
  }
  \item{
    Every non-empty set of ideals of $A$ contains a maximal element.
  }
\end{itemize}
\end{prop}

\begin{proof}
We proceed in a circle of implications.
\begin{itemize}
  \item[(def) $\implies$ (1)]
  {
    Suppose every ideal in $A$ is finitely generated and let
    $I_1 \subset I_2 \subset \cdots \subset I_n \subset \cdots$ be a
    chain of ideals. Consider $I = \bigcup_{n \geq 1} I_n$. Then
    $I = (a_1, \cdots, a_r)$ for some $a_i \in I$. Then there exists
    some $n_0$ such that $a_i \in I_{n_0}$, so $I \subset
    I_{n_0}$. Then $I \subset I_{n_0}$ and so $I_n = I_{n+1}$ for all
    $n \geq n_0$ as desired.
  }
  \item[(1) $\implies$ (2)]
  {
    Take $S = \varnothing$ to be a set of ideals of $A$. By Zorn's
    lemma it is enough to prove that every chain of ideals has an
    upper bound in $S$. But from (1), $I_{n_0}$ is an upper bound.
  }
  \item[(2) $\implies$ (def)]
  {
    Consider an ideal $I$ in $A$. Let $S$ be the set of all ideals
    of the form $(a_1, \dots, a_r)$ for some $r \geq 1$ which are in
    $I$. We claim that $I$ is a maximal element in $S$, and so
    $I$ is of this fomr.

    If not, $M = (a_1, \dots, a_r) \subsetneq I$. Then there exists
    $a_{r+1} \in I - M$, so  $(a_1, \dots, a_r, a_{r+1}) \subset I$,
    violating the maximality of $M$. Therefore $I$ is finitely generated.
  }
\end{itemize}
\end{proof}

\begin{xmpl}
Let $A$ be the ring of continuous functions from $[0,1] \to
\mathbb{R}$. Consider the ideal
$$
I_n = \{ f \in A \mid f(x) = 0, \forall 0 \leq x \leq \frac{1}{n} \}.
$$
Notice that $I_n \subsetneq I_{n+1}$ for all $n$, so this chain does
not stabilize and therefore $A$ is not Noetherian.
\end{xmpl}

\begin{theorem}[Hilbert Basis Theorem]
If $A$ is Noetherian, then the polynomial ring $A[x]$ is Noetherian.
\end{theorem}

\begin{proof}
Let $\mathcal{A}$ be an ideal of $A[x]$. We show that $\mathcal{A}$ is
finitely generated. Create a sequence of ideals $I_n \subset A$ given
by
$$
I_n = \{
         a \in A
      \mid
         \exists f(x) \in \mathcal{A} \text{ of degree } n,
                 f(x) = ax^n + \cdots
      \} \cup \{ 0 \}.
$$
Check that $I_n$ is an ideal of $A[x]$. Take
$a, b \in I_n$. Then there exist
$$
a x^n + \cdots, bx^n + \cdots \in \mathcal{A}
$$
so that $(a \pm b)x^n + \cdots \in \mathcal{A}$ since $\mathcal{A}$ is
an ideal. But then $a \pm b \in I_n$, so $I_n$ is an ideal. Consider
also $c \in A \subset A[x]$ so that $ca x^n + \cdots \in \mathcal{A}$
and then $ca \in I_n$.

Observe that $I_n \subseteq I_{n+1}$. Since $x \in A[x]$ we have
$x (a x^n + \cdots) = a x^{n+1} + \cdots \in \mathcal{A}$. But then
we have a chain
$$
I_0 \subseteq I_1 \subseteq \cdots \subseteq I_n \subseteq \cdots
$$
of ideals of $A$, a Noetherian ring,
so $\forall n \geq r$ we have $I_{n+1} = I_n$ for some $r$.
Also all $I_n$ are finitely generated, so we have
\begin{align*}
I_0 &= (a_{01}, a_{02}, \cdots, a_{0n_0}) \\
I_1 &= (a_{11}, a_{12}, \cdots, a_{1n_1}) \\
\cdots \\
I_r &= (a_{r1}, a_{r2}, \cdots, a_{rn_r})
\end{align*}
with $a_{ij} \in A$ and nonzero. Set
$f_{ij}(x)$ to be a polynomial of degree $i$ in the ideal
$\mathcal{A}$ whose leading coefficient is $a_{ij}$. We claim that
the $f_{ij}(x)$ generate $\mathcal{A} \subset A[x]$.

Consider some $f(x) \in \mathcal{A}$. We want to write
$f(x)$ as a combination of $f_{ij}$ with coefficients other
polynomials. Proceed by induction on $\mathrm{deg}(f(x)) = d$.

When $d = 0$, $f(x) = a$ with $a \in I_0 = (a_{01}, \dots,
a_{0n_0})$. Assume the claim for polynomials of degree less than
$d$. There are two cases:
\begin{enumerate}
\item{
  $d > r$. Then $f(x) = ax^d + \cdots$. Then
  $a \in I_d = I_r = (a_{r1}, \dots, a_{r n_r})$, so
  $$
  a = c_1 a_{r1} + \cdots + c_{n_r} a_{r n_r}, c_j \in A
  $$
  which gives
  \begin{align*}
     & f(x) - x^{d-r}(c_1 f_{r1}(x) + \cdots + c_{n_r} f_{r n_r}(x))\\
   = & f(x) - x^{d-r}(c_1 (a_{r1} x^r + \cdots )
                     + \cdots
                     + c_{n_r}(a_{rn_r} x^r + \cdots)).
  \end{align*}
  of degree less than $d$, since this cancels the leading
  terms. Therefore by induction this is a combination of $f_{ij}(x)$,
  so $f(x)$ is a combination of $f_{ij}(x)$ with coefficients other polynomials.
}
\item{
  $d \leq r$. Then $f(x) = a x^d + \cdots$,
  $a \in I_d = (a_{d1} + \cdots + a_{dn_d})$. For each of these we
  have $f_{dj}(x) = a_{dj}x^d + \cdots$. Since
  $a = c_1 a_{d1} + \cdots + c_{n_d} a_{d n_d}$, $c_j \in A$ we have
  $$
  f(x) - (c_1 f_{d1}(x) + \cdots + c_{n_d} f_{dn_d(x)})
  $$
  has degree less than $d$. Then by induction we are done.
}
\end{enumerate}
\end{proof}

\begin{corol}
For $A$ a Noetherian ring, then $A[x_1, \dots, x_n]$ is Noetherian
where $x_i$ is a finite list of variables.
\end{corol}
\begin{proof}
We can regard $A[x_1, x_2] = (A[x_1])[x_2]$, i.e. polynomials in $x_2$
with coefficients in $A[x_1]$. Applying induction on $n$ and the
Hilbert basis theorem, the claim follows since
$A[x_1, \dots, x_n] = (A[x_1, \dots x_{n-1}])[x_n]$.
\end{proof}

In particular, since every field and every PID are Noetherian,
all polynomials with coefficients in a field are Noetherian.

\begin{prop}
Let $A$ be Noetherian, $I$ an ideal of $A$. Then the quotient
$A / I$ is Noetherian.
\end{prop}
\begin{proof}
Let $J \subset A / I$ be an ideal in the quotient ring. Consider
$\pi^{-1}(J) \subset A$, where $\pi : A \to A / I$ is the natural
quotient map. Then
$$
\pi^{-1}(J) = \{ a \in A \mid a + I \in J \}.
$$
But the inverse image of an ideal under a ring homomorphism is always
an ideal. Since $A$ is Noetherian it is finitely generated, and so
$\pi^{-1}(J)$ is finitely generated by $a_1, \dots, a_n \in
\pi^{-1}(J)$. Then since $\pi$ is surjective we have $a_i + I$
generate $J$.
\end{proof}

\begin{xmpl}
If $K$ is a field, then $K[x_1, \dots, x_n] / I$ is Noetherian.
\end{xmpl}

\begin{xmpl}
A polynomial ring in an infinite set of variables
$K[x_1, x_2, \dots, x_n, \dots]$ is not Noetherian. This is easy to
check since $(x_1) \subsetneq (x_1, x_2) \subsetneq \cdots$ is an
infinite chain of proper subset ideals.
\end{xmpl}

\subsection{Unique Factorization}
\begin{prop}
Suppose $A$ is a unique factorization domain. Then an element $a \in
A$ which is nonzero is irreducible if and only if $(a)$ is a prime ideal.
\end{prop}
\begin{proof}
We proved already that in an integral domain, $(a)$ prime implies that
$a$ is irreducible.

Assume $a$ is irreducible. Suppose $x, y \in A$ such that $xy \in
(a)$, i.e. $xy = za$ for some $z \in A$. Then since $A$ is a UFD and
since $a$ is irreducible,
$xy = za$ implies that $a$ appears in the factorization of $x$ or of
$y$, so $x \in (a)$ or $y \in (a)$. Therefore $(a)$ is a prime ideal.
\end{proof}

Recall that a principal ideal domain is a unique factorization domain.

\begin{theorem}
If $K$ is a field, then $K[x]$ is a UFD.
\end{theorem}
\begin{proof}
The polynomials are a principal ideal domain. Let $I \subset K[x]$ be
an ideal. Let $f(x)$ be a polynomial of smallest degree in $I$, and
let $g(x)$ be another polynomial in $I$. Then we have
$g(x) = f(x) q(x) + r(x)$, where $r(x) = 0$ or
$\mathrm{deg} r(x) < \mathrm{deg} f(x)$. Then since $f, g \in I$ we
have $r \in I$, so $r(x) = 0$ since $f(x)$ was chosen to have smallest
degree. Then $g(x) \in (f(x))$.
\end{proof}

\begin{theorem}
For any $n \geq 1$, if $K$ is a field, then
$K[x_1, \dots, x_n]$ is a UFD.
\end{theorem}
This will follow from another theorem:
\begin{theorem}
If $A$ is a UFD, then $A[x]$ is a UFD.
\end{theorem}
\begin{corol}
If $A$ is a UFD, then $A[x_1, \dots, x_n]$ is a UFD, $\forall n \geq
1$. This is because $A[x_1, \dots, x_n] = (A[x_1, \dots, x_{n-1}])[x_n]$.
\end{corol}

Let $A$ be a UFD, and let $K$ be the fraction field of $A$:
$$
K = \left\{ \left. \frac{a}{b} \right| a, b \in A, b \neq 0 \right\}
  = S^{-1} A
$$
where $S = A - \{ 0 \}$.

If $p$ is an irreducible (``prime'') element of $A$, then there is a
function, called the \emph{order of $p$},
$\mathrm{ord}_p : K - \{ 0 \} \to \mathbb{Z}$ defined by
$\frac{a}{b} \mapsto r,$
where
$$
\frac{a}{b} = p^r \frac{a^\prime}{b^\prime}
$$
with $p \not\vert a^\prime$, $p \not\vert b^\prime$. Observe that the
order of the product
$\mathrm{ord}_p(xy) = \mathrm{ord}_p(x) + \mathrm{ord}_p(y)$.

Suppose $f(x) \in K[x]$, $f(x) \neq 0$. Then
$$
f(x) = a_n x^n + \cdots + a_1 x + a_0, \quad a_n \neq 0.
$$
Define
$$
\mathrm{ord}_p(f(x)) = \min \{
                               \mathrm{ord}_p(a_i),
                            \mid
                               a_i \neq 0
                            \}
$$
For example take
$$
f(x) = \frac{x^3}{15} + 4x^2 + \frac{1}{4}
     = 3^{-1} \cdot 5^{-1} \cdot x^3 + 2^{-2} \cdot 1.
$$
Then
$$
\mathrm{ord}_2(f) = -2, \quad
\mathrm{ord}_3(f) = -1, \quad
\mathrm{ord}_5(f) = -1, \quad
\mathrm{ord}_7(f) = 0.
$$

Notice that
$$
f(x) = \frac{1}{15} x^3 + 4x^2 + \frac{1}{4}
     = \frac{1}{60} (4x^3 + 240x^2 + 15)
$$
with $\mathrm{gcd}(4, 240, 15) = 1$. This is the form we wish to
derive in general.

Define the \emph{content} of $f(x)$ by
$$
\mathrm{cont}(f(x)) = \prod_p p^{\mathrm{ord}_p(f)}.
$$
We must define a non-repeated product over all irreducible elements
-- i.e., not repeating two irreducibles when they differ by a
unit. Therefore $p$ ranges over representatives of equivalence classes
of prime elements where $p \sim p^\prime$ if $p^\prime = u p$ for some
$u \in A^\ast$. Therefore the content is only defined up to a
multiplication by a unit.

Observe that we can write
$$
f(x) = \mathrm{cont}(f) \cdot f_1(x)
$$
with $f_1(x) \in A[x]$ and content 1 (i.e. the GCD of the coefficients
of $f_1$ is 1). Conversely, if $f(x) = c f_1(x)$ with $c \in K - \{ 0
\}$ and $f(x)$ having coefficients with GCD 1, then $c$ is equal to
the content of $f(x)$ (up to a unit).

\begin{proof}
We will use the fact that $K[x]$ is a UFD. An important step is that
$f(x) \in A[x]$ is irreducible in $A[x]$ if and only if $f(x)$ is
irreducible in $K[x]$ and has content 1.

\begin{lemma}[Gauss's Lemma]
Let $f, g \in K[x]$, $f, g \neq 0$. Then
$$
\mathrm{cont}(f \cdot g) = \pm \mathrm{cont}(f) \cdot \mathrm{cont}(g)
$$
where $\pm$ is taken to mean up to a unit.
\end{lemma}
\begin{proof}
Write $f = \mathrm{cont}(f) f_1$, where $f_1 \in A[x]$,
$\mathrm{cont}(f_1) = 1$, and similarly $g = \mathrm{cont}(g) g_1$.
Then
$$
f \cdot g = \mathrm{cont}(f) \mathrm{cont}(g) f_1 g_1.
$$
Since $f_1 g_1 \in A[x]$, it is sufficient to show that
$\mathrm{cont}(f_1 g_1) = 1$. Let $p$ be a prime element of $A$. We
want to show that $p$ does not divide all the coefficients of $f_1
g_1$.

We know $p$ does not divide all the coefficients of $f_1$ and that $p$
does not divide all the coefficients of $g_1$. Consider the quotient
ring $\bar{A} = A / (p)$. Since $p$ is prime and $A$ is a UFD, this
means $(p)$ is a prime ideal, and then $\bar{A}$ is an integral
domain. Then $\bar{A}[x]$ is an integral domain.
Consider $\bar{f}_1 = f_1 \bmod (p)$ in $\bar{A}[x]$. These are the
polynomials with coefficients modulo $p$, i.e.
$$
  a_n x^n + \cdots + a_1 x_1 + a_0 \bmod p
= (a_n \bmod p) x^n + \cdots + (a_1 \bmod p) x_1 + a_0 \bmod p.
$$
Since $p$ does not divide all coefficients of $f_1, g_1$, this means
$\bar{f}_1, \bar{g}_1 \neq 0$ and so $\bar{f}_1 \bar{g}_1  \neq 0$.
But $\bar{f}_1 \bar{g}_1 = \bar{f_1g_1}$ since taking an element
modulo $(p)$ is a ring homomorphism (canonical projection).
Therefore $p$ does not divide all the coefficients of $f_1 g_1$.
\end{proof}

\begin{itemize}
  \item[$\implies$]{
    Let $f$ be irreducible in $A[x]$. Then
    $f = \mathrm{cont}(f) f_1$, so $\mathrm{cont}(f)$ is a unit (in which
    case we are done, since the content is only defined up to a unit)
    or $f_1$ is a unit. But the units of a polynomial ring are just
    constants, which are units in $A$, i.e. $(A[x])^\ast =
    A^\ast$. Therefore $\mathrm{cont}(f) = 1$.

    It remains to show that $f$ is irreducible in $K[x]$. Assume
    $f = gh$, $g, h \in K[x]$ non-constant. Note that
    $(K[x])^\ast = K^\ast = K - \{ 0 \}$ and that
    $g = \mathrm{cont}(g) g_1$, $h = \mathrm{cont}(h) h_1$. Then
    $$
      f
    = \mathrm{cont}(g)\mathrm{cont}(h) g_1 h_1
    = \mathrm{cont}(f) g_1 h_1
    = g_1 h_1,
    $$
    since $\mathrm{cont}(f) = 1$ was already shown. But $f$ is irreducible
    in $A[x]$, and since $g_1 h_1 \in A[x]$ this means $g_1$ or $h_1$ is a
    unit, so constant, and then $g$ or $h$ is constant too.
    Therefore $f$ is irreducible in $K[x]$.
  }
  \item[$\impliedby$]{
    Assume $f(x)$ is irreducible in $K[x]$ and $\mathrm{cont}(f) = 1$.
    Let $f = gh$, $g, h \in A[x]$. Since $f$ is irreducible in $K[x]$,
    one of $g$, $h$ is a constant. Say $g$ is constant in $A$. Then
    $f = gh$ where $g \in A$, $h \in A[x]$, and so
    $\mathrm{cont}(f) = g \mathrm{cont}(h)$. But $\mathrm{cont}(f) =
    1$ by assumption, and this product is in $A$, so $g$ is a unit in $A$.
  }
\end{itemize}

Now we can show that if $A$ is a UFD then so is $A[x]$. Let
$f \in A[x]$. Since $K[x]$ is a UFD, $f = q_1 q_2 \cdots q_r$,
$q_i \in K[x]$ irreducible in $K[x]$. Write
$q_i = \mathrm{cont}(q_i) \cdot p_i$, $p_i \in A[x]$,
$\mathrm{cont}(p_i) = 1$. Set $c_i = \mathrm{cont}q_i$. Then we write
$$
f = c_1 \cdots c_r p_1 \cdots p_r = c p_1 \cdots p_r
$$
with $c = \mathrm{cont}(f) \in A$ by Gauss's lemma and since $f$ has
integral coefficients.
Observe that $p_i$ are irreducible in $K[x]$ since
$q_i$ are and $p_i = q_i / c_i$, and that $p_i$ have content 1, so
$p_i$ are irreducible in $A[x]$ as above. Then since $c \in A$ and $A$
is a UFD, we can factor $c$ into prime elements $c = \omega_1 \cdots
\omega_n$. Then
$$
f = \omega_1 \cdots \omega_n p_1(x) \cdots p_r(x)
$$
Since $\omega_i$ are irreducible as constants they are trivially
irreducible as polynomials as well. Uniqueness follows from uniqueness
of the factorization in $A$ and in $K[x]$.
\end{proof}

\begin{theorem}[Eisenstein's Criterion]
Let $A$ be a UFD, $K = \mathrm{Frac}(A)$, and
$$
f(x) = a_n x^n + \cdots + a_1 x + a_0, \quad
n \geq 1, a_i \in A, a_n \neq 0.
$$
Assume there is an irreducible element $p \in A$ such that
$p \not\vert a_n$, $p \vert a_i$ for $i \neq n$, $p^2 \not\vert
a_0$. Then $f(x)$ is irreducible in $K[x]$.
\end{theorem}
\begin{proof}
Assume $f(x) = g(x) h(x)$, where $g(x), h(x) \in K[x]$, and with
$\deg g$, $\deg h \geq 1$. We have
$$
f(x) = \mathrm{cont}(f) f_1(x), \quad
g(x) = \mathrm{cont}(g) g_1(x), \quad
h(x) = \mathrm{cont}(h) h_1(x),
$$
where $f_1, g_1, h_1$ have integral coefficients with GCD 1.
It is enough to consider $f_1$, so assume $\mathrm{cont}(f) = 1$. Then
applying Gauss's lemma gives
$$
  f(x)
= g(x)h(x)
= \mathrm{cont}(gh) g_1(x) h_1(x)
= \mathrm{cont}(f) g_1(x) h_1(x)
= g_1(x) h_1(x)
$$
where $g_1, h_1 \in A[x]$. Then we can assume
$f(x) = g(x) h(x)$ where $g, h \in A[x]$ with
$\deg g, \deg h \geq 1$.

Now let
$$
g(x) = b_r x^r + \cdots + b_1 x + b_0, \quad
h(x) = c_s x^s + \cdots + c_1 x + c_0
$$
with $r, s \geq 1$ and $r + s = n$. Then
$a_0 = b_0 c_0$, so since $p \vert a_0$ and
$p^2 \not\vert a_0$, suppose $p \vert b_0$ and
$p \not\vert c_0$. Furthermore
$a_n = b_r c_s$, so since $p \not\vert a_n$ we have
$p \not\vert b_r$, $p \not\vert c_s$.

Let $b_t$ be the smallest coefficient of $g(x)$ such that
$p \not\vert b_t$, so that
$p \vert b_{t-1}, \dots, p \vert b_0$. Note that
$r \geq t > 0$.
Then
$$
a_t = b_0 c_t + b_1 c_{t-1} + \cdots + b_t c_0.
$$
We have $t \leq r < n$, and since $r + s = n$, $s \geq 1$
we know $t < n$. This means $p \vert a_t$, but this is a contradiction
since $p \not\vert b_t$ and $p \not\vert c_0$.
\end{proof}

\begin{xmpl}
$x^5 - 18$ is irreducible in $\mathbb{Q}[x]$. Take
$p = 2$. Then $2^2 \not\vert 18$ and $2 \vert 18$, so this is
irreducible.

Let $p$ be prime and
$f(x) = x^{p-1} + \cdots + x + 1$ (a \emph{cyclotomic polynomial}).
We can write
$$
x^p - 1 = (x - 1)(x^{p-1} + \cdots + x + 1)
$$
and so $f(x) = \frac{x^p - 1}{x - 1}$. Note that the zeros of $f$
are the $p$-th roots of unity, i.e. $x = \omega^i$ where
$\omega = e^{2\pi i}{p}$. We claim that $f(x)$ is irreducible in
$\mathbb{Q}[x]$. It is enough to show that $f(x + 1)$ is irreducible,
since if $g(x + 1)h(x + 1)$ is irreducible then one of $g(x+1),
h(x+1)$ is a constant, in which case $g(x)$ (resp. $h(x)$) is a
constant.

Now
$$
  f(x + 1)
= \frac{(x + 1)^p - 1}{x + 1 - 1}
= \frac{x^p + p x^{p-1} + {p \choose 2} x^{p-2} + \cdots + p x + 1 - 1}{x}
$$
since
$$
(x + y)^p = \sum_{n=0}^p {p \choose n} x^{p-n} y^n
$$
so that
$$
f(x + 1) = x^{p-1} + p x^{p-2} + {p \choose 2} x^{p-3} + \cdots + p.
$$
Observe that
$$
p \vert {p \choose i} = \frac{p!}{i!(p-i)!}
$$
and $p^2 \not\vert p$, so $f(x+1)$ is irreducible by Eisenstein's criterion.
\end{xmpl}

\begin{remark}
Let $A$ be a UFD and $p$ a prime element. Then $(p)$ is a prime ideal
of $A$. Consider $\bar{A} = A / (p)$, which is therefore a domain.
Let $L$ be the fraction field of $\bar{A}$.

Assume $f(x) \in A[x]$ so $f(x) = g(x) h(x)$ with $g, h \in A[x]$ and
$\deg g, h \geq 1$. Assume also that $g, h$ are monic. Apply the ring
homomorphism
$\pi : A \to A / (p)$ by $\pi(a) = \bar{a} = a \bmod (p)$. Then
$$
\bar{f(x)} = \bar{g(x)}\bar{h(x)}
$$
where $\bar{g}$ and $\bar{h}$ are again non-constant and monic
(since their preimages were monic and 1 goes to 1). If
$\bar{f}$ is irreducible in $L[x]$, then one of $\bar{g}$,
$\bar{h}$ must be constant, a contradiction.

Therefore if $f(x) \in A[x]$ is monic and $\bar{f}(x)$ is irreducible
in $L[x]$, then $f(x)$ is irreducible in $K[x]$, where
$K = \mathrm{Frac}(A)$.
\end{remark}
