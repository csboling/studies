\documentclass{article}

\usepackage{amsmath}
\usepackage{amsfonts}
\usepackage{mathrsfs}
\usepackage{amssymb}
\usepackage{amsthm}
\usepackage{mathtools}

\usepackage{enumitem}
\usepackage{pictexwd,dcpic}
\usepackage{hyperref}
\usepackage[lastexercise]{exercise}

\newcommand\dif{\mathop{}\!\mathrm{d}}

\newcounter{Problem}
\newenvironment{Problem}{\begin{Exercise}[counter={exer}]}
                        {\end{Exercise}}

\theoremstyle{plain}
\newtheorem{theorem}{Theorem}
\newtheorem{corol}{Corollary}
\newtheorem{lemma}{Lemma}

\theoremstyle{definition}
\newtheorem{defn}{Definition}
\newtheorem{xmpl}{Example}
\newtheorem{prop}{Proposition}

\theoremstyle{remark}
\newtheorem{remark}{Remark}
\newtheorem{obsv}{Observation}
\newtheorem{exer}{Exercise}
\newtheorem{recap}{Remark}


\begin{document}

\section{Group Actions}

\begin{defn}[Group Action]
Let $G$ be a group, $X$ be a set. Then a 
\emph{left $G$-action on $X$} is a map
$G \times X \to X$, written $(g,x) \mapsto g \cdot x$, such that
\begin{itemize}
  \item{$\forall x \in X$, $1 \cdot x = x$,}
  \item{$\forall g_1, g_2 \in G$, $\forall x \in X$, we have
        $(g_1 g_2) \cdot x = g_1 (g_2 \cdot x)$.
       }
\end{itemize}
\end{defn} 

\begin{xmpl}
\begin{enumerate}
  \item{
        Take $X = G$, $\cdot : G \times G \to G$ to be the group
        operation (called the \emph{translation action}).
       }
  \item{For $X = G$, conjugation $(g, h) \mapsto g h g^{-1}$ is a
        group action.
       }
  \item{Take $G = \mathrm{GL}_n(\mathbb{R})$, $X = \mathbb{R}^n$. For
        $v \in \mathbb{R}^n$ there is an action $A \cdot v = Av$, the
        matrix product.
       }
  \item{Take $X = \{z \in \mathbb{C} | \mathrm{Im}(z) > 0\}$, so
        $z = x + iy$ for $x,y \in \mathbb{R}$, $y > 0$ 
        (the upper-half plane). Take 
        $G = \mathrm{SL}_2(\mathbb{R})$. The action is given by
        $$
        \left[\begin{array}{c c}
          a & b \\ c & d
        \end{array}\right] \cdot z
        = \frac{az + b}{cz + d}.
        $$
        We need to check
        \begin{enumerate}
          \item{$cz + d \neq 0$,
               }
          \item{$\mathrm{Im}\left(\frac{az + b}{cz + d}\right) >
                0$. (Exercise) This can be done by writing
                $\mathrm{Im}(w) = w - \bar{w}$.
               }
          \item{We have a group action.}
        \end{enumerate}
        This group action is very important in number theory.
      }
\end{enumerate}
\end{xmpl}

\begin{remark}
Giving a $G$-action on $S$ is equivalent to giving a group
homomorphism $G \to \mathrm{Perm}(S)$. Why is this? Let
$\cdot : G \times S \to S$ be a group action. Then to this we can
associate a map $\Phi : G \to \mathrm{Perm}(S)$ by the rule
$\Phi(g)(x) = g \cdot x$. We can observe that $\Phi(g)$ is a bijection
since its inverse is given by $\Phi(g^{-1})$. 

Also, starting from $\Phi : G \to \mathrm{Perm}(S)$ we can define
$G \times S \to S$ by $g \cdot x = \Phi(g)(x)$. Observe that the types
of this isomorphism are related tensor-hom adjunction.
\end{remark}

\begin{defn}[Orbit, Stabilizer]
Fix some $x \in S$. The \emph{orbit} of $x$ is the set
$$
G \cdot x = \{ y \in S | \exists g \in G . y = g \cdot x \} 
          = \{ g \cdot x | g \in G \}.
$$

The \emph{stabilizer} of $x$ is the set 
$Gx = \{ g \in G | g \cdot x = x \}$. Observe that 
$\forall x \in S$, $Gx < G$:
\begin{align*}
         & g \cdot x = x \\
\implies & g^{-1} \cdot (g \cdot x) = g^{-1} \cdot x \\
\implies & (g^{-1} g) x = g^{-1} x \\
\implies & x = g^{-1} \cdot x,
\end{align*}
so $g^{-1} \cdot x \in Gx$.
\end{defn}

\begin{prop}
Take a group action $G \times S \to S$. There is a bijection between
the coset $G / Gx$ and the elements of the orbit of $x$, i.e.
$$
G / Gx \simeq G \cdot x.
$$
\end{prop}
\begin{proof}
Consider the map given by
$$
gGx \in G / Gx \mapsto g \cdot x \in S,
$$
where by definition $g \cdot x$ is in the orbit. This is a
well-defined map, by which we mean for $g, g^\prime \in G$ such that
$g \cdot Gx = g^\prime \cdot Gx$, $g^\prime = g \cdot h$, $h \in
Gx$. Then 
$$
g^\prime \cdot x = (g h) \cdot x 
\implies g^\prime \cdot x = g(hx) = g \cdot x.
$$
This is surjective by definition because everything in the orbit is an
image of the map given. It is injective since $g, g^\prime \in G$ such
that $g^\prime \cdot x = g \cdot x$ means $g^{-1} \cdot g^\prime) x =
x$ so $g^{-1} g^\prime \in G x$ and thus $g^\prime \in g G x$.
\end{proof}

We can therefore ``divide up'' $S$ into $G$-orbits:
$$
G = \coprod_{x \in S} G \cdot x
$$
We can also say that the $G$-action gives an equivalence relation on
the set by $x \equiv x^\prime$ when $x^\prime \in G \cdot x$, or
$x^\prime = g \cdot x$. We also observe that
$$
\# S = \sum_{x \text{ a representative}} [G : G \cdot x].
$$
For example, apply this to the action of $G$ on itself by conjugation
$$
(g, x) \mapsto g x g^{-1}.
$$
Take $x \in G$. Then the stabilizer is
$$
G x = \{ g \in G | gxg^{-1} = x \}
    = \{ g \in G | gx = xg \},
$$
the \emph{centralizer} of $x$ denoted $C_x$, while the orbit of $x$ in $G$ is
$$
\{ y = gxg^{-1} | g \in G \},
$$
the \emph{conjugacy class} of $x$ in $G$.

Note that the proposition proved above implies that $G / C_x$ is
isomorphic to the conjugacy class of $x$, that the number of elements
in the conjugacy class of $x$ is $[G : C_x]$, and that this number
always divides the order of the group. 

Observe that if $x \in Z(G)$, then $C_x = G$ or equivalently 
$[G : C_x] = 1$ and that
$$
\# G = \sum_{[G : C_x] > 1} [G : C_x] + \# Z(G).
$$
This is sometimes called the \emph{class equation}.

\begin{corol}
Suppose $\# G = p^n$, for $p$ prime, $n \geq 1$. Such a group is called a
$p$-group. Then the center $Z(G)$ is nontrivial.
\end{corol}

\begin{proof}
The class equation gives
$$
p^n = \sum_{[G : C_x] > 1} [G : C_x] + \# Z(G).
$$
But notice that each $[G : C_x] > 1$ divides
$\# G = p^n$, so $[G : C_x] = p^{m_x}$ for some $m_x \geq 1$
and then
$$
p^m = \sum p^{m_x} + \# Z(G),
$$
so $p$ divides $\# Z(G)$ and thus $Z(G)$ is nontrivial.
\end{proof}

\begin{corol}
If $G$ is a $p$-group then $G$ is solvable.
\end{corol}
\begin{proof}
We want to construct an abelian normal tower ending at $\{ 1 \}$.
\end{proof}
Start from the bottom. Since $G$ is a $p$-group, $Z(G) \neq \{1\}$, so
$$
\{1\} \triangleleft Z(G) \triangleleft G.
$$
Now look at $G^\prime = G / Z(G)$. This is still a $p$-group because
its order is some factor of the order of $G$, but it is of smaller
order. Since it is a $p$-group, $Z(G^\prime) \neq \{1\}$. We get
$$
\{ 1 \} \triangleleft Z(G^\prime) \triangleleft G^\prime = G / Z(G).
$$
This yields
$$
\{1\} 
  \triangleleft Z(G)
  \triangleleft \pi^{-1}(Z(G^\prime))
  \triangleleft G.
$$
where the quotients are all abelian since they are constructed as
centers.

Argue by induction on the exponent $n$, where $\# G = p^n$.
\begin{itemize}
  \item{ We desire
  $$
  \# G = p^n \implies G \text{ solvable}.
  $$
  }
  \item{$n=0$ is trivial.}
  \item{$n=1$ gives $\# G = p$, so $G$ is cyclic and isomorphic to
        $\mathbb{Z}_p$, and thus solvable.}
  \item{Assume the statement for all $n^\prime < n$. Look at
        $G$. We assume $Z(G) \neq \{ 1 \}$, $G / Z(G) = p^{n^\prime}$ for
        all $n^\prime < n$. We have from a previous proposition that
        if $H \triangleleft G$, $H, G/H$ solvable, then $G$ is
        solvable.
        }
\end{itemize}

\begin{xmpl}
Let $G$ be a group and $S$ be the set of all subgroups of $G$. Then
we have a group action given by conjugation:
$$
(g, H) \mapsto gHg^{-1} = \{ g h g^{-1} : h \in H \}.
$$
Let $x \in S$, so that $x = H < G$. Then $\mathrm{Orb}(x)$ is the set of
all subgroups that are conjugate to $H$, while the stabilizer $G_x$ is
$$
\{ g \in G : g \cdot x = x \} = \{ g \in G : g H g^{-1} = H \},
$$
the normalizer $N(H)$ of $H$ in $G$. Then
$$
\# \mathrm{Orb}(H) = [G : N(H)].
$$
So the number of subgroups of $G$ which are conjugate to $H$ divides
the group order $|G|$.
\end{xmpl}

\begin{prop}
Let $G$ be a group, $H < G$ such that $[G : H] = 2$. 
Then $H \triangleleft G$.
\end{prop}
\begin{proof}
We have $H \triangleleft N(H) < G$. Then
$[G : H] = [G : N(H)][N(H) : H]$ (exercise, by looking at the possible
cosets like in Lagrange's theorem).
But $[G : H] = 2$. So then either $H = N(H)$ or $G = N(H)$ since 2 is
prime. In the latter case we are done because this implies 
$H \triangleleft G$. We then want to rule out $H = N(H)$.

If $H = N(H)$ then 
$$
\# \mathrm{Orb}(H) = [G : N(H)] = [G : H] = 2,
$$
so $\mathrm{Orb}(H) = \{H, H^\prime \}$ for some $H^\prime$ conjugate
to $H$. Then the conjugation action permutes $\mathrm{Orb}(H)$, so 
we have a map 
$$
\varphi : G \to \mathrm{Perm}(\{H, H^\prime\}) = S_2 = \{(1), (1 2)\}
$$
which is a group homomorphism because it is equivalent to the group
action. But since $H, H^\prime$ are conjugate by assumption, so there
exists a $g \in G$ with $H^\prime = g H g^{-1}$, so $\varphi(g) = (1 2)$
and thus $\varphi$ is surjective. Furthermore, consider
$K = \ker \varphi \triangleleft G$. Observe that $g \in \ker \varphi$
means $\varphi(g) = g H g^{-1} = H$ and thus $g \in N(H)$. Therefore
$K < N(H)$.

Therefore
\begin{itemize}
  \item{$K \triangleleft G$ and $[G : K] = 2$ since 
        $G/K \simeq \mathrm{Im}(\varphi) = S_2$,}
  \item{$K < N(H)$,}
  \item{$H = N(H) < G, [G : H] = 2$,}
\end{itemize}
so $K < N(H) = H < G$ and thus $H = N(H) = K$ and
$K \triangleleft G$ since it is the kernel of a homomorphism.
This is formally a contradiction since we assume $H = N(H)$.

Alternatively, $H < K$ and so for $h \in H$, 
$h H h^{-1} = H$ so $H \triangleleft K$.
\end{proof}

\begin{exer}
Let $H < G$ such that $[G : H] = p$ is the smallest prime number that
divides the order of the group. Then $H \triangleleft G$.
\end{exer}


\end{document}