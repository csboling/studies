\documentclass{article}

\usepackage{amsmath}
\usepackage{amsfonts}
\usepackage{mathrsfs}
\usepackage{amssymb}
\usepackage{amsthm}
\usepackage{mathtools}

\usepackage{enumitem}
\usepackage{pictexwd,dcpic}
\usepackage{hyperref}
\usepackage[lastexercise]{exercise}

\newcommand\dif{\mathop{}\!\mathrm{d}}

\newcounter{Problem}
\newenvironment{Problem}{\begin{Exercise}[counter={exer}]}
                        {\end{Exercise}}

\theoremstyle{plain}
\newtheorem{theorem}{Theorem}
\newtheorem{corol}{Corollary}
\newtheorem{lemma}{Lemma}

\theoremstyle{definition}
\newtheorem{defn}{Definition}
\newtheorem{xmpl}{Example}
\newtheorem{prop}{Proposition}

\theoremstyle{remark}
\newtheorem{remark}{Remark}
\newtheorem{obsv}{Observation}
\newtheorem{exer}{Exercise}
\newtheorem{recap}{Remark}


\begin{document}

\section{Group Actions}

\begin{defn}[Group Action]
Let $G$ be a group, $X$ be a set. Then a 
\emph{left $G$-action on $X$} is a map
$G \times X \to X$, written $(g,x) \mapsto g \cdot x$, such that
\begin{itemize}
  \item{$\forall x \in X$, $1 \cdot x = x$,}
  \item{$\forall g_1, g_2 \in G$, $\forall x \in X$, we have
        $(g_1 g_2) \cdot x = g_1 (g_2 \cdot x)$.
       }
\end{itemize}
\end{defn} 

\begin{xmpl}
\begin{enumerate}
  \item{
        Take $X = G$, $\cdot : G \times G \to G$ to be the group
        operation (called the \emph{translation action}).
       }
  \item{For $X = G$, conjugation $(g, h) \mapsto g h g^{-1}$ is a
        group action.
       }
  \item{Take $G = \mathrm{GL}_n(\mathbb{R})$, $X = \mathbb{R}^n$. For
        $v \in \mathbb{R}^n$ there is an action $A \cdot v = Av$, the
        matrix product.
       }
  \item{Take $X = \{z \in \mathbb{C} | \mathrm{Im}(z) > 0\}$, so
        $z = x + iy$ for $x,y \in \mathbb{R}$, $y > 0$ 
        (the upper-half plane). Take 
        $G = \mathrm{SL}_2(\mathbb{R})$. The action is given by
        $$
        \left[\begin{array}{c c}
          a & b \\ c & d
        \end{array}\right] \cdot z
        = \frac{az + b}{cz + d}.
        $$
        We need to check
        \begin{enumerate}
          \item{$cz + d \neq 0$,
               }
          \item{$\mathrm{Im}\left(\frac{az + b}{cz + d}\right) >
                0$. (Exercise) This can be done by writing
                $\mathrm{Im}(w) = w - \bar{w}$.
               }
          \item{We have a group action.}
        \end{enumerate}
        This group action is very important in number theory.
      }
\end{enumerate}
\end{xmpl}

\begin{remark}
Giving a $G$-action on $S$ is equivalent to giving a group
homomorphism $G \to \mathrm{Perm}(S)$. Why is this? Let
$\cdot : G \times S \to S$ be a group action. Then to this we can
associate a map $\Phi : G \to \mathrm{Perm}(S)$ by the rule
$\Phi(g)(x) = g \cdot x$. We can observe that $\Phi(g)$ is a bijection
since its inverse is given by $\Phi(g^{-1})$. 

Also, starting from $\Phi : G \to \mathrm{Perm}(S)$ we can define
$G \times S \to S$ by $g \cdot x = \Phi(g)(x)$. Observe that the types
of this isomorphism are related tensor-hom adjunction.
\end{remark}

\begin{defn}[Orbit, Stabilizer]
Fix some $x \in S$. The \emph{orbit} of $x$ is the set
$$
G \cdot x = \{ y \in S | \exists g \in G . y = g \cdot x \} 
          = \{ g \cdot x | g \in G \}.
$$

The \emph{stabilizer} of $x$ is the set 
$Gx = \{ g \in G | g \cdot x = x \}$. Observe that 
$\forall x \in S$, $Gx < G$:
\begin{align*}
         & g \cdot x = x \\
\implies & g^{-1} \cdot (g \cdot x) = g^{-1} \cdot x \\
\implies & (g^{-1} g) x = g^{-1} x \\
\implies & x = g^{-1} \cdot x,
\end{align*}
so $g^{-1} \cdot x \in Gx$.
\end{defn}

\begin{prop}
Take a group action $G \times S \to S$. There is a bijection between
the coset $G / Gx$ and the elements of the orbit of $x$, i.e.
$$
G / Gx \simeq G \cdot x.
$$
\end{prop}
\begin{proof}
Consider the map given by
$$
gGx \in G / Gx \mapsto g \cdot x \in S,
$$
where by definition $g \cdot x$ is in the orbit. This is a
well-defined map, by which we mean for $g, g^\prime \in G$ such that
$g \cdot Gx = g^\prime \cdot Gx$, $g^\prime = g \cdot h$, $h \in
Gx$. Then 
$$
g^\prime \cdot x = (g h) \cdot x 
\implies g^\prime \cdot x = g(hx) = g \cdot x.
$$
This is surjective by definition because everything in the orbit is an
image of the map given. It is injective since $g, g^\prime \in G$ such
that $g^\prime \cdot x = g \cdot x$ means $g^{-1} \cdot g^\prime) x =
x$ so $g^{-1} g^\prime \in G x$ and thus $g^\prime \in g G x$.
\end{proof}

We can therefore ``divide up'' $S$ into $G$-orbits:
$$
G = \coprod_{x \in S} G \cdot x
$$
We can also say that the $G$-action gives an equivalence relation on
the set by $x \equiv x^\prime$ when $x^\prime \in G \cdot x$, or
$x^\prime = g \cdot x$. We also observe that
$$
\# S = \sum_{x \text{ a representative}} [G : G \cdot x].
$$
For example, apply this to the action of $G$ on itself by conjugation
$$
(g, x) \mapsto g x g^{-1}.
$$
Take $x \in G$. Then the stabilizer is
$$
G x = \{ g \in G | gxg^{-1} = x \}
    = \{ g \in G | gx = xg \},
$$
the \emph{centralizer} of $x$ denoted $C_x$, while the orbit of $x$ in $G$ is
$$
\{ y = gxg^{-1} | g \in G \},
$$
the \emph{conjugacy class} of $x$ in $G$.

Note that the proposition proved above implies that $G / C_x$ is
isomorphic to the conjugacy class of $x$, that the number of elements
in the conjugacy class of $x$ is $[G : C_x]$, and that this number
always divides the order of the group. 

Observe that if $x \in Z(G)$, then $C_x = G$ or equivalently 
$[G : C_x] = 1$ and that
$$
\# G = \sum_{[G : C_x] > 1} [G : C_x] + \# Z(G).
$$
This is sometimes called the \emph{class equation}.

\begin{corol}
Suppose $\# G = p^n$, for $p$ prime, $n \geq 1$. Such a group is called a
$p$-group. Then the center $Z(G)$ is nontrivial.
\end{corol}

\begin{proof}
The class equation gives
$$
p^n = \sum_{[G : C_x] > 1} [G : C_x] + \# Z(G).
$$
But notice that each $[G : C_x] > 1$ divides
$\# G = p^n$, so $[G : C_x] = p^{m_x}$ for some $m_x \geq 1$
and then
$$
p^m = \sum p^{m_x} + \# Z(G),
$$
so $p$ divides $\# Z(G)$ and thus $Z(G)$ is nontrivial.
\end{proof}

\begin{corol}
If $G$ is a $p$-group then $G$ is solvable.
\end{corol}
\begin{proof}
We want to construct an abelian normal tower ending at $\{ 1 \}$.
\end{proof}
Start from the bottom. Since $G$ is a $p$-group, $Z(G) \neq \{1\}$, so
$$
\{1\} \triangleleft Z(G) \triangleleft G.
$$
Now look at $G^\prime = G / Z(G)$. This is still a $p$-group because
its order is some factor of the order of $G$, but it is of smaller
order. Since it is a $p$-group, $Z(G^\prime) \neq \{1\}$. We get
$$
\{ 1 \} \triangleleft Z(G^\prime) \triangleleft G^\prime = G / Z(G).
$$
This yields
$$
\{1\} 
  \triangleleft Z(G)
  \triangleleft \pi^{-1}(Z(G^\prime))
  \triangleleft G.
$$
where the quotients are all abelian since they are constructed as
centers.

Argue by induction on the exponent $n$, where $\# G = p^n$.
\begin{itemize}
  \item{ We desire
  $$
  \# G = p^n \implies G \text{ solvable}.
  $$
  }
  \item{$n=0$ is trivial.}
  \item{$n=1$ gives $\# G = p$, so $G$ is cyclic and isomorphic to
        $\mathbb{Z}_p$, and thus solvable.}
  \item{Assume the statement for all $n^\prime < n$. Look at
        $G$. We assume $Z(G) \neq \{ 1 \}$, $G / Z(G) = p^{n^\prime}$ for
        all $n^\prime < n$. We have from a previous proposition that
        if $H \triangleleft G$, $H, G/H$ solvable, then $G$ is
        solvable.
        }
\end{itemize}

\begin{xmpl}
Let $G$ be a group and $S$ be the set of all subgroups of $G$. Then
we have a group action given by conjugation:
$$
(g, H) \mapsto gHg^{-1} = \{ g h g^{-1} : h \in H \}.
$$
Let $x \in S$, so that $x = H < G$. Then $\mathrm{Orb}(x)$ is the set of
all subgroups that are conjugate to $H$, while the stabilizer $G_x$ is
$$
\{ g \in G : g \cdot x = x \} = \{ g \in G : g H g^{-1} = H \},
$$
the normalizer $N(H)$ of $H$ in $G$. Then
$$
\# \mathrm{Orb}(H) = [G : N(H)].
$$
So the number of subgroups of $G$ which are conjugate to $H$ divides
the group order $|G|$.
\end{xmpl}

\begin{prop}
Let $G$ be a group, $H < G$ such that $[G : H] = 2$. 
Then $H \triangleleft G$.
\end{prop}
\begin{proof}
We have $H \triangleleft N(H) < G$. Then
$[G : H] = [G : N(H)][N(H) : H]$ (exercise, by looking at the possible
cosets like in Lagrange's theorem).
But $[G : H] = 2$. So then either $H = N(H)$ or $G = N(H)$ since 2 is
prime. In the latter case we are done because this implies 
$H \triangleleft G$. We then want to rule out $H = N(H)$.

If $H = N(H)$ then 
$$
\# \mathrm{Orb}(H) = [G : N(H)] = [G : H] = 2,
$$
so $\mathrm{Orb}(H) = \{H, H^\prime \}$ for some $H^\prime$ conjugate
to $H$. Then the conjugation action permutes $\mathrm{Orb}(H)$, so 
we have a map 
$$
\varphi : G \to \mathrm{Perm}(\{H, H^\prime\}) = S_2 = \{(1), (1 2)\}
$$
which is a group homomorphism because it is equivalent to the group
action. But since $H, H^\prime$ are conjugate by assumption, so there
exists a $g \in G$ with $H^\prime = g H g^{-1}$, so $\varphi(g) = (1 2)$
and thus $\varphi$ is surjective. Furthermore, consider
$K = \ker \varphi \triangleleft G$. Observe that $g \in \ker \varphi$
means $\varphi(g) = g H g^{-1} = H$ and thus $g \in N(H)$. Therefore
$K < N(H)$.

Therefore
\begin{itemize}
  \item{$K \triangleleft G$ and $[G : K] = 2$ since 
        $G/K \simeq \mathrm{Im}(\varphi) = S_2$,}
  \item{$K < N(H)$,}
  \item{$H = N(H) < G, [G : H] = 2$,}
\end{itemize}
so $K < N(H) = H < G$ and thus $H = N(H) = K$ and
$K \triangleleft G$ since it is the kernel of a homomorphism.
This is formally a contradiction since we assume $H = N(H)$.

Alternatively, $H < K$ and so for $h \in H$, 
$h H h^{-1} = H$ so $H \triangleleft K$.
\end{proof}

\begin{exer}
Let $H < G$ such that $[G : H] = p$ is the smallest prime number that
divides the order of the group. Then $H \triangleleft G$.
\end{exer}

\subsection{Sylow Theorems}
Let $p$ be prime and recall that a $p$-group is a group of order
$p^n$.

\begin{defn}[$p$-Sylow subgroups]
A subgroup $H$ of $G$ is called a $p$-Sylow subgroup of $G$ if the
order of $H$ is the largest power of $p$ that divides $|G|$, that is
$|H| = p^a$ if $|G| = p^a m$, $|H| = p^a$, and $p$ does not divide $m$.
\end{defn}

\begin{xmpl}
Let $|G| = 36$. There is a 2-Sylow group of order 4 and a 3-Sylow
group of order 9. There are no other groups of Sylow type.
\end{xmpl}

\begin{theorem}[Sylow Theorems]
  Let $G$ be a finite group, $p$ prime, $p$ divides $|G|$
  ($n = |G| = p^a n^\prime$, $a \geq 1$, $p$ does not divide $n^\prime$)
  \begin{enumerate}
    \item{$G$ has at least one $p$-Sylow subgroup.
         }
    \item{All $p$-Sylow subgroups of $G$ for fixed $p$ are conjugate
          to each other. Also, every $p$-subgroup of $G$ is contained
          in a $p$-Sylow subgroup. 
         }
    \item{The number of $p$-Sylow subgroups is congruent to $1 \mod p$
          and also divides $|G|$.
         }
  \end{enumerate}
\end{theorem}

\begin{defn}[Group Exponent]
The \emph{exponent} of a group $G$ is the smallest integer $N \geq 1$
such that $a^N = 1,$ $\forall a \in G$.
\end{defn}

\begin{lemma}
If $G$ is a finite abelian group of order $n$ and $p$ is a prime that
divides $n$, then $G$ has a an element of order $p$.
\end{lemma}

\begin{proof}
First prove that if $N$ is an exponent of $G$ then $|G|$ divides a
power of $N$. Proceed by induction on $|G|$. Consider $b \in G$, $b
\neq 1$ and take $H = \langle b \rangle < G$. We have $b^N = 1$, so
$|b|$ divides $N$ and then $|H|$ divides $N$.

Look at $G / H$. This also has $N$ as an exponent, i.e.
$(aH)^N = 1 \cdot H$ and $|G / H| < |G|$. From our inductive
hypothesis we have $|G / H|$ divides a power of $N$. But
$|H|$ divides $N$ since $H$ is cyclic, and thus 
$|G| = |G / H| |H|$ which divides $N^a N = N^{a+1}$.

Now suppose $|G| = p^a m$ such that $p$ does not divide $m$
and that $N$ is the exponent of $G$. Then $|G|$ divides a power of $N$
and $p$ divides a power of $N$, so $p$ divides $N$ or $N = p
N^\prime$.

Now pick $x \in G$, $x \neq 1$. We have $x^N = 1$ so $x^{pN^\prime} =
1$ so $(x^{N^\prime})^p = 1$. Let $y = x^{N^\prime}$. Then $y^p = 1$
(either the identity or of order $p$) and there is an $x$ such that
$y = x^{N^\prime}$. We require also that $N^\prime$ is less than the
exponent of $G$ and therefore does not kill $x$.
\end{proof}

\begin{proof}[Sylow Theorems]
We proceed by application of ``counting with group actions''.
\begin{enumerate}
  \item{Proceed by induction on $|G|$. Let $n = 1$ or $n = p$. Then
        the whole group is Sylow.

        Assume all $G^\prime$ with $|G^\prime| < n$ have at least one
        $p$-Sylow subgroup. There are two cases:
        \begin{enumerate}
          \item{$G$ contains a proper subgroup $G^\prime$ with index
                $[G : G^\prime]$ prime to $p$. Then by Lagrange's
                theorem, $|G| = [G : G^\prime]|G^\prime|$ so                
                $|G^\prime| = p^a n^{\prime\prime}$ where $p$ does not
                divide $n^{\prime\prime}$. But then $|G^\prime| < n$
                and induction applies to get $H^\prime < G^\prime$
                with $|H^\prime| = p^a$. Now observe 
                $H^\prime < G^\prime < G$ and $|H^\prime| = p^a$, so 
                $H^\prime$ is $p$-Sylow.
               }
          \item{Next consider proper subgroups of $G$  with index
                divisible by $p$. Use the class equation, letting the
                group $G$ act on itself by conjugation and break it up
                into orbits (in this case conjugacy classes).                
                $$
                |G| = \sum_{x} [G : C_x]
                    = \sum_{\{x : [G : C_x] > 1\}} [G : C_x] + |Z(G)|
                $$
                where $x$ is taken by selecting a representative of
                each conjugacy class. Under our assumption, since
                these are indices of proper subgroups they are all
                divisible by $p$, so then $p$ divides $|Z(G)|$. 

                Now apply the lemma from before to the center, since
                $Z(G)$ is abelian. Then $\exists y \in Z(G)$ such that
                $y \neq 1$, $y^p = 1$. Consider
                $G^\prime = G / \langle y \rangle$, since
                $\langle y \rangle \triangleleft G$ since $y \in
                Z(G)$. Then
                $$
                \frac{|G^\prime|}{|\langle y \rangle| 
              = \frac{|G|}{p}
              = p^{a-1} n^\prime < |G|.
                $$
                Note that if $a = 1$, $p$ does not divide $|G^\prime|$,
                but then $[G : \langle y \rangle] = n^\prime$ which is
                prime to $p$. But by assumption all proper subgroups
                are divisible by $p$, and then $\langle y \rangle$
                cannot be a proper subgroup so 
                $\langle y \rangle = G$ is $p$-Sylow.

                Then applying the induction hypothesis to $G^\prime$,
                there exists an $H^\prime < G^\prime$ with 
                $|H^\prime| = p^{a-1}$. Consider the quotient map
                $\pi : G \to G / \langle y \rangle$. Then
                we have $H = \pi^{-1}(H^\prime) \to H^\prime$ and
                this induces an isomorphism
                $H / \langle y \rangle \simeq H^\prime$ since
                $\ker \pi|_{H} = \langle y \rangle$. Then
                $$
                |H| = |\langle y \rangle||H^\prime|
                    = p p^{a-1} = p^a,
                $$
                so $H$ is a $p$-Sylow of $G$.
               }
          \item{Before theorems 2 and 3, let us make some more
                comments about group actions. Let $G$ act on a set
                $X$. We write
                $$
                X^G \triangleq \{ x \in X : g \cdot x = x, 
                                  \forall g \in G \}
                $$
                to denote the \emph{fixed points} of the action. For example,
                the fixed points for a group acting on itself by
                conjugation are the elements of the center.
                \begin{lemma}
                  Suppose $G$ is a $p$-group for some prime $p$ and
                  $X$ is a finite set. Then 
                  $$
                  |X^G| \equiv |X| \mod p.
                  $$
                \end{lemma}
                \begin{proof}
                  Decompose $X$ into $G$-orbits.
                  $$
                  S =    \coprod_{x} G \cdot x
                  \simeq \coprod_{x} G / G_x,
                  $$
                  where $x$ is chosen among representatives of each
                  orbit. Then
                  $$
                  |X| = \sum_x [G : G_x] 
                      = \sum_{\{x : [G : G_x] > 1\} + |X^G|.
                  $$
                  Note that orbits with exactly one element are
                  exactly the fixed points of the action. Since
                  $G$ is a $p$-group, $[G : G_x]$ is a power of $p$
                  and so $|X| \equiv |X^G| \mod p$.
                \end{proof}
                \begin{corol}
                  If a $p$-group acts on a set whose cardinality is
                  prime to $p$, then this action has a fixed point
                  since
                  $$
                  |X^G| \equiv |X| \mod p
                  $$
                  so $|X^G|$ cannot be zero since $|X|$ is prime to $p$.
                \end{corol}
              }
          \item{Next take $G$ such that 
                $|G| = p^a n^\prime = n$, $a \geq 1$. By theorem 1
                there is a $p$-Sylow $P$ with $|P| = p^a$. Suppose
                $H < G$ is a $p$-group, $|H| = p^{a^\prime}$. We claim
                that $H$ is contained in a conjugate of $P$, i.e.
                $\exists g \in G$ such that $H < g P g^{-1}$.

                Observe that to show that $H$ is contained in $P$, it
                is enough to show $H$ is contained in the normalizer
                $N(P)$, i.e. that $H$ normalizes $P$. Assume that
                $H < N(P)$ is in the normalizer. Then recall from the
                3rd isomorphism theorem that $H P$ is a subgroup
                of $G$ and $P \triangleleft HP$, and
                $H \cap \triangleleft H$. Then
                $$
                HP / P \simeq H / (H \cap P)
                $$
                and so
                $$
                [HP : P] = [H : H \cap P]
                $$
                and then $[H : H \cap P]$ divides $|H|$, which is a
                power of $p$ and thus $[H P : P]$ is a power of $p$.
                Then $|HP|$ is a power of $p$ such that $|HP| \geq
                |P|$, but since $P$ is $p$-Sylow then
                $|P|$ is the highest power of $p$ so the orders match
                and therefore $HP < P$ implies $H < P$.
               }
          \item{
                We can now show Theorem 2, that when $p$ divides
                $|G|$, every $p$-subgroup of $G$ is contained in a
                conjugate of $P$, with $P$ a $p$-Sylow subgroup of
                $G$. Take $X$ to be the set of all subgroups of $G$
                that are conjugate to $P$:
                $$
                X = \{ g P g^{-1} : g \in G \}.
                $$
                Take $H$ to be the $p$-subgroup of the hypothesis. We
                want to show $H \in X$. Let $H$ act on $X$ by
                conjugation, so that
                $$
                h \cdot (g P g^{-1}) 
              = h (g P g^{-1}) h^{-1}
              = hg P (hg)^{-1} \in X.
                $$
                Now notice that $X$ is an orbit of the action of $g$
                on $P$ by conjugation, so
                $$
                |X| = [G : G_P] = [G : N(P)],
                $$
                because we can produce a map $G / N(P) \to X$ given by
                $g N(P) \mapsto g P g^{-1}$ which is bijective since
                conjugation of $N(P)$ does not affect the normalizer.

                Indeed, suppose $g, g^\prime$ such that
                $g P g^{-1} = g^\prime P g^\prime^{-1}$
                $$P = g^{-1} g^\prime P g^\prime^{-1} g
                    = (g^{-1} g^\prime) P (g^{-1} g^\prime)^{-1}$$
                so $g^{-1} g^\prime \in N(P)$ so $g^\prime \in g
                N(P)$, so they belong to the same coset. 

                The orbit of  an element is always in bijection with the quotient by
                the stabilizer
                Let $\cdot : G \times X \to X$ be any group
                action. The orbit
                $$
                G \cdot x = \{ g \cdot x : g \in G \}.
                $$
                and the stabilizer $G_x = \{ g \in G : g \cdot x = x
                \}$ and in general
                $$
                G / G_x \to G \cdot x
                $$
                is bijective by
                $$
                g G_x \mapsto g \cdot x.
                $$

                The action of $G$ on $X$ is transitive and the
                stabilizer $G_P$ is the normalizer $N(P)$.
                
                Furthermore $p$ does not divide $[G : N(P)]$ because
                $p$ does not divide $[G : P] = \frac{|G|}{|P|}$ and
                since $P \subset N(P) \subset G$ this means
                $[G : N(P)] | [G : P]$. Then $|X|$ is prime to $p$.
                
                Now apply
                $$
                |X| = |X^H| \mod p
                $$
                to see that $|X^H|$ is prime to $p$ and so $|X^H| \neq
                0$, so there is a fixed point. Therefore there exists
                a conjugate $P^\prime = g P g^{-1}$ of $P$ such that 
                $\forall h \in H$, $h \cdot P^\prime = P^\prime$, i.e.
                $h P^\prime h^{-1} = P^\prime$. But this means 
                $H \subset N(P^\prime)$, and we conclude that
                $H \subset P^\prime = g P g^{-1}$.

                Then every $p$-subgroup is contained in a 
                $p$-Sylow, since the conjugate of a $p$-Sylow has the
                same order and is thus $p$-Sylow. Furthermore all
                $p$-Sylows are conjugate, since a $p$-Sylow is again a
                $p$-subgroup and therefore is contained in a conjugate
                of $P$, i.e. $P^\prime \subset g P g^{-1}$
                for some $g \in G$. But $|g P g^{-1}| = p^a$ and 
                $|P^\prime| = p^a$, so $P^\prime = g P g^{-1}$.
               }
          \item{Next, prove that the number of $p$-Sylows is congruent
                to $1 \mod p$.

                We know from Theorem 2 that all $p$-Sylows are
                conjugate. The number of $p$-Sylows conjugate to a
                fixed $p$-Sylow $P$ is the number of elements in the
                orbit of $P$ under the action of conjugation, $[G : N(P)]$.
                
                Consider the action of the group $P$ on the set $X$ of
                all $p$-Sylow subgroups of $G$ by conjugation that
                maps $g P g^{-1} \mapsto y (g P g^{-1}) y^{-1}$ for
                some $y \in P$. Since $P$ is a $p$-group, we have
                $$
                |X| \equiv |X^P| \mod p,
                $$
                where $|X|$ is the number of all $p$-Sylows. But
                $$
                X^P = \{ P^\prime :  
                         \forall y \in P, y P^\prime y^{-1} = P^\prime
                      \}
                    = \{ P^\prime : P \subset N(P^\prime) \}
                    = \{ P^\prime : P \subset P^\prime \}
                    = \{ P \}.
                $$
                Therefore the action has a single fixed point and so
                $|X| \equiv 1 \mod p$ as desired.
               }
        \end{enumerate}
      }
  \item{
       }
  \item{
       }
\end{enumerate}
\end{proof}

\subsection{Problem \#5}
Transitive actions
Let $G = S_n$, $X = \{1, 2, \cdots, n\}$. Then we can get from any
element to any other element by applying the group action. This is
what is meant by the action being transitive -- the entire set $X$
lies on the stabilizer.

\begin{xmpl}
\begin{enumerate}
  \item{
    Show that a group of order 63 is not simple. 
    $(63 = 3 \cdot 3 \cdot 7)$. Recall that a \emph{simple} group is
    one with no proper normal subgroups.

    Observe that if there is only one $p$-Sylow (i.e. $N_p = 1$) then
    there is a normal subgroup of $G$. Why? Let $g \in G$, $P$ be
    the only $p$-Sylow. Then since all $p$-Sylows are conjugate, let
    $P^\prime = gPg^{-1}$ be another $p$-Sylow. But $P^\prime = P$ by
    assumption, so $gPg^{-1} = P$ and thus $P$ is normal.

    We want to show a group of order 63 has a proper normal subgroup.
    We know that the number of 3-Sylows $N_3$ should have
    $N_3 \equiv 1 \mod 3$ and $N_3$ divides 63, so $N_3$ divides
    7. Then $N_3 = 1$ or $N_3 = 7$. We know that the number of
    7-Sylows $N_7$ should have $N_7 \equiv 1 \mod 7$ and $N_7$ divides
    63, so $N_7$ divides 9, and thus $N_7 = 1$. Therefore there is a
    unique 7-Sylow normal in $G$.

    Call this 7-Sylow $H$ ($H \triangleleft G$). Then $|H| = 7$ since
    this is the highest power of $7$ that divides 63, and since $|H|$
    is prime this means 
    $H = \triangleleft a \triangleright \simeq \mathbb{Z}_7$. Then
    consider $|G / H| = 9$. This is abelian since it has a nontrivial
    center (since it is a $3$-group). 
    Then $|Z(G)| = |G|$ or $|G / Z(G)| = p$. But if $|G / Z(G)| = p$ then
    $G / Z(G)$ is cyclic, so $G$ is abelian.

    This means our group $G$ of order 63 is solvable, since $G / H$ is abelian.
  }
  \item{
    Show that a group of order $56$ is not simple.
    $|G| = 2^3 \cdot 7$, so $N_7 \equiv 1 \mod 7$ and $N_7$ divides $7
    \cdot 8$, so either $N_7 = 1$ or $N_7 = 8$. $N_2 \equiv 1 \mod 2$
    and $N_2$ divides $7 \cdot 8$, so $N_7$ divides 7 and thus $N_7 =
    1$ or $N_7 = 7$.

    Suppose $N_7 = 8$. These then have $6 \times 8 + 1 = 49$ elements
    in total. If $H, H^\prime$ are two 7-Sylows with $H \cap H^\prime
    \neq \{ 1 \}$, then $H = H^\prime$. But $H \simeq \mathbb{Z}_7
    \simeq H^\prime$, so for any $a \neq 1$ this means $a$ has to
    generate both $H$ and $H^\prime$ and then $H =
    H^\prime$. Therefore there are $56 - 49 = 7$ elements left
    out. There exists a 2-Sylow of order 8 which has to be unique
    since it just covers the remaining elements.

    We conclude that either $N_7 = 1$ or $N_2 = 1$.
  }
  \item{
    Show that a group of order $pq$, with $p$, $q$ distinct primes, is
    solvable. 

    Say $p < q$. We have $N_q \equiv 1 mod q$ and $N_q$
    divides $pq$, so since $p < q$ it is not possible that $N_q =
    q+1$, $2q + 1$, etc. Then $N_q = 1$, so this is a normal subgroup,
    call it $H$. Then there is a tower with $G / H \simeq
    \mathbb{Z}_p$ and $H / \{ 1 \} \simeq \mathbb{Z}_q$, so this tower
    is abelian.

    Since $H$ is then of prime order, let $H = \langle a \rangle$, $a^q
    = 1$. Consider a $p$-Sylow $K = \langle b \rangle$, $b^p =
    1$. Then observe
    \begin{enumerate}
      \item{$H \cap K = \{ 1 \}$,}
      \item{$H \triangleleft K$,}
      \item{$H K = G$.}
    \end{enumerate}
    Since $H$ is normal in $G$, $K$ acts on $H$ by conjugation and we
    get $\psi : K \to \mathrm{Aut}(H)$ given by 
    $\psi(k) = \lambda h . k h k^{-1}$. In this particular case we
    have
    $$
    \psi : \mathbb{Z}_p \to \mathrm{Aut}(\mathbb{Z}_q).
    $$
    We have $b a b^{-1} \in H = \langle a \rangle$, so
    $b a b^{-1} = a^i$ and then $ba = a^i b$. Knowing $i$ determines
    the group $G$:
    $$
    G = \{ a^n b^m \mid 0 \leq n < q, 0 \leq m < p \}
    $$
    so that
    $$
    a^n b^m a^{n^\prime} b^{m^\prime} = a^N b^M.
    $$
    We need to know the homomorphism $\psi$.

    We know that
    $$
    \mathrm{Aut}(\mathbb{Z}_q) \simeq \mathbb{Z}_q^\ast
    $$
    of order $q - 1$. Assume in addition that $q \not{\equiv} 1 \mod
    p$. Then
    $$
    \psi : \mathbb{Z}_p \to \mathrm{Aut}(\mathbb{Z}_q) = \mathbb{Z}_q^\ast
    $$
    has to be trivial $(\psi = \mathrm{id})$ since
    $|\mathrm{Im}(\psi)|$ divides $|\mathbb{Z}_q^\ast| = q-1$ but
    $\mathrm{Im}(\psi) \simeq \mathbb{Z}_p / \ker \psi$ so that
    $|\mathrm{Im}(\psi)|$ divides $p$. Since $p$ does not divide $q-1$
    this means $\mathrm{Im}(\psi) = \{ \mathrm{id} \}$.

    This means $i = 1$ above and so $ab = ba$. Therefore $G$ is abelian.
  }
\end{enumerate}
\end{xmpl}

\end{document}