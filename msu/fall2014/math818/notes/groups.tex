\documentclass{article}

\usepackage{amsmath}
\usepackage{amsfonts}
\usepackage{amsthm}
\usepackage{hyperref}

\theoremstyle{plain}
\newtheorem{theorem}{Theorem}

\theoremstyle{definition}
\newtheorem{defn}{Definition}
\newtheorem{xmpl}{Example}
\newtheorem{prop}{Proposition}

\theoremstyle{remark}
\newtheorem{remark}{Remark}
\newtheorem{exer}{Exercise}

\begin{document}

\section{\S 1.2 -- Groups}

\subsection{Definition}

\begin{defn}[Group]
  \label{def:group}
  A \emph{group} $(G, \cdot)$ is a monoid such that all elements have an inverse.
  That is to say
  \begin{enumerate}
    \item{$G$ is a set,
         }
    \item{$\cdot : G \times G \to G$ is an associative binary operation on $G$,
         }
    \item{There exists a unique $e \in G$ such that 
          $e \cdot g = g \cdot e = g, \forall g \in G$,
         }
    \item{For every $g \in G$, there exists a unique $g^{-1} \in G$ such that
          $g \cdot g^{-1} = g^{-1} \cdot g = e$.
         }
  \end{enumerate}
\end{defn}

\begin{xmpl}
  \begin{enumerate}
    \item{The integers $\mathbb{Z}$ form a group with $\cdot = +$.}
    \item{The rationals $\mathbb{Q}$ form a group with $\cdot = +$.}
    \item{The reals except zero $\mathbb{R}^\ast = \mathbb{R} - {0}$ 
          form a group with respect to multiplication.}
    \item{Given a set $X$, 
          $$
          \mathrm{Perm}(X) = \{f : X \to X | f \text{ is bijective}\}
          $$
          is a group with respect to composition ($\cdot = \circ$).
          }
     \item{When $X$ is a finite set $X = \{1, \dots, n\}, n \in \mathbb{N}$,
           then $\mathrm{Perm}(X) = S_n$, the symmetric group with $n$ letters.
           The order of $S_n$ is $n!$.
          }
     \item{For $n \geq 1$, the set of $n \times n$ matrices $A$ 
           with real entries such that $\det(A) \neq 0$ is a
           group with respect to multiplication of matrices, denoted
           $GL_n(\mathbb{R})$ (the \emph{general linear group}). 
           It is closed under this operation since
           $\det(A \cdot B) = \det(A) \det(B) \neq 0$. This product is
           associative since linear maps are associative, and thus so is
           multiplication of their corresponding matrices. It possesses 
           an identity element $e = I_n$ and inverses
           $A^{-1} = \frac{1}{\det A} A^{\mathrm{adj}}$, where
           the $(i, j)$ element of $A^{\mathrm{adj}}$ is $(-1)^{i + j}$ times
           the minor determinant of the $(n-1) \times (n-1)$ matrix obtained
           by erasing the $j$-th row and the $i$-th column.
           }
  \end{enumerate}
\end{xmpl}

\begin{defn}[Abelian Group]
  \label{def:abelian-group}
  A group $(G, \cdot)$ is \emph{Abelian} or \emph{commutative} when
  $\forall x, y \in G$, $x \cdot y = y \cdot x$.
\end{defn}

\begin{exer}
  For $n > 1$, show $GL_n(\mathbb{R})$ is not abelian.
  For $n > 2$, show $S_n$ is not abelian.
  Hint: It is enough to show that $S_3$ is not abelian because
  $S_n$ is a subgroup of $S_{n+1}$ for all $n$.
\end{exer}

\begin{defn}[Subgroup]
  \label{def:subgroup}
  A subset $H \subset G$ is called a \emph{subgroup} of $G$ when
  \begin{enumerate}
    \item{$e \in H$,}
    \item{$x, y \in H \implies x \cdot y \in H$,}
    \item{$x \in H \implies x^{-1} \in H.$}
  \end{enumerate}
  We write $H < G$.
\end{defn}

\begin{xmpl}
  \item{$\mathbb{Z} < \mathbb{Q}$ with $+$.}
  \item{$\{A \in GL_n(\mathbb{R}) | \det A = 1\} < GL_n(\mathbb{R})$
        because $\det (A \cdot A^{-1}) = \det I = 1$, 
        so $\det(A) = 1 \implies \det(A^{-1}) = 1$. This is the
        \emph{special linear group} $SL_n(\mathbb{R})$.
       }
\end{xmpl}

\begin{defn}[Direct Product Group]
  Given groups $G_1, G_2$, 
  $$
  G_1 \times G_2 = \{(g_1, g_2) | g_1 \in G_1, g_2 \in G_2 \}
  $$
  is a group, with product given by
  $$
  (g_1, g_2) \cdot (g_1^\prime, g_2^\prime) 
  = (g_1 \cdot_1 g_1^\prime, g_2 \cdot_2 g_2^\prime).
  $$
  (Exercise.)
\end{defn}

% 9/5
\subsection{Generalities}
Let $G$ be a group, and let $S \subset G$. Consider
$$
\langle S \rangle = \bigcap_{\substack{H < G \\ S \subset H}} H,
$$
the intersection of all subgroups that contain $S$.

\begin{prop}
$\langle S \rangle$ is a subgroup of $G$, and is the smallest
subgroup of $G$ that contains $S$. Every element of $\langle S \rangle$
can be written as a product $x_1 \cdots x_n$ with either $x_i \in S$ or
$x_i^{-1} S$, where the empty product is taken to be 1. Then
$\langle S \rangle \equiv$ the subgroup of $G$ generated by $S$.
\end{prop}

\begin{proof}
First, every intersection of any family of subgroups of a group is a
subgroup (Exercise). Observe $S \subset \langle S \rangle$, and also if
$H^\prime < G$ such that $S \subset H^\prime$ then 
$\langle S \rangle \subset H^\prime$, since 
$\langle S \rangle = \bigcap_{\substack{H < G \\ S \subset H}} H \subset H^\prime$.

Next, to show
$\langle S \rangle = \{ x_1 \cdots x_n | x_i \in S \lor x_i^{-1} \in S \},$
observe that $\{ x_1 \cdots x_n | x_i \in S \lor x_i^{-1} \in S \} < G$ and
contains $S$, and in fact each subgroup of $G$ that contains $S$ has to
contain this set, and thus so does their intersection.
\end{proof}

\begin{defn}[Generating Set, Cyclic Group]
  \begin{enumerate}
    \item{If for some $S \subset G$, $\langle S \rangle = G$, we say that
          $S$ \emph{generates} $G$.
         }
    \item{If $G$ is generated by a single element $a \in G$, we say $G$ is
          \emph{cyclic} with generator $A$. Then 
          $$
          G = \langle a \rangle = \{ \dots, a^{-2}, a^{-1}, 1, a, a^2, \dots \}.
          $$
         }
  \end{enumerate}
\end{defn}

\subsection{Homomorphisms}

\begin{defn}[Group Homomorphism]
For $G, G^\prime$ groups, a map $f : G \to G^\prime$ is a \emph{homomorphism}
when $\forall x, y \in G$, $f(x \cdot y) = f(x) \cdot f(y)$. Then
$$
f(e) = f(e \cdot e) = f(e) \cdot f(e) 
\implies e^\prime = f(e)^{-1} \cdot f(e) \cdot f(e) = f(e)
$$
and
$$
e^\prime = f(e) = f(x \cdot x^{-1}) = f(x) \cdot f(x^{-1})
\implies f(x^{-1}) = f(x)^{-1},
$$
so a group homomorphism preserves identities and inverses.
\end{defn}

\begin{prop}
Given homomorphisms $f : G \to G^\prime$ 
and $g : G^\prime \to G^{\prime\prime}$,
$g \circ f$ is a homomorphism.
\end{prop}

\begin{defn}[Group Isomorphism]
A bijective group homomorphism is called a \emph{group isomorphism}.
Then the inverse map $f^{-1} : G^\prime \to G$ is also a group isomorphism.
We say that groups $G, G^\prime$ are isomorphic if there exists an isomorphism
between them. We write $G \simeq G^\prime$.
\end{defn}

\begin{xmpl}
For $n \geq 2$, $\mathbb{Z}_n = \{ 0, 1, \dots, n-1 \}$ (the integers mod $n$)
with addition
and $G^\prime = \{ z \in \mathbb{C} | z^n = 1 \}$ with multiplication
are isomorphic, with an isomorphism $f$ given by
$$
f(a \mod n) = e^{\frac{2 \pi i a}{n}}.
$$
\end{xmpl}

Consider the automorphisms 
$\mathrm{Aut}(G) = \{f : G \to G | f \text{ is an isomorphism}\}$.
Then $\mathrm{Aut}(G)$ is a group with $\circ$. This is a subgroup
of $\mathrm{Perm}(G)$.

\begin{exer}
What is $\mathrm{Aut}(\mathbb{Z}_n)$? 
Hint: $f : \mathbb{Z}_n \to \mathbb{Z}_n$ is determined by $f(1 \mod n)$.
Consider integers prime to $n$.
\end{exer}

\begin{defn}
Let $f: G \to G^\prime$ be a group homomorphism. Then the \emph{kernel} of
$f$ $\ker(f) \equiv \{ x \in G | f(x) = e^\prime \}$ and the \emph{image} of $f$
$\mathrm{Im}(f) \equiv \{ y \in G^\prime | \exists x \in G . f(x) = y \}$.
\end{defn}

\begin{prop}
$\ker(f), \mathrm{Im}(f) < G$.
\end{prop}
\begin{proof}
\begin{align*}
  x, y \in \ker(f) 
    & \implies f(x) = e^\prime, f(y) = e^\prime \\
    & \implies f(x \cdot y) = f(x) \cdot f(y)   
                            = e^\prime \cdot e^\prime 
                            = e^\prime          \\
    & \implies x \cdot y \in \ker(f).
\end{align*}
Similarly $x \in \ker(f) \implies x^{-1} \in ker(f)$,
and $e \in ker(f)$ since homomorphisms preserve identity.
\end{proof}

\begin{prop}
$\ker(f)$ is trivial iff. $f$ is injective.
\end{prop}
\begin{proof}
$$
f(x) = f(y) \iff f(x) f(y)^{-1} = e^\prime 
            \iff f(xy^{-1}) = e^\prime 
            \iff xy^{-1} \in \ker(f).
$$
Thus if $\ker(f) = \{ e \}$ then $x = y$. The converse is similar.
\end{proof}

\begin{xmpl}
Consider the groups
$G = GL_n (\mathbb{R})$, 
$G^\prime = \mathbb{R}^\ast = \mathbb{R} - \{0\}$ with multiplication.
$\det$ is a group homomorphism, and $\ker(\det) = SL_n(\mathbb{R})$.
\end{xmpl}

\end{document}