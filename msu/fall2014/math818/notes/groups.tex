\documentclass{article}

\usepackage{amsmath}
\usepackage{amsfonts}
\usepackage{mathrsfs}
\usepackage{amssymb}
\usepackage{amsthm}
\usepackage{mathtools}

\usepackage{enumitem}
\usepackage{pictexwd,dcpic}
\usepackage{hyperref}
\usepackage[lastexercise]{exercise}

\newcommand\dif{\mathop{}\!\mathrm{d}}

\newcounter{Problem}
\newenvironment{Problem}{\begin{Exercise}[counter={exer}]}
                        {\end{Exercise}}

\theoremstyle{plain}
\newtheorem{theorem}{Theorem}
\newtheorem{corol}{Corollary}
\newtheorem{lemma}{Lemma}

\theoremstyle{definition}
\newtheorem{defn}{Definition}
\newtheorem{xmpl}{Example}
\newtheorem{prop}{Proposition}

\theoremstyle{remark}
\newtheorem{remark}{Remark}
\newtheorem{obsv}{Observation}
\newtheorem{exer}{Exercise}
\newtheorem{recap}{Remark}


\begin{document}

\section{\S 1.2 -- Groups}

\subsection{Definition}

\begin{defn}[Group]
  \label{def:group}
  A \emph{group} $(G, \cdot)$ is a monoid such that all elements have an inverse.
  That is to say
  \begin{enumerate}
    \item{$G$ is a set,
         }
    \item{$\cdot : G \times G \to G$ is an associative binary operation on $G$,
         }
    \item{There exists a unique $e \in G$ such that 
          $e \cdot g = g \cdot e = g, \forall g \in G$,
         }
    \item{For every $g \in G$, there exists a unique $g^{-1} \in G$ such that
          $g \cdot g^{-1} = g^{-1} \cdot g = e$.
         }
  \end{enumerate}
\end{defn}

\begin{xmpl}
  \begin{enumerate}
    \item{The integers $\mathbb{Z}$ form a group with $\cdot = +$.}
    \item{The rationals $\mathbb{Q}$ form a group with $\cdot = +$.}
    \item{The reals except zero $\mathbb{R}^\ast = \mathbb{R} - {0}$ 
          form a group with respect to multiplication.}
    \item{Given a set $X$, 
          $$
          \mathrm{Perm}(X) = \{f : X \to X | f \text{ is bijective}\}
          $$
          is a group with respect to composition ($\cdot = \circ$).
          }
     \item{When $X$ is a finite set $X = \{1, \dots, n\}, n \in \mathbb{N}$,
           then $\mathrm{Perm}(X) = S_n$, the symmetric group with $n$ letters.
           The order of $S_n$ is $n!$.
          }
     \item{For $n \geq 1$, the set of $n \times n$ matrices $A$ 
           with real entries such that $\det(A) \neq 0$ is a
           group with respect to multiplication of matrices, denoted
           $GL_n(\mathbb{R})$ (the \emph{general linear group}). 
           It is closed under this operation since
           $\det(A \cdot B) = \det(A) \det(B) \neq 0$. This product is
           associative since linear maps are associative, and thus so is
           multiplication of their corresponding matrices. It possesses 
           an identity element $e = I_n$ and inverses
           $A^{-1} = \frac{1}{\det A} A^{\mathrm{adj}}$, where
           the $(i, j)$ element of $A^{\mathrm{adj}}$ is $(-1)^{i + j}$ times
           the minor determinant of the $(n-1) \times (n-1)$ matrix obtained
           by erasing the $j$-th row and the $i$-th column.
           }
  \end{enumerate}
\end{xmpl}

\begin{defn}[Abelian Group]
  \label{def:abelian-group}
  A group $(G, \cdot)$ is \emph{Abelian} or \emph{commutative} when
  $\forall x, y \in G$, $x \cdot y = y \cdot x$.
\end{defn}

\begin{exer}
  For $n > 1$, show $\mathrm{GL}_n(\mathbb{R})$ is not abelian.
  For $n > 2$, show $S_n$ is not abelian.
  Hint: It is enough to show that $S_3$ is not abelian because
  $S_n$ is a subgroup of $S_{n+1}$ for all $n$.
\end{exer}

\begin{defn}[Subgroup]
  \label{def:subgroup}
  A subset $H \subset G$ is called a \emph{subgroup} of $G$ when
  \begin{enumerate}
    \item{$e \in H$,}
    \item{$x, y \in H \implies x \cdot y \in H$,}
    \item{$x \in H \implies x^{-1} \in H.$}
  \end{enumerate}
  We write $H < G$.
\end{defn}

\begin{xmpl}
  \item{$\mathbb{Z} < \mathbb{Q}$ with $+$.}
  \item{$\{A \in \mathrm{GL}_n(\mathbb{R}) | \det A = 1\} < \mathrm{GL}_n(\mathbb{R})$
        because $\det (A \cdot A^{-1}) = \det I = 1$, 
        so $\det(A) = 1 \implies \det(A^{-1}) = 1$. This is the
        \emph{special linear group} $SL_n(\mathbb{R})$.
       }
\end{xmpl}

\begin{defn}[Direct Product Group]
  Given groups $G_1, G_2$, 
  $$
  G_1 \times G_2 = \{(g_1, g_2) | g_1 \in G_1, g_2 \in G_2 \}
  $$
  is a group, with product given by
  $$
  (g_1, g_2) \cdot (g_1^\prime, g_2^\prime) 
  = (g_1 \cdot_1 g_1^\prime, g_2 \cdot_2 g_2^\prime).
  $$
  (Exercise.)
\end{defn}

% 9/5
\subsection{Generalities}
Let $G$ be a group, and let $S \subset G$. Consider
$$
\langle S \rangle = \bigcap_{\substack{H < G \\ S \subset H}} H,
$$
the intersection of all subgroups that contain $S$.

\begin{prop}
$\langle S \rangle$ is a subgroup of $G$, and is the smallest
subgroup of $G$ that contains $S$. Every element of $\langle S \rangle$
can be written as a product $x_1 \cdots x_n$ with either $x_i \in S$ or
$x_i^{-1} S$, where the empty product is taken to be 1. Then
$\langle S \rangle \triangleq$ the subgroup of $G$ generated by $S$.
\end{prop}

\begin{proof}
First, every intersection of any family of subgroups of a group is a
subgroup (Exercise). Observe $S \subset \langle S \rangle$, and also if
$H^\prime < G$ such that $S \subset H^\prime$ then 
$\langle S \rangle \subset H^\prime$, since 
$\langle S \rangle = \bigcap_{\substack{H < G \\ S \subset H}} H \subset H^\prime$.

Next, to show
$\langle S \rangle = \{ x_1 \cdots x_n | x_i \in S \lor x_i^{-1} \in S \},$
observe that $\{ x_1 \cdots x_n | x_i \in S \lor x_i^{-1} \in S \} < G$ and
contains $S$, and in fact each subgroup of $G$ that contains $S$ has to
contain this set, and thus so does their intersection.
\end{proof}

\begin{defn}[Generating Set, Cyclic Group]
  \begin{enumerate}
    \item{If for some $S \subset G$, $\langle S \rangle = G$, we say that
          $S$ \emph{generates} $G$.
         }
    \item{If $G$ is generated by a single element $a \in G$, we say $G$ is
          \emph{cyclic} with generator $A$. Then 
          $$
          G = \langle a \rangle = \{ \dots, a^{-2}, a^{-1}, 1, a, a^2, \dots \}.
          $$
         }
  \end{enumerate}
\end{defn}

\subsection{Homomorphisms}

\begin{defn}[Group Homomorphism]
For $G, G^\prime$ groups, a map $f : G \to G^\prime$ is a \emph{homomorphism}
when $\forall x, y \in G$, $f(x \cdot y) = f(x) \cdot f(y)$. Then
$$
f(e) = f(e \cdot e) = f(e) \cdot f(e) 
\implies e^\prime = f(e)^{-1} \cdot f(e) \cdot f(e) = f(e)
$$
and
$$
e^\prime = f(e) = f(x \cdot x^{-1}) = f(x) \cdot f(x^{-1})
\implies f(x^{-1}) = f(x)^{-1},
$$
so a group homomorphism preserves identities and inverses.
\end{defn}

\begin{prop}
Given homomorphisms $f : G \to G^\prime$ 
and $g : G^\prime \to G^{\prime\prime}$,
$g \circ f$ is a homomorphism.
\end{prop}

\begin{defn}[Group Isomorphism]
A bijective group homomorphism is called a \emph{group isomorphism}.
Then the inverse map $f^{-1} : G^\prime \to G$ is also a group isomorphism.
We say that groups $G, G^\prime$ are isomorphic if there exists an isomorphism
between them. We write $G \simeq G^\prime$.
\end{defn}

\begin{xmpl}
For $n \geq 2$, $\mathbb{Z}_n = \{ 0, 1, \dots, n-1 \}$ (the integers mod $n$)
with addition
and $G^\prime = \{ z \in \mathbb{C} | z^n = 1 \}$ with multiplication
are isomorphic, with an isomorphism $f$ given by
$$
f(a \mod n) = e^{\frac{2 \pi i a}{n}}.
$$
\end{xmpl}

Consider the automorphisms 
$\mathrm{Aut}(G) = \{f : G \to G | f \text{ is an isomorphism}\}$.
Then $\mathrm{Aut}(G)$ is a group with $\circ$. This is a subgroup
of $\mathrm{Perm}(G)$.

\begin{exer}
What is $\mathrm{Aut}(\mathbb{Z}_n)$? 
Hint: $f : \mathbb{Z}_n \to \mathbb{Z}_n$ is determined by $f(1 \mod n)$.
Consider integers prime to $n$.
\end{exer}

\begin{defn}
Let $f: G \to G^\prime$ be a group homomorphism. Then the \emph{kernel} of
$f$ $\ker(f) \triangleq \{ x \in G | f(x) = e^\prime \}$ and the \emph{image} of $f$
$\mathrm{Im}(f) \triangleq \{ y \in G^\prime | \exists x \in G . f(x) = y \}$.
\end{defn}

\begin{prop}
$\ker(f), \mathrm{Im}(f) < G$.
\end{prop}
\begin{proof}
\begin{align*}
  x, y \in \ker(f) 
    & \implies f(x) = e^\prime, f(y) = e^\prime \\
    & \implies f(x \cdot y) = f(x) \cdot f(y)   
                            = e^\prime \cdot e^\prime 
                            = e^\prime          \\
    & \implies x \cdot y \in \ker(f).
\end{align*}
Similarly $x \in \ker(f) \implies x^{-1} \in ker(f)$,
and $e \in ker(f)$ since homomorphisms preserve identity.
\end{proof}

\begin{prop}
$\ker(f)$ is trivial iff. $f$ is injective.
\end{prop}
\begin{proof}
$$
f(x) = f(y) \iff f(x) f(y)^{-1} = e^\prime 
            \iff f(xy^{-1}) = e^\prime 
            \iff xy^{-1} \in \ker(f).
$$
Thus if $\ker(f) = \{ e \}$ then $x = y$. The converse is similar.
\end{proof}

\begin{xmpl}
Consider the groups
$G = \mathrm{GL}_n (\mathbb{R})$, 
$G^\prime = \mathbb{R}^\ast = \mathbb{R} - \{0\}$ with multiplication.
$\det$ is a group homomorphism, and $\ker(\det) = SL_n(\mathbb{R})$.
\end{xmpl}

\subsection{Cosets}

\begin{defn}[Coset]
Let $H < G$. A \emph{left coset} of $H$ in $G$ is a subset in $G$ of the form
$$
aH \triangleq \{ ah | h \in H \}
$$
for some $a \in G$. A \emph{right coset} is defined similarly.
\end{defn}

\begin{remark}
\begin{enumerate}
  \item{Two left cosets are either disjoint or identical. Take $aH$,
        $a^\prime H$ for $a, a^\prime \in G$. Suppose 
        $aH \cap a^\prime H \neq \varnothing$. Then 
        $\exists x \in aH \cap a^\prime H$ so that 
        $x = ah_1 = a^\prime h_2$, $h_1, h_2 \in H$ so 
        $a^\prime = a h_1 h_2^{-1} \implies a^\prime \in aH$,
        so $a^\prime h \in aH \forall h \in H$. Thus 
        $a^\prime H \subset aH$. Symmetrically 
        $aH \subset a^\prime H$.
        Thus $aH = a^\prime H$.
       }
  \item{ There is a bijection $H \to aH$ given by $h \mapsto ah$. Thus
         all cosets have the same cardinality as $H$.
       }
\end{enumerate}
\end{remark}

\begin{defn}[Group Index]
The number of left cosets of $H$ in $G$, called the \emph{index} of
$H$ in $G$, is denoted $[G:H]$.
\end{defn}

\begin{corol}
\begin{enumerate}
  \item{$G$ is the disjoint union of all its left cosets.}
  \item{If $G$ is finite (meaning the underlying set is finite), then
        $|G| = [G:H] |H|$.}
  \item{If $G$ is finite and $H < G$, the order of $H$ divides the
        order of $G$ (Lagrange's theorem).}
\end{enumerate}
\end{corol}

\subsubsection{Cyclic Groups}
\begin{defn}[Cyclic Group]
$G$ is \emph{cyclic} when $\exists a \in G$ s.t. 
$G = \langle a \rangle$.
\end{defn}

\begin{prop}
\begin{enumerate}
  \item{If $G$ is infinite cyclic, then $G \simeq \mathbb{Z}$.}
  \item{If $G$ is finite cyclic, then $G \simeq \mathbb{Z}_{|G|}$.}
\end{enumerate}
\end{prop}
\begin{proof}
Pick a generator $a$ of $G$, so that 
$G = \{ a^n | n \in \mathbb{Z} \}$, and construct a group homomorphism
$f : \mathbb{Z} \to G$ given by $k \mapsto a^k$. Show if $G$ is
infinite then $f$ is an isomorphism, and if $|G| = n < \infty$ then
$f$ gives an isomorphism $\bar{f} : \mathbb{Z}_n \to G$ given by
$k \mod n \mapsto a^k$.
\end{proof}

\begin{prop}[Corollary of Lagrange's Theorem]
If $|G| = p$ for some $p$ prime, then $G$ is cyclic and so 
$G \simeq \mathbb{Z}_p$.
\end{prop}
\begin{proof}
$|G| > 1$ so $\exists a \in G. a \neq 1$. Consider 
$H = \langle a \rangle$. Then $|H|$ divides $|G| = p$ and 
$|H| \neq 1$, so $|H| = |G|$ and thus $H = G$, so 
$G \langle a \rangle$ and $G$ is cyclic. Thus $G \simeq \mathbb{Z}_p$.
\end{proof}

This fact allows us to classify some groups (up to isomorphism) from
their order alone -- this is a major goal of group theory.

\subsection{Normal Subgroups}

\begin{defn}
Let $H < G$. Then $H$ is called \emph{normal} in $G$ 
($H \triangleleft G$) when $\forall a \in G$, $aH = Ha$. 
\end{defn}

Equivalently,
\begin{itemize}
  \item{$\forall a \in G, a H a^{-1} = H$}
  \item{$\forall a \in G, a H a^{-1} \subset H$}
  \item{$\forall a \in G, a H \subset Ha$}
  \item{$\forall a \in G, a H a \subset a H$}
\end{itemize}
\begin{proof}
Note that, for instance
\begin{align*}
a H & = \{ x \in G | \exists h \in H . x = a h \}
\end{align*}
so that
\begin{align*}
a^{-1} a H & = \{ x^\prime \in G | \exists x \in a H . x^\prime =
a^{-1} x = x a^{-1} a h = h \} \\
          & = H,
\end{align*}
so we may manipulate cosets algebraically.

\begin{align*}
           a^{-1} H (a^{-1})^{-1} \subset H 
& \implies a^{-1} H a \subset H  \\
& \implies H \subset a H a^{-1}, \\
           a^{-1} H a \subset H
& \implies a^{-1} h a \in H \\
& \implies a (a^{-1} h a) a^{-1} \in a H a^{-1} \\
& \implies \forall h \in H . h \in a H a^{-1} \\
& \implies H \subset a H a^{-1}.
\end{align*}
\end{proof}

\begin{remark}
\begin{enumerate}
  \item{Normality is a relative notion to the ambient group.}
  \item{If $G$ is abelian, all subgroups are normal.}
\end{enumerate}
\end{remark}

\begin{prop}
Let $f : G \to G^\prime$ be a group homomorphism. Then 
$\ker(f) \triangleleft G$.
\end{prop}
\begin{proof}
Show $a \ker(f) a^{-1} \subset \ker(f), \forall G$.
Let $x \in \ker(f).$ Then 
$f(a x a^{-1}) = f(a) f(x) f(a)^{-1} = 1$, so
$a x a^{-1} \in \ker(f)$.
\end{proof}

\begin{xmpl}
\begin{enumerate}
  \item{$G = \mathrm{\mathrm{GL}}_n(\mathbb{R})$, $\det : G \to \mathbb{R}^\ast$.
        $\ker(\det) = \mathrm{SL}_n(\mathbb{R}) 
                    \triangleleft \mathrm{\mathrm{GL}}_n(\mathbb{R})$.
       }
  \item{The \emph{center} of $G$, $Z(G)$ is defined to be
        $$
        Z(G) = \{ z \in G | z g = g z, \forall g \in G \}.
        $$
        $Z(G) \triangleleft G$, since $g Z(G) g^{-1} \subset Z(G)$
        since $g z g^{-1} = z g g^{-1} = z \in Z(G)$. If 
        $Z(G) = \{1\}$, then $G$ is said to be \emph{perfect}.
        }
\end{enumerate}
\end{xmpl}

\end{document}