\documentclass{article}

\usepackage{amsmath}
\usepackage{amsfonts}
\usepackage{mathrsfs}
\usepackage{amssymb}
\usepackage{amsthm}
\usepackage{mathtools}

\usepackage{enumitem}
\usepackage{pictexwd,dcpic}
\usepackage{hyperref}
\usepackage[lastexercise]{exercise}

\newcommand\dif{\mathop{}\!\mathrm{d}}

\newcounter{Problem}
\newenvironment{Problem}{\begin{Exercise}[counter={exer}]}
                        {\end{Exercise}}

\theoremstyle{plain}
\newtheorem{theorem}{Theorem}
\newtheorem{corol}{Corollary}
\newtheorem{lemma}{Lemma}

\theoremstyle{definition}
\newtheorem{defn}{Definition}
\newtheorem{xmpl}{Example}
\newtheorem{prop}{Proposition}

\theoremstyle{remark}
\newtheorem{remark}{Remark}
\newtheorem{obsv}{Observation}
\newtheorem{exer}{Exercise}
\newtheorem{recap}{Remark}


\begin{document}

\section{\S 1.2 -- Groups}

\subsection{Definition}

\begin{defn}[Group]
  \label{def:group}
  A \emph{group} $(G, \cdot)$ is a monoid such that all elements have an inverse.
  That is to say
  \begin{enumerate}
    \item{$G$ is a set,
         }
    \item{$\cdot : G \times G \to G$ is an associative binary operation on $G$,
         }
    \item{There exists a unique $e \in G$ such that 
          $e \cdot g = g \cdot e = g, \forall g \in G$,
         }
    \item{For every $g \in G$, there exists a unique $g^{-1} \in G$ such that
          $g \cdot g^{-1} = g^{-1} \cdot g = e$.
         }
  \end{enumerate}
\end{defn}

\begin{xmpl}
  \begin{enumerate}
    \item{The integers $\mathbb{Z}$ form a group with $\cdot = +$.}
    \item{The rationals $\mathbb{Q}$ form a group with $\cdot = +$.}
    \item{The reals except zero $\mathbb{R}^\ast = \mathbb{R} - {0}$ 
          form a group with respect to multiplication.}
    \item{Given a set $X$, 
          $$
          \mathrm{Perm}(X) = \{f : X \to X | f \text{ is bijective}\}
          $$
          is a group with respect to composition ($\cdot = \circ$).
          }
     \item{When $X$ is a finite set $X = \{1, \dots, n\}, n \in \mathbb{N}$,
           then $\mathrm{Perm}(X) = S_n$, the symmetric group with $n$ letters.
           The order of $S_n$ is $n!$.
          }
     \item{For $n \geq 1$, the set of $n \times n$ matrices $A$ 
           with real entries such that $\det(A) \neq 0$ is a
           group with respect to multiplication of matrices, denoted
           $GL_n(\mathbb{R})$ (the \emph{general linear group}). 
           It is closed under this operation since
           $\det(A \cdot B) = \det(A) \det(B) \neq 0$. This product is
           associative since linear maps are associative, and thus so is
           multiplication of their corresponding matrices. It possesses 
           an identity element $e = I_n$ and inverses
           $A^{-1} = \frac{1}{\det A} A^{\mathrm{adj}}$, where
           the $(i, j)$ element of $A^{\mathrm{adj}}$ is $(-1)^{i + j}$ times
           the minor determinant of the $(n-1) \times (n-1)$ matrix obtained
           by erasing the $j$-th row and the $i$-th column.
           }
  \end{enumerate}
\end{xmpl}

\begin{defn}[Abelian Group]
  \label{def:abelian-group}
  A group $(G, \cdot)$ is \emph{Abelian} or \emph{commutative} when
  $\forall x, y \in G$, $x \cdot y = y \cdot x$.
\end{defn}

\begin{exer}
  For $n > 1$, show $\mathrm{GL}_n(\mathbb{R})$ is not abelian.
  For $n > 2$, show $S_n$ is not abelian.
  Hint: It is enough to show that $S_3$ is not abelian because
  $S_n$ is a subgroup of $S_{n+1}$ for all $n$.
\end{exer}

\begin{defn}[Subgroup]
  \label{def:subgroup}
  A subset $H \subset G$ is called a \emph{subgroup} of $G$ when
  \begin{enumerate}
    \item{$e \in H$,}
    \item{$x, y \in H \implies x \cdot y \in H$,}
    \item{$x \in H \implies x^{-1} \in H.$}
  \end{enumerate}
  We write $H < G$.
\end{defn}

\begin{xmpl}
  \item{$\mathbb{Z} < \mathbb{Q}$ with $+$.}
  \item{$\{A \in \mathrm{GL}_n(\mathbb{R}) | \det A = 1\} < \mathrm{GL}_n(\mathbb{R})$
        because $\det (A \cdot A^{-1}) = \det I = 1$, 
        so $\det(A) = 1 \implies \det(A^{-1}) = 1$. This is the
        \emph{special linear group} $SL_n(\mathbb{R})$.
       }
\end{xmpl}

\begin{defn}[Direct Product Group]
  Given groups $G_1, G_2$, 
  $$
  G_1 \times G_2 = \{(g_1, g_2) | g_1 \in G_1, g_2 \in G_2 \}
  $$
  is a group, with product given by
  $$
  (g_1, g_2) \cdot (g_1^\prime, g_2^\prime) 
  = (g_1 \cdot_1 g_1^\prime, g_2 \cdot_2 g_2^\prime).
  $$
  (Exercise.)
\end{defn}

% 9/5
\subsection{Generalities}
Let $G$ be a group, and let $S \subset G$. Consider
$$
\langle S \rangle = \bigcap_{\substack{H < G \\ S \subset H}} H,
$$
the intersection of all subgroups that contain $S$.

\begin{prop}
$\langle S \rangle$ is a subgroup of $G$, and is the smallest
subgroup of $G$ that contains $S$. Every element of $\langle S \rangle$
can be written as a product $x_1 \cdots x_n$ with either $x_i \in S$ or
$x_i^{-1} S$, where the empty product is taken to be 1. Then
$\langle S \rangle \equiv$ the subgroup of $G$ generated by $S$.
\end{prop}

\begin{proof}
First, every intersection of any family of subgroups of a group is a
subgroup (Exercise). Observe $S \subset \langle S \rangle$, and also if
$H^\prime < G$ such that $S \subset H^\prime$ then 
$\langle S \rangle \subset H^\prime$, since 
$\langle S \rangle = \bigcap_{\substack{H < G \\ S \subset H}} H \subset H^\prime$.

Next, to show
$\langle S \rangle = \{ x_1 \cdots x_n | x_i \in S \lor x_i^{-1} \in S \},$
observe that $\{ x_1 \cdots x_n | x_i \in S \lor x_i^{-1} \in S \} < G$ and
contains $S$, and in fact each subgroup of $G$ that contains $S$ has to
contain this set, and thus so does their intersection.
\end{proof}

\begin{defn}[Generating Set, Cyclic Group]
  \begin{enumerate}
    \item{If for some $S \subset G$, $\langle S \rangle = G$, we say that
          $S$ \emph{generates} $G$.
         }
    \item{If $G$ is generated by a single element $a \in G$, we say $G$ is
          \emph{cyclic} with generator $A$. Then 
          $$
          G = \langle a \rangle = \{ \dots, a^{-2}, a^{-1}, 1, a, a^2, \dots \}.
          $$
         }
  \end{enumerate}
\end{defn}

\subsection{Homomorphisms}

\begin{defn}[Group Homomorphism]
For $G, G^\prime$ groups, a map $f : G \to G^\prime$ is a \emph{homomorphism}
when $\forall x, y \in G$, $f(x \cdot y) = f(x) \cdot f(y)$. Then
$$
f(e) = f(e \cdot e) = f(e) \cdot f(e) 
\implies e^\prime = f(e)^{-1} \cdot f(e) \cdot f(e) = f(e)
$$
and
$$
e^\prime = f(e) = f(x \cdot x^{-1}) = f(x) \cdot f(x^{-1})
\implies f(x^{-1}) = f(x)^{-1},
$$
so a group homomorphism preserves identities and inverses.
\end{defn}

\begin{prop}
Given homomorphisms $f : G \to G^\prime$ 
and $g : G^\prime \to G^{\prime\prime}$,
$g \circ f$ is a homomorphism.
\end{prop}

\begin{defn}[Group Isomorphism]
A bijective group homomorphism is called a \emph{group isomorphism}.
Then the inverse map $f^{-1} : G^\prime \to G$ is also a group isomorphism.
We say that groups $G, G^\prime$ are isomorphic if there exists an isomorphism
between them. We write $G \simeq G^\prime$.
\end{defn}

\begin{xmpl}
For $n \geq 2$, $\mathbb{Z}_n = \{ 0, 1, \dots, n-1 \}$ (the integers mod $n$)
with addition
and $G^\prime = \{ z \in \mathbb{C} | z^n = 1 \}$ with multiplication
are isomorphic, with an isomorphism $f$ given by
$$
f(a \mod n) = e^{\frac{2 \pi i a}{n}}.
$$
\end{xmpl}

Consider the automorphisms 
$\mathrm{Aut}(G) = \{f : G \to G | f \text{ is an isomorphism}\}$.
Then $\mathrm{Aut}(G)$ is a group with $\circ$. This is a subgroup
of $\mathrm{Perm}(G)$.

\begin{exer}
What is $\mathrm{Aut}(\mathbb{Z}_n)$? 
Hint: $f : \mathbb{Z}_n \to \mathbb{Z}_n$ is determined by $f(1 \mod n)$.
Consider integers prime to $n$.
\end{exer}

\begin{defn}
Let $f: G \to G^\prime$ be a group homomorphism. Then the \emph{kernel} of
$f$ $\ker(f) \triangleq \{ x \in G | f(x) = e^\prime \}$ and the \emph{image} of $f$
$\mathrm{Im}(f) \triangleq \{ y \in G^\prime | \exists x \in G . f(x) = y \}$.
\end{defn}

\begin{prop}
$\ker(f), \mathrm{Im}(f) < G$.
\end{prop}
\begin{proof}
\begin{align*}
  x, y \in \ker(f) 
    & \implies f(x) = e^\prime, f(y) = e^\prime \\
    & \implies f(x \cdot y) = f(x) \cdot f(y)   
                            = e^\prime \cdot e^\prime 
                            = e^\prime          \\
    & \implies x \cdot y \in \ker(f).
\end{align*}
Similarly $x \in \ker(f) \implies x^{-1} \in ker(f)$,
and $e \in ker(f)$ since homomorphisms preserve identity.
\end{proof}

\begin{prop}
$\ker(f)$ is trivial iff. $f$ is injective.
\end{prop}
\begin{proof}
$$
f(x) = f(y) \iff f(x) f(y)^{-1} = e^\prime 
            \iff f(xy^{-1}) = e^\prime 
            \iff xy^{-1} \in \ker(f).
$$
Thus if $\ker(f) = \{ e \}$ then $x = y$. The converse is similar.
\end{proof}

\begin{xmpl}
Consider the groups
$G = \mathrm{GL}_n (\mathbb{R})$, 
$G^\prime = \mathbb{R}^\ast = \mathbb{R} - \{0\}$ with multiplication.
$\det$ is a group homomorphism, and $\ker(\det) = SL_n(\mathbb{R})$.
\end{xmpl}

\subsection{Cosets}

\begin{defn}[Coset]
Let $H < G$. A \emph{left coset} of $H$ in $G$ is a subset in $G$ of the form
$$
aH \triangleq \{ ah | h \in H \}
$$
for some $a \in G$. A \emph{right coset} is defined similarly.
\end{defn}

\begin{remark}
\begin{enumerate}
  \item{Two left cosets are either disjoint or identical. Take $aH$,
        $a^\prime H$ for $a, a^\prime \in G$. Suppose 
        $aH \cap a^\prime H \neq \varnothing$. Then 
        $\exists x \in aH \cap a^\prime H$ so that 
        $x = ah_1 = a^\prime h_2$, $h_1, h_2 \in H$ so 
        $a^\prime = a h_1 h_2^{-1} \implies a^\prime \in aH$,
        so $a^\prime h \in aH \forall h \in H$. Thus 
        $a^\prime H \subset aH$. Symmetrically 
        $aH \subset a^\prime H$.
        Thus $aH = a^\prime H$.
       }
  \item{ There is a bijection $H \to aH$ given by $h \mapsto ah$. Thus
         all cosets have the same cardinality as $H$.
       }
\end{enumerate}
\end{remark}

\begin{defn}[Group Index]
The number of left cosets of $H$ in $G$, called the \emph{index} of
$H$ in $G$, is denoted $[G:H]$.
\end{defn}

\begin{corol}
\begin{enumerate}
  \item{$G$ is the disjoint union of all its left cosets.}
  \item{If $G$ is finite (meaning the underlying set is finite), then
        $|G| = [G:H] |H|$.}
  \item{If $G$ is finite and $H < G$, the order of $H$ divides the
        order of $G$ (Lagrange's theorem).}
\end{enumerate}
\end{corol}

\subsubsection{Cyclic Groups}
\begin{defn}[Cyclic Group]
$G$ is \emph{cyclic} when $\exists a \in G$ s.t. 
$G = \langle a \rangle$.
\end{defn}

\begin{prop}
\begin{enumerate}
  \item{If $G$ is infinite cyclic, then $G \simeq \mathbb{Z}$.}
  \item{If $G$ is finite cyclic, then $G \simeq \mathbb{Z}_{|G|}$.}
\end{enumerate}
\end{prop}
\begin{proof}
Pick a generator $a$ of $G$, so that 
$G = \{ a^n | n \in \mathbb{Z} \}$, and construct a group homomorphism
$f : \mathbb{Z} \to G$ given by $k \mapsto a^k$. Show if $G$ is
infinite then $f$ is an isomorphism, and if $|G| = n < \infty$ then
$f$ gives an isomorphism $\bar{f} : \mathbb{Z}_n \to G$ given by
$k \mod n \mapsto a^k$.
\end{proof}

\begin{prop}[Corollary of Lagrange's Theorem]
If $|G| = p$ for some $p$ prime, then $G$ is cyclic and so 
$G \simeq \mathbb{Z}_p$.
\end{prop}
\begin{proof}
$|G| > 1$ so $\exists a \in G. a \neq 1$. Consider 
$H = \langle a \rangle$. Then $|H|$ divides $|G| = p$ and 
$|H| \neq 1$, so $|H| = |G|$ and thus $H = G$, so 
$G \langle a \rangle$ and $G$ is cyclic. Thus $G \simeq \mathbb{Z}_p$.
\end{proof}

This fact allows us to classify some groups (up to isomorphism) from
their order alone -- this is a major goal of group theory.

\subsection{Normal Subgroups}

\begin{defn}
Let $H < G$. Then $H$ is called \emph{normal} in $G$ 
($H \triangleleft G$) when $\forall a \in G$, $aH = Ha$. 
\end{defn}

Equivalently,
\begin{itemize}
  \item{$\forall a \in G, a H a^{-1} = H$}
  \item{$\forall a \in G, a H a^{-1} \subset H$}
  \item{$\forall a \in G, a H \subset Ha$}
  \item{$\forall a \in G, a H a \subset a H$}
\end{itemize}
\begin{proof}
Note that, for instance
\begin{align*}
a H & = \{ x \in G | \exists h \in H . x = a h \}
\end{align*}
so that
\begin{align*}
a^{-1} a H & = \{ x^\prime \in G | \exists x \in a H . x^\prime =
a^{-1} x = x a^{-1} a h = h \} \\
          & = H,
\end{align*}
so we may manipulate cosets algebraically.

\begin{align*}
           a^{-1} H (a^{-1})^{-1} \subset H 
& \implies a^{-1} H a \subset H  \\
& \implies H \subset a H a^{-1}, \\
           a^{-1} H a \subset H
& \implies a^{-1} h a \in H \\
& \implies a (a^{-1} h a) a^{-1} \in a H a^{-1} \\
& \implies \forall h \in H . h \in a H a^{-1} \\
& \implies H \subset a H a^{-1}.
\end{align*}
\end{proof}

\begin{remark}
\begin{enumerate}
  \item{Normality is a relative notion to the ambient group.}
  \item{If $G$ is abelian, all subgroups are normal.}
\end{enumerate}
\end{remark}

\begin{prop}
Let $f : G \to G^\prime$ be a group homomorphism. Then 
$\ker(f) \triangleleft G$.
\end{prop}
\begin{proof}
Show $a \ker(f) a^{-1} \subset \ker(f), \forall G$.
Let $x \in \ker(f).$ Then 
$f(a x a^{-1}) = f(a) f(x) f(a)^{-1} = 1$, so
$a x a^{-1} \in \ker(f)$.
\end{proof}

\begin{xmpl}
\begin{enumerate}
  \item{$G = \mathrm{\mathrm{GL}}_n(\mathbb{R})$, $\det : G \to \mathbb{R}^\ast$.
        $\ker(\det) = \mathrm{SL}_n(\mathbb{R}) 
                    \triangleleft \mathrm{\mathrm{GL}}_n(\mathbb{R})$.
       }
  \item{The \emph{center} of $G$, $Z(G)$ is defined to be
        $$
        Z(G) = \{ z \in G | z g = g z, \forall g \in G \}.
        $$
        $Z(G) \triangleleft G$, since $g Z(G) g^{-1} \subset Z(G)$
        since $g z g^{-1} = z g g^{-1} = z \in Z(G)$. If 
        $Z(G) = \{1\}$, then $G$ is said to be \emph{perfect}.
        }
\end{enumerate}
\end{xmpl}

% 9/10/2014

\begin{defn}[Affine Map]
An \emph{affine map} is a sum of linear maps with a translation. In
the case $\mathbb{R} \to \mathbb{R}$, the functions $f(x) = ax + b$,
$a \neq 0$ is affine.
\end{defn}

\begin{xmpl}[Normal Subgroup]
Let $G$ be the set of all non-constant \emph{affine} maps 
$\mathbb{R} \to \mathbb{R}$ with operation $\circ$. That is,
$$
G = \{ f : \mathbb{R} \to \mathbb{R} | f(x) = ax + b, 
                                       a \neq 0,
                                       a, b \in \mathbb{R}
    \}.
$$
For simplicity, set $L_{a,b}(x) = ax + b$. There are two standard
subgroups of $G$. 

$A \triangleq \{ f | f(x) = ax, a \neq 0 \} = \{L_{a, 0} | a \neq 0\}$
is a subgroup of $G$ which is isomorphic to the multiplicative group
$\mathbb{R}^\ast$.

$T \triangleq \{ f | f(x) = x + b \} = \{ L_{1,b} | b \in \mathbb{R}
\}$ is a subgroup of $G$. 

These subgroups are normal in $G$. Consider the map 
$f : G \to \mathbb{R}^\ast \simeq A$ given by 
$$
L_{a,b} \mathrel{\mathop{\mapsto}^{f}} a
$$. 
This is a group homomorphism because
\begin{align*}
(L_{a,b} \circ L_{a^\prime, b^\prime})(x) 
 & = L_{a,b}(L_{a^\prime, b^\prime}(x)) \\
 & = L_{a,b}(a^\prime x + b^\prime) \\
 & = a(a^\prime x + b^\prime) + b \\
 & = aa^\prime x + ab^\prime + b \\
 & = L_{a a^\prime, a b^\prime + b}(x),
\end{align*}
so $L_{a,b} \circ L_{a^\prime, b^\prime} = L_{a a^\prime, a b^\prime +
  b}$, and thus
$$
L_{a,b} \circ L_{a^\prime, b^\prime} \mathrel{\mathop{\mapsto}^{f}} a a^\prime,
$$ 
so this is a group homomorphism. The kernel is then 
$\{L_{1,b} | b \in \mathbb{R} \} = T$, so $T$ is normal in $G$.
\end{xmpl}

Note that
$$
G \simeq \left\{
\left.\left(\begin{array}{c c}
a & b \\ 0 & 1
\end{array}\right)\right| a, b \in R, a \neq 0
\right\} < GL_2(\mathbb{R})
$$
by the group homomorphism
$$
L_{a,b} \mapsto \left(\begin{array}{c c}
a & b \\ 0 & 1
\end{array}\right).
$$

\subsection{Quotient Groups and Isomorphism Theorems}

\begin{defn}[Quotient Group]
Let $G$ be a group and $H \triangleleft G$, so that $aHa^{-1} = H$,
$\forall a \in G$, or equivalently $aH = Ha$.
The \emph{quotient group} is the set of all (left = right) cosets
$$
G / H = \{ aH | a \in G \}
$$
with a natural operation induced by the operation in the group:
$$
aH \cdot bH \triangleq aHbH
$$
or for $S, T \subset G$,
$$
S \cdot T \{ s \cdot t | s \in S, t \in T \}.
$$
This is indeed another coset because, by normality of $H$,
$aHbH = abHH = abH$. We need to check that it is associative, contains
identity ($H$), and contains all inverses ($(aH)^{-1} = a^{-1}H$). 
Similarly there is the
\emph{canonical homomorphism} $\pi : G \to G/H$ given by 
$\pi(a) = aH$. $\pi$ is surjective and
$\ker(\pi) = \{ a \in G | aH = H \} = H$, so $\pi$ is an isomorphism.
\end{defn}

Note that this quotient can also be written as a quotient by an
equivalence relation $G/\sim$ where
$$
a \sim b \triangleq aH = bH
$$
or by a function as $G/\pi$.

\subsection{Isomorphism Theorems}

\begin{theorem}[First Isomorphism Theorem]
Let $f : G \to G^\prime$ be a group homomorphism. There are two
associated subgroups $\mathrm{Im}(f) < G^\prime$ and 
$\ker(f) \triangleleft G$. Then
$G / \ker(f) \simeq \mathrm{Im}(f)$, with an isomorphism given by
$a \ker(f) \mapsto f(a)$.
\end{theorem}
\begin{proof}
Set $\bar{f}(a \ker (f)) = f(a)$. We check that $\bar{f}$ is a
well-defined map $G / \ker(f) \to \mathrm{Im}(f)$ by showing that
$a \ker f = b \ker f \implies f(a) = f(b)$. This condition however
implies that 
$$
b^{-1} a \in \ker(f) 
  \implies f(b^{-1} a) = 1
  \implies f(b)^{-1} f(a) = 1
  \implies f(a) = f(b).
$$

Next, we prove $\bar{f}$ is a group homomorphism.
$$
\bar{f}(a \ker f b \ker f) 
  = \bar{f} (a b \ker f) 
  = f(ab) 
  = f(a)f(b)
  = \bar{f}(a \ker f) \bar{f}(b \ker f)
$$
since $f$ is a homomorphism.

$\bar{f}$ is surjective by definition since it is onto
$\mathrm{Im}(f)$.

\begin{align*}
\ker(\bar{f}) & = \{ a \ker f | \bar{f}(a \ker f) = f(a) = 1 \} \\
              & = \{ a \ker f | a \in \ker f \} \\
              & \simeq \{ \ker f \},
\end{align*}
with the isomorphism given by 
$\left(\pi|_{\ker f}\right)^{-1}$,
the inverse of the restriction to $\ker f$ 
of the canonical isomorphism $\pi : G \to G / \ker f$.
This means by definition that $\ker(\bar{f})$ has the same cardinality
as a singleton set so it must be trivial, thus $\bar{f}$ is injective.
\end{proof}

\begin{theorem}[Second Isomorphism Theorem]
Let $G$ be a group, $K \subset H$, and both $K$ and $H$ normal in
$G$. Then $K$ is normal in $H$ and we can consider $H / K$, which is
then a normal subgroup in $G / K$. Then $G / H \simeq (G / K) / (H / K)$.
\end{theorem}
\begin{proof}
$K \triangleleft G \iff \forall g \in G, g K g^{-1} = K$. Since
$H \subset G$, $h K h^{-1} = K$, so $K \triangleleft H$. Consider the
canonical maps $\pi_K : G \to G / K$ and
$\pi_H : G \to G / H$. We claim that there exists a homomorphism
$\varphi : G / K \to G / H$ given by $\varphi(gK) = gH$, and that
$\varphi \circ \pi_K = \pi_H$. (Exercise.) Note that $\varphi$ is
surjective and 
$$
\ker(\varphi) 
  = \{ gK | gH = H \} 
  = \{ gK | g \in H \}
  = H / K.
$$
Then $H / K$ is normal in $G / K$, as the kernel of a group
homomorphism, and from the first isomorphism theorem
$(G / K) / \ker(\varphi) = \mathrm{Im}(\varphi)$
or $(G / H) / (H / K) \simeq G / H$.
\end{proof}

\begin{theorem}[Third Isomorphism Theorem]
Let $H < G$, $K < G$ such that $H$ \emph{normalizes} $K$, i.e.
$\forall h \in H, h K h^{-1} = K$ or
$\forall h \in H, h K = K h$. Then
$$
H \cap K \triangleleft H, 
\quad HK < G,
\quad H / (H \cap K) \simeq HK / K.
$$
\end{theorem}
\begin{proof} There are three assertions to prove.
\begin{itemize}
  \item[($H \cap K \triangleleft H$.)]{
        $h \in H$, $\kappa \in H \cap K$. Then
        $h \kappa h^{-1} \in H$ since $h, \kappa \in H$
        and $h \kappa h^{-1} \in K$ since $h \in H$, $H$ normalizes
        $K$.
        Then $h \kappa h^{-1} \in H \cap K$.
       }
  \item[($HK < G$)]{
        $h_1 \kappa_1 h_2 \kappa_2 
       = h_1 h_2 \kappa_1^\prime \kappa_2
       \in HK$, since $H$ normalizes $K$ and thus $Kh = hK$. 
       This proves $HK$ is closed.

       $(h \kappa)^{-1} = \kappa^{-1} h^{-1} = h^{-1} \kappa^\prime \in HK$, since
       $H$ normalizes $K$ and thus $K h^{-1} = h^{-1} K$.

       Also $K \triangleleft HK$. Indeed, 
       $$
       h \kappa \kappa^\prime (h \kappa)^{-1} 
     = h \kappa \kappa^\prime \kappa^{-1} h^{-1} \in K,
       $$
       since $\kappa \kappa^\prime \kappa^{-1} \in K$ and
       $h K h^{-1} \subset K$.
       }
       \item[($H / H \cap K \simeq HK / K$)]{
             Consider $H \hookrightarrow HK \to^{\pi} HK / K$ that behaves as
             $$
             h \mapsto h \cdot 1 = h \mapsto h K.
             $$
             Notice that $\varphi = \pi \circ j$ is a homomorphism, and is
             surjective since $h \kappa K = h K$, and has the kernel
             $$
             \ker \varphi = \{ h | hK = K \} \implies
             \ker \varphi = \{ h \in H | h \in K \} = H \cap K.
             $$
             Then $H / \ker \varphi \simeq \mathrm{Im} \varphi$, which
             is exactly
             $H / H \cap K \simeq HK / K$.
            }
\end{itemize}
\end{proof}

\begin{defn}[Normalizer]
Let $G$ be a group, $K < G$. Then the \emph{normalizer}
$N_K$ of $K$ in G is
$$
N_K = \{ g \in G | g K g^{-1} = K \}.
$$
\end{defn}

\begin{exer}
Show that
\begin{itemize}
  \item{$K < N_K < G$.}
  \item{$K \triangleleft N_K$, and in fact $N_K$ is the largest
      subgroup of $G$ that contains $K$ such that $K$ is normal in
      that subgroup.}
\end{itemize}

\end{exer}

\begin{remark}
The assumption in the 3rd isomorphism theorem can be written $H < N_K$.
\end{remark}

\subsubsection{Special Case}
\begin{defn}[Semidirect Product]
Let $K \triangleleft G$, $H \cap K = \{ 1 \}$, and $HK = G$.
Then $G$ is the \emph{semidirect product} of $H$ and $K$ and we write
$G = H \ltimes K$.
\end{defn}


The third isomorphism theorem applies in this situation to give
$$
HK / K \simeq H / H \cap K \implies
G / K  \simeq H / \{ 1 \} \simeq H.
$$

If $H$ normalizes $K$, then
$h \in H, k \in K$ gives $h k h^{-1} = k^\prime \in K$. This defnes a
group homomorphism $\psi : H \to \mathrm{Aut}(K)$ given by
$\psi(h)(k) = h k h^{-1} \in K$. We call $\psi(h)$ the
\emph{conjugation} by $h$. Now describe the group operation on 
$G = HK$, using the operations of $H$ and $K$ and the homomorphism $\psi$.

Take $g_1, g_2 \in G$. Then $g_1 = h_1 k_1$, $g_2 = h_2 k_2$ Is this
unique? Can we write $g = hk = h^\prime k^\prime$? This would mean
$$
hk = h^\prime k^\prime 
  \implies k = h^{-1} h^\prime k^\prime
  \implies k (k^\prime)^{-1} = h^{-1} h^\prime,
$$
but since $H \cap K$ is trivial this means
$hk = h^\prime k^\prime$.

$$g_1 \cdot g_2 = h_1 k_1 h_2 k_2 = h_1 h_2 k_1^\prime k_2$$
because $k_1 h_2 = h_2 k_1^\prime$ because $kH = Hk$, so
\begin{align*}
h_2^{-1} k_1 h_2 = k_1^\prime
& \implies
  \psi(h_2^{-1})(k_1) = k_1^\prime \\
& \implies
  \psi(h_2)^{-1}(k_1) = k_1^\prime \\
& \implies g_1 g_2 = h_1 h_2 \psi(h_2)^{-1} (k_1) k_2.
\end{align*}

% 9/15

We can express this construction differently. Start with
$H, K$ groups and a homomorphism $\psi : H \to \mathrm{Aut}(K)$.
Consider the direct product 
$$
K \times H = \{(k, h) | k \in K, h \in H \}
$$
as a set but with the operation
$$
(k_1, h_1) \cdot (k_2, h_2) = (k_1 \psi(h_1)(k_2), h_1 h_2).
$$
Call this $G$.

\begin{exer}
Show that $G$ is a group with this operation. This is an alternative
definition of the semidirect product $K \rtimes_\psi H$. Show also
that the previously defined $G$ is isomorphic to this. (Exercise 12 in Lang).
\end{exer}

\begin{xmpl}[Semidirect Products]
\begin{enumerate}
  \item{$G = S_3$, the symmetric group in three letters (permutations
      of $\{1,2,3\}$. We have elements $\sigma = (1 2 3)$ and
      $\tau = (1 2)$ where these notations indicate $1 \to 2 \to 3 \to
      1$ and $1 \to 2 \to 1, 3 \to 3$, respectively. Consider
      $K = \langle \sigma \rangle = \{1, \sigma, \sigma^2\}$ and
      $H = \langle \tau \rangle = \{1, \tau\}$. Then
      $$
      G = \{1, \sigma, \sigma^2, \tau, \tau \sigma, \tau \sigma^2\}
        = H \ltimes K.
      $$
      The conjugation 
      $\psi : \langle \tau \rangle \to Aut(\langle \sigma \rangle)$
      is given by
      $$
      \psi(\tau)(\sigma) = \tau \sigma \tau^{-1}
                         = \sigma^2 \tau \tau^{-1} 
                         = \sigma^2,
      $$
      because $\tau \sigma = \sigma^2 \tau$ because
      $$
      (1 2) \cdot (1 2 3) = (2 3), \quad
      (1 2 3) \cdot (1 2 3) \cdot (1 2)
        = (1 3 2) \cdot (1 2) = (2 3).
      $$
      We can also realize that these subgroups are cyclic
      groups of order 2 and 3, or
      $$
      \langle \tau \rangle \simeq \mathbb{Z}_2, \quad
      \langle \sigma \rangle \simeq \mathbb{Z}_3
      $$
      and
      $$
      \psi : \mathbb{Z}_2 
           \to \mathrm{Aut}(\mathbb{Z}_3)
           =   \mathbb{Z}_3^* \simeq \mathbb{Z}_2 
      $$
      so $\psi$ is identity under these identifications.
      Then we can rewrite 
      $$S \simeq \mathbb{Z}_3 \rtimes_\psi \mathbb{Z}_2.$$
      }
  \item{Let $G = \mathrm{GL}_2(\mathbb{R})$. Consider
        $K = \mathrm{SL}_2(\mathrm{R})$ and
        $\mathrm{R}^* = \mathrm{R} - \{ 0 \}.$
        Recall that 
        $\mathrm{SL}_2(\mathbb{R}) \triangleleft
        \mathrm{GL}_2(\mathbb{R})$
        because $\mathrm{SL}_2 = \ker \det$.
        We can embed 
        $\mathbb{R}^* \hookrightarrow \mathrm{GL}_2(\mathbb{R})$
        by $a \mapsto \left(\begin{array}{c c} a & 0 \\ 0 &
            1\end{array}\right)$ and call this subgroup $H$.

        Note that $H \cap K = \{ 1 \}$ and
        $\mathrm{GL}_2(\mathbb{R}) = H K$ because
        given $A \in \mathrm{GL}_2(\mathbb{R})$ we can write
        $$
        A = \left(\begin{array}{c c}
              \det A & 0 \\ 0 & 1
            \end{array}\right) \cdot A^\prime
        $$
        with $\det A^\prime = 1$.

        $$
        \psi : \mathbb{R}^*
           \to \mathrm{Aut}(\mathrm{SL}_2(\mathbb{R})
        $$
        is given by
        $$
        \psi(a)(A) = 
          \left(\begin{array}{c c}
            a & 0 \\ 0 & 1
          \end{array}\right) \cdot A \cdot
          \left(\begin{array}{c c}
            a & 0 \\ 0 & 1
          \end{array}\right)^{-1}.
          $$
       }
\end{enumerate}
\end{xmpl}

\subsection{Decomposition of Groups}

\begin{defn}[Tower of Subgroups]
Let $G$ be a group. A sequence of subgroups
$$
G = G_0 > G_1 > \cdots > G_i > \cdots
$$ 
will be called a \emph{tower} of subgroups of $G$.
It is called a \emph{normal tower} if 
$\forall i, G_{i+1} \triangleleft G_i$, an
\emph{abelian tower} if it is a normal tower such that
$G_{i} / G_{i+1}$ is abelian, and a \emph{cyclic tower}
when $G_{i} / G_{i+1}$ is cyclic.
\end{defn}

\begin{defn}[Solvable Group]
A group is called \emph{solvable} if it has a finite
abelian tower ending in $\{ 1 \}$.
\end{defn}

This terminology comes from a Galois group being solvable, in which
case corresponding polynomials can be solved by radicals.

\begin{theorem}[Feit-Thompson]
  Every finite group of odd order is solvable.
\end{theorem}

\begin{xmpl}
\begin{enumerate}
  \item{Consider the subgroup $G < \mathrm{GL}_3(\mathbb{R})$ given by
      $$
      G = \left\{\left.
          \left(\begin{array}{c c c}
            a & d & e \\ 0 & b & f \\ 0 & 0 & c
          \end{array}\right)\right|a, b, c \neq 0\right\},
      $$
      the \emph{parabolic subgroup}. $G$ is solvable.

      \begin{proof}
        Consider $\phi : G 
                     \to \mathbb{R}^* \times \mathbb{R}^* \times \mathbb{R}^*$
        given by
        $$
        \left(\begin{array}{c c c}
          a & d & e \\ 0 & b & f \\ 0 & 0 & c
        \end{array}\right) \mapsto (a,b,c).
        $$
        This is a homomorphism (exercise) which is
        surjective. Take $\ker \varphi = G_1 \triangleleft G$.
        This $\ker \varphi$ is the matrices which have ones on the
        diagonal. Then the quotient, by the first isomorphism theorem, is
        $$
        G / \ker \varphi 
          \simeq \mathrm{Im}(\varphi)
          = \mathbb{R}^* \times \mathbb{R}^* \times \mathbb{R}^*.
        $$
        Note that this image is abelian.

        Next take
        $$
         G_2 = 
          \left\{\left.
          \left(\begin{array}{c c c}
            1 & 0 & x \\ 0 & 1 & 0 \\ 0 & 0 & 1
          \end{array}\right)\right| x \in \mathbb{R}\right\}.
        $$
        Check that $G_2$ is normal in $G_1$, and that
        $G_1 / G_2 \simeq \mathbb{R} \times \mathbb{R}$, since it is
        determined by the two zeroed entries.

        Observe that $G_2 \simeq \mathbb{R}$ by the map taking its
        top-right element. So we have the abelian tower
        $$
        \left(\begin{array}{c c c}
          a & d & f \\ 0 & b & e \\ 0 & 0 & c
        \end{array}\right)
        \triangleright
        \left(\begin{array}{c c c}
          1 & d & f \\ 0 & 1 & e \\ 0 & 0 & 1
        \end{array}\right)
        \triangleright
        \left(\begin{array}{c c c}
          1 & 0 & x \\ 0 & 1 & 0 \\ 0 & 0 & 1
        \end{array}\right)
        \triangleright \{ 1 \}.
        $$
        This is true for any field.
      \end{proof}
    }
\end{enumerate}
\end{xmpl}

% 9/17

\begin{prop}
Let $G$ be a group and $H \triangleleft G$. Then $G$ is solvable if
and only if both $H$ and the quotient $G / H$ are solvable.
\end{prop}
\begin{proof}
Assume $H, G / H$ are solvable. Then there are towers
$$
H = H_0 \triangleright \cdots \triangleright H_n = \{ 1 \}, \quad
H/H_{i+1} \text{ abelian} 
$$
$$
G/H = K_0 \triangleright \cdots \triangleright K_m = \{ 1 \}, \quad
K/K_{i+1} \text{ abelian}.
$$
Confirm that the inverse image $f^{-1}(K_i)$ of a normal subgroup is
normal under any group homomorphism $f$. Note that
$$
\pi^{-1}(K_i) / \pi^{-1}(K_{i+1}) \simeq K_i / K_{i+1}
$$
By restricting $\pi$ we have a surjective map
$\pi|_{\pi^{-1}(K_i)} : \pi^{-1}(K_i) \to K_i$ with kernel $H$,
so
$$
\pi^{-1}(K_i) / H \simeq K_i.
$$
Now
$$
K_i / K_{i+1} \simeq (\pi^{-1}(K_i) / H) / (\pi^{-1}(K_{i+1}) / H)
             \simeq \pi^{_1}(K_i) / \pi^{-1}(K_{i+1}).
$$
So $\forall i$, $\pi^{-1}(K_i) / \pi^{-1}(K_{i+1})$ are all abelian.

Next, assume $G$ is solvable and show $H, G/H$ are solvable.
We may form the tower
$$
H \triangleright H \cap G_1 
  \triangleright H \cap G_2 
  \triangleright \cdots 
  \triangleright H \cap G_n = \{ 1 \}
$$
since
\begin{enumerate}
  \item{
    $$
    \forall i . H \cap G_{i+1} \triangleleft H \cap G_i.
    $$
    Take $h \in H \cap G_i$ and $g \in H \cap G_{i+1}$.
    Then $h g h^{-1} \in G_{i+1}$ and $h, g \in H$, so
    $h g h^{-1} \in H \cap G_{i+1}$.
  }
  \item{
    Construct a homomorphism
    $$
    \varphi : H \cap G_i / H \cap G_{i+1} \to G_i / G_{i+1}
    $$
    given by 
    $$
    \varphi(h \mod H \cap G_{i+1}) \triangleq h \mod G_{i+1},
    $$
    which is injective since if $h \in H \cap G_i$,
    $\varphi(h \mod H \cap G_{i+1}) = 1 \mod G_{i+1}$ then
    $h \in H \cap G_{i+1}$ so
    $h \mod H \cap G_{i+1} = 1,$
    thus $\ker \varphi$ is trivial.
    Check that this is well-defined (show that it is independent of
    the representative).

    $\varphi$ gives an isomorphism between 
    $H \cap G_i / H \cap G_{i+1}$ and a subgroup of the abelian group
    $G_i / G_{i+1}$, so $H \cap G_i / H \cap G_{i+1}$ is abelian.
  }
\end{enumerate}
Similarly we can show that $G / H$ is solvable (exercise).
\end{proof}

Let $H \triangleleft G$. Consider the quotient $G / H$. Then we have
two groups ``smaller'' than the original group. We can then decompose
our understanding of $G$ by considering $H$ and $G/H$ similarly.

\begin{defn}[Simple Group]
A group that has no normal proper subgroup is called \emph{simple},
i.e. if the only normal subgroups of $G$ are $\{1\}$ and $G$.
\end{defn}

\begin{theorem}[Jordan-H\"older Theorem]
Let $G$ be a finite group. Suppose 
$$
G = G_0 \triangleright G_1 
        \triangleright \cdots
        \triangleright G_r = \{1\}
$$
and
$$
G = H_0 \triangleright H_1 
        \triangleright \cdots
        \triangleright H_s = \{1\}
$$
are two normal towers such that
$G_i / G_{i+1}$ and $H_j / H_{j+1}$ are simple groups
$\forall i, j$. Then $r = s$ and there is a permutation
on the indices $i \to i^\prime$ such that
$G_i / G_{i+1} \simeq H_{i^\prime} / H_{i^\prime + 1}, \forall
i$. This is analogous to prime factorization of positive integers.
\end{theorem}

\begin{remark}
Every finite group admits a normal tower ending at $\{ 1 \}$ such that
the quotients are simple, because a group $G$ is either simple itself
or has a nontrivial normal subgroup by definition, and thus admits a
refinement. This procedure will stop eventually because the group is finite.
\end{remark}

\begin{defn}[Refinement]
A tower of $G$ is called a \emph{refinement} of another tower of $G$
when it is obtained by inserting additional subgroups.
\end{defn}

\begin{defn}[Equivalent Towers]
Two normal towers of $G$ 
$$
G = G_0 \triangleright G_1 
        \triangleright \cdots
        \triangleright G_r = \{1\}
$$
and
$$
G = H_0 \triangleright G_1 
        \triangleright \cdots
        \triangleright H_s = \{1\}
$$
are \emph{equivalent} if their quotients are isomorphic up to a permutation.
\end{defn}

\begin{theorem}[Schreier]
Any two normal towers of $G$ that both end with $\{1\}$ have
equivalent (isomorphic up to permutation) refinements.
\end{theorem}

\begin{proof}
Let 
\begin{align*}
G &= G_0 \triangleright G_1 \triangleright 
     \cdots \triangleright G_r = \{ 1\}, \\
G &= H_0 \triangleright H_1 \triangleright 
     \cdots \triangleright H_s = \{ 1\}. \\
\end{align*}
Consider that
$$
H_0 \cap G_i > H_1 \cap G_i > \cdots > H_s \cap G_i = \{1\}.
$$
Define $G_{i,j} = G_{i+1}(H_j \cap G_i)$,
$H_{j,i} = (H_j \cap G_i)H_{j+1}$. Then
$$
G_{i+1} \subset G_{i+1} (H_j \cap G_i) \subset G_i
$$
for all $i=0,\dots,r$ and $j=0,\dots,s$. 

It can be shown that these
constructions are both normal towers of $G$ that end in $\{1\}$ and
have the same number of steps (because we have the same number of
pairs for each tower). Furthermore $\{G_{i,j}\}$ is a refinement of 
$\{G_i\}$ and $\{H_{j,i}\}$ is a refinement of $\{H_j\}$.

We claim that there are isomorphisms 
$G_{i,j} / G_{i,j+1} \simeq H_{j,i} / H_{j, i+1}$. This implies the
theorem. Let's write it in all its disgusting glory. This says
$$
\frac{G_{i+1} (H_j \cap G_i)}{G_{i+1} (H_{j+1} \cap G_i)}
\simeq
\frac{(H_j \cap G_i) H_{j+i}}{(H_j \cap G_{i+1}) H_{j+1}}
$$
We then want to show that
$$
\frac{u(U \cap V)}{u(U \cap v)}
\simeq
\frac{U \cap V}{(u \cap V)(U \cap v)}
\simeq
\frac{(V \cap U)v}{(V \cap u)v},
$$
(see diagram) where

$$
U = G_i, \quad
u = G_{i+1}
V = H_j, \quad
v = H_{j+1}.
$$

We proceed by the third isomorphism theorem
$$
HK / K \simeq H / (H \cap K)
$$
if $H$ normalizes $K$. We set first
$H = U \cap V$ and $K = u(U \cap v)$, then
$K = (V \cap u)v$. Check that 
\begin{itemize}
  \item{$H$ normalizes $K$,}
  \item{$HK$ is the numerator we want,}
  \item{and $H \cap K = (u \cap V)(U \cap v)$.}
\end{itemize}

This is also called the \emph{Butterfly lemma}.
\end{proof}

This immediately implies the Jordan-H\"older theorem because a normal
tower with simple quotients has no nontrivial refinements.

\end{document}
& \implies g_1 g_2 = h_1 h_2 \psi(h_1)^{-1} (k_1) k_2.
\end{align*}

\end{document}
>>>>>>> presentation for 9/15, notes, me851 and mth818 hw
