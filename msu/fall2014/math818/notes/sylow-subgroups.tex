\documentclass{article}

\usepackage{amsmath}
\usepackage{amsfonts}
\usepackage{mathrsfs}
\usepackage{amssymb}
\usepackage{amsthm}
\usepackage{mathtools}

\usepackage{enumitem}
\usepackage{pictexwd,dcpic}
\usepackage{hyperref}
\usepackage[lastexercise]{exercise}

\newcommand\dif{\mathop{}\!\mathrm{d}}

\newcounter{Problem}
\newenvironment{Problem}{\begin{Exercise}[counter={exer}]}
                        {\end{Exercise}}

\theoremstyle{plain}
\newtheorem{theorem}{Theorem}
\newtheorem{corol}{Corollary}
\newtheorem{lemma}{Lemma}

\theoremstyle{definition}
\newtheorem{defn}{Definition}
\newtheorem{xmpl}{Example}
\newtheorem{prop}{Proposition}

\theoremstyle{remark}
\newtheorem{remark}{Remark}
\newtheorem{obsv}{Observation}
\newtheorem{exer}{Exercise}
\newtheorem{recap}{Remark}


\begin{document}

\section{Sylow Theorems}
Let $p$ be prime and recall that a $p$-group is a group of order
$p^n$.

\begin{defn}[$p$-Sylow subgroups]
A subgroup $H$ of $G$ is called a $p$-Sylow subgroup of $G$ if the
order of $H$ is the largest power of $p$ that divides $|G|$, that is
$|H| = p^a$ if $|G| = p^a m$, $|H| = p^a$, and $p$ does not divide $m$.
\end{defn}

\begin{xmpl}
Let $|G| = 36$. There is a 2-Sylow group of order 4 and a 3-Sylow
group of order 9. There are no other groups of Sylow type.
\end{xmpl}

\begin{theorem}[Sylow Theorems]
  Let $G$ be a finite group, $p$ prime, $p$ divides $|G|$
  ($n = |G| = p^a n^\prime$, $a \geq 1$, $p$ does not divide $n^\prime$)
  \begin{enumerate}
    \item{$G$ has at least one $p$-Sylow subgroup.
         }
    \item{All $p$-Sylow subgroups of $G$ for fixed $p$ are conjugate
          to each other. Also, every $p$-subgroup of $G$ is contained
          in a $p$-Sylow subgroup. 
         }
    \item{The number of $p$-Sylow subgroups is congruent to $1 \mod p$
          and also divides $|G|$.
         }
  \end{enumerate}
\end{theorem}

\begin{defn}[Group Exponent]
The \emph{exponent} of a group $G$ is the smallest integer $N \geq 1$
such that $a^N = 1,$ $\forall a \in G$.
\end{defn}

\begin{lemma}
If $G$ is a finite abelian group of order $n$ and $p$ is a prime that
divides $n$, then $G$ has a an element of order $p$.
\end{lemma}

\begin{proof}
First prove that if $N$ is an exponent of $G$ then $|G|$ divides a
power of $N$. Proceed by induction on $|G|$. Consider $b \in G$, $b
\neq 1$ and take $H = \langle b \rangle < G$. We have $b^N = 1$, so
$|b|$ divides $N$ and then $|H|$ divides $N$.

Look at $G / H$. This also has $N$ as an exponent, i.e.
$(aH)^N = 1 \cdot H$ and $|G / H| < |G|$. From our inductive
hypothesis we have $|G / H|$ divides a power of $N$. But
$|H|$ divides $N$ since $H$ is cyclic, and thus 
$|G| = |G / H| |H|$ which divides $N^a N = N^{a+1}$.

Now suppose $|G| = p^a m$ such that $p$ does not divide $m$
and that $N$ is the exponent of $G$. Then $|G|$ divides a power of $N$
and $p$ divides a power of $N$, so $p$ divides $N$ or $N = p
N^\prime$.

Now pick $x \in G$, $x \neq 1$. We have $x^N = 1$ so $x^{pN^\prime} =
1$ so $(x^{N^\prime})^p = 1$. Let $y = x^{N^\prime}$. Then $y^p = 1$
(either the identity or of order $p$) and there is an $x$ such that
$y = x^{N^\prime}$. We require also that $N^\prime$ is less than the
exponent of $G$ and therefore does not kill $x$.
\end{proof}

\begin{proof}[Sylow Theorems]
We proceed by application of ``counting with group actions''.
\begin{enumerate}
  \item{Proceed by induction on $|G|$. Let $n = 1$ or $n = p$. Then
        the whole group is Sylow.

        Assume all $G^\prime$ with $|G^\prime| < n$ have at least one
        $p$-Sylow subgroup. There are two cases:
        \begin{enumerate}
          \item{$G$ contains a proper subgroup $G^\prime$ with index
                $[G : G^\prime]$ prime to $p$. Then by Lagrange's
                theorem, $|G| = [G : G^\prime]|G^\prime|$ so                
                $|G^\prime| = p^a n^{\prime\prime}$ where $p$ does not
                divide $n^{\prime\prime}$. But then $|G^\prime| < n$
                and induction applies to get $H^\prime < G^\prime$
                with $|H^\prime| = p^a$. Now observe 
                $H^\prime < G^\prime < G$ and $|H^\prime| = p^a$, so 
                $H^\prime$ is $p$-Sylow.
               }
          \item{Next consider proper subgroups of $G$  with index
                divisible by $p$. Use the class equation, letting the
                group $G$ act on itself by conjugation and break it up
                into orbits (in this case conjugacy classes).                
                $$
                |G| = \sum_{x} [G : C_x]
                    = \sum_{\{x : [G : C_x] > 1\}} [G : C_x] + |Z(G)|
                $$
                where $x$ is taken by selecting a representative of
                each conjugacy class. Under our assumption, since
                these are indices of proper subgroups they are all
                divisible by $p$, so then $p$ divides $|Z(G)|$. 

                Now apply the lemma from before to the center, since
                $Z(G)$ is abelian. Then $\exists y \in Z(G)$ such that
                $y \neq 1$, $y^p = 1$. Consider
                $G^\prime = G / \langle y \rangle$, since
                $\langle y \rangle \triangleleft G$ since $y \in
                Z(G)$. Then
                $$
                \frac{|G^\prime|}{|\langle y \rangle|}
              = \frac{|G|}{p}
              = p^{a-1} n^\prime < |G|.
                $$
                Note that if $a = 1$, $p$ does not divide $|G^\prime|$,
                but then $[G : \langle y \rangle] = n^\prime$ which is
                prime to $p$. But by assumption all proper subgroups
                are divisible by $p$, and then $\langle y \rangle$
                cannot be a proper subgroup so 
                $\langle y \rangle = G$ is $p$-Sylow.

                Then applying the induction hypothesis to $G^\prime$,
                there exists an $H^\prime < G^\prime$ with 
                $|H^\prime| = p^{a-1}$. Consider the quotient map
                $\pi : G \to G / \langle y \rangle$. Then
                we have $H = \pi^{-1}(H^\prime) \to H^\prime$ and
                this induces an isomorphism
                $H / \langle y \rangle \simeq H^\prime$ since
                $\ker \pi|_{H} = \langle y \rangle$. Then
                $$
                |H| = |\langle y \rangle||H^\prime|
                    = p p^{a-1} = p^a,
                $$
                so $H$ is a $p$-Sylow of $G$.
               }
          \item{Before theorems 2 and 3, let us make some more
                comments about group actions. Let $G$ act on a set
                $X$. We write
                $$
                X^G \triangleq \{ x \in X : g \cdot x = x, 
                                  \forall g \in G \}
                $$
                to denote the \emph{fixed points} of the action. For example,
                the fixed points for a group acting on itself by
                conjugation are the elements of the center.
                \begin{lemma}
                  Suppose $G$ is a $p$-group for some prime $p$ and
                  $X$ is a finite set. Then 
                  $$
                  |X^G| \equiv |X| \mod p.
                  $$
                \end{lemma}
                \begin{proof}
                  Decompose $X$ into $G$-orbits.
                  $$
                  S =    \coprod_{x} G \cdot x
                  \simeq \coprod_{x} G / G_x,
                  $$
                  where $x$ is chosen among representatives of each
                  orbit. Then
                  $$
                  |X| = \sum_x [G : G_x] 
                      = \sum_{\{x : [G : G_x] > 1\}} [G : G_x] + |X^G|.
                  $$
                  Note that orbits with exactly one element are
                  exactly the fixed points of the action. Since
                  $G$ is a $p$-group, $[G : G_x]$ is a power of $p$
                  and so $|X| \equiv |X^G| \mod p$.
                \end{proof}
                \begin{corol}
                  If a $p$-group acts on a set whose cardinality is
                  prime to $p$, then this action has a fixed point
                  since
                  $$
                  |X^G| \equiv |X| \mod p
                  $$
                  so $|X^G|$ cannot be zero since $|X|$ is prime to $p$.
                \end{corol}
              }
          \item{Next take $G$ such that 
                $|G| = p^a n^\prime = n$, $a \geq 1$. By theorem 1
                there is a $p$-Sylow $P$ with $|P| = p^a$. Suppose
                $H < G$ is a $p$-group, $|H| = p^{a^\prime}$. We claim
                that $H$ is contained in a conjugate of $P$, i.e.
                $\exists g \in G$ such that $H < g P g^{-1}$.

                Observe that to show that $H$ is contained in $P$, it
                is enough to show $H$ is contained in the normalizer
                $N(P)$, i.e. that $H$ normalizes $P$. Assume that
                $H < N(P)$ is in the normalizer. Then recall from the
                3rd isomorphism theorem that $H P$ is a subgroup
                of $G$ and $P \triangleleft HP$, and
                $H \cap \triangleleft H$. Then
                $$
                HP / P \simeq H / (H \cap P)
                $$
                and so
                $$
                [HP : P] = [H : H \cap P]
                $$
                and then $[H : H \cap P]$ divides $|H|$, which is a
                power of $p$ and thus $[H P : P]$ is a power of $p$.
                Then $|HP|$ is a power of $p$ such that $|HP| \geq
                |P|$, but since $P$ is $p$-Sylow then
                $|P|$ is the highest power of $p$ so the orders match
                and therefore $HP < P$ implies $H < P$.
               }
          \item{
                We can now show Theorem 2, that when $p$ divides
                $|G|$, every $p$-subgroup of $G$ is contained in a
                conjugate of $P$, with $P$ a $p$-Sylow subgroup of
                $G$. Take $X$ to be the set of all subgroups of $G$
                that are conjugate to $P$:
                $$
                X = \{ g P g^{-1} : g \in G \}.
                $$
                Take $H$ to be the $p$-subgroup of the hypothesis. We
                want to show $H \in X$. Let $H$ act on $X$ by
                conjugation, so that
                $$
                h \cdot (g P g^{-1}) 
              = h (g P g^{-1}) h^{-1}
              = hg P (hg)^{-1} \in X.
                $$
                Now notice that $X$ is an orbit of the action of $g$
                on $P$ by conjugation, so
                $$
                |X| = [G : G_P] = [G : N(P)],
                $$
                because we can produce a map $G / N(P) \to X$ given by
                $g N(P) \mapsto g P g^{-1}$ which is bijective since
                conjugation of $N(P)$ does not affect the normalizer.

                Indeed, suppose $g, g^\prime$ such that
                $g P g^{-1} = g^\prime P (g^\prime)^{-1}$
                $$P = g^{-1} g^\prime P (g^\prime)^{-1} g
                    = (g^{-1} g^\prime) P (g^{-1} g^\prime)^{-1}$$
                so $g^{-1} g^\prime \in N(P)$ so $g^\prime \in g
                N(P)$, so they belong to the same coset. 

                The orbit of  an element is always in bijection with the quotient by
                the stabilizer
                Let $\cdot : G \times X \to X$ be any group
                action. The orbit
                $$
                G \cdot x = \{ g \cdot x : g \in G \}.
                $$
                and the stabilizer $G_x = \{ g \in G : g \cdot x = x
                \}$ and in general
                $$
                G / G_x \to G \cdot x
                $$
                is bijective by
                $$
                g G_x \mapsto g \cdot x.
                $$

                The action of $G$ on $X$ is transitive and the
                stabilizer $G_P$ is the normalizer $N(P)$.
                
                Furthermore $p$ does not divide $[G : N(P)]$ because
                $p$ does not divide $[G : P] = \frac{|G|}{|P|}$ and
                since $P \subset N(P) \subset G$ this means
                $[G : N(P)] | [G : P]$. Then $|X|$ is prime to $p$.
                
                Now apply
                $$
                |X| = |X^H| \mod p
                $$
                to see that $|X^H|$ is prime to $p$ and so $|X^H| \neq
                0$, so there is a fixed point. Therefore there exists
                a conjugate $P^\prime = g P g^{-1}$ of $P$ such that 
                $\forall h \in H$, $h \cdot P^\prime = P^\prime$, i.e.
                $h P^\prime h^{-1} = P^\prime$. But this means 
                $H \subset N(P^\prime)$, and we conclude that
                $H \subset P^\prime = g P g^{-1}$.

                Then every $p$-subgroup is contained in a 
                $p$-Sylow, since the conjugate of a $p$-Sylow has the
                same order and is thus $p$-Sylow. Furthermore all
                $p$-Sylows are conjugate, since a $p$-Sylow is again a
                $p$-subgroup and therefore is contained in a conjugate
                of $P$, i.e. $P^\prime \subset g P g^{-1}$
                for some $g \in G$. But $|g P g^{-1}| = p^a$ and 
                $|P^\prime| = p^a$, so $P^\prime = g P g^{-1}$.
               }
          \item{Next, prove that the number of $p$-Sylows is congruent
                to $1 \mod p$.

                We know from Theorem 2 that all $p$-Sylows are
                conjugate. The number of $p$-Sylows conjugate to a
                fixed $p$-Sylow $P$ is the number of elements in the
                orbit of $P$ under the action of conjugation, $[G : N(P)]$.
                
                Consider the action of the group $P$ on the set $X$ of
                all $p$-Sylow subgroups of $G$ by conjugation that
                maps $g P g^{-1} \mapsto y (g P g^{-1}) y^{-1}$ for
                some $y \in P$. Since $P$ is a $p$-group, we have
                $$
                |X| \equiv |X^P| \mod p,
                $$
                where $|X|$ is the number of all $p$-Sylows. But
                $$
                X^P = \{ P^\prime :  
                         \forall y \in P, y P^\prime y^{-1} = P^\prime
                      \}
                    = \{ P^\prime : P \subset N(P^\prime) \}
                    = \{ P^\prime : P \subset P^\prime \}
                    = \{ P \}.
                $$
                Therefore the action has a single fixed point and so
                $|X| \equiv 1 \mod p$ as desired.
               }
        \end{enumerate}
      }
  \item{
       }
  \item{
       }
\end{enumerate}
\end{proof}

\subsection{Problem \#5}
Transitive actions
Let $G = S_n$, $X = \{1, 2, \cdots, n\}$. Then we can get from any
element to any other element by applying the group action. This is
what is meant by the action being transitive -- the entire set $X$
lies on the stabilizer.

\begin{xmpl}
\begin{enumerate}
  \item{
    Show that a group of order 63 is not simple. 
    $(63 = 3 \cdot 3 \cdot 7)$. Recall that a \emph{simple} group is
    one with no proper normal subgroups.

    Observe that if there is only one $p$-Sylow (i.e. $N_p = 1$) then
    there is a normal subgroup of $G$. Why? Let $g \in G$, $P$ be
    the only $p$-Sylow. Then since all $p$-Sylows are conjugate, let
    $P^\prime = gPg^{-1}$ be another $p$-Sylow. But $P^\prime = P$ by
    assumption, so $gPg^{-1} = P$ and thus $P$ is normal.

    We want to show a group of order 63 has a proper normal subgroup.
    We know that the number of 3-Sylows $N_3$ should have
    $N_3 \equiv 1 \mod 3$ and $N_3$ divides 63, so $N_3$ divides
    7. Then $N_3 = 1$ or $N_3 = 7$. We know that the number of
    7-Sylows $N_7$ should have $N_7 \equiv 1 \mod 7$ and $N_7$ divides
    63, so $N_7$ divides 9, and thus $N_7 = 1$. Therefore there is a
    unique 7-Sylow normal in $G$.

    Call this 7-Sylow $H$ ($H \triangleleft G$). Then $|H| = 7$ since
    this is the highest power of $7$ that divides 63, and since $|H|$
    is prime this means 
    $H = \triangleleft a \triangleright \simeq \mathbb{Z}_7$. Then
    consider $|G / H| = 9$. This is abelian since it has a nontrivial
    center (since it is a $3$-group). 
    Then $|Z(G)| = |G|$ or $|G / Z(G)| = p$. But if $|G / Z(G)| = p$ then
    $G / Z(G)$ is cyclic, so $G$ is abelian.

    This means our group $G$ of order 63 is solvable, since $G / H$ is abelian.
  }
  \item{
    Show that a group of order $56$ is not simple.
    $|G| = 2^3 \cdot 7$, so $N_7 \equiv 1 \mod 7$ and $N_7$ divides $7
    \cdot 8$, so either $N_7 = 1$ or $N_7 = 8$. $N_2 \equiv 1 \mod 2$
    and $N_2$ divides $7 \cdot 8$, so $N_7$ divides 7 and thus $N_7 =
    1$ or $N_7 = 7$.

    Suppose $N_7 = 8$. These then have $6 \times 8 + 1 = 49$ elements
    in total. If $H, H^\prime$ are two 7-Sylows with $H \cap H^\prime
    \neq \{ 1 \}$, then $H = H^\prime$. But $H \simeq \mathbb{Z}_7
    \simeq H^\prime$, so for any $a \neq 1$ this means $a$ has to
    generate both $H$ and $H^\prime$ and then $H =
    H^\prime$. Therefore there are $56 - 49 = 7$ elements left
    out. There exists a 2-Sylow of order 8 which has to be unique
    since it just covers the remaining elements.

    We conclude that either $N_7 = 1$ or $N_2 = 1$.
  }
  \item{
    Show that a group of order $pq$, with $p$, $q$ distinct primes, is
    solvable. 

    Say $p < q$. We have $N_q \equiv 1 mod q$ and $N_q$
    divides $pq$, so since $p < q$ it is not possible that $N_q =
    q+1$, $2q + 1$, etc. Then $N_q = 1$, so this is a normal subgroup,
    call it $H$. Then there is a tower with $G / H \simeq
    \mathbb{Z}_p$ and $H / \{ 1 \} \simeq \mathbb{Z}_q$, so this tower
    is abelian.

    Since $H$ is then of prime order, let $H = \langle a \rangle$, $a^q
    = 1$. Consider a $p$-Sylow $K = \langle b \rangle$, $b^p =
    1$. Then observe
    \begin{enumerate}
      \item{$H \cap K = \{ 1 \}$,}
      \item{$H \triangleleft K$,}
      \item{$H K = G$.}
    \end{enumerate}
    Since $H$ is normal in $G$, $K$ acts on $H$ by conjugation and we
    get $\psi : K \to \mathrm{Aut}(H)$ given by 
    $\psi(k) = \lambda h . k h k^{-1}$. In this particular case we
    have
    $$
    \psi : \mathbb{Z}_p \to \mathrm{Aut}(\mathbb{Z}_q).
    $$
    We have $b a b^{-1} \in H = \langle a \rangle$, so
    $b a b^{-1} = a^i$ and then $ba = a^i b$. Knowing $i$ determines
    the group $G$:
    $$
    G = \{ a^n b^m \mid 0 \leq n < q, 0 \leq m < p \}
    $$
    so that
    $$
    a^n b^m a^{n^\prime} b^{m^\prime} = a^N b^M.
    $$
    We need to know the homomorphism $\psi$.

    We know that
    $$
    \mathrm{Aut}(\mathbb{Z}_q) \simeq \mathbb{Z}_q^\ast
    $$
    of order $q - 1$. Assume in addition that $q \not{\equiv} 1 \mod
    p$. Then
    $$
    \psi : \mathbb{Z}_p \to \mathrm{Aut}(\mathbb{Z}_q) = \mathbb{Z}_q^\ast
    $$
    has to be trivial $(\psi = \mathrm{id})$ since
    $|\mathrm{Im}(\psi)|$ divides $|\mathbb{Z}_q^\ast| = q-1$ but
    $\mathrm{Im}(\psi) \simeq \mathbb{Z}_p / \ker \psi$ so that
    $|\mathrm{Im}(\psi)|$ divides $p$. Since $p$ does not divide $q-1$
    this means $\mathrm{Im}(\psi) = \{ \mathrm{id} \}$.

    This means $i = 1$ above and so $ab = ba$. Therefore $G$ is abelian.
  }
\end{enumerate}
\end{xmpl}

\begin{prop}
Let $G$ be a group of order $p \cdot q$, with $p$, $q$ distinct
primes.
\begin{enumerate}
  \item{$G$ is solvable.}
  \item{If $p < q$, then $G$ is isomorphic to a semidirect product
        $$
        G \simeq \mathbb{Z}_p \ltimes \mathbb{Z}_q.
        $$
       }
  \item{If $p < q$ and $p$ does not divide $q - 1$, then
        $$
        G \simeq \mathbb{Z}_p \times \mathbb{Z}_q \simeq \mathbb{Z}_{pq},
        $$
        so $G$ is cyclic.
       }
\end{enumerate}
\end{prop}

\begin{proof}
  \begin{enumerate}
    \item{
      Observe that a unique $q$-Sylow is normal, since
      $|N_q| \equiv 1 \mod q$ and $N_q$ divides $pq$ so
      $N_q = 1$.

      Let $Q$ be the $q$-Sylow with $|Q| = q$. Then
      $Q \simeq \mathbb{Z}_q$, and $|G / Q| = p$, so
      $G / Q  \simeq \mathbb{Z}_p$. Then there is a normal
      tower for $G$ with abelian quotients.
    }
    \item{
      Let $P$ be a $p$-Sylow. We observe that
      $$
      P \cap Q = \{1\}, \quad
      Q \triangleleft G, \quad
      G = PQ
      $$
      and so $G = P \ltimes Q$. This product is determined by an
      associated homomorphism
      $$
      \psi : P \to \mathrm{Aut}(Q)
      $$
      given by conjugation
      $$
      \psi(x)(y) = xyx^{-1}.
      $$
    }
    \item{
      Under the condition that $p$ does not divide $q - 1$,
      $\psi$ is trivial, i.e. $\psi(x) = \mathrm{id}$, so
      $x y x^{-1} = y$, so $xy = yx$.

      This is because $P \simeq \mathbb{Z}_p$,
      $Q \simeq \mathbb{Z}_q$, so up to isomorphism
      $$
      \psi : \mathbb{Z}_p \to \mathrm{Aut}(\mathbb{Z}_q) \simeq
      \mathbb{Z}_q^\ast
      $$
      Then $|\mathrm{Im}(\psi)|$ divides both $p$ and $q - 1$ by the
      first isomorphism theorem. But since $p$ does not divide $q -
      1$, this requires that $\mathrm{Im}(\psi) = \{ \mathrm{id} \}$.

      Now write the map
      $$
      \phi : P \times Q \to G
      $$
      defined by
      $$
      \phi((x, y)) = xy,
      $$
      which is a group isomorphism since it is invertible and
      $$
      \phi((x_1, y_1) \cdot (x_2, y_2)) 
    = \phi((x_1, y_1) \cdot \phi((x_2, y_2))
    = x_1 y_1 x_2 y_2
      $$
      because $P$ commutes with $Q$ since $\psi$ is trivial.

      Therefore
      $$
      G \simeq P \times Q \simeq \mathbb{Z}_p \times \mathbb{Z}_q.
      $$

      Finally, we use
      \begin{lemma}
        If $\mathrm{gcd}(n, m) = 1$, then
        $\mathbb{Z}_n \times \mathbb{Z}_m \simeq \mathbb{Z}_{nm}$.
      \end{lemma}
      \begin{proof}
        We claim that $a = (1 \mod n, 1 \mod m)$ is a generator of
        $\mathbb{Z}_n \times \mathbb{Z}_m$. It is enough to show that
        the order of $a$ is at least $nm$.
        
        Take $N \geq 1$ such that $N \cdot a = (0, 0)$. Then
        $n$ divides $N$ and $m$ divides $N$, so $nm$ divides $N$ since
        $n$, $m$ are relatively prime. Therefore $N \geq n m$.
      \end{proof}
    }
  \end{enumerate}
\end{proof}

\begin{theorem}[Burnside's Theorem]
  A group of order $p^a \cdot q^b$ is solvable.
\end{theorem}

\begin{prop}
  Let $G$ be a finite group, $H$ a subgroup of index
  $[G : H] = p$, where $p$ is the smallest prime that divides
  $G$. Then $H$ is normal in $G$.
\end{prop}
\begin{proof}
  Consider the normalizer $N(H)$,
  $$
  N(H) = \{ g \in G :  gHg^{-1} \subset H \}.
  $$
  We have
  $$
  H \triangleleft N(H) < G
  $$
  and so
  $$
  [G : H] = [G : N(H)] [N(H) : H].
  $$
  If $N(H) = G$, then $H \triangleleft G$ as desired.
  Otherwise $N(H) = H$ because
  $$
  H < N(H) < G
  $$
  and $[G : H] = p$.

  Recall that the number of conjugates of $H$ in $G$ is always equal
  to the index of the normalizer -- the orbit of an action has the
  same order as the quotient of a group by the stabilizer:
  $$
  [G : N(H)] = [G : H] = p.
  $$
  So there are $p$ conjugates of $H$ in $G$. $G$ acts on these by
  $$
  \varphi : G \to \mathrm{Perm}(C) = S_p
  $$
  where $C$ is the set of conjugates of $H$ in $G$. Then $\varphi$ is
  a group homomorphism since conjugation is a group action.

  Consider
  $$
  K = \ker \varphi \triangleleft G.
  $$
  Then $G / K \simeq \mathrm{Im}(\varphi) < S_p$ so
  $[G : K]$ divides $|S_p| = p!$, and thus $[G : K]$ divides $|G|$.
  Then $[G:K]$ must divide $p$ as well, so $[G : K] = 1$ or
  $[G : K] = p$. But also $x \in K = \ker \varphi$ implies
  $\varphi(x) = \mathrm{id}$, so $x H x^{-1} = H$ and thus
  $x \in N(H)$. Therefore $K < N(H) = H$, so since 
  $[G : H] = p$ we get $H = K$. But 
  $K = \ker \varphi \triangleleft G$ so $H \triangleleft G$ as desired.
\end{proof}

Group actions often give normal subgroups. A group that acts on a
small set is usually not simple.

\begin{xmpl}
  Show that a group of order 
  $552 = 23 \cdot 24 = 23 \cdot 2^3 \cdot 3$ is not simple. 
\end{xmpl}

\end{document}