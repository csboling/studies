\documentclass{article}

\usepackage{amsmath}
\usepackage{amsfonts}
\usepackage{amssymb}
\usepackage{enumerate}
\usepackage[lastexercise]{exercise}

\newcounter{Problem}
\newenvironment{Problem}{\begin{Exercise}[name={Problem},
                                          counter={Problem}]}
                        {\end{Exercise}}
\title{MATH 818 Homework \#1}
\date{September 19, 2014}
\author{Sam Boling}

\begin{document}

\begin{titlepage}
\maketitle
\end{titlepage}

\begin{Problem}
Show that every group of order $\leq 5$ is abelian.
\end{Problem}

\begin{Answer}
Let $G$ be a group of order $\leq 5$. There are the following possibilities:
\begin{itemize}
  \item{$|G| = 1$. Then $G$ must be trivial, and therefore abelian.}
  \item{$|G| \in \{2, 3, 5\}$. Then $|G|$ is prime, so $G$ is cyclic.

        Let $H$ be any cyclic group, and let $h, h^\prime \in
        H$. Since $H$ is cyclic $H = \langle x \rangle$ for some $x
        \in G$, so $h = x^n$ and $h^\prime =
        x^{n^\prime}$ for some $n, n^\prime \in \mathbb{Z}$. Therefore
        $$
        h \cdot h^\prime 
      = x^n \cdot x^{n^\prime} 
      = x^{n + n^\prime}
      = x^{n^\prime} \cdot x^{n}
      = h^\prime \cdot h,
        $$
        so $H$ is abelian.

        Therefore $G$ is abelian if $|G| \in \{2, 3, 5\}$.
      }
  \item{$|G| = 4$. Then since $G$ contains an odd number of non-unit
      elements, there must exist an $x \in G$ such that $x = x^{-1}$.
%      Then consider the function $f(g) = xggx$. This is a homomorphism
%      since $$f(ab) = xabx = xaxxbx = f(a)f(b)$$
      There are two cases:
      \begin{itemize}
        \item{$g = g^{-1}$ for each $g \in G$. Then since $xy \in G$
              and thus $(xy)^{-1} = xy$,
              $$
              xy = (xy)^{-1} = y^{-1} x^{-1} = yx
              $$
              for any $x, y \in G$, so $G$ is abelian.
              }
        \item{$x = x^{-1}$ for one $x \neq e$, and the remaining two
              elements can be labeled $y$ and its distinct inverse.
              Then $G = \{e, x, y, y^{-1}\}$. Since $y \neq x^{-1}$,
              $xy \neq e$, and since $y \neq e \neq x$ this means 
              $x \neq xy \neq y$. Therefore the only remaining option
              is $xy = y^{-1}$, and the same argument means
              $yx = y^{-1}$. This further means
              \begin{align*}
              x y^{-1} &= xxy = y = (y^{-1})^{-1} \\
                      &= (xy)^{-1} = y^{-1} x^{-1} = y^{-1} x,
              \end{align*}
              so $x$ commutes with $y$ and $y^{-1}$. Each other pair
              of elements either contains $e$ or consists of an
              element and its inverse, so all pairs of elements commute.              
             }
      \end{itemize}
       Therefore $G$ is abelian when $|G| = 4$.
       }
\end{itemize}
Hence $G$ is abelian when $|G| \leq 5$.
\end{Answer}

\pagebreak

\begin{Problem}
Show that if $N$ is a normal subgroup of order 2 in a group $G$ then
$N$ is contained in the center $Z(G)$.
\end{Problem}

\begin{Answer}
A group of order 2 has the form $\{ e, x \}$, where $x =
x^{-1}$. Then let $N = \{ e, x \}$ of this form be a normal subgroup
of $G$.

Let $a \in G$ and consider the cosets
$$aN = \{ an | n \in N \} = \{ ae, ax \} = \{ a, ax \}$$
and
$$Na = \{ na | n \in N \} = \{ ea, xa \} = \{ a, xa \}.$$
Since $N$ is normal, $aN = Na$ and thus $ax = xa$, so $x$
commutes with $a$. But $a$ was chosen arbitrarily in $G$, so $x$ commutes
with every $a \in G$ and thus $x \in Z(G)$. Since $e \in Z(G)$ this
means $N \subset Z(G)$, as desired.
\end{Answer}

\pagebreak

\begin{Problem}
Let $G^c$ be the subgroup of $G$ generated by all its commutators.

\begin{enumerate}[(a)]
  \item{Show that $G^c$ is normal and that the quotient $G / G^c$ is
        abelian.}
  \item{Show that every homomorphism $f : G \to G^\prime$ where
      $G^\prime$ is an abelian group factors through the quotient 
      $G / G^c$, i.e. it can be written as a composition
      $G \to G / G^c \to G^\prime$.}
  \item{Suppose that $N$ is a normal subgroup of $G$ such that
      $N \cap G^c = \{ 1 \}$. Show that $N$ is contained in the center
      $Z(G)$ of $G$.}
\end{enumerate}
\end{Problem}

\begin{Answer}
\begin{enumerate}[(a)]
  \item{To show that $G^c$ is normal, we show that 
        $g G^c g^{-1} \subset G^c$. 

        Let $c \in G^c$. Then
        $$
        c = a_1 b_1 a_1^{-1} b_1^{-1}
            a_2 b_2 a_2^{-1} b_2^{-1}
            \cdots
            a_n b_n a_n^{-1} b_n^{-1}
        $$
        for some $n \in \mathbb{Z}_+$ and some $a_i, b_i \in G$.

        Note that conjugation
        distributes over both multiplication and commutation, i.e.
        $$
        (g x g^{-1}) (g y g^{-1}) = g x y g^{-1}
        $$
        and
        $$
        (g a g^{-1})
        (g b g^{-1})
        (g a^{-1} g^{-1})
        (g b^{-1} g^{-1})
      = g a b a^{-1} b^{-1} g^{-1}.
        $$
        Furthermore,
        $$
        (g x g^{-1})(g x^{-1} g^{-1}) = g x x^{-1} g^{-1} = g g^{-1} = e,
        $$
        so $(g x g^{-1})^{-1} = g x^{-1} g^{-1}.$
        But this means
        \begin{align*}
        g c g^{-1} &= (g a_1 g^{-1})(g b_1 g^{-1})
                     (g a_1^{-1} g^{-1})(g b_1^{-1} g^{-1})
                     \cdots
                     (g a_n g^{-1})(g b_n g^{-1})
                     (g a_n^{-1} g^{-1})(g b_n^{-1} g^{-1}) \\
                  &= a_1^g b_1^g (a_1^g)^{-1} (b_1^g)^{-1}
                     \cdots
                     a_n^g b_n^g (a_n^g)^{-1} (b_n^g)^{-1},
        \end{align*}
        where $x^g = g x g^{-1}$, so $g c g^{-1} \in G^c$. This is
        true for any $c \in G^c$, so $g G^c g^{-1} \subset G^c$ and
        thus $G^c$ is normal in $G$.

        To show that $G / G^c$ is abelian, let $x, y \in G$ and
        consider $\pi(x y x^{-1} y^{-1})$, where $\pi$ is the
        canonical isomorphism $\pi : G \to G / G^c$.
        Denote
        $a b a^{-1} b^{-1} = [a, b]$. Let $h \in \pi(x y x^{-1} y^{-1})$. Then
        $$
        h = x y x^{-1} y^{-1} 
            a_1 b_1 a_1^{-1} b_1^{-1}
            \cdots
            a_n b_n a_n^{-1} b_n^{-1}
          = [x, y] [a_1, b_1] \cdots [a_n, b_n] \in G^c
        $$
        so $\pi(x y x^{-1} y^{-1}) = G^c$. Then 
        $x y x^{-1} y^{-1} G^c = G^c$ so $x y G^c = y x G^c$, so
        $G / G^c$ is abelian.
       }
  \item{Let $G^\prime$ be abelian and let $f : G \to G^\prime$ be a
        homomorphism. Let $c \in G^c$, so that
        $$
        c = a_1 b_1 a_1^{-1} b_1^{-1} \cdots a_n b_n a_n^{-1} b_n^{-1}
        $$
        for some $n \in \mathbb{Z}_+$, $a_i, b_i \in G$ for 
        $i \in \{1, \dots, n\}$, and let $e^\prime$ denote the unit
        element in $G^\prime$. Then
        \begin{align*}
        f(c) &= f(a_1 b_1 a_1^{-1} b_1^{-1}
                  \cdots
                  a_n b_n a_n^{-1}b_n^{-1}) \\
             &= f(a_1)f(b_1)f(a_1)^{-1}f(b_1)^{-1} 
                \cdots
                f(a_n)f(b_n)f(a_n)^{-1}f(b_n)^{-1} \\
             &= f(a_1)f(a_1)^{-1}f(b_1)f(b_1)^{-1}
                \cdots
                f(a_1)f(a_1)^{-1}f(b_1)f(b_1)^{-1} \\
             &= e^\prime,
        \end{align*}
        since $f$ is a homomorphism and the codomain of $f$ is
        abelian. Therefore $G^c \subset \ker f$. 
%Indeed, let 
%        $x \in \ker f$. Then 
%        $$
%        f(x) = e^\prime = e^\prime e^\prime 
%                        (e^\prime)^{-1} (e^\prime)^{-1}
%             = f(e)f(e)f(e)^{-1}f(e)^{-1}
%             = f(eeee) 
%        
%        $$

        Then $f(xc) = f(x)$ for any $c \in G^c$ and for any choice of $x$, so
        the function $\varphi : G / G^c \to G^\prime$ given by
        by $\varphi(xG^c) = f(x)$ is well-defined. Then 
        $(\varphi \circ \pi)(x) = \varphi(xG^c) = f(x)$,
        where $\pi$ is the canonical homomorphism
        $\pi : G \to G/G^c$, so $f = \varphi \circ \pi$.
       }
  \item{Let $n \in N$ and $g \in G$. We have that $g n \in gN$, and
        since $N$ is normal in $G$, $gN = Ng$ so $g n \in N g$ and
        therefore $g n = n^\prime g$ for some $n^\prime \in N$. But
        this means that $n^\prime = g n g^{-1}$ and so
        $n^\prime n^{-1} = g n g^{-1} n^{-1}$. But 
        $n^\prime n^{-1} \in N$ since $N$ is a subgroup
        and $g n g^{-1} n^{-1} \in G^c$ by definition, 
        so $n^\prime n^{-1} \in N \cap G^c = \{ 1 \}$ by
        assumption. Then $n^\prime n^{-1} = 1$, so 
        $n^\prime = (n^{-1})^{-1} = n$. Therefore $g n = n g$, so
        since $g$ is arbitrary this means $n \in Z(G)$. Therefore
        $N \subset Z(G)$.
       }
\end{enumerate}
\end{Answer}

\pagebreak

\begin{Problem}
Let $G$ be a group. For an element $a \in G$, consider the map 
$f_a : G \to G$ given by $f_a(x) = axa^{-1}$ (``conjugation by a'').

\begin{enumerate}[(a)]
  \item{Show that $f_a$ is an automorphism of $G$ and so it gives an
      element in the group of automorphisms $\mathrm{Aut}(G)$.}
  \item{By definition, an automorphism of $G$ which has the form $f_a$
      for some $a \in G$ is called ``inner''. Show that the set 
      $\mathrm{Inn}(G)$ of inner automorphisms is a subgroup of
      $\mathrm{Aut}(G)$.} \item{Show that the group $\mathrm{Inn}(G)$ of inner automorphisms
      is isomorphic to the quotient $G / Z(G)$ where $Z(G)$ is the
      center of $G$.}
  \item{Show that $\mathrm{Inn}(G)$ is normal in $\mathrm{Aut}(G)$.}
\end{enumerate}
\end{Problem}

\begin{Answer}
\begin{enumerate}[(a)]
  \item{$f_a$ is a homomorphism because
        $$
        f_a(x) f_a(y) = a x a^{-1} a y a^{-1} 
                      = a x y a^{-1}
                      = f_a(xy).
        $$
        Furthermore
        $$
        f_{a^{-1}}(f_a(x)) = a^{-1} f_a(x) a
                         = a^{-1} a x a^{-1} a
                         = x,
        $$
        so $f_{a^{-1}} \circ f_a = \mathrm{id}$ and since $a$ is
        arbitrary $f_a \circ f_{a^{-1}} = \mathrm{id}$, so $f_a$ is
        invertible. Therefore $f_a : G \to G$ is an isomorphism and
        so $f_a \in \mathrm{Aut}(G)$.
        }
      \item{Next we require that $\mathrm{Inn}(G)$ is a subgroup
            of $\mathrm{Aut}(G)$.
        \begin{itemize}
          \item[(Unit)]{
            $f_e(x) = e x e^{-1} = x$, so
            $f_e = \mathrm{id}$ and so $\mathrm{id} \in
            \mathrm{Inn}(G)$. Thus $\mathrm{Inn}(G)$ inherits the unit
            from the ambient group.
            }
          \item[(Closure)]{
              Let $f_a, f_b \in \mathrm{Inn}(G)$. Then
              $$
              (f_a \circ f_b)(x) = f_a( b x b^{-1})
                                 = a b x b^{-1} a^{-1}
                                 = (ab) x (ab)^{-1}
                                 = f_{ab}(x),
              $$
              so $f_a \circ f_b \in \mathrm{Inn}(G)$.
            }
          \item[(Inverses)]{
              Let $f_a \in \mathrm{Inn}(G)$.
              As in (a), $f_{a^{-1}} \circ f_a = \mathrm{id}$, so
              $f_a^{-1} = f_{a^{-1}}$, so $f_a^{-1} \in \mathrm{Inn}(G)$.
            }
        \end{itemize}
        Therefore $\mathrm{Inn}(G) < \mathrm{Aut}(G)$.
      }
      \item{Consider the function 
            $f: G \to \mathrm{Inn}(G)$ given by $f(g) = f_g$. Then
            $$
            f(gh) = f_{gh} = f_g \circ f_h = f(g)f(h),
            $$
            where $f_{gh} = f_g \circ f_h$ was shown above when
            demonstrating that $\mathrm{Inn}(G)$ is closed under composition.
            Therefore $f$ is a group homomorphism.

            Let $g \in Z(G)$. Then 
            $$
            f(g)(x) = g x g^{-1} = g g^{-1} x = x = \mathrm{id}(x),
            $$
            so $f(g) = \mathrm{id}$ and thus $g \in \ker f$, so $Z(G)
            \subset \ker f$.

            Next let $g \in \ker f$. Then $f(g) = \mathrm{id}$. We have
            $$
            f(g)(x) = f_g(x) = g x g^{-1}.
            $$
            But $f(g)(x) = \mathrm{id}(x) = x$, so $g x g^{-1} =
            x$. Then $g x = x g$, so $g \in Z(G)$
            and therefore $\ker f \subset Z(G)$.

            Therefore $f: G \to \mathrm{Inn}(G)$ is a homomorphism
            with $\ker f = Z(G)$, so
            $$
            G / Z(G) = G / \ker f \simeq \mathrm{Im}(f) = \mathrm{Inn}(G)
            $$
            by the first isomorphism theorem.
            Thus $G / Z(G) \simeq \mathrm{Inn}(G)$ as desired.
           }
      \item{
        Let $g \in \mathrm{Aut}(G)$. Then the coset
        $g \cdot \mathrm{Inn}(G) = \{ g \circ f_a | a \in G \}$.
        Then consider an arbitrary $a \in G$. We write
        \begin{align*}
          (g \circ f_a)(x) 
            &= g(f_a(x)) = g(a x a^{-1}) \\
            &= g(a) g(x) g(a^{-1})
             = g(a) g(x) g(a)^{-1} \\
            &= f_{g(a)}(g(x))
             = (f_{g(a)} \circ g)(x),
        \end{align*}
        and $f_{g(a)} \circ g \in \mathrm{Inn}(G) g$. Therefore
        $g \cdot \mathrm{Inn}(G) \subset \mathrm{Inn}(G) \cdot g$ for any
        $g \in \mathrm{Aut}(G)$, so $\mathrm{Inn}(G)$ is normal
        in $\mathrm{Aut}(G)$.
      }
\end{enumerate}
\end{Answer}

\pagebreak

\begin{Problem}
Let $G$ be a group and $A$ and abelian normal subgroup of $G$. Give a
group homomorphism $f : G/A \to \mathrm{Aut}(A)$.
\end{Problem}

\begin{Answer}
The constant function $f(xA) = \mathrm{id}$ is a group homomorphism
because
$$
f(xyA) = \mathrm{id}
       = \mathrm{id} \circ \mathrm{id}
       = f(xA) \circ f(yA).
$$
Furthermore 
$$
\mathrm{id}(x y) = x y = \mathrm{id}(x) \mathrm{id}(y)
$$
and $\mathrm{id} \circ \mathrm{id} = \mathrm{id}$, so
$\mathrm{id}^{-1} = \mathrm{id}$. Therefore $\mathrm{id}$ is an
invertible group homomorphism so $\mathrm{id} \in \mathrm{Aut}(A)$.
\end{Answer}

\pagebreak

\begin{Problem}
Find the automorphism groups $\mathrm{Aut}(\mathbb{Z}_{45})$,
$\mathrm{Aut}(\mathbb{Z}_3 \times \mathbb{Z}_5)$, and
$\mathrm{Aut}(\mathbb{Z}_5 \times \mathbb{Z}_5)$.
\end{Problem}

\begin{Answer}
First consider finite cyclic groups more generally. Suppose 
$G = \langle a \rangle$ for some $a \in G$. Let $x \in G$, and let
$f \in \mathrm{Aut}(G)$. Since $f$ is surjective, $x = f(x^\prime)$
for some $x^\prime \in G$. Since $a$ generates $G$, we may write
$x^\prime = a^n$ for some $n$, so 
$$
x = f(x^\prime) = f(a^n) = f(a)^n
$$
since $f$ is a homomorphism. But $x$ is chosen arbitrarily in $G$, so
this means $G = \langle f(a) \rangle$. Therefore any automorphism maps
generators to generators. Furthermore we note that since
$f(x) = f(a^n) = f(a)^n$ for any $x \in \langle a \rangle$ and any
homomorphism $f : \langle a \rangle \to \langle a \rangle$, any such
$f$ is defined by its action on a generator.

Next let $S$ be the set of all generators of $G$. Define a function
$\varphi : S \to (G \to G)$ by 
$$
\varphi(t)(x) = \varphi(t)(s^n) = t^n,
$$
noting that this is independent of choice of representative since any
$x \in G$ admits a representation $x = s^n$ for some $s \in S$. Then
we observe that
$$
(\varphi(s) \circ \varphi(t))(s^n)
= \varphi(s)(\varphi(t)(s^n))
= \varphi(s)(t^n)
= s^n
= \mathrm{id}(s^n),
$$
so $\varphi(s) \circ \varphi(t) = \mathrm{id}$ and thus
$(\varphi(t))^{-1} = \varphi(s)$, so $\varphi(t)$ is bijective for
each $t \in S$. Furthermore
$$
\varphi(t)(xy) 
= \varphi(t)(s^m s^n) 
= \varphi(t)(s^{m+n})
= t^{m+n} = t^m t^n
= \varphi(t)(s^m) \cdot \varphi(t)(s^n)
= \varphi(t)(x) \cdot \varphi(t)(y),
$$
so $\varphi(t)$ is a homomorphism for each $t \in S$. Therefore
for each $t$, $\varphi(t)$ is an automorphism, or $\varphi$ is a map
$\varphi : S \to \mathrm{Aut}(G)$.

The automorphisms of $G$ are then in correspondence with the generators of $G$.
We can now specify automorphism groups of of these groups by
listing their generators.

\begin{enumerate}
  \item{The integers modulo $n$. This is a finite cyclic group, so let
        $a$ generate $\mathbb{Z}_n$, so
        that $\mathbb{Z}_n = \langle a \rangle$. Then this means that
        $\forall x \in \mathbb{Z}_n$, there exists a $b$ such that
       \begin{align*}
        x 
        &=
          \overbrace{a + a + \cdots + a}^{b \text{ times}} \mod n \\
        &=
          b \cdot a \mod n
        \end{align*}
        For $n \geq 2$, we have $1 \in \mathbb{Z}_n$, so taking $x =
        1 \mod n$ this means
        $1 \equiv b a \mod n$ for some $b$. But this is equivalent to
        the statement that $a$ is coprime to $n$. Furthermore
        $1 \equiv a \mod n$ for some $a$ means that
        $x = x \cdot 1 \equiv x \cdot a \mod n$ for any 
        $x \in \mathbb{Z}_n$, so the generators for $\mathbb{Z}_n$ are
        precisely the congruence classes of integers coprime to
        $n$. In the case of $\mathbb{Z}_{45}$ these are the congruence
        classes of the following integers:
        $$
        1, 2, 4, 7, 8, 11, 13, 14, 16, 17, 19, 22, 23, 26, 28, 29,
        31, 32, 34, 37, 38, 41, 43, 44.
        $$
      }
  \item{Since e.g. $(1 \mod 3, 1 \mod 5)$ generates 
        $\mathbb{Z}_3 \times \mathbb{Z}_5$, this is a finite cyclic
        group and by the same argument as above, we have that the
        generators $(a, b)$ are such that $a$ is coprime to 3 and
        $b$ is coprime to 5. The generators are thus the 
        following elements, as a brute-force computation will show:
        $$
        (1,1),(1,2),(1,3),(1,4),(2,1),(2,2),(2,3),(2,4).
        $$
       }
  \item{$\mathbb{Z}_5 \times \mathbb{Z}_5$ is not cyclic. However, we
        observe that for 
        $(x \mod 5, y \mod 5),
        \in \mathbb{Z}_5 \times \mathbb{Z}_5$, the matrix product
        \begin{align*}
        &\left[\begin{array}{c c}
           a \mod 5 & b \mod 5 \\
           c \mod 5 & d \mod 5
         \end{array}\right]
         \left[\begin{array}{c}
           x + x^\prime \mod 5 \\ y + y^\prime \mod 5
         \end{array}\right] \\
        =& 
        \left[\begin{array}{c}
          a(x + x^\prime) + b(y + y^\prime) \mod 5 \\ 
          c(x + x^\prime) + d(y + y^\prime) \mod 5
        \end{array}\right] \\
        =&
        \left[\begin{array}{c c}
          a \mod 5 & b \mod 5 \\
          c \mod 5 & d \mod 5
        \end{array}\right]
        \left[\begin{array}{c}
          x \\ y
        \end{array}\right]
        \left[\begin{array}{c c}
          a \mod 5 & b \mod 5 \\
          c \mod 5 & d \mod 5
        \end{array}\right]
        \left[\begin{array}{c}
          x^\prime \\ y^\prime
        \end{array}\right],
       \end{align*}
        so the map given by left-multiplication by this matrix is
        a group homomorphism 
        $\mathbb{Z}_5 \times \mathbb{Z}_5 \to 
         \mathbb{Z}_5 \times \mathbb{Z}_5$,
        and when this matrix is invertible then such a map represents an
        automorphism. Indeed for any homomorphism from this group to
        itself we require that 
        $$
        f((x,y))
        = x f((1, 0)) + y f((0, 1)),
        $$
        so such homomorphisms are determined by their actions on the
        corresponding basis vectors.
        Therefore we have
        $$
        \mathrm{Aut}(\mathbb{Z}_5 \times \mathbb{Z}_5)
        \simeq \mathrm{GL}_2(\mathbb{Z}_5)
        $$
       }
\end{enumerate}

\end{Answer}

\pagebreak

\begin{Problem}
Show that $SL_2(\mathbb{R})$ is perfect.
\end{Problem}

\begin{Answer}

\begin{enumerate}[(a)]
  \item{It is observed that
       \begin{align*}
         & \left[\begin{array}{c c}
           1               & \sqrt{a} + 1 \\
           0               & 1
         \end{array}\right]
         \left[\begin{array}{c c}
           1               & 0 \\
           \sqrt{a} - 1    & 1
         \end{array}\right]
         \left[\begin{array}{c c}
           1               & -\frac{\sqrt{a} + 1}{a} \\
           0               & 1
         \end{array}\right]
         \left[\begin{array}{c c}
           1               & 0 \\
           a(1 - \sqrt{a}) & 1
         \end{array}\right] \\
       = &
       \left[\begin{array}{c c}
         1 + a - 1    & \sqrt{a} + 1 \\
         \sqrt{a} - 1 & 1
       \end{array}\right]
       \left[\begin{array}{c c}
         1 + (a - 1)     & -\frac{\sqrt{a} + 1}{a} \\
         a(1 - \sqrt{a}) & 1
       \end{array}\right] \\
       = &
       \left[\begin{array}{c c}
           a            & 1 + \sqrt{a} \\
           \sqrt{a} - 1 & 1
       \end{array}\right]
       \left[\begin{array}{c c}
         a               & -\frac{\sqrt{a} + 1}{a} \\
         a(1 - \sqrt{a}) & 1
       \end{array}\right] \\
        =& 
        \left[\begin{array}{c c}
          a^2 + a(1 - a)                    & -(\sqrt{a} + 1) + 1 + \sqrt{a} \\
          a(\sqrt{a} - 1) + a(1 - \sqrt{a}) & -\frac{a - 1}{a} + 1
        \end{array}\right] \\
        =&
        \left[\begin{array}{c c}
          a & 0 \\ 0 & a^{-1}
        \end{array}\right],
       \end{align*}
       so
       $
        \left[\begin{array}{c c}
          a & 0 \\ 0 & a^{-1}
        \end{array}\right]
       $
       can be written as a product of elementary matrices $\forall a
       \in \mathbb{R} - \{ 0 \}$.
      }
  \item{It is observed that
        \begin{align*}
          &\left[\begin{array}{c c}
            a & 0 \\ 0 & a^{-1}
          \end{array}\right]
          \left[\begin{array}{c c}
            1 & s \\ 0 & 1
          \end{array}\right]
          \left[\begin{array}{c c}
            a & 0 \\ 0 & a^{-1}
          \end{array}\right]^{-1}
          \left[\begin{array}{c c}
            1 & s \\ 0 & 1
          \end{array}\right]^{-1} \\
       = &\left[\begin{array}{c c}
            a & 0 \\ 0 & a^{-1}
          \end{array}\right]
          \left[\begin{array}{c c}
            1 & s \\ 0 & 1
          \end{array}\right]
          \left[\begin{array}{c c}
            a^{-1} & 0 \\ 0 & a
          \end{array}\right]^{-1}
          \left[\begin{array}{r r}
            1 & -s \\ 0 & 1
          \end{array}\right]^{-1} \\
        = &\left[\begin{array}{c c}
            a & as \\ 0 & a^{-1}
          \end{array}\right]
          \left[\begin{array}{r r}
            a^{-1} & -a^{-1}s \\ 0 & a
          \end{array}\right] \\
       = &\left[\begin{array}{c c}
            1 & s(a^2 - 1) \\ 0 & 1
          \end{array}\right],
       \end{align*}
       and so letting $a = \sqrt{2}$ we have
       \begin{align*}
          \left[\begin{array}{c c}
            1 & s \\ 0 & 1
          \end{array}\right]
       &= \left[\begin{array}{c c}
            \sqrt{2} & 0 \\ 0 & \frac{1}{\sqrt{2}}
          \end{array}\right]
          \left[\begin{array}{c c}
            1 & s \\ 0 & 1
          \end{array}\right]
          \left[\begin{array}{c c}
            \sqrt{2} & 0 \\ 0 & \frac{1}{\sqrt{2}}
          \end{array}\right]^{-1}
          \left[\begin{array}{c c}
            1 & s \\ 0 & 1
          \end{array}\right]^{-1}.
       \end{align*}
       Since
       $$
       \det 
         \left[\begin{array}{c c}
           \sqrt{2} & 0 \\ 0 & \frac{1}{\sqrt{2}}
         \end{array}\right]
       = \det
         \left[\begin{array}{c c}
           1 & s \\ 0 & 1
         \end{array}\right]
       = 1
       $$
       for any $s \in \mathbb{R}$,
       these matrices both belong to $\mathrm{SL}_2(\mathbb{R})$,
       which means the product written above belongs to the commutator subgroup of 
       $\mathrm{SL}_2(\mathbb{R})$. Then 
       $
       \left[\begin{array}{c c}
         1 & s \\ 0 & 1
       \end{array}\right] \in (\mathrm{SL}_2(\mathbb{R})^c
       $ for all $s \in \mathbb{R}$. Similarly,
       \begin{align*}
         &\left[\begin{array}{c c}
            a^{-1} & 0 \\ 0 & a
          \end{array}\right]
          \left[\begin{array}{c c}
            1 & 0 \\ t & 1
          \end{array}\right]
          \left[\begin{array}{c c}
            a^{-1} & 0 \\ 0 & a
          \end{array}\right]^{-1}
          \left[\begin{array}{c c}
            1 & 0 \\ t & 1
          \end{array}\right]^{-1} \\
        =&\left[\begin{array}{c c}
            a^{-1} & 0 \\ 0 & a
          \end{array}\right]
          \left[\begin{array}{c c}
            1 & 0 \\ t & 1
          \end{array}\right]
          \left[\begin{array}{c c}
            a & 0 \\ 0 & a^{-1}
          \end{array}\right]^{-1}
          \left[\begin{array}{c c}
            1 & 0 \\ -t & 1
          \end{array}\right]^{-1} \\
        =&\left[\begin{array}{c c}
            a^{-1} & 0 \\ at & a
          \end{array}\right]
          \left[\begin{array}{c c}
            a & 0 \\ -a^{-1} t & a^{-1}
          \end{array}\right] \\
        =&\left[\begin{array}{c c}
            1 & 0 \\ t(a^2 - 1) & 1
          \end{array}\right],
        \end{align*}
        so
        \begin{align*}
          \left[\begin{array}{c c}
            1 & 0 \\ t & 1
          \end{array}\right]
        &=\left[\begin{array}{c c}
            \frac{1}{\sqrt{2}} & 0 \\ 0 & \sqrt{2}
          \end{array}\right]
          \left[\begin{array}{c c}
            1 & 0 \\ t & 1
          \end{array}\right]
          \left[\begin{array}{c c}
            \frac{1}{\sqrt{2}} & 0 \\ 0 & \sqrt{2}
          \end{array}\right]^{-1}
          \left[\begin{array}{c c}
            1 & 0 \\ t & 1
          \end{array}\right]^{-1}
        \end{align*}
        and thus
        $
          \left[\begin{array}{c c}
            1 & 0 \\ t & 1
          \end{array}\right] \in (\mathrm{SL}_2(\mathbb{R}))^c,
        $
        for any $t \in \mathbb{R}$.

        Therefore each elementary matrix is a commutator in the group
        $\mathrm{SL}_2(\mathbb{R})$.
    }
  \item{
      Let $\left[\begin{array}{c c}
            a & b \\ c & d
           \end{array}\right] \in \mathrm{SL}_2(\mathbb{R})$,
      so that $|ad - bc| = 1$. 

      Suppose $ad - bc = 1$. Then
      \begin{align*}
       &\left[\begin{array}{r r}
          1   & 0 \\
          -ac & 1
        \end{array}\right]
        \left[\begin{array}{c c}
          a & 0     \\
          0 & a^{-1}
        \end{array}\right]
        \left[\begin{array}{c c}
          a & b \\ c & d
        \end{array}\right]        
        \left[\begin{array}{c c}
          1 & -\frac{b}{a} \\
          0 & 1
        \end{array}\right] \\
      =&
        \left[\begin{array}{r r}
          1   & 0 \\
          -ac & 1
        \end{array}\right]
        \left[\begin{array}{c c}
          1 & \frac{b}{a} \\ ac & ad
        \end{array}\right]
        \left[\begin{array}{c c}
          1 & -\frac{b}{a} \\
          0 & 1
        \end{array}\right] \\
      =&\left[\begin{array}{r r}
          1   & 0 \\
          -ac & 1
        \end{array}\right]
        \left[\begin{array}{r r}
          1 & 0 \\ ac & ad - bc
        \end{array}\right] \\
      =&
        \left[\begin{array}{r r}
          1 & 0 \\ 0 & ad - bc
        \end{array}\right],
      \end{align*}
      and since $ad - bc = 1$ this is the identity matrix. Therefore
      multiplying by appropriate inverses (which are themselves
      elementary matrices) we see that the original matrix can be
      written as a product of elements of the commutator subgroup and
      thus belongs to the commutator subgroup itself.

      Similarly, if $ad - bc = -1$, then 
      \begin{align*}
       &\left[\begin{array}{r r}
          1   & 0 \\
          -ac & 1
        \end{array}\right]
        \left[\begin{array}{c c}
          -a & 0     \\
           0 & -a^{-1}
        \end{array}\right]
        \left[\begin{array}{c c}
          a & b \\ c & d
        \end{array}\right]        
        \left[\begin{array}{c c}
          1 & \frac{b}{a} \\
          0 & 1
        \end{array}\right] \\
      =&
        \left[\begin{array}{r r}
          1   & 0 \\
          -ac & 1
        \end{array}\right]
        \left[\begin{array}{c c}
          -1 & -\frac{b}{a} \\ -ac & -ad
        \end{array}\right]
        \left[\begin{array}{c c}
          1 & \frac{b}{a} \\
          0 & 1
        \end{array}\right] \\
      =&
        \left[\begin{array}{r r}
          1 & -\frac{b}{a} \\ 0 & bc - ad
        \end{array}\right]
        \left[\begin{array}{r r}
          1 & \frac{b}{a} \\ 0 & 1
        \end{array}\right] \\
      =&
        \left[\begin{array}{r r}
          1 & 0 \\ 0 & bc - ad
        \end{array}\right] = I.
      \end{align*}
        
      Then any element of 
      $\mathrm{SL}_2(\mathbb{R})$ can be decomposed as a product of
      elements of the commutator subgroup. Therefore 
      $\mathrm{SL}_2(\mathbb{R}) \subset (\mathrm{SL}_2(\mathbb{R}))^c$,
      and since $(\mathrm{SL}_2(\mathbb{R}))^c$ is a subgroup this
      means that $\mathrm{SL}_2(\mathbb{R})$ is equal to its
      commutator subgroup and thus perfect.
      }
\end{enumerate}

\end{Answer}

\end{document}
