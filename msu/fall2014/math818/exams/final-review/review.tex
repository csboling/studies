\documentclass{article}

\usepackage{amsmath, amssymb}

\begin{document}

\section{The Integers}
\begin{itemize}
  \item{
    Euclidean division says that for any integers $a, b$, there exist unique
    integers $q$, $r$ such that $a = bq + r$ and $0 \leq r < |b|$.

    Common in algebra are arguments like the following: take some algebraic
    object $X$ and let $f: X \to \mathbb{Z}_+$ be a map such that
    $f(X) \neq \varnothing$. Then $\mathrm{Im}(f)$ is a nonempty
    subset of the positive integers, so it has a smallest element
    $a_0$. Now let $a$ be arbitrary in $X$ and write
    $f(a) = q a_0 + r$. Then $r < a_0$ and so $r = 0$, which often
    tells us something nice about how $a$ relates to $f^{-1}(a_0)$.
  }
  \item{
    The GCD of two numbers $\mathrm{gcd}(a, b) = g$ is the number such
    that $x \vert a$ and $x \vert b$ implies $x \vert g$. If two
    integers have GCD one, they are said to be coprime. We may always
    write $g = ac + bd$ for some $c, d$.
  }
  \item{
    If $a$, $n$ are relatively prime and $n \vert ab$, then $n \vert
    b$. In particular if a prime $p \vert ab$, then $p \not{\vert} a$
    implies $p \vert b$.
  }
\end{itemize}

\section{Groups}
\subsection{Constructions}
  \begin{itemize}
    \item{
      Given a group $G$ and a set $S \subset G$,
      $$
      \langle S \rangle = \bigcap_{\substack{H < G \\ S \subset H}} H
      $$
      is the smallest subgroup that contains $S$, and every element of
      this set can be written as a product of elements of $S$ and
      their inverses. This is called the \emph{subgroup of $G$
        generated by $S$}. Note that every intersection of subgroups
      is a subgroup. We say that $S$ generates $G$.
    }
    \item{
      A group generated by a single element is a \emph{cyclic group}.
    }
    \item{
      The \emph{normalizer} $N_G(H)$ is
      $\{ g \in G \mid g H g^{-1} = H \}$, the set of all elements
      that normalize $H$. This is the largest group that contains $G$
      as a normal subgroup. $N(H) = G$ if and only if $H$ is normal.
    }
    \item{
      Let $N \triangleleft G$, $H < G$, $H \cap N = \{ 1 \}$, and $HN
      = G$. Then $G$ is the \emph{semidirect product} $H \rtimes N$.
      Equivalently there is a homomorphism $\varphi : H \to
      \mathrm{Aut}(N)$. If this homomorphism is injective then $G = H
      \times N$, and $H$ commutes with $N$, which is also the case
      when $H \triangleleft G$ as well. Indeed in this case
      $(n, h) \mapsto nh$ is an isomorphism. Notice also that from the
      second isomorphism theorem we have $G / N \simeq H$.

      Let $\psi : K \to \mathrm{Aut}(L)$ be a homomorphism. Then the operation
      $$
      (l, k) \cdot (l \psi(k) l^\prime, k k^\prime)
      $$
      defines a group, which is precisely the semidirect product. This
      makes it clear that when $\psi = \mathrm{const}~\mathrm{id}$
      that $L \ltimes K \simeq L \times K$.

      Note that whenever we have the ability to change the order of
      elements while causing some mutation, e.g. $xy = yx^\prime$ for
      $x \in H$, $y \in N$, this is a semidirect product with
      nontrivial kernel, because $\varphi : H \to \mathrm{Aut}(N)$
      is given by $\varphi(h)(n) = hnh^{-1} \neq n$. Equivalently
      we have $G = HN = NH$.
    }
    \item{
      For a group $G$, the \emph{commutator subgroup} $[G, G]$ is the
      subgroup generated by all elements of the form
      $ghg^{-1}h^{-1}$. Notice that $[G, G]$ is normal, $G / [G, G] = G^{ab}$
      is abelian, and every homomorphism $G \to A$ into an abelian
      group $A$ factors through the abelianization $G^{ab}$. Every
      normal subgroup of $G$ with trivial intersection with $[G, G]$
      is in $Z(G)$.
    }
    \item{
      The group of inner automorphisms $\mathrm{Inn}(G)$ is isomorphic
      to the quotient $G / Z(G)$ since the conjugation action
      $G \to \mathrm{Inn}(G)$ has quotient $Z(G)$.
    }
  \end{itemize}

\subsection{Fundamental Results}
Let $K < H < G$. All cosets $xHy$ have the same order,
$|H|$, and these partition the group $G$. A consequence of this is
that both the order of $H$ and the index of $H$ divide the order of
the group, namely
$$
|G| = [G : H]|H|
$$
or more generally
$$
[G : K] = [G : H][H : K].
$$
If $H$ is normal in $G$, this means $|G / H| = [G : H]$.

Every element of a finite group $G$ has an order that divides
the order of the group, since $\langle g \rangle$ is a subgroup
and because of Lagrange's theorem.

\begin{enumerate}
  \item{
    For any group homomorphism $\varphi : G \to H$,
    $$
    \frac{G}{\ker(\varphi)} \simeq \mathrm{Im}(\varphi)
    $$
    under the isomorphism
    $$
    \bar{\varphi}(g \ker(\varphi)) = \varphi(g).
    $$
    Indeed $\varphi = \bar{\varphi} \circ \pi_{\ker(\varphi)}$.
    This means every group homomorphism $\varphi $ factors
    through the quotient, i.e.
    $$
    G \to \frac{G}{\ker(\varphi)} \to H
    $$
    where the first homomorphism is surjective and the second is a group
    isomorphism.
  }
  \item{
    Let $K \triangleleft G$, $H \triangleleft G$, and $K < H$. Then
    $$
    \frac{G / H}{H / K} \simeq G / K
    $$
    since there is a homomorphism $\varphi : G / H \to G / K$ given by
    $$
    \varphi(gH) = gK
    $$

  }
  \item{
    Let $H, K < G$ such that $H$ \emph{normalizes} $K$, i.e. $H <
    N(K)$ or $hKh^{-1} = K$ for all $h \in H$. Then
    $$
    \frac{HK}{K} \simeq \frac{H}{H \cap K}
    $$
    with isomorphism given by $\pi_K \circ i$, where $i : H \to HK$ is
    the inclusion map.
  }
\end{enumerate}

\begin{itemize}
  \item{
    Every group homomorphism $\varphi : G \to H$ has associated
    subgroups $\ker(\varphi) \triangleleft G$ and
    $\mathrm{Im}(\varphi) < H$. Every normal subgroup of $G$ is the
    kernel of a group homomorphism out of $G$. The preimage of a
    normal subgroup under a group homomorphism is normal.
    $\ker(\varphi)$ is trivial iff. $\varphi$ is injective.
  }
  \item{
    A subgroup of index $p$, where $p$ is the smallest prime factor of
    $|G|$, is normal.
  }
\end{itemize}

\subsection{Solvable Groups}
\begin{itemize}
  \item{
    A group $G$ is solvable if it has a normal tower
    $$
    G
  = G_0    \triangleright
    G_1    \triangleright
    \cdots \triangleright
    G_n
  = \{ 1 \}
    $$
    such that every quotient $G_i / G_{i+1}$ is abelian.
  }
  \item{
    A group $G$ with normal subgroup $H$ is solvable if and only if
    $H$ and $G / H$ are both solvable.
  }
  \item{
    Any two normal towers of $G$ that both end with $\{1\}$ have
    refinements that are isomorphic up to permutation.
  }
\end{itemize}

\section{Group Actions}
The quotient
$$
G / G_x \simeq G . x
$$
for every $x$, so we can partition a set $S$ acted on by a group $G$
into $G$-orbits. Consequences of this include
$$
|S| = \sum_{x} [G : G_x]
$$
for representatives $x$ and
$$
G = \sum_{[G : C_x] > 1} [G : C_x] + |Z(G)|,
$$
where $C_x = \{gxg^{-1} \mid g \in G\}$ is the conjugacy class of
$x$. In particular, if $G$ is a $p$-group, this means $|Z(G)|$ is
nontrivial. A consequence of this is that a $p$-group is solvable.

The number of conjugates of a subgroup $H$ is given by $[G : N(H)]$.
Conjugate subgroups occupy the same conjugacy class.

\section{Sylow Theorems}
Suppose $p$ divides the order of $G$.
\begin{enumerate}
  \item{
    $G$ has at least one $p$-Sylow subgroup.
  }
  \item{
    All $p$-Sylows for fixed $p$ are conjugate, and every $p$-subgroup
    of $G$ is contained in a $p$-Sylow.
  }
  \item{
    The number of $p$-Sylows $N_p$ is congruent to $1 \bmod p$ and
    divides the order of $G$.
  }
\end{enumerate}

If $G$ is a finite abelian group of order $n$ and $p$ is a prime that
divides $n$, then $G$ has an element of order $p$.

If $|G| = pq$ with $p, q$ distinct primes then:
\begin{itemize}
  \item{
    $G$ is solvable.
  }
  \item{
    If $p < q$ then $G \simeq \mathbb{Z}_p \ltimes \mathbb{Z}_q$.
  }
  \item{
    If $p < q$ and $p \not{\vert} q - 1$ then
    $G \simeq \mathbb{Z}_p \times \mathbb{Z}_q
       \simeq \mathbb{Z}_{pq}$
  }
\end{itemize}

\section{Permutation Groups}
\begin{itemize}
  \item{
    Every $k$-cycle is of order $k$.
  }
  \item{
    Every permutation is a product of transpositions.
  }
  \item{
    Let $w : S_n \to \mathrm{S^\ast L}_n(\mathbb{Z})$ (matrices with
    determinant $\pm 1$) be given by $w(\sigma)_{ij} =
    \delta_i^{\sigma(j)}$. Note that this is precisely the matrix that,
    when left-multiplied by a vector, permutes its elements according to
    the rule $\sigma$. The determinant of this matrix is the sign
    homomorphism $\xi : S_n \to \{ \pm 1 \}$, i.e. $\xi = \det \circ
    w$. If $\xi(\sigma) = 1$ then $\sigma$'s disjoint composition
    contains an even number of cycles, otherwise $\xi(\sigma) = -1$.
  }
  \item{
    For any $\sigma \in S_n$ and any $k$-cycle $(a_1 \cdots a_k)$ we
    have
    $$
    \sigma (a_1 \cdots a_k) \sigma^{-1}
  = (\sigma(a_1) \cdots \sigma(a_k)).
    $$
  }
  \item{
    All $k$-cycles for fixed $k$ are conjugate.
    The conjugacy classes of $S_n$ are in correspondence with the
    partitions of $n$.
  }
  \item{
    If $n \geq 5$, $S_n$ is not solvable. For $n \neq 4$,
    $\ker(\xi)$ is simple.
  }
\end{itemize}

\section{Abelian Groups}

\subsection{Free Abelian Groups}
An abelian group $A$ is \emph{free} if it is isomorphic to
$$
\mathbb{Z}[I] \triangleq \oplus_{i \in I} \mathbb{Z}
$$
for some set $I$, or equivalently if there is a subset (basis) $B$
of $A$ such that every element of $A$ can be written as a cofinite
linear combination with integer coefficients of elements of $B$.

\begin{itemize}
  \item{
    For any abelian group $A$, set $I$ and set map $f : I \to A$
    there is a unique homomorphism $\bar{f} : \mathbb{Z}[I] \to A$ such that
    $f(\mathrm{inj}_i(1)) = f(i)$ for all $i \in I$.
  }
  \item{
    A surjective homomorphism $f$ into a free abelian group $A^\prime$ has a right inverse
    given by making a choice $g(a_i) = f^{-1}(a_i)$ for each $a_i$ in the basis of
    the target of $f$. Furthermore $A = \ker(f) \oplus \mathrm{Im}(g)$ and the
    restriction $f|_{\mathrm{Im}(g)} : \mathrm{Im}(g) \to A^\prime$ is an isomorphism, so
    $A = \ker{f} \oplus A^\prime$.
  }
  \item{
    In an abelian group all subgroups are normal, so when $A = B + C$ and
    $B \cap C = \{0\}$ we have $A \simeq B \times C = B \oplus C$.

    In a free abelian group all subgroups are free.
  }
  \item{
    Every basis of a free abelian group $A$ has the same cardinality,
    called the \emph{rank} of $A$.
  }
\end{itemize}

\subsection{Finitely Generated Abelian Groups}
Terminology:
\begin{itemize}
  \item{
    An abelian group $A$ is \emph{torsion} when every element has finite
    order. A group is torsion-free if it is not torsion, i.e. $na = 0$
    implies $n = 0$ or $a = 0$.
  }
  \item{
    An abelian group is \emph{finitely generated} when there is a
    finite subset of $A$ such that every element can be written as
    linear combinations of this subset. Note that for a free abelian group we
    only require that each element admits some representation as a
    finite linear combination of elements from some basis set -- in a
    free abelian group the basis may be infinite. An abelian group $A$
    is finitely generated if and only if there is a surjective map
    $\mathbb{Z}^n \to A$ for some $n$.
  }
  \item{
    An abelian group that is both torsion and finitely generated is
    finite. A finite group is of course torsion and finitely
    generated.

    An abelian group that is torsion-free and finitely generated is
    free. The map $A \to A$ given by $a \mapsto Na$ is injective
    since $A$ is torsion-free, and has image $NA$ so $NA \simeq A$ in
    this case, for any $N$.
  }
  \item{
    The \emph{torsion part} of an abelian group consists of all its
    torsion elements. The \emph{cotorsion} or \emph{torsion-free part}
    is $A / A_{tor}$. This is torsion free since the only torsion free
    element is $0 + A_{tor}$.
  }
  \item{
    If $A$ is finitely generated then its cotorsion is finitely
    generated and therefore free. In this case, since the canonical
    projection $\pi : A \to A / A_{tor} = \mathbb{Z}^n$ is surjective,
    $A \simeq \ker(f) \oplus \mathbb{Z}^n$.
  }
  \item{
    The $m$ torsion $A[m]$ consists of all elements whose order divides
    $m$. The $p$-power torsion $A[p^\infty]$ consists of all elements
    whose order divides a power of $p$. For coprime $r, s$,
    $A[rs] = A[r] \oplus A[s]$. This means in particular that the
    $m$ torsion has the decomposition
    $$
    A[m] = A[p_1^{k_1}] \oplus \cdots \oplus A[p_n^{k_n}]
    $$
    where $m = \prod_i p_i^{k_i}$ is the prime factorization of
    $m$. Indeed this factorization applies for any finite abelian
    group of order $m$.
  }
  \item{
    For a finite abelian group $A$ with
    $|A| = \prod_i p_i^{k_i}$, we have
    $A[p_i^\infty] = A[p_i^k]$. Note that $A$ is abelian and so every
    subgroup is normal, and therefore each $p_i$-Sylow of $A$ is
    unique.

    Let $P$ be the $p_i$-Sylow of $A$ and let $x \in P$. Then
    $|x|$ divides $|P| = p_i^{k_i}$, so $x \in A[p_i^{k_i}]$.

    Take $x \neq 0 \in A[p_i^{k_i}]$ and note that $|x|$ divides
    $p_i^{k_i}$ by definition, so $|x| = p_i^k$ for some
    $k \leq k_i$. But $|x|$ divides $|A[p_i^{k_i}]|$, so
    $A[p_i^{k_i}]$ is a $p_i$-group and therefore contained in the
    unique $p_i$-Sylow $P$. We conclude that $A[p_i^{k_i}] = P$ so
    that $|A[p_i^{k_i}]| = p_i^{k_i}$.

    Furthermore, $A[p_i^{k_i}] < A[p_i^\infty]$, so $A[p_i^\infty]$ is
    a $p$-group with order $\geq |A[p_i^{k_i}]|$ and therefore its
    order is the same.
  }
  \item{
    Indeed, any torsion abelian group $A$ is isomorphic to
    $\oplus_p A[p^\infty]$, where $p$ is taken over all primes. The
    isomorphism $\oplus_p A[p^\infty] \to A$ is given by
    $(a_p)_p \mapsto \sum_p a_p$.
  }
  \item{
    For any finite abelian group we can write down the isomorphism
    classes by:
    \begin{itemize}
      \item{factoring the order of the group,}
      \item{taking multiplicative partitions of each $p$-power
        factor,}
      \item{assembling each unique combination that takes one choice
        from each of these sets,}
      \item{simplifying the expression for each isomorphism class by
        multiplying coprime factors.}
    \end{itemize}

    For any finitely generated abelian group we can extend this
    procedure by observing that the torsion part is finitely generated
    and torsion and thus finite. The cotorsion is finitely generated
    and torsion-free and thus free, so it is isomorphic to
    $\mathbb{Z}^r$ for some $r$. Therefore every finitely generated
    abelian group is isomorphic to a direct sum of cyclic groups. This
    is the \emph{structure theorem for finitely generated abelian groups.}
  }
\end{itemize}

\section{Free Groups}

\section{Rings}
\begin{itemize}
  \item{
    It is possible for a ring element to have a left inverse but no
    right inverse, in which case it is not a unit. If both inverses
    exist they are equal. The units $A^\ast$ of a ring $A$ form a
    group under multiplication.
  }
  \item{
    A \emph{division ring} is a nonzero ring such that $A^\ast = A -
    \{ 0 \}$. A \emph{field} is a commutative divison ring.
  }
  \item{
    An ideal $I$ is an additive subgroup of $A$ such that
    $AI \subset I$. An ideal $Ax = \{ax \mid a \in A\}$ is the ideal
    generated by $x$. $I + J$ and $I \cap J$ are ideals for any ideals
    $I$, $J$.
  }
  \item{
    The quotient ring $A / I$ is the additive quotient group with
    multiplication $(x + I)(y + I) = xy + I$. The kernel of any ring
    homomorphism $A \to B$ is a two-sided ideal and factors as
    $A \to A / \ker(f) \to B$.
  }
  \item{
    A prime ideal is an ideal such that $ab \in I$ implies $a \in I$
    or $b \in I$. A maximal ideal $M$ is an ideal such that for any
    ideal $J$, $M \subset J \subset A = (1)$ implies $J = M$ or
    $J = A$. A maximal ideal is prime. Every proper ideal in $A$ is
    contained in a maximal ideal. The radical of $A$ (elements of
    finite multiplicative order) is the intersection of all prime
    ideals of $A$.
  }
  \item{
    A domain is a nonzero ring such that $xy = 0$ implies $x = 0$ or
    $y = 0$. $I$ is a prime ideal if and only if $A / I$ is a
    domain. $M$ is a maximal ideal if and only if $A / M$ is a field.
  }
  \item{
    Every multiplicative subset of a ring has empty intersection with
    some prime ideal of $A$. In a domain, $S = A - \{ 0 \}$ is a
    multiplicative set. In this case the localization $S^{-1}A$ is
    called the fraction field of $A$.
  }
  \item{
    Every PID has GCDs -- the GCD of two elements $a, b$ is the
    element $c$ such that $(a, b) = (c)$.
  }
  \item{
    In any domain, every principal ideal is generated by an
    irreducible element. The rings where the converse is true are the
    unique factorization domains, since in this case $xy = za$ means
    $a$ must be a factor in $x$ or in $y$.
  }
  \item{
    A PID is a UFD. Euclid's lemma (a prime dividing a product divides
    one of the products) holds for irreducible elements in PIDs (since
    GCDs exist).
  }
\end{itemize}

A partial order $\leq$ is a relation that is reflexive, transitive, and
antisymmetric. If either $x \leq y$ or $y \leq x$, this relation is a
total order. A poset is \emph{inductively ordered} if every totally
ordered subset $T \subset S$ has an upper bound in $S$. A maximal
element $m$ is one such that $m$ is not $\leq x$ for any $x$.

Zorn's lemma: Every nonempty inductively ordered poset has at least
one maximal element.


\section{Noetherian Rings}
A ring is Noetherian if every ideal is finitely generated, or
equivalently if every chain of ideals eventually stabilizes, or if
every non-empty set of ideals contains a maximal element. All PIDs
are Noetherian since every ideal is generated by 1 element.

Given a Noetherian ring $A$, a polynomial ring $A[x_1, \dots, x_n]$ is
Noetherian for a finite list of variables. Any quotient ring $A / I$
is Noetherian because the quotient map is surjective and $\pi^{-1}(J)$
is an ideal for any $J \subset A / I$.

\section{UFDs and Polynomials}
The content of a polynomial is a function $A[x] \to S^{-1}A$ where
$S = A - \{ 0 \}$, i.e. $S^{-1} A = K = \mathrm{Frac}(A)$, defined by
$$
\prod_p p^{\mathrm{ord}_p(f(x))}
$$
where the order of a prime element is the minimum exponent $n$
(including negative) such that $p^n$ appears in a coefficient of
$f(x)$. Note that $p$ ranges over equivalence classes of prime
elements and that the content is only defined up to a unit. Then we
can write $f(x) = \mathrm{cont}(f(x)) f_1(x)$ where $f_1(x)$ has
coefficients with GCD 1, i.e. $\mathrm{cont}(f_1(x)) = 1$.

To show that the content of the product is the product of the
contents, take an irreducible $p$ and notice that $A / (p)$ is an
integral domain. Then since $p$ does not divide all the coefficients
of $f_1$ or $g_1$, then $f_1$, $g_1$ are not 0 mod $p$ and so neither
is $f_1 g_1$.

$f$ is irreducible in $A[x]$ if and only if it is irreducible in
$K[x]$ where $K = \mathrm{Frac}(A)$. If an irreducible $p$ does not
divide the leading coefficient of $f$, but divides every other
coefficient of $f$ and $p^2$ does not divide the constant coefficient
$a_0$, then $f$ is irreducible.
If $f$ is monic and $f \bmod p$ is irreducible in the fraction field
of $A / (p)$, then $f$ is irreducible in the fraction field of $A$.

\end{document}
