\documentclass{article}

\begin{document}

\section{Groups}
\begin{defn}[Constructions]
  \begin{itemize}
    \item{
      Given a group $G$ and a set $S \subset G$,
      $$
      \langle S \rangle = \bigcap_{\substack{H < G \\ S \subset H}} H
      $$
      is the smallest subgroup that contains $S$, and every element of
      this set can be written as a product of elements of $S$ and
      their inverses. This is called the \emph{subgroup of $G$
        generated by $S$}. Note that every intersection of subgroups
      is a subgroup. We say that $S$ generates $G$.
    }
    \item{
      A group generated by a single element is a \emph{cyclic group}.
    }
    \item{
      The \emph{normalizer} $N_G(H)$ is
      $\{ g \in G \mid g H g^{-1} = H \}$, the set of all elements
      that normalize $H$. This is the largest group that contains $G$
      as a normal subgroup. $N(H) = G$ if and only if $H$ is normal.
    }
    \item{
      Let $N \triangleleft G$, $H < G$, $H \cap N = \{ 1 \}$, and $HN
      = G$. Then $G$ is the \emph{semidirect product} $H \rtimes N$.
      Equivalently there is a homomorphism $\varphi : H \to
      \mathrm{Aut}(N)$. If this homomorphism is injective then $G = H
      \times N$, and $H$ commutes with $N$, which is also the case
      when $H \triangleleft G$ as well. Notice also that from the
      second isomorphism theorem we have $G / N \simeq H$.

      Note that whenever we have the ability to change the order of
      elements while causing some mutation, e.g. $xy = yx^\prime$ for
      $x \in H$, $y \in N$, this is a semidirect product with
      nontrivial kernel, because $\varphi : H \to \mathrm{Aut}(N)$
      is given by $\varphi(h)(n) = hnh^{-1} \neq n$. Equivalently
      we have $G = HN = NH$.
    }
  \end{itemize}
\end{defn}


\subsection{Fundamental Results}
\begin{theorem}[Lagrange's Theorem[
Let $K < H < G$. All cosets $xHy$ have the same order,
$|H|$, and these partition the group $G$. A consequence of this is
that both the order of $H$ and the index of $H$ divide the order of
the group, namely
$$
|G| = [G : H]|H|
$$
or more generally
$$
[G : K] = [G : H][H : K].
$$
If $H$ is normal in $G$, this means $|G / H| = [G : H]$.
\end{theorem}

\begin{enumerate}
  \item{
    For any group homomorphism $\varphi : G \to H$,
    $$
    \frac{G}{\ker(\varphi)} \simeq \mathrm{Im}(\varphi)
    $$
    under the isomorphism
    $$
    \bar{\varphi}(g \ker(\varphi)) = \varphi(g).
    $$
  }
  \item{
    Let $K \triangleleft G$, $H \triangleleft G$, and $K < H$. Then
    $$
    \frac{G / H}{H / K} \simeq G / K
    $$
    since there is a homomorphism $\varphi : G / H \to G / K$ given by
    $$
    \varphi(gH) = gK
    $$

  }
  \item{
    Let $H, K < G$ such that $H$ \emph{normalizes} $K$, i.e. $H <
    N(K)$ or $hKh^{-1} = K$ for all $h \in H$. Then
    $$
    \frac{HK}{K} \simeq \frac{H}{H \cap K}
    $$
    with isomorphism given by $\pi_K \circ i$, where $i : H \to HK$ is
    the inclusion map.
  }
\end{enumerate}

\begin{itemize}
  \item{
    Every group homomorphism $\varphi : G \to H$ has associated
    subgroups $\ker(\varphi) \triangleleft G$ and
    $\mathrm{Im}(\varphi) < H$. Every normal subgroup of $G$ is the
    kernel of a group homomorphism out of $G$. The preimage of a
    normal subgroup under a group homomorphism is normal.
    $\ker(\varphi)$ is trivial iff. $\varphi$ is injective.
  }
  \item{
    A subgroup of index $p$, where $p$ is the smallest prime factor of
    $|G|$, is normal.
  }
\end{itemize}

\subsection{Solvable Groups}
\begin{itemize}
  \item{
    A group $G$ is solvable if it has a normal tower
    $$
    G
  = G_0    \triangleright
    G_1    \triangleright
    \cdots \triangleright
    G_n
  = \{ 1 \}
    $$
    such that every quotient $G_i / G_{i+1}$ is abelian.
  }
  \item{
    A group $G$ with normal subgroup $H$ is solvable if and only if
    $H$ and $G / H$ are both solvable.
  }
  \item{
    Any two normal towers of $G$ that both end with $\{1\}$ have
    refinements that are isomorphic up to permutation.
  }
\end{itemize}

\section{Group Actions}
The quotient
$$
G / G_x \simeq G . x
$$
for every $x$, so we can partition a set $S$ acted on by a group $G$
into $G$-orbits. Consequences of this include
$$
|S| = \sum_{x} [G : G_x]
$$
for representatives $x$ and
$$
G = \sum_{[G : C_x] > 1} [G : C_x] + |Z(G)|,
$$
where $C_x = \{gxg^{-1} \mid g \in G\}$ is the conjugacy class of
$x$. In particular, if $G$ is a $p$-group, this means $|Z(G)|$ is
nontrivial. A consequence of this is that a $p$-group is solvable.

The number of conjugates of a subgroup $H$ is given by $[G : N(H)]$.
Conjugate subgroups occupy the same conjugacy class.

\section{Sylow Theorems}
Suppose $p$ divides the order of $G$.
\begin{enumerate}
  \item{
    $G$ has at least one $p$-Sylow subgroup.
  }
  \item{
    All $p$-Sylows for fixed $p$ are conjugate, and every $p$-subgroup
    of $G$ is contained in a $p$-Sylow.
  }
  \item{
    The number of $p$-Sylows $N_p$ is congruent to $1 \bmod p$ and
    divides the order of $G$.
  }
\end{enumerate}

If $G$ is a finite abelian group of order $n$ and $p$ is a prime that
divides $n$, then $G$ has an element of order $p$.

If $|G| = pq$ with $p, q$ distinct primes then:
\begin{itemize}
  \item{
    $G$ is solvable.
  }
  \item{
    If $p < q$ then $G \simeq \mathbb{Z}_p \ltimes \mathbb{Z}_q$.
  }
  \item{
    If $p < q$ and $p \not{\vert} q - 1$ then
    $G \simeq \mathbb{Z}_p \times \mathbb{Z}_q
       \simeq \mathbb{Z}_{pq}$
  }
\end{itemize}

\section{Permutation Groups}
\begin{itemize}
  \item{
    Every $k$-cycle is of order $k$.
  }
  \item{
    Every permutation is a product of transpositions.
  }
  \item{
    Let $w : S_n \to \mathrm{S^\ast L}_n(\mathbb{Z})$ (matrices with
    determinant $\pm 1$) be given by $w(\sigma)_{ij} =
    \delta_i^{\sigma(j)}$. Note that this is precisely the matrix that,
    when left-multiplied by a vector, permutes its elements according to
    the rule $\sigma$. The determinant of this matrix is the sign
    homomorphism $\xi : S_n \to \{ \pm 1 \}$, i.e. $\xi = \det \circ
    w$. If $\xi(\sigma) = 1$ then $\sigma$'s disjoint composition
    contains an even number of cycles, otherwise $\xi(\sigma) = -1$.
  }
  \item{
    For any $\sigma \in S_n$ and any $k$-cycle $(a_1 \cdots a_k)$ we
    have
    $$
    \sigma (a_1 \cdots a_k) \sigma^{-1}
  = (\sigma(a_1) \cdots \sigma(a_k)).
    $$
  }
  \item{
    All $k$-cycles for fixed $k$ are conjugate.
    The conjugacy classes of $S_n$ are in correspondence with the
    partitions of $n$.
  }
  \item{
    If $n \geq 5$, $S_n$ is not solvable. For $n \neq 4$,
    $\ker(\xi)$ is simple.
  }
\end{itemize}

\section{Abelian Groups}

\section{Free Groups}

\section{Rings}

\section{Noetherian Rings}

\section{UFDs and Polynomials}

\end{document}
