\section{Calculus on manifolds}
Let $M$ be a smooth manifold and $p \in M$, with
$C_p^\infty(M)$ the set of germs of smooth functions at $p$, i.e. the
$\mathbb{R}$-algebra of equivalence classes of pairs $(U, f)$, where
U is open, $f: U \to \mathbb{R}$ is smooth and $p \in U$, where two
such pairs are equivalent when
$\exists p \in U \subset U_1 \cap U_2$ with $f_1 = f_2$ on $U$.

Point derivations of $C_p^\infty(M)$ are $\mathbb{R}$-linear maps
$D: C_p^\infty(M) \to \mathbb{R}$ such that
$D(fg)(p) = D(f) g(p) + f(p) D(g)$.

The tangent space $T_pM$ of $M$ at $p$ is the $\mathbb{R}$-vector space of
point derivations of $C_p^\infty(M)$.

\begin{defn}[Differential]
The \emph{differential} of a smooth map $F: N \to M$ at $p$ is a
smooth map
$$
(F_\ast)_p = F_{\ast,p} = D_p F : T_p N \to T_{F(p)} M.
$$
For any $X_p \in T_pN$, $g \in C_{F(p)}^\infty M$, this acts as
$$
F_{\ast,p}(X_p)(g) \triangleq X_P(g \circ F).
$$
\end{defn}

\begin{prop}
  \begin{enumerate}
    \item{
      Given $F: N \to M$ and $G: M \to P$,
      $$
        (G \circ F)_{\ast, p}
      = G_{\ast, F(p)} \circ F_{\ast, p}.
      $$
    }
    \item{
      The differential of $\id_M$ is $\id_{T_pM}$.
    }
  \end{enumerate}
\end{prop}
\begin{proof}
  \begin{enumerate}
    \item{
      Let $X_p \in T_pM$ and $g \in C_{f(p)}^\infty(P)$. Then
      \begin{align*}
         G_{\ast, f(p)} \circ F_{\ast, p} (X_p) g
      &= G_{\ast, f(p)} (F_{\ast, p} (X_p)) g \\
      &= F_{\ast p}(X_p) (g \circ G) \\
      &= X_p(g \circ G \circ F) \\
      &= (G \circ F)_{\ast, p}(X_p)(g).
      \end{align*}
    }
    \item{
      $$
        \id_{\ast,p}(X_p)(g)
      = X_p(g \circ \id_M)
      = X_p(g).
      $$
    }
  \end{enumerate}
\end{proof}

\begin{remark}
For $p \in U \subset M$ with $U$ open,
$C_p^\infty(U) = C_p^\infty(M)$, so $T_pU = T_pM$.
\end{remark}

\begin{corol}
If $F: N \to M$ is a local diffeomorphism, then
$F_{\ast, p} : T_p N \to T_{f(p)} M$ is an isomorphism of vector spaces.
\end{corol}
\begin{proof}
If $F$ has a $C^\infty$ inverse $F^{-1} : M \to N$ at $p$, then
$$
    (F \circ F^{-1})_{\ast, F(p)}
  = (\id_M)_{\ast, F(p)}
  = \id_{T_{F(p)}}
$$
and likewise $(F^{-1} \circ F)_{\ast, p} = \id_{T_pN}$.
\end{proof}

\subsection{Basis for $T_pM$}
Let $(U, x^1, \dots, x^m)$ be a coordinate chart about $p$
and $\phi$ be a chart to $(\mathbb{R}^m, r^1, \dots, r^m)$.
Recall that $T_{\phi(p)} \mathbb{R}^m$ has basis
$\left.\frac{\partial}{\partial r^i}\right|_{\phi(p)}$.
Then form
$\left\{ \left.\frac{\partial}{\partial x^i}\right|_p \right\}_{i=1}^m$.
Since $\phi$ is a diffeomorphism onto its image,
$\phi_\ast : T_pM \to T_{\phi(p)} \mathbb{R}^m$ is an isomorphism. We
compute
\begin{align*}
   \phi_{\ast,p}
     \left(
       \left.
         \frac{\partial}{\partial x^i}
       \right|_p
     \right)(g)
&= \left.
     \frac{\partial}{\partial x^i}
   \right|_p
     (g \circ \phi) \\
&= \frac{\partial (g \circ \phi \circ \phi^{-1})}
        {\partial r^i}
   (\phi(p)) \\
&= \frac{\partial g}{\partial r^i}(\phi(p)) \\
&= \left.\frac{\partial}{\partial r^i}\right|_{\phi(p)} g.
\end{align*}
Therefore
$$
  \phi_\ast
    \left(
      \left.
        \frac{\partial}{\partial x^i}
      \right|_p
    \right)
= \left.
    \frac{\partial}{\partial r^i}
  \right|_{\phi(p)}.
$$
This shows that
$\left\{ \left.\frac{\partial}{\partial x^i}\right|_p
\right\}_{i=1}^m$
is an induced basis for $T_pM$.

\begin{corol}
$\dim T_p M = \dim M = m$. Consequently
$\mathbb{R}^n$ and $\mathbb{R}^m$ are not diffeomorphic if
$n \neq m$. They are not homeomorphic either, but this is more
difficult to show.
\end{corol}

\subsection{Change of coordinates}
Let $p \in M$ and $(U, x^1, \dots, x^m)$, $(V, y^1, \dots, y^m)$ be
coordinates about $p$, with bases $\{\partial_{x^i}|_p\}$ and
$\{\partial_{y_i}|_p\}$. Then we can write
$$
  \left.
    \frac{\partial}{\partial x^j}
  \right|_p
= \sum_{k=1}^m
    a_j^k
    \left.\frac{\partial}{\partial y^k}\right|_p
$$
for some unique coefficients $a_j^k$.

Evaluating both sides on $y^i: U \cap V \to \mathbb{R}$ gives
\begin{align*}
   \sum_{k=1}^n
     a_j^k
     \frac{\partial y^i}{\partial y^k}(p)
&= \sum_k a_j^k \delta_k^i = a_j^i
\end{align*}
so that
$$
  \left.\frac{\partial}{\partial x^j}\right|_p
= \sum_{i=1}^m
    \frac{\partial y^i}{\partial x^j}
    \left.
      \frac{\partial}{\partial y^i}
    \right|_p.
$$

\subsection{Differential in coordinates}
Let $F: N \to M$ with a chart $(U, x, \phi)$ at
$p$ and $(V, y, \psi)$ at $F(p)$. We see that
$$
  F_{\ast,p}
  \left(
    \left.
      \frac{\partial}{\partial x^j}
    \right|_p
  \right)
= \sum_{k=1}^n
    a_m^k
    \left.
      \frac{\partial}{\partial y^k}
    \right|_{F(p)}
$$
for some $a_m^k$. To determine $a_m^k$, let both sides act as a
derivation on $y^i: V \to \mathbb{R}$ so that
\begin{align*}
   F_{\ast, p}
     \left(
       \left.
         \frac{\partial}{\partial x^j}
       \right|_p
     \right)(y^i)
&= \left.
     \frac{\partial}{\partial x^j}
   \right|_p
     (y^i \circ F)
 = \frac{\partial F^i}{\partial x^j}(p)
\end{align*}
whereas
$$
  \left(
    \sum_{k=1}^m
      a_j^k
      \left.
        \frac{\partial}{\partial y^k}
      \right|_{F(p)}
  \right)
    (y^i)
= \sum_{k=1}^n
    a_j^k
    \frac{\partial y^i}{\partial y^k}
= a^i_j.
$$
Therefore in coordinates $F_{\ast,p} = J(F)(p).$

\begin{theorem}[Coordinate-free inverse function theorem]
$F: N \to M$ is a local diffeomorphism at $p$ if and only if
$F_{\ast,p}$ is invertible.
\end{theorem}
