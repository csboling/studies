\section{de Rham theory}
We have a cochain complex
$$
\cdots \to \{ 0 \} \to \{ 0 \} \to
\Omega^0(M) = C^\infty(M) \to
\Omega^1(M) \to \cdots \to
\Omega^{n-1}(M) \to
\Omega^n(M) \to
\{0\} \to \{0\} \to \cdots,
$$
known as the \emph{de Rham complex}
$(\Omega^k(M), \dif^k)$, which satisfies $\dif \circ \dif = 0$.
This means that
$$
        \mathrm{im}(d^{k-1}: \Omega^{k-1}(M) \to \Omega^k(M))
\subset \ker(d^k : \Omega^k(M) \to \Omega^{k+1}(M)).
$$
We call $Z^k(M) = \ker(\dif^k)$, i.e. the closed $k$-forms,
the \emph{cocycles}, and $B^k(M) = \mathrm{im}(\dif^{k-1})$, i.e. the
exact $k$-forms, the \emph{coboundaries}, and define the
\emph{de Rham cohomology groups}
$$
  H^k_{dR}(M)
= Z^k(M) / B^k(M).
$$
Elements in $H^k_{dR}(M)$ are equivalence classes $[\omega]$ for the
equivalence relation $\sim$ on $Z^k(M)$ given by
$$
     \omega_1 \sim \omega_2
\iff \omega - \omega_2 \in B^k(M)
\iff \omega_1 - \omega_2 = \dif \alpha, \alpha \in \Omega^{k-1}(M).
$$
These are called the \emph{cohomology classes}.

\begin{prop}
$H^0(M) = \mathbb{R}^l$, where $l$ is the number of connected
components of $M$.
\end{prop}

\begin{proof}
$H^0(M) = \ker \dif^0 / \{ 0 \} = \ker \dif^0$, i.e.
$\{ f : C^\infty(M, \mathbb{R}) ~\vert~ \dif f = 0 \}$.
Let $e_1, \dots, e_l$ be the connected components of $M$,
choose elements $c_1 \in e_1$, and define a function
$H^0(M) = Z^0(M) \to \mathbb{R}^l$ given by
$f \mapsto (f(c_1), \dots, f(c_n))$. This is $\mathbb{R}$-linear and
onto, and one-to-one since
$\dif f = \sum \frac{\partial f}{\partial x^i} \dif x = 0$ so that
$\frac{\partial f}{\partial x^i} = 0$ for all $i$, i.e. each
$f$ is locally constant.
\end{proof}

\begin{remark}
$H^k(M) = \{ 0 \}$ for $k > \dim M$.
\end{remark}

\begin{xmpl}
$H^0(\mathbb{R}) = \mathbb{R}$, $H^k(\mathbb{R}) = \{ 0 \}$ for $k >
1$.

$Z^1(\mathbb{R}) = \Omega^1(\mathbb{R}) = \{ f \dif t ~\vert~ f \in
C^\infty(\mathbb{R})\}$.
We claim that $\dif(\Omega^0(\mathbb{R})) = Z^1(\mathbb{R}) =
\Omega^1(\mathbb{R})$.

Let $\alpha = f \dif t$. We wish to show that there exists a
$g \in C^\infty(\mathbb{R})$ with $\dif g = \alpha = f \dif t$.
Define
$$
  g(t)
= \int_0^t
    f(u) \dif u
\in C^\infty(\mathbb{R}).
$$
By the fundamental theorem of calculus, $\dot{g}(t) = f(t)$, so
$\dif g = \dot{g}(t) \dif t = \alpha$.
\end{xmpl}

\begin{xmpl}[Cohomology of $S^1$]
We will see that
$$
  H_{dR}^k(S^1)
= \left\{
    \begin{array}{c c}
      \mathbb{R}, & k = 0 \\
      \mathbb{R}, & k = 1 \\
      0,          & k > 1 \\
    \end{array}
  \right.
$$
and more generally
$$
  H_{dR}^k(S^n)
= \left\{
    \begin{array}{c c}
      \mathbb{R}, & k = 0 \\
      \mathbb{R}, & k = n \\
      0,          & k \neq 0, k \neq 1 \\
    \end{array}
  \right.
$$
using the Mayer-Vietoris sequence.

The 1-cocycles
$Z^1(S^1) = \ker \dif_1 = \Omega^1(S^1)$ are all the top forms on
$S^1$, so we have the integration map
$I: Z^1(S^1) \to \mathbb{R}$ given by
$\alpha \mapsto \int_{S^1} \alpha$. We also have a map
$h : \mathbb{R} \to S^1$ given by $h(t) = (\cos t, \sin t)$,
inclusion $i: [0, 2\pi] \to \mathbb{R}$, and an orientation form
$\omega = (-y \dif x + x \dif y)\restrict_{S^1}$. We have seen
$h^\ast \omega = \dif t$ and $h_\ast\left(\frac{\dif}{\dif t}\right) =
X$. The map $H = h \circ i : [0, 2\pi] \to S^1$ is a diffeomorphism
onto its interior, so this is a parameterization of $S^1$.

To show that $I$ is surjective,
$$
  I(\omega)
= \int_{S^1} \omega
= \int_0^{2\pi} H^\ast \omega
= \int_0^{2\pi} \dif t
= 2 \pi > 0
$$
and since $I$ is $\mathbb{R}$-linear, any real number is in the image
of $I$. By the first isomorphism theorem,
$$
\mathbb{R} \simeq Z^1(S^1) / \ker(I),
$$
so we wish to show that $\ker(I)$ consists of exactly the
1-coboundaries on $S^1$, i.e. the exact 1-forms.

Let $\dif g \in B^1(S^1)$. Then
$$
I(\dif g) = \int_{S^1} \dif g = \int_{\partial S^1} g = 0.
$$

Let $\alpha \in \ker I$. Then $\alpha$ is a top form, so
$\alpha = f \omega$. Let
$\bar{f} = h^\ast f = f \circ h$, and note that
$h^\ast \alpha = \bar{f} h^\ast \omega = \bar{f} \dif t$. Since
$h$ is $2\pi$-periodic, the pullback of $f$ by a periodic map
$\bar{f}$ is $2\pi$-periodic. Since $\alpha \in \ker I$,
$$
0
= \int_{S^1} \alpha
= \int_{S^1} f \omega
= \int_0^{2\pi} \bar{f}(t) \dif t.
$$
Let
$$
  \bar{g}(t)
= \int_0^t
    \bar{f}(u) \dif u.
$$
Then
$$
  \dif \bar{g}
= \dot{\bar{g}}(t) \dif t
= \bar{f}(t) \dif t
= h^\ast(\alpha).
$$
But
$$
  \bar{g}(t + 2 \pi)
= \int_0^{t + 2 \pi}
    \bar{f}(u) \dif u
= \int_0^{2\pi}
    \bar{f}(u) \dif u
+ \int_{2\pi}^{2\pi + t}
    \bar{f}(u) \dif u
= \int_0^t \bar{f}(u) \dif t
= \bar{g}(t).
$$
Therefore there exists a smooth function $g$ on the circle whose pullback
is the $2\pi$-periodic function $\bar{g}$, i.e.
$\dif (h^\ast g) = h^\ast (\dif g)$, so $\alpha = \dif g$ since
$h^\ast : \Omega^1(S^1) \to \Omega^1(\mathbb{R})$ is one-to-one.
To see this, let $0 = h^\ast(\phi \omega)$ so that
$$
  0
= h^\ast(\phi \omega) \left(\frac{\dif}{\dif t}\right)
= \phi \omega \left(h_\ast \frac{\dif}{\dif t}\right)
= \phi(\omega(X))
= \phi.
$$
\end{xmpl}

\subsection{Action of smooth maps on cohomology}
Let $F: N \to M$, so that $F^\ast : \Omega^k(M) \to \Omega^k(N)$. Then
\begin{enumerate}
  \item{
    $F^\ast(Z^k(M)) \subset Z^k(N)$, since if $\alpha \in Z^k(M)$,
    $\dif \alpha = 0$, so $\dif(F^\ast \alpha) = F^\ast (\dif \alpha)
    = 0$.
  }
  \item{
    $F^\ast(B^k(M)) \subset B^k(N)$, since if $\alpha \in B^k(M)$ with
    $\alpha = \dif \beta$ then
    $$
    F^\ast(\alpha) = F^\ast(\dif \beta) = \dif (F^\ast \beta) \in B^k(N).
    $$
  }
  \item{
    By (1) and (2),
    $F^\ast : \Omega^k(M) \to \Omega^k(N)$ induces a well-defined
    linear map $F^\ast : H^k(M) \to H^k(N)$ given by
    $F^\ast([\omega]) = [F^\ast \omega]$.

    This shows that cohomology is a contravariant functor, i.e.
    \begin{enumerate}
      \item{
        $(\mathrm{Id}_M)^\ast = \mathrm{Id}_{H^k(M)}$
      }
      \item{
        For $F: N \to M$ and $G: M \to P$,
        $(G \circ F)^\ast = F^\ast \circ G^\ast : H^k(P) \to H^k(N)$.
      }
    \end{enumerate}
  }
  \item{
    Cohomology groups are thus invariant under diffeomorphisms. If
    $F: N \to M$ is a diffeomorphism, then $F^\ast: H^k(M) \to H^k(N)$
    is an isomorphism. This is because
    $$
      \mathrm{Id}_{H^k(M)}
    = (\mathrm{Id}_M)^\ast
    = (F \circ F^{-1})^\ast
    = (F^{-1})^\ast \circ F^\ast
    $$
    and $\mathrm{Id}_{H^k(N)} = F^\ast \circ (F^{-1})^\ast$, so
    $F^\ast$ is an isomorphism.
  }
  \item{
    Indeed, de Rham cohomology is invariant under homotopy
    equivalence. A smooth homotopy of smooth maps
    $f, g: N \to M$ is a smooth map $H : N \times I \to M$ with
    $H(\cdot, 0) = f(\cdot)$ and $H(\cdot, 1) = g(\cdot)$.
    We have an equivalence relation given by $f \sim g$ if there
    exists a smooth homotopy from $f$ to $g$.

    A \emph{homotopy equivalence} between $N$ and $M$ is a map $f: N \to M$ with a
    homotopy inverse, i.e. a map $g : M \to N$ such that
    $g \circ f$ is homotopic to $\mathrm{Id}_N$. If such an
    equivalence exists, we say $M$ and $N$ are homotopy equivalent --
    this is an equivalence relation on spaces.

    \begin{theorem}
      Maps in cohomology only depend on the homotopy class of a map.
    \end{theorem}

    \begin{corol}
      If $f: N \to M$ is a homotopy equivalence, then
      $f^\ast : H^k(M) \to H^k(N)$ is an isomorphism for all $k$. This
      argument works just the same as the argument for diffeomorphisms.
    \end{corol}
  }
\end{enumerate}

\begin{defn}
  A space is \emph{contractible} if it is homotopy equivalent to a point.
\end{defn}

\begin{xmpl}[Homotopy equivalent spaces]
  \begin{enumerate}
    \item{
      $\mathbb{R}^n$ is contractible. Let $0 \in \mathbb{R}^n$. Then
      the point map $p : \mathbb{R}^n \to \{ 0 \}$ has a homotopy
      inverse given by $i: \{0\} \to \mathbb{R}^n$. Clearly
      $p \circ i : \{0\} \to \{0\} = \mathrm{Id}$.
      $i \circ p : \mathbb{R}^n \to \mathbb{R}^n$ is given by
      $x \mapsto 0$, i.e. $i \circ p = p$. Define a homotopy
      $H: \mathbb{R}^n \times I \to \mathbb{R}^n$ given by
      $H(x, t) = t x$. Then $H(x,1) = x =
      \mathrm{Id}_{\mathbb{R}^n}(x)$ and $H(x, 0) = 0 = p(x)$.
      This shows that
      $$
             H^k(\mathbb{R})^n
      \simeq H^k(\{\ast\})
      =      \left\{
               \begin{array}{c c}
                 \mathbb{R}, & \quad k = 0 \\
                 0,          & \quad k > 0.
               \end{array}
             \right.
     $$
     }
  \end{enumerate}
\end{xmpl}

\section{Homological Algebra}

\begin{defn}
A sequence of vector spaces
$A \xrightarrow{f} B \xrightarrow{g} C$ is said to be
\emph{exact at $B$} if $\mathrm{im}(f) = \ker(g)$.

A sequence
$$
  A_0    \xrightarrow{f_0}
  A_1    \xrightarrow{f_1}
  A_2    \xrightarrow{f_2}
  \cdots \xrightarrow{f_{n-2}}
  A_{n-1} \xrightarrow{f_{n-1}}
  A_n
$$
is an \emph{exact sequence} if it is exact at
$A_1, \dots A_{n-1}$.

An exact sequence of the form
$$
0 \xrightarrow
A \xrightarrow{f}
B \xrightarrow{g}
C \xrightarrow
0
$$
is called a \emph{short exact sequence}.
\end{defn}

Note that for the short exact sequence above,
$g$ is onto since $\mathrm{im}(g) = \ker(x \mapsto 0) = C$, and
$f$ is injective since $\ker f = \mathrm{im}(\{ 0 \}) = \{ 0 \}$.
Furthermore the first isomorphism theorem says that
$\ker(g) = \mathrm{im}(f) \simeq A$ so
$C \simeq B / \mathrm{im}(f)  \simeq B / A$.

\begin{defn}
A \emph{cochain complex} is a sequence of vector spaces
$\mathscr{C} = \{C_k\}_{k \in \mathbb{Z}}$ with the
linear maps $\dif = \dif_k : C_k \to C_{k+1}$ with
$\dif_{k+1} \circ \dif_k = 0$ for all $k$.
\end{defn}

For example, the de Rham complex is a cochain complex
with $\dif$ the exterior derivative.

In a cochain complex
$\mathrm{im}(\dif_{k+1}) \subset \ker(\dif_k)$ so
$$
  H^k(\mathscr{C})
= \frac{\ker(C^k \xrightarrow{\dif} C^{k+1})}
       {\mathrm{im}(C^{k-1} \xrightarrow{\dif} C^{k})}.
$$

\begin{defn}
Given cochain complexes
$(\mathscr{A}, \dif)$ and
$(\mathscr{B}, \dif^\prime)$, a \emph{chain map}
$\varphi: (\mathscr{A}, \dif) \to (\mathscr{B}, \dif^\prime)$
is a sequence of maps $\varphi_k : A_k \to B_k$ such that the cochain
diagrams commute, i.e. $\dif^\prime \varphi = \varphi \dif$.
\end{defn}

A cochain map induces maps
$\varphi^\ast : H^k(\mathscr{A}) \to H^k(\mathscr{B})$ in
cohomology given by $\varphi^\ast([a]) = [\varphi(a)]$. This is
well-defined because if $a \in \ker(\dif)$, then
$\varphi(a) \in \ker \dif^\prime$ since
$\dif^\prime \varphi a = \varphi \dif a = \varphi 0 = 0$. Furthermore
$\varphi a = \dif^\prime(\beta)$ means
$$
  \varphi a
= \varphi \dif \beta
= \dif^\prime \varphi b
= \dif^\prime \beta.
$$
In de Rham cohomology, given $F: N \to M$, the pullback
$F^\ast : (\Omega^\ast(M), \dif) \to (\Omega^\ast N, \dif)$ is a
cochain map because the pullback commutes with the exterior
derivative, and so $F^\ast$ induces a map in cohomology
$H_{dR}^k(M) \to H_{dR}^k(N)$.

\begin{defn}
A sequence of cochain maps
$$
0 \xrightarrow{}
A \xrightarrow{i}
B \xrightarrow{j}
C \xrightarrow{}
0
$$
is called short exact if $\forall k$,
$$
0   \xrightarrow{}
A_k \xrightarrow{i}
B_k \xrightarrow{j}
C_k \xrightarrow{}
0
$$
is a short exact sequence of vector spaces.
\end{defn}

For a short exact sequence of chain maps,
we define the \emph{connecting homomorphism}
$\dif^\ast : H^k(\mathscr{C}) \to H^{k+1}(\mathscr{A})$.

Let $[c] \in H^k(\mathscr{C})$. Choose
$c \in [c]$, i.e. $c \in C^k$ with $\dif c = 0$. Since
$j$ is surjective, choose $b \in B^k$ such that
$j(b) = c$. Then
$j(\dif b) = \dif(j b) = \dif c = 0$, so
$\dif b \in \ker(j) = \mathrm{im}(i)$. Therefore
$\dif b = i(a)$, and indeed since $i$ is injective
this $a \in A^{k+1}$ is unique.

We also have that $\dif a = 0$.
$i(\dif a) = \dif(i a) = \dif(\dif b) = 0$, so
$\dif a \in \ker(i)$. But $i$ is injective, so
$\dif a = 0$. Therefore $a$ defines a class
$[a] \in H^{k+1}(\mathscr{A})$.

Therefore we set
$\dif^\ast([c]) = [a] = [i^{-1} \dif j^{-1} c]$,
and the above argument shows that this choice is well-defined.
Choosing a different $c \in [c]$ and a different $b \in B^k$ would
yield the same result.

All this means that
\begin{lemma}[Zig-zag lemma]
A short exact sequence
$
0           \xrightarrow{}
\mathscr{A} \xrightarrow{i}
\mathscr{B} \xrightarrow{j}
\mathscr{C} \xrightarrow{}
0
$
induces a long sequence in cohomology
$$
\cdots              \xrightarrow{j^\ast}
H^{k-1}(\mathscr{C}) \xrightarrow{\dif^\ast}
H^{k}(\mathscr{A})   \xrightarrow{i^\ast}
H^{k}(\mathscr{B})   \xrightarrow{j^\ast}
H^{k}(\mathscr{C})   \xrightarrow{\dif^\ast}
\cdots
$$
This sequence is in fact exact.
\end{lemma}

\subsection{Mayer-Vietoris}
Let $M$ be a smooth manifold and $\{U, V\}$ be an open cover.
Then we have maps
$$
U \cap V \xrightarrow{j_U}
U        \xrightarrow{i_U}
M
$$
and
$$
U \cap V \xrightarrow{j_V}
V        \xrightarrow{i_V}
M
$$
which induce restriction maps
\begin{align*}
  i_U^\ast &: \Omega^k(M) \to \Omega^k(U) \\
  i_V^\ast &: \Omega^k(M) \to \Omega^k(V) \\
  j_U^\ast &: \Omega^k(U) \to \Omega^k(U \cap V) \\
  j_V^\ast &: \Omega^k(V) \to \Omega^k(U \cap V).
\end{align*}
These in turn induce maps
$$
  i
:   \Omega^k(M)
\to \Omega^k(U) \oplus \Omega^k(V)
=   \Omega^k(U \coprod V)
$$
given by
$$
        \sigma
\mapsto (i_U^\ast \sigma, i_V^\ast \sigma)
=       (\sigma\restrict_U, \sigma\restrict_V)
$$
and
$$
  j
:   \Omega^k(U \coprod V)
=   \Omega^k(U) \oplus \Omega^k(V)
\to \Omega^k(U \cap V)
$$
given by
$$
        (\alpha, \beta)
\mapsto j_V^\ast \beta - j_U^\ast \alpha
=       \beta\restrict_{U \cap V} - \alpha\restrict_{U \cap V}.
$$

The maps $i$ and $j$ commute with $\dif$ because the pullback does, so
we have a sequence of chain maps
$$
  0                                  \xrightarrow{}
  \Omega^\ast(M)                      \xrightarrow{i}
  \Omega^\ast(U \coprod V)
= \Omega^\ast(U) \oplus \Omega^\ast(V) \xrightarrow{j}
  \Omega^\ast(U \cap V)               \xrightarrow{}
  0
$$

\begin{prop}
The sequence above is a short exact sequence.
\end{prop}

\begin{proof}
\begin{itemize}
  \item{
    The sequence is exact at $\Omega^k(M)$ since $i$ is injective.
    Suppose $i\omega = (\omega\restrict_U, \restrict_V) = (0,
    0)$. Then $\omega\restrict_U = 0 = \omega\restrict_V$. But
    $U$ and $V$ cover $M$, so this means $\omega = 0$.
  }
  \item{
    Let $\omega \in \Omega^k(M)$. Then
    $$
      j(i \omega)
    = j(\omega\restrict_U, \omega\restrict_V)
    = (\omega\restrict_V)\restrict_{U \cap V}
    - (\omega\restrict_U)\restrict_{U \cap V}
    = 0,
    $$
    so $\mathrm{im}(i) \subset \ker(j)$.

    Let $(\alpha, \beta) \in \ker(j)$, so that
    $$
      0
    = j(\alpha, \beta)
    = \beta\restrict_{U \cap V}
    - \alpha\restrict_{U \cap V}
    $$
    which shows that the form
    $$
      \omega_x
    = \left\{
        \begin{array}{c c}
          \alpha_x, & x \in U \\
          \beta_x,  & x \in V
        \end{array}
      \right.
     $$
     is well-defined, and so $i \omega = (\alpha, \beta)$.
  }
  \item{
    To show exactness at $\Omega^k(U \cap V)$, we need to show
    $j$ is onto. Let $\sigma \in \Omega^k(U \cap V)$, and let
    $\{ \rho_U, \rho_V \}$ be a partition of unity subordinate to the
    cover $\{ U, V \}$.

    We construct smooth forms
    $$
      \sigma_U
    = \left\{
        \begin{array}{c c}
          \rho_V \sigma, & \quad x \in U \cap V \\
          0              & \quad x \in U \setminus (U \cap V)
        \end{array}
      \right. \quad
      \sigma_V
    = \left\{
        \begin{array}{c c}
          \rho_U \sigma, & \quad x \in U \cap V \\
          0              & \quad x \in V \setminus (U \cap V)
        \end{array}
      \right.
    $$
    with $\sigma_U \in \Omega^k(U)$ and
    $\sigma_V \in \Omega^k(V)$. But then
    $$
      j(-\sigma_U, \sigma_V)
    = \sigma_V \restrict_{U \cap V}
    + \sigma_U \restrict_{U \cap V}
    = (\rho_U + \rho_V)\sigma
    = \sigma.
    $$
  }
\end{itemize}
\end{proof}

Therefore this sequence induces a long exact sequence in cohomology
$$
\cdots               \xrightarrow{}
H^{k-1}(U \cap V)     \xrightarrow{\dif^\ast}
H^k(M)               \xrightarrow{i^\ast}
H^k(U) \oplus H^k(V) \xrightarrow{j^\ast}
H^k(U \cap V)        \xrightarrow{}
\cdots
$$
called the Mayer-Vietoris sequence. Here
\begin{align*}
   i^\ast([\omega])
&= ([\omega\restrict_U], [\omega\restrict_V]), \\
   j^\ast([\alpha], [\beta])
&= [\beta\restrict_{U \cap V} - \alpha\restrict_{U \cap V}],
\end{align*}
and $\dif^\ast : H^k(U \cap V) \to H^{k+1}(M)$ is the connecting
homomorphism
\begin{align*}
   \dif^\ast([\sigma])_x
&= \left\{
     \begin{array}{c c}
       -\dif \sigma_U, & \quad x \in U \\
       \dif \sigma_V,  & \quad x \in V
     \end{array}
   \right.
\end{align*}

\begin{xmpl}[Cohomology of $S^n$]
We claim that
$$
  H^k_{dR}(S^n)
= \left\{
    \begin{array}{c c}
      \mathbb{R}, & \quad k = 0, k = n \\
      0,          & \quad \text{otherwise}
    \end{array}
  \right.
$$
for $n \geq 1$, while
$$
  H^0_{dR}(S^0)
= \mathbb{R}^2
$$
since $S^0 = \{ \pm 1 \}$ has 2 connected components.

\begin{lemma}
$\mathbb{R}^n \setminus \{ 0 \}$ is homotopy equivalent to $S^{n-1}$,
so $H_{dR}^k(\mathbb{R}^n \setminus \{ 0 \}) \simeq H^k_{dR}(S^{n-1})$.
\end{lemma}

\begin{proof}
We have a homotopy equivalence between the inclusion
$i : S^{n-1} \hookrightarrow \mathbb{R}^n \setminus \{ 0 \}$ and the
retraction map
$r : \mathbb{R}^n \setminus \{ 0 \} \to S^{n-1}$ given by
$x \mapsto \frac{x}{\| x \|}$. $r \circ i = \mathrm{id}_{S^{n-1}}$,
and
$i \circ r : \mathbb{R}^n \setminus \{ 0 \} \to \mathbb{R}^n \setminus
\{ 0 \}$
is given by $x \mapsto \frac{x}{\| x \|}$. Define a homotopy
$
  H
: (\mathbb{R} \setminus \{ 0 \}) \times I \to \mathbb{R}^n \setminus
\{ 0 \}
$
by
$$
H(x, t) = x + t(r(x) - x),
$$
and observe that $H(x, 0) = x$,
$H(x, 1) = r(x)$, proving the lemma.
\end{proof}

\begin{lemma}
Let $\{ U, V \}$ be an open cover of $M$, with
$U$, $V$, and $U \cap V$ connected and nonempty. Then
$M$ is connected and we may start the Mayer-Vietoris sequence at
$$
0 \to H^1(M) \to H^1(U) \oplus H^1(V)
$$
\end{lemma}

\begin{proof}
Observe that $X$ is connected if and only if all continuous maps
$f : X \to \{ 0, 1 \}$ are constant. Let $f : M \to \{ 0, 1 \}$ be
continuous. Then $f\restrict_U$ is constant because $U$ is connected,
$f\restrict_V$ is constant because $V$ is connected, and since
$U \cap V$ is nonempty,
$f\restrict_U = f\restrict_{U \cap V} = f\restrict_V$, so
$f$ is constant.

Next, we must check that the Mayer-Vietoris sequence is exact at
$H^1(M)$, i.e. that $i^\ast: H^1(M) \to H^1(U) \oplus H^1(V)$ is
injective. We have that $\ker(i^\ast) = \mathrm{im}(\dif^\ast)$, so we
wish to show that $\dif^\ast : H^0(U \cap V) \to H^1(M)$ is the zero
map, i.e. that
$\ker(\dif^\ast) = H^0(U \cap V) = \mathrm{im}(j^\ast)$. Therefore we
wish to show that $j^\ast$ is surjective. But this is true since
$U$, $V$, $U \cap V$ are connected, and so
$$
  H^0(U \cap V) \simeq \mathbb{R} \simeq H^0(U) \simeq H^0(V),
$$
and clearly the map $j^\ast : \mathbb{R}^2 \to \mathbb{R}$ given by
$(a, b) \mapsto b - a$ is surjective.
\end{proof}

We have seen previously that
$$
  H^k(S^1)
= \left\{
    \begin{array}{c c}
      \mathbb{R}, & \quad k = 0, 1 \\
      0,          & \quad \text{otherwise}
    \end{array}
  \right.
$$
so assume that the claim holds for $S^n$ as an inductive
hypothesis. Then we have diffeomorphisms given by the
stereographic projections
\begin{align*}
   \pi_N
&: S^n \setminus \{ N \} = U \to \mathbb{R}^n, \\
   \pi_S
&: S^n \setminus \{ S \} = V \to \mathbb{R}^n.
\end{align*}
$U$, $V$ forms an open cover of $S^n$, and so
$$
       H^k(U)
\simeq H^k(V)
\simeq H^k(\mathbb{R}^n)
\simeq H^k(\{ \ast \}).
$$
We see also that
$
  \pi_N \restrict_{U \cap V}
: U \cap V \to \mathbb{R}^n \setminus \{ 0 \}
$
as a diffeomorphism. But we have seen
$\mathbb{R}^n \setminus \{ 0 \}$ is homotopy equivalent to
$S^{n-1}$, so
$$
       H^k(U \cap V)
\simeq H^k(\mathbb{R}^n \setminus \{ 0 \})
\simeq H^k(S^{n - 1}).
$$

We also note that $U$, $V$, $U \cap V$ are each nonempty and
connected. By our previous lemma, we have an exact sequence
$$
0                    \xrightarrow{\alpha}
H^1_{dR}(S^n)         \xrightarrow{\beta}
H^1(U) \oplus H^1(V) \xrightarrow{}
$$
But $U$, $V$ are homotopy equivalent to a point, so
$H^1(U) \oplus H^1(V) = 0$. Furthermore
$\ker(\beta) = \mathrm{im}(\alpha) = 0$, so $\beta$ is
injective. Therefore
$$
       H^1(S^n)
\simeq \beta(H^1(S^n))
=      0.
$$

Now consider $k \geq 2$. We have
$$
0 = H^{k-1}(U) \oplus H^{k-1}(V) \xrightarrow{}
H^{k-1}(U \cap V)     \xrightarrow{}
H^k(S^n)             \xrightarrow{}
H^k(U) \oplus H^k(V) = 0,
$$
i.e.
$$
       H^k(S^n)
\simeq H^{k-1}(U \cap V)
\simeq H^{k-1}(\mathbb{R}^n \setminus \{0\})
\simeq H^{k-1}(S^{n-1})
$$
so we have the cohomology of $S^n$ by induction.
\end{xmpl}

\begin{xmpl}
Let $n \geq 2$, and let
$P = \{p_1, p_2, \dots, p_l \} \subset \mathbb{R}^n$. We claim that
$$
  H^k_{dR}(\mathbb{R}^n - P)
= \left\{
    \begin{array}{c c}
      \mathbb{R},                             & k = 0 \\
      \mathbb{R}^{|P|} \setminus \mathbb{R}^l, & k = n - 1 \\
      0,                                      & \text{otherwise}
    \end{array}
  \right.
$$
It is proven in algebraic topology that de Rham cohomology is
isomorphic to other cohomology theories, e.g. singular cohomology,
etc.

For $l = 0$ the claim holds vacuously, whereas for $l = 1$ we have
$H^0(\mathbb{R}^n \setminus \{ p \}) = \mathbb{R}$ since $\mathbb{R}$ is
connected
and $H^k(\mathbb{R}^n \setminus \{ p \}) \simeq H^k(S^{n-1})$.

Let $U = \mathbb{R}^n \setminus \{p_1, \dots, p_{l-1}\}$ and
$V = \mathbb{R}^n \setminus \{ p_l \}$. Then
$U \cap V = \mathbb{R}^n \setminus P$, and since $U \cap V$ is
connected we have
$$
H^0(U \cap V) = H^0(\mathbb{R}^n - P) = \mathbb{R}.
$$

For $k \geq 1$ we have
$$
0 = H^k(\mathbb{R}^n) \xrightarrow{}
H^k(U) \oplus H^k(V)  \xrightarrow{}
H^k(U \cap V)         \xrightarrow{}
H^{k+1}(\mathbb{R}^n) = 0,
$$
so
$$
       H^k(U) \oplus H^k(V)
\simeq H^k(U \cap V)
$$
or
$$
\simeq H^k(\mathbb{R}^n \setminus \{p_1, \dots, p_{l-1}\})
         \oplus
       H^k(\mathbb{R}^n \setminus \{p_l\})
\simeq H^k(\mathbb{R}^n \setminus \{ p_1, \dots, p_l \}).
$$
But
$$
       H^k(\mathbb{R}^n \setminus \{p_1, \dots, p_{l-1}\})
         \oplus
       H^k(\mathbb{R}^n \setminus \{p_l\})
\simeq \mathbb{R}^{l-1} \oplus \mathbb{R}^1
\simeq \mathbb{R}^l.
$$
This shows that cohomology \emph{counts holes in Euclidean space}.

\begin{corol}
If $\mathbb{R}^n$ and $\mathbb{R}^m$ are homeomorphic, then $n = m$.
\end{corol}
\begin{proof}
Let $f : \mathbb{R}^n \to \mathbb{R}^m$ be a homeomorphism. Then
$\mathbb{R}^n \setminus \{0\}$ is homeomorphic to $\mathbb{R}^m
\setminus \{f(0)\}$, so
$$
       \mathbb{R}
=      H^{n-1}(\mathbb{R}^n \setminus \{ 0 \})
\simeq H^{n-1}(\mathbb{R}^m \setminus \{ f(0) \})
$$
and therefore $m = n$.
\end{proof}

\end{xmpl}

In algebraic topology we will see that if we have several holes in a
Euclidean space, we can draw a sphere around each of them, and there exists a
retraction
$\mathbb{R}^3 \setminus \{p_1, \dots, p_n\} \to S^2 \vee S^2 \cdots
\vee S^2$.

\begin{theorem}
Every continuous map from the disk $\mathbb{D}^n$ to itself has a fixed
point.
\end{theorem}

\begin{proof}
This follows once we show that there does not exist a retraction
$r: \mathbb{D}^n \to \mathbb{S}^{n-1}$. Suppose that
$r : \mathbb{D}^{n} \to S^{n-1}$, i.e. $r(x) = x$ for every
$x \in \partial \mathbb{D}^{n} = S^{n-1}$. Another way of saying
this is that $r \circ i = \mathrm{id}_{S^{n-1}}$, where
$i: S^{n-1} \to \mathbb{D}^{n}$ is the inclusion map. We then have a
map in cohomology
$$
    \mathrm{id}_{S^{n-1}}^\ast
:   \mathbb{R} = H^{n-1}(S^{n-1})
\to H^{n-1}(S^{n-1}) = \mathbb{R}.
$$
But
$r^\ast : \mathbb{H}^{n-1}(S^{n-1}) \to H^{n-1}(\mathbb{D}^n) = 0$
must be the zero map, so
$\mathrm{id}_{S^{n-1}}^\ast = i^\ast \circ 0 = 0$, a contradiction.
\end{proof}

\begin{xmpl}
The cohomology of the 2-torus is
$$
  H_{dR}^k(T^2)
= \left\{
    \begin{array}{c c}
      \mathbb{R},   & k = 0 \\
      \mathbb{R}^2, & k = 1 \\
      \mathbb{R},   & k = 2
    \end{array}
  \right..
$$

Cut the 2-torus along two closed curves around the ``tube'', and take
open neighborhoods of these curves. Take $U$ and $V$ to be half-tori
such that their intersection consists of two connected components
$A$, $B$, each
homeomorphic to a cylinder. Indeed each of the maps
$A \hookrightarrow U$,
$B \hookrightarrow V$,
$A \hookrightarrow U$,
$B \hookrightarrow V$
are homotopy equivalences, and therefore induce isomorphisms in
cohomology
$H^\ast(U) \to H^\ast(A)$, etc. All of these sets are in turn homotopy
equivalent to the circle $S^1$.

Since $T^2$ is connected,
$H^0(T^2) = \mathbb{R}$, so write the Mayer-Vietoris sequence
\begin{align*}
\mathbb{R} = H^0(T^2)                                           &\xrightarrow{}
H^0(U) \oplus H^0(V) = \mathbb{R} \oplus \mathbb{R}             &\xrightarrow{\alpha}
H^0(U \cap V) = H^0(A \coprod B) = \mathbb{R} \oplus \mathbb{R} &\xrightarrow{\dif_0^\ast} \\
H^1(T^2)                                                        &\xrightarrow{i^\ast}
H^1(U) \oplus H^1(V) = \mathbb{R} \oplus \mathbb{R}             &\xrightarrow{\beta}
H^1(U \cap V) = H^1(A \coprod B) = \mathbb{R} \oplus \mathbb{R} &\xrightarrow{\dif_1^\ast} \\
H^2(T^2)                                                        &\xrightarrow{}
H^2(U) \oplus H^2(V) = 0
\end{align*}
Working backwards,
$$
       H^2(T^2)
=      \frac{\mathbb{R} \oplus \mathbb{R}}
            {\ker(\dif_1^\ast}
\simeq \frac{\mathbb{R} \oplus \mathbb{R}}
            {\mathrm{im}(\beta)}.
$$
By the rank-nullity theorem and exactness,
\begin{align*}
   \dim(H^1(T^2))
&= \dim(\mathrm{im}(i^\ast))
 + \dim(\ker(i^\ast)) \\
&= \dim(\ker(\beta))
 + \dim(\ker(i^\ast)) \\
&= \dim(\ker(\beta))
 + \dim(\mathrm{im}(\dif_0^\ast)) \\
&= \dim(\ker(\beta))
 + (2 - \dim(\ker(\dif_0^\ast))) \\
&= \dim(\ker(\beta))
 + 2
 - \dim(\mathrm{im}(\alpha)).
\end{align*}
since $2 - \dim \ker(\dif_0^\ast) = \dim(\mathrm{im}(\dif_0^\ast))$.

$\alpha : H^0(U) \oplus H^0(V) \to H^0(U \cap V)$ is the difference
  map
$$
\alpha(a, b) = (b - a, b - a)
$$
on constant functions $a$, $b$. This map has
$\dim(\mathrm{im}(\alpha)) = \dim(\ker(\alpha)) = 1$

For $\beta$, take generators
$\omega_U \in H^1(U) \simeq \mathbb{R}$ and
$\omega_V \in H^1(V) \simeq \mathbb{R}$. Then
$ H^1(U) \oplus H^1(V)
= \{(r_1 \omega_U, r_2 \omega_V) | r_1, r_2 \in \mathbb{R}\}$.
The fact that $A \hookrightarrow U$ is a homotopy equivalence means
that
$\omega_U \restrict_U$ is a generator on $U$ and
$\omega_V \restrict_V$ is a generator on $V$, so
$\beta : \mathbb{R}^2 \to \mathbb{R}^2$ is the map
$(r_1, r_2) \mapsto (r_2 - r_1, r_2 - r_1)$, so
$\dim(\ker(\beta)) = 1$ as well.
\end{xmpl}
