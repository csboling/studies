\section{Quotient spaces}

\begin{defn}[Open relation]
An equivalence relation $\sim$ is said to be \emph{open} when the
canonical surjection $\pi: X \to X / \sim$ is an open map, i.e.
$\pi(U)$ is open for every open set $U \subset X$, or equivalently
$\pi^{-1}(\pi(U)) = \cup_{u \in U} [u]$ is open.
\end{defn}

\begin{prop}
  Quotients by open relations preserve second countability.
\end{prop}

\begin{proof}
Let $[p] \in V$. Then $p \in \pi^{-1}(V)$, and there exists some
$U_i$ in the countable base of
$X$ such that $p \in U_i \subset \pi^{-1}(V)$, so $[p] \in \pi(U_i)$.
\end{proof}

\begin{defn}
The \emph{graph} of a relation $\sim$ is
$$
  \Gamma_\sim
= \{ (s_1, s_2) \vert s_1 \sim s_2 \}
\subset S \times S.
$$
\end{defn}

\begin{prop}
$S / \sim$ is Hausdorff if and only if $\Gamma_\sim$ is closed in
$S \times S$.
\end{prop}
\begin{corol}
$S$ is Hausdorff if and only if
$\Delta = \Set{(s, s) | s \in S}$ is closed in $S \times S$.
\end{corol}

\begin{proof}
$\Gamma_\sim$ is closed if and only if
$S \times S \setminus \Gamma_\sim$ is open, or equivalently
$\forall (s_1, s_2)$ such that $s_1 \nsim s_2$, $\exists U_i$ open in
$S$ with $s_i \in U_i$ such that
$U_1 \times U_2 \subset S \times S \setminus \Gamma_\sim$. This is
equivalent to the statement that $\forall s_1, s_2 \in S$ such that
$s_1 \nsim s_2$, $\exists U_i$ with $s_i \in U_i$ and
$x \nsim y$ $\forall x \in U_1$, $\forall y \in U_2$. Equivalently,
$[s_1]$ and $[s_2]$ belong to disjoint open sets $\pi(U_1)$,
$\pi(U_2)$ since $\pi$ is open. The corollary follows since
$\Delta$ is $S$ under the quotient $x \sim y$ when $x = y$.
\end{proof}

\begin{xmpl}[Real projective space]
The real projective space $\mathbb{R}P^n$ is the space of lines in
$\mathbb{R}^{n+1}$ that pass through the origin, i.e.
$(\mathbb{R}^{n+1} \setminus \{ 0 \}) / \sim$ under the relation
$x \sim y \iff x = ty$ for some $t \in \mathbb{R} \setminus \{ 0 \}$.
There are so-called \emph{homogeneous coordinates} on $\mathbb{R}P^n$
given by $\pi(x^0, \dots, x^n) \mapsto [x^0 : \cdots : x^n]$,
well-defined only up to scalar multiplication.

\begin{prop}
$\mathbb{R} P^n$ is a topological manifold.
\end{prop}

\begin{proof}
  \begin{enumerate}
    \item{
      Let $U \subset \mathbb{R}^{n+1} \setminus \{ 0 \}$. Then
      $$
        \pi^{-1}(\pi(U))
      = \cup_{u \in U} [u]
      = \cup_{t \in \mathbb{R} \setminus \{ 0 \}}
          t U
      $$
      and for each $tU = \Set{ t u | u \in U }$, there is a
      homeomorphism
      $tU \to U$ given by $x \mapsto \frac{x}{t}$, so
      $\pi^{-1}(\pi(U))$ is open in
      $\mathbb{R}^{n+1} \setminus \{ 0 \}$. Therefore
      $\mathbb{R}P^n$ is second countable.
    }
    \item{
      Take
      $$
        X
      = \Set{(x, y) | x = ty}
      \subset \mathbb{R}^{n+1} \times \mathbb{R}^{n+1}.
      $$
      Define
      $   F
      :   \mathbb{R}^{n+1} \times \mathbb{R}^{n+1}
      \to \mathbb{R}^{{n+1 \choose 2}}$
      by
      $$
        (x, y)
      \mapsto
        (\dots, x^i y^j - x^j y^i, \dots).
      $$
      Then $X = F^{-1}(\{0 \})$ is closed, so
      $X \cap (\mathbb{R}^{n+1} \setminus \{0\})^2 = \Gamma_\sim$ is
      as well. Therefore $\mathbb{R} P^n$ is Hausdorff.
    }
    \item{
      Let $[x] = [x^0 : \cdots : x^n] \in \mathbb{R} P^n$. Either all
      elements in $\pi^{-1}([x])$ have $x^0 = 0$ or all have
      $x^0 \neq 0$. Let $U_i = \{ x^i \neq 0 \} \subset
      \mathbb{R}P^n$. Then $\mathbb{R}P^n = \cup_i U_i$ and
      $\pi^{-1}(U_i) = \Set{ (x^0, \dots, x^n) | x^i \neq 0 }$. Define
      $\varphi_i : U_i \to \mathbb{R}^n$ by
      $$
        [x^0 : \cdots : x^n]
      \mapsto
        \left(
          \frac{x^0}{x^i},
          \dots,
          \frac{x^n}{x^i}
        \right).
      $$
      Then $\varphi_i$ is continuous and has continuous inverse
      $(x^1, \dots, x^n) \mapsto [x^1: \cdots : 1 : \cdots : x^n]$, so
      this provides an $n$-dimensional atlas on $\mathbb{R}P^n$.

      Finally we confirm that these charts are compatible. We have
      $$
        (\varphi_1 \circ \varphi_0^{-1})(x^1, \dots, x^n)
      = \left(\frac{1}{x^1}, \frac{x^2}{x^1}, \dots, \frac{x^n}{x^1}\right).
      $$
    }
  \end{enumerate}
\end{proof}

We will see in homework that $\mathbb{R}P^n \simeq S^n / (x \sim -x)$,
while $\mathbb{R} P^1 \simeq S^1$. In the second semester course, we
will see that the fundamental group of $\mathbb{R} P^n$ is
$\mathbb{Z} / 2\mathbb{Z}$.

\end{xmpl}
