\section{Differential Forms}

In coordinates $(U, \phi) = (U, x^1, \dots, x^n)$, a
1-form $\omega : U \to T^\ast U$ can be written as
$$
\omega = \sum_i a_i \dif x^i
$$
where $a_i : U \to \mathbb{R}$ are real functions and
$\{ \dif x^i_u \}$ is the dual basis of $T_u^\ast M$ to the basis
$\left\{ \frac{\partial}{\partial x^i} \right\}$ of $T_u M$.
Due to smoothness in coordinates, each $a_i$ is $C^\infty$.

The differential of $f \in C^\infty(M)$
is a one-form $\dif f: M \to T^\ast M$ with
$(\dif f)_p(X_p) = X_p(f)$. Denote by
$\Omega^0(M) = C^\infty(M)$ the space of smooth 0-forms, and check
in coordinates that $\dif : \Omega^0(M) \to \Omega^1(M)$ is given by
$\dif f = \sum_i a_i \dif x^i$. Pairing these with
$\frac{\partial}{\partial x^j} : U \to TU$ gives
\begin{align*}
   \frac{\partial f}{\partial x^j}
&= \dif f \left( \frac{\partial}{\partial x^j} \right) \\
&= \left(
     \sum_i a_i \dif x^i
   \right)
   \left(
     \frac{\partial}{\partial x^j}
   \right) \\
&= \sum_i a_i \dif x^i
     \left(
       \frac{\partial}{\partial x^j}
     \right) \\
&= \sum_i
     a_i \delta^i_j = a_j
\end{align*}
so $\dif f = \sum_i \frac{\partial f}{\partial x^i} \dif x^i$ is
$C^\infty$.

\subsection{1-forms and vector fields}
\begin{prop}
Given a 1-form $\omega: M \to T^\ast M$ and $X: M \to TM$ we have a
function
$\omega(X) : M \to \mathbb{R}$ given by
$\omega(X)(p) = \omega_p (X_p)$.
$\omega : M \to T^\ast M$ is $C^\infty$ if and only if
$\forall X \in \chi(M)$,
$\omega(X) \in C^\infty(M)$, where
$\chi(M)$ is the vector space/$C^\infty(M)$-module of smooth vector fields.
\end{prop}

\begin{proof}
  \begin{itemize}
    \item[($\implies$)]{
      Let $(U, x^i)$ be coordinates so that
      $\omega_i = \sum a_i \dif x^i$ and
      $X|_u = \sum c^i \frac{\partial}{\partial x^i}$
      and then
      \begin{align*}
         \omega(X)\restrict_U
      &= \sum_i a_i \dif x^i (X) \\
      &= \sum_i
           a_i \dif x^i
           \left(
             \sum_j
               c^j \frac{\partial}{\partial x^j}
           \right) \\
      &= \sum_{i,j}
           a_i c^j \dif x^i
           \left(
             \frac{\partial}{\partial x^j}
           \right) \\
      &= \sum_{i, j}
           a_i c^j \delta^i_j \\
      &= \sum_i
           a_i c^i
      \end{align*}
      is smooth.
    }
    \item[($\impliedby$)]{
      Next we check that $\omega$ is smooth in a chart
      $(U, x^i)$.
      Let $p \in U$. We wish to show that each
      $a_i$ is smooth at $p$.
      First, extend $\frac{\partial}{\partial x^i}$ to
      $X_i \in \chi(M)$ that agrees with $\frac{\partial}{\partial
        x^i}$ in a neighborhood of $p$.
      $\omega(X_i)$ is smooth by assumption, so for $p \in V \subset U$
      \begin{align*}
         \omega(X_i)\restrict_V
      &= \sum_j
           a_j \dif x^j \\
      &= \sum_j
           a_j \dif x^j(X_i) \\
      &= \sum_j
           a_j \partial^j_i
       = a_i
      \end{align*}
      since $X_i = \frac{\partial}{\partial x^i}$ on $V$.
    }
  \end{itemize}
\end{proof}

\subsection{Pullback of differential forms}
Let $F: N \to M$ be smooth. Then there is a function
$F^\ast : \Omega^0(M) \to \Omega^0(N)$ given by
$F^\ast g = g \circ F$ for any $g \in \Omega^0(M)$.
Note that this gives a contravariant functor on
the category of manifolds and smooth maps.

For a 1-form $\omega: M \to T^\ast M$, the pullback
$F^\ast \omega : N \to T^\ast M$ is defined by
$$
(F^\ast \omega)_n(X_n) = \omega_{F(n)}(F_{\ast, n} (X_n)).
$$

\subsubsection{Properties of the pullback}
\begin{enumerate}
  \item{
    Pullbacks commute with the differential, i.e.
    $\forall g \in C^\infty(M)$
    $$
    F^\ast(\dif g) = \dif(F^\ast g)
    $$
    as 1-forms on $N$.

    Let $n \in N$ and $X_n \in T_n N$. Then
    \begin{align*}
       (F^\ast(\dif g))_n(X_n)
    &= \dif g_{F(n)}(F_{\ast, n}(X_n)) \\
    &= F_{\ast, n}(X_n)(g) \\
    &= X_n(g \circ f) \\
    &= X_n(F^\ast g) \\
    &= (\dif (F^\ast g))(X_n)
    \end{align*}
    as desired.
  }
  \item{
    $F^\ast$ is linear, i.e.
    $$
      F^\ast(\omega_1 + \omega_2)
    = F^\ast(\omega_1) + F^\ast(\omega_2)
    $$
    and
    $$
    F^\ast(g \omega_1) = F^\ast g F^\ast \omega_1.
    $$
    Note that if $g$ is the constant function
    $g = a$ then $F^\ast g = a$.

    Let $n \in N$, $X_n \in T_n N$. Then
    \begin{align*}
       F^\ast(g \omega_1 + \omega_2)_n(X_n)
    &= (g \omega_1 + \omega_2)_{F(n)}(F_{\ast, n}X_n) \\
    &= F^\ast g(n)(\omega_1)_{F(n)}(F_{\ast,n}X_n)
     + (\omega_2)_{F(n)}(F_{\ast,n}X_n) \\
    &= (F^\ast g F^\ast \omega_1 + F^\ast \omega_2)_n(X_n)
    \end{align*}
  }
  \item{
    $F^\ast$ carries smooth 1-forms to smooth 1-forms.
    Let $\omega = \sum_i a_i \dif y^i$ be smooth. Then
    \begin{align*}
       F^\ast \omega
    &= F^\ast
       \left(
         \sum_i a_i \dif y^i
       \right) \\
    &= \sum_i
         F^\ast a_i
         F^\ast \dif y^i \\
    &= \sum_i
         F^\ast a_i
         \dif(F^\ast y^i) \\
    &= \sum_i
         F^\ast a_i
         \dif (y^i \circ F) \\
    &= \sum_i
         F^\ast a_i
         \dif(F^i) \\
    &= \sum_i
         F^\ast a_i
         \sum_j
           \frac{\partial F^i}{\partial x^j}
           \dif x^j \\
    &= \sum_{i,j}
         F^\ast a_i
         \frac{\partial F^i}{\partial x^j}
         \dif x^j.
    \end{align*}
    The pullback of a smooth map is smooth, as is the derivative, so
    this is smooth.
  }
  \item{
    To summarize, for $F: N \to M$,
    $F^\ast$ is a chain map from the deRham cochain for $M$ to the
    deRham cochain for $N$.
  }
\end{enumerate}

\subsection{Restricting 1-forms to submanifolds}
Let $i : S \hookrightarrow M$ be an immersion.
For any $s \in S$,
$i_{\ast, s} : T_s S \to T_{i(s)} M$ is injective by definition.
Identify
$T_s S \simeq i_{\ast, s}(T_s S) \subset T_{i(s)}(M)$. People usually
write $T_s S \subset T_s M$ for this. If we have a 1-form
$\omega : M \to T^\ast M$, then we can restrict
$\omega\restrict_S = i^\ast(\omega)$.

\begin{xmpl}
Let $S^1 \subset \mathbb{R}^2$. We will show that $S^1$ admits a
nowhere-vanishing 1-form called \emph{orientation}.

Let $\omega : \mathbb{R}^2 \to T^\ast \mathbb{R}^2$ be dual to the
rotational vector field, i.e.
$\omega = -y \dif x + x \dif y$ which is dual to
$X = -y \frac{\partial}{\partial x} + x \frac{\partial}{\partial y}$.
$X$ restricts to a vector field on the circle.

Then
\begin{align*}
   \omega(X)
&= (-y \dif x + x \dif y)
     \left(
     - y \frac{\partial}{\partial x}
     + x \frac{\partial}{\partial y}
     \right) \\
&= - y \dif x
   \left(
   - y \frac{\partial}{\partial x}
   + x \frac{\partial}{\partial y}
   \right)
   + x \dif y
   \left(
   - y \frac{\partial}{\partial x}
   + x \frac{\partial}{\partial y}
   \right) \\
&= (-y)(-y) + (x)(x) \\
&= x^2 + y^2 = 1.
\end{align*}
Pull this back under $F: \mathbb{R} \to S^1$ given by
$F(t) = (\cos t, \sin t)$. Then
\begin{align*}
   F^\ast \omega
&= F^\ast(-y \dif x + x \dif y) \\
&= -F^\ast y F^\ast \dif x
 + F^\ast x F^ast \dif y \\
&= -(y \circ F) \dif (F^\ast x)
 + (x \circ F) \dif(F^\ast y) \\
&= -\sin t \dif (\cos t)
 + \cos t \dif (\sin t) \\
&= -\sin t
    \frac{\partial \cos t}{\partial t}
    \dif t
 + \cos t
    \frac{\partial \sin t}{\partial t}
    \dif t \\
&= \dif t
\end{align*}
which is a nowhere vanishing 1-form (i.e. an orientation) on $\mathbb{R}$.
\end{xmpl}

\subsection{$k$-forms}
First, some notation for the \emph{exterior powers} of $T_p^\ast M$.
\begin{itemize}
  \item{
    $\Lambda^0(T_p^\ast M) = \mathbb{R}$.
  }
  \item{
    $\Lambda^1(T_p^\ast M) = A_1(T_p M) = T_p^\ast M$.
  }
  \item{
    $\Lambda^k(T_p^\ast M) = A_k(T_p M)$,
    the space of alternating multilinear maps
    $\prod_{i=1}^k T_pM \to \mathbb{R}$. This is called the
    $k$-th exterior bundle, and as a set is given by
    $$
      \Lambda^k(T^\ast M)
    = \coprod_{p \in M} \Lambda^k(T_p^\ast M)
    $$
    which is equipped with the projection
    $\pi: \Lambda^k(T^\ast M) \to M$ given by
    $(p, \omega_p) \mapsto p$.

    This bundle has as sections maps
    $\omega : M \to \Lambda^k(T^\ast M)$ given by
    $p \mapsto \omega_p \in \Lambda^k(T^\ast M)$, with
    $\pi \circ \omega = \mathrm{id}_\omega$.
  }
\end{itemize}
The \emph{coordinate $k$-forms} for a coordinate chart
$(U, x^i)$ in $M$ are the set
$\{ \dif x^i \}_{i=1}^n$, which forms a basis for
$T_p^\ast M = A_1(T_p M)$.

Let $I = (i_1, \dots, i_k)$ be an increasing multi-index. Then
there are ${n \choose k}$ $k$-covectors
$$
  \dif x_u^I
= \dif_u^{i_1} \wedge \cdots \dif x_u{i_k}
\in \Lambda^k(T_u^\ast M).
$$
These form a basis for $\Lambda^k(T_u^\ast M)$.

Given $\omega_u \in \Lambda^k(T_u^\ast M)$, there exist scalars
$a_I(\omega_u) \in \mathbb{R}$ such that
$\omega_u = \sum_{|I| = k} a_I(\omega_u) \dif x_u^I$.

Given a section $\omega: U \to \Lambda^k(T^\ast U)$
we can write $\omega = \sum a_I \dif x^I$ where
$a_I: U \to \mathbb{R}$ are functions.

\subsubsection{$\Lambda^k(T^\ast M)$ as a vector bundle}
Let $(U, \phi) = (U, x^i)$ be a chart. Then
$\tilde{\phi} : \Lambda^k(T^\ast U) \to \phi(U) \times
\mathbb{R}^{{n \choose k}}$ is a map given by
$$
  \tilde{\phi}(u, \omega_u)
= (\phi(u), (a_I(\omega_u))).
$$
This is a bijection with inverse
$$
  (\tilde{\phi})^{-1}(\phi(\omega), (a_I))
= (u, \sum a_I \dif x^I).
$$
Defining $\tilde{\phi}$ to be a homeomorphism chooses a topology for
$\Lambda^k(T^\ast U)$, and we can follow the construction for
endowing $TM$ with a $C^\infty$ manifold vector bundle structure.

\subsubsection{Smooth $k$-forms}
Let $\Omega^k(M)$ be the $\mathbb{R}$-vector space
and $C^\infty(M)$-module of all smooth sections of
$\Lambda^k(T^\ast M)$ over $M$.
We let $\Omega^0(M) = C^\infty(M)$, and note that
$\Omega^k(M) = \{ 0 \}$ for $k > \dim M$, since in this case there are
no basis elements.

A $k$-form $\omega$ in coordinates $\omega_u = \sum a_I \dif x^I$ is
smooth if and only if $a_I$ is smooth for all $I$.

\subsubsection{Wedge product}
Let $\alpha: M \to \Lambda^k(T^\ast M)$ be a $k$-form and
$\beta : M \to \Lambda^l(T^\ast M)$ be an $l$-form. Then we define
$$
  (\alpha \wedge \beta)_p
= \alpha_p \wedge \beta_p
\in \Lambda^k(T_p^\ast M)
$$
for all $p$. Concretely,
$$
  (a \wedge \beta)_p(x_1, \dots, x_{k+l})
= \frac{1}{k!l!}
  A(\alpha \otimes \beta)(x_1, \dots, x_{k+l})
= \frac{1}{k!l!}
  \sum_{\sigma \in S_{k+l}}
    \mathrm{sgn}(\sigma)
    \alpha(x_{\sigma(1)}, \dots, x_{\sigma(k)})
    \beta(x_{\sigma(k+1)}, \dots, x_{\sigma(k+l)})
$$
which reduces to the sum over all $(k,l)$-shuffles,
i.e. those permutations with
$$
\sigma(1) < \cdots < \sigma(k),
\sigma(k+1) < \cdots < \sigma(k + l).
$$

The wedge product has the following properties:
\begin{enumerate}
  \item{
    $\alpha \wedge \beta$ is $C^\infty(M)$-bilinear:
    $$
      (f\alpha_1 + \alpha_2) \wedge \beta
    = f \alpha_1 \wedge \beta + \alpha_2 \wedge \beta
    $$
    and similarly in the second coordinate.
  }
  \item{
    The wedge product is not commutative, but is graded commutative:
    $\alpha \wedge \beta = (-1)^{kl} \beta \wedge \alpha$.
  }
  \item{
    The wedge product is associative.
  }
  \item{
    If $\alpha, \beta$ are smooth, then so is
    $\alpha \wedge \beta$. The wedge product thus defines a bilinear
    map
    $\wedge: \Omega^k(M) \times \Omega^l(M) \to \Omega^{k+l}(M)$.

    Take $p \in M$ and $U$ a coordinate chart at $p$. Write
    $$
      \alpha
    = \sum a_I \dif x^I,
      \beta
    = \sum b_I \dif x^I
    $$
    so that
    \begin{align*}
       \alpha \wedge \beta
    &= \sum_{I, J}
         a_I b_J \dif x^I \wedge \dif x^J \\
    &= \sum_{|K| = k + l}
         \sum_{I \cup J = K}
           \pm a_I b_J \dif x^K
    \end{align*}
    so this has smooth coordinates.
  }
\end{enumerate}

\begin{xmpl}
Let $M$ be a smooth 3-manifold, and
$f_1, f_2 : M \to \mathbb{R}$ be $C^\infty$ functions. Then
$$
  \dif f^1
= \frac{\partial f^1}{\partial x^1} \dif x^1
+ \frac{\partial f^1}{\partial x^2} \dif x^2
+ \frac{\partial f^1}{\partial x^3} \dif x^3
$$
and
$$
  \dif f^2
= \frac{\partial f^2}{\partial x^1} \dif x^1
+ \frac{\partial f^2}{\partial x^2} \dif x^2
+ \frac{\partial f^2}{\partial x^3} \dif x^3
$$
so that
\begin{align*}
   \dif f^1 \wedge \dif f^2
&= \left(
     \frac{\partial f^1}{\partial x^1}
     \frac{\partial f^2}{\partial x^2}
   - \frac{\partial f^1}{\partial x^2}
     \frac{\partial f^2}{\partial x^1}
   \right)
   \dif x^1 \wedge \dif x^2
 +  \left(
     \frac{\partial f^1}{\partial x^1}
     \frac{\partial f^2}{\partial x^3}
   - \frac{\partial f^1}{\partial x^3}
     \frac{\partial f^2}{\partial x^1}
   \right)
   \dif x^1 \wedge \dif x^3
 +  \left(
     \frac{\partial f^1}{\partial x^2}
     \frac{\partial f^2}{\partial x^3}
   - \frac{\partial f^1}{\partial x^3}
     \frac{\partial f^2}{\partial x^2}
   \right)
   \dif x^2 \wedge \dif x^3.
\end{align*}

\end{xmpl}

\subsection{The $\mathbb{R}$-algebra of $C^\infty$ forms}
We have a real vector space
$$
  \Omega^\ast(M)
= \bigoplus_{k=0}^n
    \Omega^k(M)
$$
whose elements are finite sums $a_0 + \cdots + a_n$ with
$a_i \in \Omega^i(M)$. A vector is \emph{homogeneous}
of degree $k$ if it is of the form
$\alpha \in \Omega^k(M)$. This is an algebra with multiplication given
by $\wedge$. This is called a \emph{graded real algebra}, where $k$ is
the grading.

\begin{defn}
In general if $A = \bigoplus_{k=0}^\infty A_k$ is a graded
$\mathbb{R}$-algebra, then a linear map
$D: A \to A$ is an \emph{antiderivation of $A$} if whenever
$\alpha \in A_k$ and $\beta \in A_l$,
$$
  D(\alpha \cdot \beta)
= D(\alpha)\cdot\beta + (-1)^k \alpha \cdot D(\beta).
$$
An antiderivation is of degree $m$ if for all homogeneous
$\alpha$, the degree of $D(\alpha)$ is the degree of $\alpha$ plus
$m$.
\end{defn}

For example, the exterior derivative
$\dif : \Omega^\ast(U) \to \Omega^\ast(U)$ is an antiderivation.

\begin{defn}
An \emph{exterior derivative} on a smooth manifold $M$ is a linear map
$D: \Omega^\ast(M) \to \Omega^\ast(M)$ that
\begin{itemize}
  \item{is an antiderivation of degree zero,}
  \item{squares to 0,}
  \item{
    $D$ agrees with $\dif$ on $\Omega^0(M)$, i.e.
    $\forall f \in \Omega^0(M) = C^\infty(M)$,
    $X \in \chi(M)$,
    $D(f)(X) = X(f)$.
  }
\end{itemize}
\end{defn}

We wish to show that there exists an exterior derivative and that it
is unique.

\begin{lemma}
Let $f^1, \dots, f^k$ be $C^\infty$ functions on $M$, so that
$\dif f^i : M \to T^\ast M$ are $C^\infty$ differential 1-forms. If
$D$ is an exterior derivative, then
$D(\dif f^1 \wedge \cdots \wedge \dif f^k) = 0$.
\end{lemma}

\begin{proof}
\begin{itemize}
  \item{
    For $k = 1$, $D(\dif f) = D(D f) = 0$.
  }
  \item{
    Suppose the proposition holds for $k - 1$. Then
    \begin{align*}
       D(\dif f^1 \wedge \cdots \wedge \dif f^k)
    &= D(\dif f^1) \wedge (\dif f^2 \wedge \cdots \wedge \dif f^k)
     + (-1) \dif f^1 \wedge D(\dif f^2 \wedge \cdots \wedge \dif ^k)
      \\
    &= 0
    \end{align*}
    using the case $k = 1$ for the first term and the inductive
    hypothesis for the second.
  }
\end{itemize}
\end{proof}

Let $(U, x^i)$ be local coordinates on a manifold $M$. We wish to show
that there exists an exterior derivative
$d_U : \Omega^\ast(U) \to \Omega^\ast(U)$ that is unique.

Suppose $D : \Omega^\ast(U) \to \Omega^\ast(U)$ is an exterior
derivative. For any $\omega \in \Omega^k(U)$,
$$
  \omega
= \sum_I a_I \dif x^I
$$
so that
\begin{align*}
   D(\omega)
&= D
   \left(
     \sum_I a_I \dif x^I
   \right) \\
&= \sum_I
     D(a_I) \wedge \dif x^I
 + (-1)^0 a_I \wedge D(\dif x^I) \\
&= \sum_I
     D(a_I) \wedge \dif x^I \\
&= \sum_I
     \dif(a_I) \wedge \dif x^I \\
&= \sum_I
     \sum_{j=1}^n
       \frac{\partial a_I}{\partial x^j}
       \dif x^j \wedge \dif x^I
\end{align*}
since $\dif x^I$ is a wedge product of exact 1-forms.
Therefore necessarily if $D$ exists, it has this formula in the
coordinate chart $U$. The fact that this formula satisfies the
properties for an exterior derivative is exactly the same proof as
when $U \subset \mathbb{R}^n$. We then change the notation so that
$\dif_U : \Omega^\ast(U) \to \Omega^\ast(U)$ is the exterior
derivative on $U$.

The same proof as in $U \subset \mathbb{R}^n$ works
to show that this is the only exterior derivative on $U$. This works
just the same because $U$ has global coordinates $(x^i)$. Any manifold
$U$ that has exactly one coordinate chart is diffeomorphic to
$\mathbb{R}^n$ since the coordinates are diffeomorphisms, so we need
to extend $D$ to an exterior derivative on all of $M$.

For $\omega \in \Omega^k(M)$, define $D \omega$ by
$$
(D\omega)_p \triangleq (d_U \omega)_p
$$
where $U$ is some open set containing $p$. This is well-defined
because if $p \in U \cap V$ then $\dif_{U \cap V}$ is unique and so
$$
  (\dif_{U \cap V} \omega)_p
= (\dif_U \omega)_p
= (\dif_V \omega)_p.
$$

We denote the exterior derivative on $M$ by $\dif$ from now on. We
need to show that there are no others on $M$.

\begin{lemma}
Suppose $D$ is an exterior derivative on $M$. and $U \subset M$ is
open. Let $\omega \in \Omega^k(M)$ such that
$\omega\restrict_U \equiv 0$. Then $D \omega\restrict_U \equiv 0$.
\end{lemma}
\begin{proof}
Let $p \in U$. We will show that $D\omega\restrict_p = 0$.

Let $f$ be a $C^\infty$ bump function on $M$ with support in $U$ and
with $f \equiv 1$ on some neighborhood $V$ of $p$ contained in $U$.
Let $\tilde{\omega} = f \cdot \omega \in \Omega^k(M)$. Note that
$\tilde{\omega} \equiv 0$ on $M$ since $\omega \equiv 0$ on $U$.
By linearity,
\begin{align*}
   0
&= D(0)
 = D(\tilde{\omega}) \\
&= D(f) \wedge \omega + (-1)^0 f \wedge D(\omega)
\end{align*}
so at $p$ we have
$$
0 = (-1)^0 f(p) (D\omega)_p = (D \omega)_p.
$$
\end{proof}

\begin{prop}[Uniquenss of $D$]
If $D: \Omega^\ast(M) \to \Omega^\ast(M)$ is an exterior derivative,
then $D = \dif$.
\end{prop}
\begin{proof}
Let $\omega \in \Omega^k(M)$. We need to show that $\forall p \in M$,
$(D\omega)_p = (\dif\omega)_p$. Let
$p \in (U, x^i)$ and write $\omega = \sum_I a_I \dif x^I$.
Extend $a_I : U \to \mathbb{R}$ and $x^i : U \to \mathbb{R}$ to smooth
functions $\bar{a}_I : M \to \mathbb{R}$ and $\bar{x}^i: M \to
\mathbb{R}$ by zero, with $a_I = \bar{a}_I$ and $x^i = \bar{x}^i$ on a
neighborhood $V$ of $p$. Let
$$
  \bar{\dif x}^I
= \dif \bar{x}^{i_1} \wedge \cdots \wedge \dif \bar{x}^{i_k}.
$$
Note that $\omega = \bar{\omega}$ on $V$ and so
$\omega - \bar{\omega} \equiv 0$ on $V$. Therefore $D(\omega -
\bar{\omega}) = 0$. But
\begin{align*}
   (D \omega)_p
&= D(\bar{\omega})_p
 = D\left(
     \sum_I
       \bar{a}_I \bar{\dif x}^I
   \right)_p \\
&= \sum_I D(\bar{a}_I)_p \wedge \bar{\dif x}^I_p
 + (-1)^0 \bar{a}_I(p) D(\dif \bar{x}^I)_p \\
&= \sum_I
     D(a_I)_p \wedge \dif x^I_p \\
&= \sum_I
     \dif(a_I)_p \wedge \dif x^I_p \\
&= (\dif_u \omega)_p
 = \dif \omega_p.
\end{align*}
\end{proof}

\subsection{Pullback of forms}
Let $F : N \to M$ and $\omega$ be a $k$-form on $M$,
i.e. $\omega: M \to \Lambda^k(T^\ast M)$.
The \emph{pullback} $F^\ast \omega$ to a $k$-form on
$N$ is defined by
\begin{align*}
   (F^\ast\omega)_p
     (X_1, \dots, X_k)
&= \omega_{F(p)}
     (F_{\ast, p} X_1, \dots, F_{\ast, p} X_k)
\end{align*}
for all $X_1, \dots, X_k \in T_p N$.

This operation has the following properties:
\begin{enumerate}
  \item{
    $F^\ast$ is $C^\infty(M)$-linear, i.e.
    $$
      F^\ast(g \omega_1 + \omega_2)
    = F^\ast g F^\ast \omega_1 + F^\ast \omega_2.
    $$
  }
  \item{
    $F^\ast$ is an algebra map on the exterior algebra, i.e.
    $$
      F^\ast(\omega \wedge \lambda)
    = F^\ast \omega \wedge F^\ast \lambda
    $$
    as $k + l$ forms on $N$.
  }
  \item{
    $F^\ast$ is a map of cochain complexes, which means it commutes
    with $\dif$, i.e.
    $$
      F^\ast(\dif \omega)
    = \dif (F^\ast \omega)
    $$
    as $k + 1$ forms.
  }
  \item{
    $F^\ast$ carries smooth forms to smooth forms.
  }
\end{enumerate}
