\section{Differential Forms}

In coordinates $(U, \phi) = (U, x^1, \dots, x^n)$, a
1-form $\omega : U \to T^\ast U$ can be written as
$$
\omega = \sum_i a_i \dif x^i
$$
where $a_i : U \to \mathbb{R}$ are real functions and
$\{ \dif x^i_u \}$ is the dual basis of $T_u^\ast M$ to the basis
$\left\{ \frac{\partial}{\partial x^i} \right\}$ of $T_u M$.
Due to smoothness in coordinates, each $a_i$ is $C^\infty$.

The differential of $f \in C^\infty(M)$
is a one-form $\dif f: M \to T^\ast M$ with
$(\dif f)_p(X_p) = X_p(f)$. Denote by
$\Omega^0(M) = C^\infty(M)$ the space of smooth 0-forms, and check
in coordinates that $\dif : \Omega^0(M) \to \Omega^1(M)$ is given by
$\dif f = \sum_i a_i \dif x^i$. Pairing these with
$\frac{\partial}{\partial x^j} : U \to TU$ gives
\begin{align*}
   \frac{\partial f}{\partial x^j}
&= \dif f \left( \frac{\partial}{\partial x^j} \right) \\
&= \left(
     \sum_i a_i \dif x^i
   \right)
   \left(
     \frac{\partial}{\partial x^j}
   \right) \\
&= \sum_i a_i \dif x^i
     \left(
       \frac{\partial}{\partial x^j}
     \right) \\
&= \sum_i
     a_i \delta^i_j = a_j
\end{align*}
so $\dif f = \sum_i \frac{\partial f}{\partial x^i} \dif x^i$ is
$C^\infty$.

\subsection{1-forms and vector fields}
\begin{prop}
Given a 1-form $\omega: M \to T^\ast M$ and $X: M \to TM$ we have a
function
$\omega(X) : M \to \mathbb{R}$ given by
$\omega(X)(p) = \omega_p (X_p)$.
$\omega : M \to T^\ast M$ is $C^\infty$ if and only if
$\forall X \in \chi(M)$,
$\omega(X) \in C^\infty(M)$, where
$\chi(M)$ is the vector space/$C^\infty(M)$-module of smooth vector fields.
\end{prop}

\begin{proof}
  \begin{itemize}
    \item[($\implies$)]{
      Let $(U, x^i)$ be coordinates so that
      $\omega_i = \sum a_i \dif x^i$ and
      $X|_u = \sum c^i \frac{\partial}{\partial x^i}$
      and then
      \begin{align*}
         \omega(X)\restrict_U
      &= \sum_i a_i \dif x^i (X) \\
      &= \sum_i
           a_i \dif x^i
           \left(
             \sum_j
               c^j \frac{\partial}{\partial x^j}
           \right) \\
      &= \sum_{i,j}
           a_i c^j \dif x^i
           \left(
             \frac{\partial}{\partial x^j}
           \right) \\
      &= \sum_{i, j}
           a_i c^j \delta^i_j \\
      &= \sum_i
           a_i c^i
      \end{align*}
      is smooth.
    }
    \item[($\impliedby$)]{
      Next we check that $\omega$ is smooth in a chart
      $(U, x^i)$.
      Let $p \in U$. We wish to show that each
      $a_i$ is smooth at $p$.
      First, extend $\frac{\partial}{\partial x^i}$ to
      $X_i \in \chi(M)$ that agrees with $\frac{\partial}{\partial
        x^i}$ in a neighborhood of $p$.
      $\omega(X_i)$ is smooth by assumption, so for $p \in V \subset U$
      \begin{align*}
         \omega(X_i)\restrict_V
      &= \sum_j
           a_j \dif x^j \\
      &= \sum_j
           a_j \dif x^j(X_i) \\
      &= \sum_j
           a_j \partial^j_i
       = a_i
      \end{align*}
      since $X_i = \frac{\partial}{\partial x^i}$ on $V$.
    }
  \end{itemize}
\end{proof}

\subsection{Pullback of differential forms}
Let $F: N \to M$ be smooth. Then there is a function
$F^\ast : \Omega^0(M) \to \Omega^0(N)$ given by
$F^\ast g = g \circ F$ for any $g \in \Omega^0(M)$.
Note that this gives a contravariant functor on
the category of manifolds and smooth maps.

For a 1-form $\omega: M \to T^\ast M$, the pullback
$F^\ast \omega : N \to T^\ast M$ is defined by
$$
(F^\ast \omega)_n(X_n) = \omega_{F(n)}(F_{\ast, n} (X_n)).
$$

\subsubsection{Properties of the pullback}
\begin{enumerate}
  \item{
    Pullbacks commute with the differential, i.e.
    $\forall g \in C^\infty(M)$
    $$
    F^\ast(\dif g) = \dif(F^\ast g)
    $$
    as 1-forms on $N$.

    Let $n \in N$ and $X_n \in T_n N$. Then
    \begin{align*}
       (F^\ast(\dif g))_n(X_n)
    &= \dif g_{F(n)}(F_{\ast, n}(X_n)) \\
    &= F_{\ast, n}(X_n)(g) \\
    &= X_n(g \circ f) \\
    &= X_n(F^\ast g) \\
    &= (\dif (F^\ast g))(X_n)
    \end{align*}
    as desired.
  }
  \item{
    $F^\ast$ is linear, i.e.
    $$
      F^\ast(\omega_1 + \omega_2)
    = F^\ast(\omega_1) + F^\ast(\omega_2)
    $$
    and
    $$
    F^\ast(g \omega_1) = F^\ast g F^\ast \omega_1.
    $$
    Note that if $g$ is the constant function
    $g = a$ then $F^\ast g = a$.

    Let $n \in N$, $X_n \in T_n N$. Then
    \begin{align*}
       F^\ast(g \omega_1 + \omega_2)_n(X_n)
    &= (g \omega_1 + \omega_2)_{F(n)}(F_{\ast, n}X_n) \\
    &= F^\ast g(n)(\omega_1)_{F(n)}(F_{\ast,n}X_n)
     + (\omega_2)_{F(n)}(F_{\ast,n}X_n) \\
    &= (F^\ast g F^\ast \omega_1 + F^\ast \omega_2)_n(X_n)
    \end{align*}
  }
  \item{
    $F^\ast$ carries smooth 1-forms to smooth 1-forms.
    Let $\omega = \sum_i a_i \dif y^i$ be smooth. Then
    \begin{align*}
       F^\ast \omega
    &= F^\ast
       \left(
         \sum_i a_i \dif y^i
       \right) \\
    &= \sum_i
         F^\ast a_i
         F^\ast \dif y^i \\
    &= \sum_i
         F^\ast a_i
         \dif(F^\ast y^i) \\
    &= \sum_i
         F^\ast a_i
         \dif (y^i \circ F) \\
    &= \sum_i
         F^\ast a_i
         \dif(F^i) \\
    &= \sum_i
         F^\ast a_i
         \sum_j
           \frac{\partial F^i}{\partial x^j}
           \dif x^j \\
    &= \sum_{i,j}
         F^\ast a_i
         \frac{\partial F^i}{\partial x^j}
         \dif x^j.
    \end{align*}
    The pullback of a smooth map is smooth, as is the derivative, so
    this is smooth.
  }
  \item{
    To summarize, for $F: N \to M$,
    $F^\ast$ is a chain map from the deRham cochain for $M$ to the
    deRham cochain for $N$.
  }
\end{enumerate}

\subsection{Restricting 1-forms to submanifolds}
Let $i : S \hookrightarrow M$ be an immersion.
For any $s \in S$,
$i_{\ast, s} : T_s S \to T_{i(s)} M$ is injective by definition.
Identify
$T_s S \simeq i_{\ast, s}(T_s S) \subset T_{i(s)}(M)$. People usually
write $T_s S \subset T_s M$ for this. If we have a 1-form
$\omega : M \to T^\ast M$, then we can restrict
$\omega\restrict_S = i^\ast(\omega)$.

\begin{xmpl}
Let $S^1 \subset \mathbb{R}^2$. We will show that $S^1$ admits a
nowhere-vanishing 1-form called \emph{orientation}.

Let $\omega : \mathbb{R}^2 \to T^\ast \mathbb{R}^2$ be dual to the
rotational vector field, i.e.
$\omega = -y \dif x + x \dif y$ which is dual to
$X = -y \frac{\partial}{\partial x} + x \frac{\partial}{\partial y}$.
$X$ restricts to a vector field on the circle.

Then
\begin{align*}
   \omega(X)
&= (-y \dif x + x \dif y)
     \left(
     - y \frac{\partial}{\partial x}
     + x \frac{\partial}{\partial y}
     \right) \\
&= - y \dif x
   \left(
   - y \frac{\partial}{\partial x}
   + x \frac{\partial}{\partial y}
   \right)
   + x \dif y
   \left(
   - y \frac{\partial}{\partial x}
   + x \frac{\partial}{\partial y}
   \right) \\
&= (-y)(-y) + (x)(x) \\
&= x^2 + y^2 = 1.
\end{align*}
Pull this back under $F: \mathbb{R} \to S^1$ given by
$F(t) = (\cos t, \sin t)$. Then
\begin{align*}
   F^\ast \omega
&= F^\ast(-y \dif x + x \dif y) \\
&= -F^\ast y F^\ast \dif x
 + F^\ast x F^ast \dif y \\
&= -(y \circ F) \dif (F^\ast x)
 + (x \circ F) \dif(F^\ast y) \\
&= -\sin t \dif (\cos t)
 + \cos t \dif (\sin t) \\
&= -\sin t
    \frac{\partial \cos t}{\partial t}
    \dif t
 + \cos t
    \frac{\partial \sin t}{\partial t}
    \dif t \\
&= \dif t
\end{align*}
which is a nowhere vanishing 1-form (i.e. an orientation) on $\mathbb{R}$.
\end{xmpl}

\subsection{$k$-forms}
First, some notation for the \emph{exterior powers} of $T_p^\ast M$.
\begin{itemize}
  \item{
    $\bigwedge^0(T_p^\ast M) = \mathbb{R}$.
  }
  \item{
    $\bigwedge^1(T_p^\ast M) = A_1(T_p M) = T_p^\ast M$.
  }
  \item{
    $\bigwedge^k(T_p^\ast M) = A_k(T_p M)$.
  }
\end{itemize}
Elements of the $k$th exterior power of $T_p^\ast M$ are called
\emph{$k$-covectors}, and are alternating multilinear maps
$(T_pM)^k \to \mathbb{R}$.
