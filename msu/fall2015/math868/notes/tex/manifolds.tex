\section{Manfiolds}

Let $M$ be a topological space.
\begin{defn}
$M$ is a \emph{Hausdorff space} if $\forall p \neq q \in M$,
there exist open sets $U, V$ containing $p$ and $q$ respectively
with $U \cap V = \varnothing$. We say that the topology
\emph{separates points}.
\end{defn}

\begin{defn}
$M$ is \emph{second-countable} if there is a countable collection
$\{U_i\}_{i=1}^\infty$ of open sets in $M$ such that for all open sets
$U$ and for any $p \in U$, there exists an $i \in \mathbb{N}$ such
that
$p \in U_i \subset U$.
\end{defn}

These properties are hereditary, e.g. subspaces of Hausdorff spaces (under the subspace
topology) are also Hausdorff.

\begin{defn}
$M$ is \emph{locally homeomorphic to $\mathbb{R}^n$} if
$\forall p \in M$, there is an open set $U$ with $p \in U$ and a
continuous map $\varphi: U \to \mathbb{R}^n$ such that
$\phi: U \to \phi(U)$ is a homeomorphism. The pair $(U, \phi)$ is
called a \emph{coordinate chart} because it gives a set of $n$ real numbers
that uniquely describe each point $p$. We write $x^i = \pi^i \circ
\phi$ where $\pi^i$ is the $i$-th projection.
\end{defn}
This property is not hereditary.

\begin{defn}[Topological manifold]
$M$ is a topological $n$-manifold if it is Hausdorff, second countable
and locally homeomorphic to $\mathbb{R}^n$. We say that the collection
of charts $(U_\alpha, \phi_\alpha)$ covering $M$ is an
\emph{atlas}. An atlas is said to be \emph{smooth} if all the
transition functions
$$
  \phi_{\beta\alpha}
\triangleq
  \phi_\beta \circ \phi_\alpha^{-1}
:   \phi_\alpha(U_\alpha \cap  U_\beta)
\to \phi_\beta(U_\alpha \cap U_\beta)
$$
are $C^\infty$.
\end{defn}
