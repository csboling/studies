\section{Vector fields (\S 14)}

Let $M$ be a smooth manifold. A vector field is a section
$X: M \to TM$ associating to each $p$ on the manifold a vector
$X_p \in T_p M$. Denote the space of smooth vector fields on $M$
by $\chi(M) = \Gamma(M, TM)$, which is a real vector space as well as
a $C^\infty(M)$-module.

Smoothness is usually checked in coordinates. For a coordinate chart
$(U, \phi) = (U, x^1, \dots, x^m)$, we can look at the restriction
$X \restrict_U$ and write in coordinates
$X = \sum_i a^i \frac{\partial}{\partial x^i}$, and we call the basis
$\left\{\frac{\partial}{\partial x^i}: U \to TM\right\}$ a
$C^\infty$-framing. Then $X$ is smooth over $U$ if and only if
$a^i : U \to \mathbb{R}$ are $C^\infty$.

Recall that $T_p M$ consists of point derivations of $C_p^\infty(M)$.
\begin{lemma}
A vector field $X$ is smooth if and only if $\forall f \in
C^\infty(M)$, the function $X(f)$ given by $X(f)(p) = X_pf$ is smooth.
\end{lemma}
\begin{proof}
  Let $p \in M$, and choose coordinates
  $(U, \phi) = (U, x^1, \dots, x^m)$ about $p$.
  \begin{itemize}
    \item[($\implies$)]{
      Let $f \in C^\infty(M)$. Then we have
      $X = \sum_i a^i \frac{\partial}{\partial x^i}$ where $a^i$ are
      smooth. Then on $U$,
      $X(f) = \sum_i a^i \frac{\partial f}{\partial x^i}$ is smooth on
      $U$.
    }
    \item[($\impliedby$)]{
      Let $X$ be a smooth vector field. Consider the coordinate
      function $x^i : U \to \mathbb{R}$. In the previous class we
      showed that there exists a smooth extension
      $\tilde{x^i} : M \to \mathbb{R}$ that agrees with
      $x^i : U \to \mathbb{R}$ on some neighborhood $p \in V \subset
      U$. Then $X(\tilde{x^i})$ is smooth, and
      \begin{align*}
         X(\tilde{x^i})
      &= \sum_j a^j \frac{\partial \tilde{x^i}}{\partial x^j} \\
      &= \sum_j a^j \delta^i_j = a^i
      \end{align*}
      near $p$, so each $a^i$ is smooth.
    }
  \end{itemize}
\end{proof}

This shows that each $X \in \chi(M)$ defines a map
$X: C^\infty(M) \to C^\infty(M)$ given by $f \mapsto X(f)$.
Furthermore, $C^\infty(M)$ is an $\mathbb{R}$-algebra, and
$X \in \chi(M)$ acts linearly on $C^\infty(M)$ and satisfies the
Leibniz rule $X(fg) = X(f) g + f X(g)$, i.e.
$X \in \chi(M)$ is a derivation of an $\mathbb{R}$-algebra of
$C^\infty(M)$. Therefore if $X, Y \in \chi(M)$, then
$XY - YX \in \chi(M)$. On functions this acts as
$(X Y - Y X)(f) = X(Y(f)) - Y(X(f))$. We write
$$
[ \cdot, \cdot ] : \chi(M) \times \chi(M) \to \chi(M)
$$
for $[X, Y] = XY - YX$ and call this the \emph{Lie bracket}.
The Lie bracket is
\begin{enumerate}
  \item{
    $\mathbb{R}$-bilinear,
  }
  \item{
    anticommutative, i.e. $[X, Y] = -[Y, X]$ and
  }
  \item{
    $[X, [Y, Z]] + [Y, [Z, X]] + [Z, [X, Y]] = 0$
  }
\end{enumerate}

\begin{defn}[Lie algebra]
A vector space $V$ equipped with a binary operation satisfying
properties 1 - 3 above is called a \emph{Lie algebra}.
\end{defn}

Typically Lie algebras are studied in the context of Lie groups as the
tangent space of a Lie group at the identity element. The Lie algebra
$(\chi(M), [\cdot, \cdot])$ is the Lie algebra to the ``Lie group'' of
all diffeomorphisms of $M$. This has the structure of an infinite
dimensional manifold.

\begin{remark}
  \begin{enumerate}
    \item{
      We will give a more geometric interpretation of the Lie bracket
      later, namely via the \emph{Lie derivative}
      $\mathcal{L}_X(Y) = [X, Y]$.
    }
    \item{
      The Lie bracket is \emph{not} $C^\infty(M)$-bilinear. Indeed for
      $f, g \in C^\infty(M)$,
      $$
      [fX, gY] = fg[X, Y] + f X(g) Y - g Y(f) X.
      $$
    }
  \end{enumerate}
\end{remark}

\subsection{$F$-related vector fields}
\begin{defn}
Let $F: N \to M$ be a smooth map. Two vector fields
$X \in \chi(N)$ and $\tilde{X} \in \chi(M)$ are
\emph{$F$-related} if $\forall n \in N$,
$$
F_{\ast, n}(X_n) = \tilde{X}_{F(n)} \in T_{F(n)} M.
$$
\end{defn}

Typically, given a smooth map $F: N \to M$ and a vector field
$X \in \chi(N)$, there will be no vector field $\tilde{X}$ on $M$ that
is $F$-related to $X$. Consider the map $F: \mathbb{R}^2 \to
\mathbb{R}$ given by projection $(x, y) \mapsto y$,
and the vector field $X(x, y) = x \frac{\partial}{\partial y}$. This
is a linear smooth map, and equals its own derivative, i.e.
$F_{\ast, (x, y)} X = x \frac{\dif}{\dif t}$. There is no well-defined
single choice of vector at $F(x, y)$ that is the pushforward of all
the vectors based at points in $F^{-1}(F(x, y))$.

\begin{prop}
If $F: N \to M$ is smooth and $X \in \chi(N)$, $\tilde{X} \in
\chi(M)$, then $X$ and $\tilde{X}$ are $F$-related if and only if
$\forall g \in C^\infty(M)$,
$X(g \circ F) = \tilde{X}(g) \circ F$ as smooth functions on $N$.
\end{prop}
\begin{proof}
Let $p \in N$. Then
$X(g \circ F)(p) = F_{\ast, p}(X_p)(g)$, and
$$
  (\tilde{X}(g) \circ F)(p)
= \tilde{X}(g)(F(p))
= \tilde{X}_{F(p)}(g).
$$
\end{proof}

\begin{prop}
Let $F: N \to M$ be $C^\infty$, $X, Y \in \chi(N)$,
and $\tilde{X}, \tilde{Y} \in \chi(M)$ such that
$X$ is $F$-related to $\tilde{X}$ and $Y$ is $F$-related to
$\tilde{Y}$.
Then $[X, Y]$ is $F$-reltaed to $[\tilde{X}, \tilde{Y}]$, i.e.
$$
F_\ast([X, Y]) = [F_\ast X, F_\ast Y] = [\tilde{X}, \tilde{Y}].
$$
\end{prop}
\begin{proof}
Let $g \in C^\infty(M)$. Then
\begin{align*}
   [X, Y](g \circ F)
&= X(Y(g \circ F)) - Y(X(g \circ F)) \\
&= X(\tilde{Y}(g) \circ F) - Y(\tilde{X}(g) \circ F) \\
&= \tilde{X}(\tilde{Y}(g)) \circ F
 - \tilde{Y}(\tilde{X}(g)) \circ F \\
&= [\tilde{X}, \tilde{Y}](g) \circ F.
\end{align*}
\end{proof}

\subsection{Integration of vector fields}
Given a smooth curve $c$ in a manifold, we can take its derivative
$c^\prime(t) \in T_{c(t)} M$. Integration is the opposite procedure of
constructing a curve given a vector field.

\begin{defn}
  \begin{enumerate}
    \item{
      Let $X \in \chi(M)$. An
      \emph{integral curve for $X$} is a smooth curve
      $c: (\alpha, \beta) \to M$ such that
      $c^\prime(t) = X_{c(t)} \in T_{c(t)} M$ for all
      $t \in (\alpha, \beta)$.
    }
    \item{
      An integral curve is \emph{maximal} if it
      cannot be extended to a larger domain, i.e. there does not exist an
      integral curve $\tilde{c}: (a, b) \to M$ with
      $(\alpha, \beta) \subsetneq (a, b)$ such that
      $c = \tilde{c}\restrict_{(\alpha, \beta)}$.
    }
    \item{
      An integral curve $c: (\alpha, \beta) \to M$ starts at
      $p \in M$ if $0 \in (\alpha, \beta)$ and $c(0) = p$.
      We write this as $c_t(p) = c(t)$ to indicate that the curve
      starts at the point $p$.
    }
  \end{enumerate}
\end{defn}

\begin{xmpl}
Take
$X = -y \frac{\partial}{\partial x} + x \frac{\partial}{\partial y}$.
We wish to find a curve $c(t)$ satisfying
$$
  c^\prime(t)
= \dot{x}(t) \frac{\partial}{\partial x}
+ \dot{y}(t) \frac{\partial}{\partial y}
$$
i.e.
$$
  X_{c(t)}
= -y(t) \frac{\partial}{\partial x}
+  x(t) \frac{\partial}{\partial y}
$$
so we wish to solve
$$
\dot{x} = -y \quad \dot{y} = x.
$$
This gives
$$
\ddot{x} + x = 0 \quad \ddot{y} + y = 0
$$
which has the general solution
$$
x(t) = \alpha \cos(t) + \beta \sin(t) \quad
y(t) = \alpha \sin(t) - \beta \cos(t).
$$
Choosing an initial condition $c(0) = (x_0, y_0)$ gives
$$
x(t) = x_0 \cos(t) - y_0 \sin(t) \quad
y(t) = x_0 \sin(t) + y_0 \cos(t)
$$
so that
$$
  c(t)
= \left[
    \begin{array}{c}
      x(t) \\
      y(t)
    \end{array}
  \right]
= \left[
    \begin{array}{r r}
      \cos(t) & -\sin(t) \\
      \sin(t) &  \cos(t)
    \end{array}
  \right]
  \left[
    \begin{array}{c}
      x_0 \\
      y_0
    \end{array}
  \right].
$$
\end{xmpl}
