\section{Integration}

\subsection{Pullback of forms}
Let $F : N \to M$ and $\omega$ be a $k$-form on $M$,
i.e. $\omega: M \to \Lambda^k(T^\ast M)$.
The \emph{pullback} $F^\ast \omega$ to a $k$-form on
$N$ is defined by
\begin{align*}
   (F^\ast\omega)_p
     (X_1, \dots, X_k)
&= \omega_{F(p)}
     (F_{\ast, p} X_1, \dots, F_{\ast, p} X_k)
\end{align*}
for all $X_1, \dots, X_k \in T_p N$.

This operation has the following properties:
\begin{enumerate}
  \item{
    $F^\ast$ is $C^\infty(M)$-linear, i.e.
    $$
      F^\ast(g \omega_1 + \omega_2)
    = F^\ast g F^\ast \omega_1 + F^\ast \omega_2.
    $$
  }
  \item{
    $F^\ast$ is an algebra map on the exterior algebra, i.e.
    $$
      F^\ast(\omega \wedge \lambda)
    = F^\ast \omega \wedge F^\ast \lambda
    $$
    as $k + l$ forms on $N$.

    Let $n \in N$, $x_1, \dots, x_{k+l} \in T_n N$.
    \begin{align*}
       F^\ast(\alpha \wedge \beta)_n(x_1, \dots, x_{k+l})
    &= (\alpha \wedge \beta)_{F(n)}
         (F_{\ast,n} x_1, \dots, F_{ast, n} x_{k+l}) \\
    &= \frac{1}{k! l!}
         A(\alpha \otimes \beta)_{F(n)}
           (F_{\ast,n} x_1, \dots, F_{ast, n} x_{k+l}) \\
    &= \frac{1}{k! l!}
         \sum_{\sigma \in S_{k+l}}
           \alpha(F_{\ast, n}(x_{\sigma(1)}),
                  \dots,
                  F_{\ast,n}(x_{\sigma(k)})
           \beta(F_{\ast, n}(x_{\sigma(k+1)}),
                 \dots,
                 F_{\ast, n}(x_{\sigma(k+l)}) \\
    &= \frac{1}{k! l!}
         \sum_{\sigma \in S_{k+l}}
           \mathrm{sgn}(\sigma)
             (F^\ast \alpha)_n
               (x_{\sigma(1)},
                \dots,
                x_{\sigma(k)}))
             (F^\ast \beta)_n
               (x_{\sigma(k+1)},
                \dots,
                x_{\sigma(k+l)}) \\
    &= (F^\ast \alpha \wedge F^\ast \beta)_n
         (x_1, \dots, x_{k+l}).
    \end{align*}
  }
  \item{
    $F^\ast$ is a map of cochain complexes, which means it commutes
    with $\dif$, i.e.
    $$
      F^\ast(\dif \omega)
    = \dif (F^\ast \omega)
    $$
    as $k + 1$ forms.

    Let $\omega \in \Omega^k(M)$ and $(V, y^i)$ be coordinates on $M$.
    On this coordinate chart,
    $$
      \omega
    = \sum_I
        a_I \dif y^I
    = \sum_I
        a_I \dif y^{i_1} \wedge \cdots \wedge \dif y^{i_k}
    $$
    so
    $$
      \dif \omega
    = \sum_I \dif(a_I) \wedge \dif y^I
    = \sum_I \dif(A_I) \wedge \dif y^{i_1} \wedge \cdots \wedge \dif y^{i_k}
    $$
    so
    \begin{align*}
       F^\ast(\dif \omega)
    &= F^\ast
         \left(
           \sum_I
             \dif(a_I) \wedge
             \dif y^{i_1} \wedge
             \cdots \wedge
             \dif y^{i_k}
         \right) \\
    &= \sum_I
         F^\ast(\dif(a_I)) \wedge
         F^\ast \dif y^{i_1} \wedge
         \cdots \wedge
         F^\ast \dif y^{i_k} \\
    &= \sum_I
         \dif(F^\ast a_I) \wedge
         \dif(F^\ast y^i) \wedge
         \cdots \wedge
         \dif(F^\ast y^{i_k})
    \end{align*}
    whereas
    \begin{align*}
       F^\ast \omega
    &= \sum_I
         F^\ast (a_I)
         F^\ast(\dif y^{i_1} \wedge \cdots \wedge \dif y^{i_k}) \\
    &= \sum_I
         F^\ast(a_I)
           \dif(F^\ast y^{i_1}) \wedge \cdots \wedge \dif(F^\ast y^{i_k})
    \end{align*}
    so that
    \begin{align*}
       \dif(F^\ast \omega)
    &= \sum_I
         \dif(F^\ast a_I) \wedge
           (\dif F^\ast y^{i_1} \wedge \cdots \wedge \dif F^\ast
      y^{i_k}) \\
    &= (-1)^0
       F^\ast(a_I) \wedge
       \dif
         (\dif F^\ast y^{i_1} \wedge \cdots \wedge \dif F^\ast
      y^{i_k}) \\
    &= F^\ast(\dif \omega)
    \end{align*}
    since the exterior derivative of a wedge product of exact 1-forms
    is 0.
  }
  \item{
    $F^\ast$ carries smooth forms to smooth forms.

    Let $(U, x^i)$ be coordinates on $N$ and
    $F^1, \dots, F^k \in C^\infty(N, \mathbb{R})$. Let
    $J = (j_1, \dots, j_k)$ be a multi-index. Then
    \begin{align*}
       \frac{\partial (F^1, \dots, F^k)}
            {\partial (x^{j_1}, \dots, x^{j_k})}
    &= \det
       \left[
         \begin{array}{c c c}
           \frac{\partial F^1}{\partial x^{j_1}}
         & \cdots
         & \frac{\partial F^1}{\partial x^{j_k}} \\
           \frac{\partial F^2}{\partial x^{j_1}}
         & \cdots
         & \frac{\partial F^2}{\partial x^{j_k}} \\
           \vdots & \vdots & \vdots \\
           \frac{\partial F^k}{\partial x^{j_1}}
         & \cdots
         & \frac{\partial F^k}{\partial x^{j_k}}
         \end{array}
       \right] \\
    &= \sum_{\sigma \in S_k}
         \prod_{i=1}^k
           \frac{\partial F^i}{\partial x^{j_{\sigma(i)}}}.
    \end{align*}

    \begin{lemma}
      \begin{align*}
         \dif F^1 \wedge \cdots \wedge \dif F^k
      &= \sum_I
           \frac{\partial (F^1, \dots, F^k)}
                {\partial (x^{i_1}, \dots, x^{i_k})}
           \dif x^I.
      \end{align*}
      As a corollary, $\dif F^1 \wedge \cdots \wedge \dif F^k$ is smooth.
    \end{lemma}
    \begin{proof}
      We know $\dif F^1 \wedge \cdots \wedge \dif F^k$
      can be written as $\sum_I a_I \dif x^I$ for some $a_I$. Then
      writing
      $
        \frac{\partial}{\partial x^J}
      = \left(
          \frac{\partial}{\partial x^{j_1}},
          \cdots,
          \frac{\partial}{\partial x^{j_k}}
        \right)
      $
      we have
      \begin{align*}
        \left(
          \sum_I
            a_I \dif x^I
        \right)
          \left(
            \frac{\partial}{\partial x^J}
          \right)
      &= \sum_I
           a_I \dif x^I
           \left(
             \frac{\partial}{\partial x^J}
           \right) \\
      &= \sum_I
           a_I \delta^I_J = a_J
      \end{align*}
      whereas
      \begin{align*}
        (\dif F^1 \wedge \cdots \wedge \dif F^k)
          \left(
            \frac{\partial}{\partial x^{j_1}},
            \dots,
            \frac{\partial}{\partial x^{j_k}}
          \right)
      &= \frac{1}{1! \cdots 1!}
           \sum_{\sigma \in S_k}
             \mathrm{sgn}(\sigma)
             \dif F^1
               \left(
                 \frac{\partial}{\partial x^{j_{\sigma(1)}}}
               \right)
             \wedge \cdots \wedge
             \dif F^k
               \left(
                 \frac{\partial}{\partial x^{j_{\sigma(k)}}}
               \right) \\
      &= \sum_{\sigma \in S_k}
           \mathrm{sgn}(\sigma)
           \frac{\partial F^1}{\partial x^{j_{\sigma(1)}}}
           \cdots
           \frac{\partial F^k}{\partial x^{j_{\sigma(k)}}} \\
      &= \frac{\partial (F^1, \cdots, F^k)}
              {\partial (x^{j_1}, \dots, x^{j_k})}.
      \end{align*}
    \end{proof}

    Now on $V$, $\omega = \sum_I a_I \dif y^I$, so
    \begin{align*}
       F^\ast \omega
    &= \sum_I
         F^\ast a_I
         (\dif y^{i_1} \wedge \cdots \wedge \dif y^{i_k}) \\
    &= \sum_I
         (a_I \circ F)
         F^\ast(\dif y^{i_1}) \wedge \cdots \wedge F^\ast(\dif
      y^{i_k}) \\
    &= \sum_I
         (a_I \circ F)
         \dif(F^\ast y^{i_1})
         \wedge \cdots \wedge
         \dif(F^\ast y^{i_k}) \\
    &= \sum_I
         (a_I \circ F)
         \dif(y^{i_1} \circ F)
         \wedge \cdots \wedge
         \dif(y^{i_1} \circ F)
    \end{align*}
    which is smooth by the lemma.
  }
\end{enumerate}

\section{Orientations}

\subsection{Via ordered bases}
Let $V$ be an $n$-dimensional $\mathbb{R}$-vector space. Let
$X$ be the collection of all ordered bases,
$$
  X
= \{ u = \{ u_1, \dots, u_n \} ~\vert~ \text{$u$ is an ordered basis} \}.
$$
Let $u, v \in X$, with $u = \{u_1, \dots, u_n\}$ and
$v = \{v_1, \dots, v_n\}$. Then there exist unique scalars $a^i_j$
such that
$$
u_j = \sum_{i=1}^n a^i_j v_i,
$$
and denote $A = [a_j^i]$. Then
$$
  \left[
    \begin{array}{c c c}
      u_1 & \cdots & u_n
    \end{array}
  \right]
=
  \left[
    \begin{array}{c c c}
      v_1 & \cdots & v_n
    \end{array}
  \right]
  A.
$$
The determinant of $A$ is nonzero since this operation is
invertible. Define a relation $\sim$ on $X$ by
$$
u \sim v \iff u = vA, \det A > 0.
$$
This is an equivalence relation on $X$:
\begin{enumerate}
  \item{
    $u \sim u$ since $u = uI$.
  }
  \item{
    If $u \sim v$, then $u = vA$ so
    $v = uA^{-1}$, and $\det A^{-1} = \frac{1}{\det A}$.
  }
  \item{
    If $u \sim v$ and $v \sim w$, then
    $u = vA = wBA$, and $\det(BA) = \det(B)\det(A) > 0$ when
    $\det(A), \det(B) > 0$.
  }
\end{enumerate}

\begin{defn}
An orientation on $V$ is an equivalence class $\mu$ for $\sim$ on $X$.
We usually write $\mu$ and $-\mu$ for these classes and say they are
opposite orientations.
\end{defn}

Note that $V$ has exactly two equivalence classes. Assume that
$u \nsim v$ and $u \nsim w$. We wish to show that $v \sim w$.
There exist matrices $A$, $B$ such that $u = vA$ and $u = w B$, and
$\det A, \det B < 0$. But $v = w BA^{-1}$ and
$\det BA^{-1} = \frac{\det B}{\det A} > 0$, so $v \sim w$.

\begin{xmpl}
\begin{itemize}
  \item{
    For $\mathbb{R}^1$, $\mu = [e_1]$ and $-u = [-e_1]$.
  }
  \item{
    For $\mathbb{R}^2$, $\mu = [\{e_1, e_2\}]$,
    $-\mu = [\{e_2, e_1\}]$.
  }
  \item{
    For $\mathbb{R}^3$, $\mu = [\{e_1, e_2, e_3\}]$
    (the ``right hand rule'') and
    $-\mu = [\{e_2, e_1, e_3\}]$ (the ``left hand rule'').
  }
\end{itemize}
\end{xmpl}

\subsection{Via $n$-covectors}
We can also construct this in terms of $n$-covectors
$\Lambda^n(V^\vee) = A_n(V)$. This space has dimension
$\{ n \choose n \} = 1$, and so is isomorphic to $\mathbb{R}$.

\begin{lemma}
Let $\beta \in A_n(V) \setminus \{ 0 \}$. Let
$u, v \in X$, so that $u = vA$. Then
$$
  \beta(u_1, \dots, u_n)
= (\det A)\beta(v_1, \dots, v_n).
$$
\end{lemma}
\begin{proof}
Let
$$
  u_j
= \sum_{i=1}^n
    a^i_j v_i.
$$
Then
\begin{align*}
   \beta(u_1, \dots, u_n)
&= \beta
     \left(
       \sum_{i=1}^n
         a_1^i v_i,
       \dots,
       \sum_{i=1}^n
         a_n^i v_i
     \right) \\
&= \sum_{I = (i_1, \dots, i_n)}
     (a_1^{i_1}, \dots, a_n^{i_n})
     \beta(v_{i_1}, \dots, v_{i_n}) \\
&= \sum_{\sigma \in S_n}
     a_1^{i_{\sigma(1)}}
     \cdots
     a_n^{i_{\sigma(n)}}
     \beta(
       v_{\sigma(1)}, \dots, v_{\sigma(n)}
     ) \\
&= \sum_{\sigma \in S_n}
     \mathrm{sgn}(\sigma)
       a_1^{i_{\sigma(1)}}
       \cdots
       a_n^{i_{\sigma(n)}}
       \beta(v_1, \dots, v_n) \\
&= (\det A)
   \beta(v_1, \dots, v_n).
\end{align*}
where $I$ ranges over \emph{all} multi-indices.
\end{proof}

\subsection{Orientation on manifolds}
A pointwise orientation $\mu = [\{X_1, \dots, X_n\}]$ is
continuous if and only if $\forall p \in M$, there exist coordinates
$(U, x^i)$ about $p$ with
$\dif x^1 \wedge \cdots \wedge \dif x^n (X_1, \dots, X_n) > 0$ on $U$.

\begin{theorem}
A smooth manifold $M$ is orientable if and only if there exists a
nowhere zero smooth $n$-form (a \emph{volume form})
$\omega : M \to \Lambda^n(T^\ast M)$ on $M$, i.e. for any
$p \in M$, $\omega_p \neq 0$ in $\Lambda^n(T_p^\ast M)$, i.e.
$\exists X_1(p), \dots, X_n(p) \in T_p M$ with
$\omega_p(X_1(p), \dots, X_n(p)) \neq 0$.
\end{theorem}

\begin{proof}
  \begin{itemize}
    \item[($\implies$)]{
      Let $\mu = [\{X_1, \dots, X_n\}]$ be a pointwise continuous
      orientation on $M$. Then there exist coordinates
      $(U, x^i)$ about $p$ with
      $\dif x^1 \wedge \cdots \wedge \dif x^n (X_1, \dots, X_n) > 0$
      on $U$. Let
      $\{(U_\alpha, x^i_\alpha)\}$ be a locally finite refinement of
      the cover given by all such charts. Let
      $\{ \rho_\alpha \}$ be a partition of unity subordinate to
      $\{U_\alpha\}$, so that $\mathrm{supp}(\rho_\alpha) \subset
      U_\alpha$ for each $\alpha$,
      $\rho_\alpha \geq 0$, and $\sum_\alpha \rho_\alpha \equiv
      1$.
      Define $\omega$ by
      $$
        \omega
      = \sum_\alpha
          \rho_\alpha
            \dif x^1_\alpha \wedge \cdots \wedge \dif x^n_\alpha.
      $$
      This is a well-defined smooth $n$-form, and for $p \in M$
      \begin{align*}
         \omega_p(X_1(p), \dots, X_n(p))
      &= \sum_\alpha
           \rho_\alpha(p)
           (\dif x^1_\alpha \wedge \cdots \wedge \dif x^n_\alpha)\
           (X_1(p), \dots, X_n(p)) \\
      &> 0
      \end{align*}
      since the $\rho_\alpha$ are nonnegative and sum to 1 and
      the wedge product evaluated on the $X_i(p)$ is strictly
      positive. Therefore $\omega_p \neq 0$.
    }
    \item[($\impliedby$)]{
      Let $\omega : M \to \Lambda^n(T^\ast M)$ be a smooth
      nowhere zero $n$-form. For any $p \in M$, there exist vectors
      $X_1(p), \dots, X_n(p) \in T_p M$ with
      $\omega_p(X_1(p), \dots, X_n(p)) > 0$. We claim that
      the pointwise orientation
      $\mu = [\{ X_1, \dots, X_n\}]$ is continuous.

      Let $p \in M$, and let $(U, z^i)$ be a connected coordinate
      chart about $p$. Then
      $\omega\restrict_U = f \dif z^1 \wedge \cdots \wedge \dif z^n$
      where $f : U \to \mathbb{R}^n$ is smooth, and since $\omega$ is
      nowhere zero $f$ is nowhere zero. Then either $f > 0$ on $U$
      or $f < 0$ on $U$.
      \begin{itemize}
        \item{
          If $f > 0$ on $U$, then let $(U, x^i) = (U, z^i)$. Then
          $$
            \dif x^1 \wedge \cdots \wedge \dif x^n
            (X_1, \dots, X_n)
          = \frac{1}{f} \omega(X_1, \dots, X_n)
          > 0.
          $$
        }
        \item{
          If $f < 0$ on $U$ then let $(U, x^i) = (U, -z^i)$ instead,
          which gives
          $$
            \dif x^1 \wedge \cdots \wedge \dif x^n
            (X_1, \dots, X_n)
          = -\frac{1}{f} \omega(X_1, \dots, X_n)
          > 0.
          $$
        }
      \end{itemize}
    }
  \end{itemize}
\end{proof}

This shows that we can think of an orientation on a connected manifold
$M$ as either an equivalence class of pointwise continuous framings
$\mu = [\{X_1, \dots, X_n\}]$ or as an
equivalence class of nowhere vanishing $n$-forms $[\omega]$, where
$\omega_1 \sim \omega_2$ when $\omega_1 = f \omega_2$ for some smooth,
strictly positive real function $f$, when $\omega_1, \omega_2$ are nowhere zero
smooth $n$-forms. This makes sense because in coordinates,
$\omega_1 = g \dif x^1 \wedge \cdots \wedge \dif x^n$ and
$\omega_2 = h \dif x^1 \wedge \cdots \wedge \dif x^n$ where
$g, h : U \to \mathbb{R}$ are smooth and nowhere zero, so
$\omega_1 = \frac{g}{h} \omega_2 = f \omega_2$ on $U$ for a smooth
nowhere zero function $f$, for any two smooth $n$-forms $\omega_1$ and
$\omega_2$. These definitions of orientation are the same because
$\omega(X_1, \dots, X_n) > 0$ independent of the choice of
representatives in each equivalence class.

\begin{defn}
A diffeomorphism
$F : (N, [\omega_N]) \to (M, [\omega_M])$ between oriented manifolds is
\emph{orientation-preserving} if $[F^\ast \omega_M] = [\omega_N]$ and
\emph{orientation-reserving} otherwise, i.e. if
$[F^\ast \omega_M] = [-\omega_N]$.
\end{defn}

Let $U, V$ be open in $\mathbb{R}^n$ with coordinates $x^i$, $y^i$
respectively. Let $F: U \to V$ .Then
\begin{align*}
   F^\ast(\dif y^1 \wedge \cdots \wedge \dif y^n)
&= \dif(F^\ast y_1) \wedge \cdots \wedge \dif (F^\ast y^n) \\
&= \dif F^1 \wedge \cdots \wedge \dif F^n \\
&= \frac{\partial (F^1 \cdots F^n)}{\partial x^1 \cdots x^n}
     \dif x^1 \wedge \cdots \wedge \dif x^n \\
&= \mathrm{Jac}(F) \dif x^1 \wedge \cdots \wedge \dif x^n,
\end{align*}
so $F$ is orientation preserving if and only if
$\mathrm{Jac}(F) > 0$ on $U$ and orientation reversing if and only if
$\mathrm{Jac}(F) < 0$ on $U$.

\begin{xmpl}
  \begin{enumerate}
    \item{
      The map $F: \mathbb{R}^n \to \mathbb{R}^n$ given by
      $x \mapsto -x$ has
      $$
        \left[
          \frac{\partial F^i}{\partial x^j}
        \right]
      = -I_n
      $$
      so $F$ is orientation reversing for $n$ even and orientation
      preserving for $n$ odd.
    }
    \item{
      Let $f: \mathbb{R}^{n+1} \to \mathbb{R}$, and let 0 be a regular
      value. Then $S = f^{-1}(0)$ is an oriented $n$-manifold since
      $$
        \omega
      = \sum_{i=1}^{n+1}
          (-1)^{i-1}
          \frac{\partial f}{\partial x^i}
          \dif x^1
          \wedge \cdots
          \wedge \hat{\dif x^i}
          \wedge \cdots
          \wedge \dif x^n
      $$
      is a nowhere zero $C^\infty$ $n$-form on $S$.

      For $f: \mathbb{R}^2 \to \mathbb{R}$ given by
      $f(x^1, x^2) = (x^1)^2 + (x^2)^2 - 1$ has
      $S^1 = f^{-1}(0)$. Then
      $\omega = x^1 \dif x^2 - x^2 \dif x^1$ is nonvanishing on the
      vector field
      $$
        X
      = - x^2 \frac{\partial}{\partial x^1}
        + x^1 \frac{\partial}{\partial x^2}
      $$
      since
      \begin{align*}
         \omega(X)
      &= x^1 \dif x^2(X) - x^2 \dif x^1(X) \\
      &= x^1(x^1) - x^2 (-x^2) \\
      &= 1
      \end{align*}
      on $S^1$.

      The map $F: \mathbb{R}^2 \to \mathbb{R}^2$
      given by $x \mapsto -x$ restricts to the
      antipodal map $F: S^1 \to S^1$. This pulls back $\omega$ as
      \begin{align*}
         F^\ast \omega
      &= F^\ast(x^1 \dif x^2 - x^2 \dif x^1) \\
      &= F^\ast x^1 \dif (F^\ast x^2)
       - F^\ast x^2 \dif (F^\ast x^1) \\
      &= (x^1 \circ F) \dif (x^2 \circ F)
       - (x^2 \circ F) \dif (x^1 \circ F) \\
      &= -x^1 \dif (-x^2) - (-x^2) \dif (-x^1) \\
      &= x^1 \dif x^2 - x^2 \dif x^1
      \end{align*}
      which preserves not only the orientation but the form $\omega$ itself.
    }
    \item{
      With $S = S^2$, let
      $$
        \omega
      = x^1 \dif x^2 \wedge \dif x^3
      - x^2 \dif x^1 \wedge \dif x^3
      + x^3 \dif x^1 \wedge \dif x^2
      $$
      so
      \begin{align*}
         F^\ast(\omega)
      &= (-1)^3 \omega = -\omega,
      \end{align*}
      i.e. this is orientation reversing.
    }
    \item{
      $\mathbb{R} P^n$ is orientable for $n$ odd and not orientable
      for $n$ even.
      Since $\mathbb{R} P^n \simeq S^n / (x \sim A(x))$.
    }
  \end{enumerate}
\end{xmpl}

We have seen two perspectives on orientation, related by the fact that
if $\mu = [\{ X_1, \dots, X_n \}]$ is an orientation, then
$\forall \{ Y_1, \dots, Y_n \} \sim \{ X_1, \dots, X_n \}$ and
$\forall \tilde{\omega} \sim \omega$,
$\tilde{\omega}(Y_1, \dots, Y_n) > 0$.
A third perspective on orientation is given by equivalence classes of
oriented atlases $[\mathcal{A} = \{(U, x^i)\}]$.

\begin{defn}[Oriented atlas]
A smooth atlas $\mathcal{A}$ is \emph{oriented} if
$\forall (U_\alpha, \phi_\alpha), (V_\beta, \psi_\beta) \in
\mathcal{A}$,
with $U_\alpha \cap V_\beta$ nonempty,
$$
  \psi_\beta \circ \phi_\alpha^{-1}
: \phi_\alpha(U \cap V) \to  \psi_\beta(U \cap V)
$$
is orientation preserving. In coordinates,
$$
  \det
    \left[
      \frac{\partial y^i}{\partial x^j}
    \right]
: U \cap V \to \mathbb{R} \setminus \{ 0 \}
$$
is positive.
\end{defn}

\begin{theorem}
$M$ is orientable if and only if it admits an oriented atlas.
\end{theorem}
\begin{proof}
  \begin{itemize}
    \item[($\implies$)]{
      Let $\mu = \{ X_1, \dots, X_n \}$ be a pointwise continuous
      orientation. Then for any $p \in M$, there exists a coordinate
      chart $(U, x^i)$ containing $p$ with
      $\dif x^1 \wedge \cdots \wedge \dif x^n(X_1, \dots, X_n) > 0$.
      Let $\mathcal{A}$ be the atlas composed of these charts for
      every point $p$.

      Suppose $(U, x^i)$ and $(V, y^i)$ are overlapping charts in
      $\mathcal{A}$. On $U \cap V$,
      $$
      \dif x^1 \wedge \cdots \wedge \dif x^n(X_1, \dots, X_n) > 0
      $$
      and
      $$
      \dif y^1 \wedge \cdots \wedge \dif y^n(X_1, \dots, X_n) > 0.
      $$
      But
      $$
        \dif y^1 \wedge \cdots \wedge \dif y^n
      = \det\left[\frac{\partial y^i}{\partial x^j}\right]
        \dif x^1 \wedge \cdots \wedge \dif x^n
      $$
      so $\det\left[\frac{\partial y^i}{\partial x^j}\right] > 0$.
    }
    \item[($\impliedby$)]{
      Let $\mathcal{A} = \{(U, x^i)\}$ be an oriented atlas for $M$.
      Then each $p \in M$ belongs to one of these charts. Define
      $$
        \mu_p
      = \left[
          \left\{
            \left.
              \frac{\partial}{\partial x^1}
            \right|_p,
            \dots
            \left.
              \frac{\partial}{\partial x^n}
            \right|_p
          \right\}
        \right].
      $$
      We must verify that when $p$ also lies in chart $(V, y^i)$ where
      $U \cap V$, we must check that the ordered basis
      $$
      \left\{
        \left.
          \frac{\partial}{\partial x^1}
        \right|_p,
        \dots
        \left.
          \frac{\partial}{\partial x^n}
        \right|_p
      \right\}
      $$
      at $p$ is
      equivalent to the ordered standard basis
      $$
      \left\{
        \left.
          \frac{\partial}{\partial y^1}
        \right|_p,
        \dots
        \left.
          \frac{\partial}{\partial y^n}
        \right|_p
      \right\}
      $$
      in the other coordinate chart. But the matrix
      $$
        \left[
          \left.
            \frac{\partial}{\partial y^1}
          \right|_p,
          \dots
          \left.
            \frac{\partial}{\partial y^n}
          \right|_p
        \right]
      = \left[
          \left.
            \frac{\partial}{\partial x^1}
          \right|_p,
          \dots
          \left.
            \frac{\partial}{\partial x^n}
          \right|_p
        \right]
        \left[
          \frac{\partial y^i}{\partial x^j}(p)
        \right]
      $$
      and since $\det\left[\frac{\partial y^i}{\partial^j}\right] >
      0$, these ordered bases are similar, so the map $\mu_p$ is
      well-defined. This $\mu$ is continuous and in the coordinate
      chart $(U, x^i)$ is represented by the $C^0$ local framing
      $
      \left\{
        \left.
          \frac{\partial}{\partial y^1}
        \right|_p,
        \dots
        \left.
          \frac{\partial}{\partial y^n}
        \right|_p
      \right\}.
      $
    }
  \end{itemize}
\end{proof}

Given two oriented atlases $\mathcal{A}$ and
$\mathcal{B}$, we define $\mathcal{A} \sim \mathcal{B}$ when
$\forall (U_\alpha, \phi_\alpha) \in \mathcal{A},
(V_\alpha, \psi_\beta) \in \mathcal{B}$ with $U_\alpha \cap V_\beta$
nonempty,
$$
  \psi_\beta \circ \phi_\alpha^{-1}
: \phi_\alpha(U_\alpha \cap V_\beta) \to \psi_\beta(U_\alpha \cap V_\beta)
$$
is orientation preserving as a map between Euclidean spaces.
This is an equivalence relation on ordered atlases, where transitivity
follows from the chain rule.

For all $\tilde{\omega} \sim \omega$ and
for all $(V, y^i) \in \mathcal{B} \sim \mathcal{A}$,
$\tilde{\omega} = f \dif y^1 \wedge \cdots \dif y^n$ on $V$
where $f > 0$ on $V$.

For all $\{Y_1, \dots, Y_n\}_p \sim \{X_1, \dots, X_n\}_p$ and any
chart in an oriented atlas,
$
  \{Y_1, \dots, Y_n\}
\sim
  \left\{
    \left.
      \frac{\partial}{\partial y^1}
    \right|_p,
    \dots
    \left.
      \frac{\partial}{\partial y^n}
    \right|_p
  \right\}.
$

\section{Manifolds with boundary}
A manifold with boundary is modeled locally by the closed upper
half-space
$$
  \mathcal{H}^n
= \{ x = (x^i) \in \mathbb{R}^n ~\vert~
     x^n \geq 0 \
  \}
$$
which is given the subspace topology in $\mathbb{R}^n$.
The interior of $\mathcal{H}^n$ consists of those points whose last
coordinate is positive, while the boundary
$\partial \mathcal{H}^n$ consists of those points
whose last coordinate is zero, which is itself a manifold of dimension
$n - 1$.

Open sets on $\mathcal{H}^n$ are unions of open balls in
$\mathrm{int}(\mathcal{H}^n)$ and half-closed half-balls lying on
$\partial \mathcal{H}^n$.

For $n = 1$, we see that
$\mathcal{L}^1 = \{ x ~\vert~ x \leq 0 \}$ has an
orientation-reversing homeomorphism with $\mathcal{H}^1$
given by $x \mapsto -x$. There is no orientation-preserving
homeomorphism $\mathcal{H}^1 \to \mathcal{L}^1$. For
$n > 1$ we can define an orientation preserving homeomorphism
$(x^1, \dots, x^n) \mapsto (-x^1, x^2, \dots, x^{n-1}, -x^n)$.

\begin{defn}
A \emph{topological $n$-manifold with boundary} is a topological
space $M$ that is
\begin{enumerate}
  \item{
    Hausdorff,
  }
  \item{
    second countable, and
  }
  \item{
    locally homeomorphism to $\mathcal{H}^n$, or possibly
    $\mathcal{L}^1$ in the case $n = 1$.

    This means for each $p \in M^n$ there exists a chart
    $(U, \phi)$, i.e. an open set $U \subset M$ containing $p$ and a
    homeomorphism $\phi: U \to \phi(U) \subset\ mathcal{H}^n$ where
    $\phi(U)$ is an open subset of $\mathcal{H}^n$.
  }
\end{enumerate}
\end{defn}

\begin{xmpl}
  \begin{enumerate}
    \item{
      Trivially, $\mathcal{H}^n$.
    }
    \item{
      The closed unit ball
      $\bar{\mathbb{D}}^n = \{ x \in \mathbb{R}^n ~\vert~ \|x\| \leq 1 \}$.
    }
  \end{enumerate}
\end{xmpl}

\subsubsection{A generalization of smoothness}
Let $N, M$ be smooth manifolds with $S \subset N$,
$T \subset M$, possibly not open.

\begin{defn}
A map $f : S \to M$ is \emph{smooth} at $p \in S$, there
exists an open $U \subset N$ with $p \in U$ and a smooth extension of
$f: U \cap S \to M$ to $\tilde{f}: U \to M$, i.e. a smooth map
$\tilde{f}: U \to M$ with $\tilde{f}\restrict_{U \cap S} =
f\restrict_{U \cap S}$.

A map $f : S \to M$ is smooth if it is smooth
at all $p \in S$.

A diffeomorphism $f: S \to T$ is smooth if
$f: S \to T \subset M$ is smooth, bijective, and has smooth whole
inverse $f^{-1}: T \to S$.
\end{defn}

\begin{remark}
A map $f: S \to M$ with $S \subset N$ is smooth if and only if there
exists an open $U \subset N$ with $S \subset U$ and a smooth map
$\tilde{f}: U \to M$ that agrees with $f$ on $S$. This follows by
summing up the local smooth extensions with a partition of unity.
\end{remark}

\begin{defn}
An atlas of charts $\mathcal{A} = \{(U_\alpha, \phi_\alpha)\}$ for a
topological $n$-manifold with boundary is smooth provided all
transition functions $\phi_\beta \circ \phi^{-1}$ are diffeomorphisms
in the above sense.

A \emph{smooth $n$-manifold with boundary} is a topological
$n$-manifold with boundary equipped with a maximal smooth atlas.
\end{defn}

\subsection{Invariance of domain}

\begin{theorem}[$C^\infty$ invariance of domain]
Let $M^n$, $N^n$ be smooth manifolds with equal dimension with subsets
$U \subset M^n$ and $S \subset N^n$. Suppose $f: U \to S$ is a
diffeomorphism. If $U$ is open in $M^n$, then $S$ is open in $N^n$.
\end{theorem}

\begin{proof}
Let $f(p) \in S$ for some $p \in U$. We wish to show that there exists
an open neighborhood $V_{f(p)} \subset N^n$ about $f(p)$ with
$V_{f(p)} \subset S$.

$f$ has smooth inverse $f^{-1}$, so there is some open subset
$V \subset N^n$ with $S \subset V$ and a map
$g: V \to M^m$ that agrees with $g = f^{-1}$ on $S$. The composite map
$g \circ f : U \to M^n$ has
$(g \circ f)(u) = f^{-1}(f(u)) = u$, i.e. $g \circ f =
\mathrm{id}_U$. In particular
$(g \circ f)_{\ast, p} : T_pM \to T_pM$ is the identity linear map.
But $(g \circ f)_{ast, p} = g_{\ast,f(p)} \circ f_{\ast, p}$ and
$f_{\ast, p} : T_p M \to T_{f(p)} N$. Since $\mathrm{id}$ is
injective, $f_{\ast, p}$ must then be injective, so since
$M$ and $N$ have the same dimension $f_{\ast,p}$ is an
isomorphism. Therefore the inverse function theorem implies that
there exist open sets $U_p \subset M^n$ and $V_{f(p)} \subset N^n$
about $p$ and $f(p)$ on which
$f\restrict_{U_p} : U_p \to V_{f(p)}$ is a diffeomorphism. Then
$V_{f(p)}$ is an open neighborhood in $M$ that lies in the image of
$f(U_p) \subset S$ as desired.
\end{proof}

\begin{prop}
Let $M$ be a manifold with boundary with
$p \in (U, \phi) \cap (V, \psi)$. Then
$\phi(p) \in \mathrm{int}(\mathcal{H}^n)$ if and only if
$\psi(p) \in \psi(U \cap V)$ is in $\mathrm{int}(\mathcal{H}^n)$.
This says that whether $p$ is in the interior or in the boundary is
independent of chart.
\end{prop}
\begin{corol}
$\phi(p) \in \partial \mathcal{H}^n$ if and only if
$\psi(p) \in \partial \mathcal{H}^n$.
\end{corol}

\begin{proof}
Suppose $\phi(p) \in \phi(U \cap V)$ lies in $\mathrm{int}
(\mathcal{H}^n)$. Then there exists an open $B$ about $\phi(p)$ such
that $B \subset \phi(U \cap V)$ is open in $\mathbb{R}^n$. Note that
$\psi \circ \phi^{-1}\restrict_B : B \to \psi(U \cap V) \subset
\mathcal{H}^n \subset \mathbb{R}^n$. By invariance of domain,
$\psi \circ \phi^{-1}(B)$ is an open neighborhood of $\psi(p)$ in
$\mathbb{R}^n$, so $\psi(p) \in \mathrm{int} \mathcal{H}^n$.
\end{proof}

\begin{defn}[Boundary of a manifold with boundary]
The \emph{boundary} of a manifold with boundary $M$ is the set
$$
  \partial M
= \{ p \in M ~\vert~
     \exists (U, \phi)
       \text{ about $p$ s.t. $\phi(p) \in \partial \mathcal{H}^n$}
  \}.
$$
\end{defn}

\begin{prop}
$\partial M$ with the subspace topology from $M$ is a $C^\infty$
$n-1$ manifold without boundary.
\end{prop}
\begin{proof}
  \begin{itemize}
    \item{
      $\partial M$ is second countable and Hausdorff when given the
      subspace topology.
    }
    \item{
      Let $\mathcal{A} = \{(U_\alpha, \phi_\alpha) ~\vert~ \alpha \in
      A\}$ be a smooth atlas for $M$. For any $\alpha$, we have
      $$
        \phi_\alpha\restrict_{U_\alpha \cap \partial M}
      : U_\alpha \cap \partial M \to \partial \mathcal{H}^n
      = \mathbb{R}^{n-1}
      $$
      which is a diffeomorphism onto its image. Therefore
      $$
        \mathcal{B}
      = \{(U_\alpha \cap \partial M,
           \phi_\alpha \restrict_{U_\alpha \cap \partial M})\}
      $$
      is a smooth atlas for $\partial M$.
    }
  \end{itemize}
\end{proof}

There is also a theory for ``manifolds with corners'', where
boundaries can have boundaries.

\subsection{Constructions on manifolds with boundary}

Let $(M, \partial M)$ be a manifold with boundary and $p \in \partial
M$. We can define several constructions on these objects.
\begin{itemize}
  \item{
    $C_p^\infty(M)$, the $\mathbb{R}$-algebra of germs of $C^\infty$
    real functions at $p$. This means that if
    $f: U \to \mathbb{R}$ and $g: V \to \mathbb{R}$ where $f \sim g$
    when there exists an open $W \subset U \cap V$ such that $f = g$
    on $W$. Smoothness is regarded in our more general sense.
  }
  \item{
    $T_pM$, the $\mathbb{R}$ vector space of point derivations of the
    algebra of germs $C_p^\infty(M)$. The dimension of $T_p M$ is
    $\dim M = n$, not the dimension of the boundary
    $\dim (\partial M) = n - 1$. In particular for $\mathcal{H}^2$, a
    point $p$ on the boundary has a well-defined tangent vector
    $-\left.\frac{\partial}{\partial y}\right|_p \in T_p
    \mathcal{H}^2$, but is not tangent to any curve in $\mathcal{H}^2$.
  }
  \item{
    There is a natural inclusion
    $i: T_p \partial M \to T_p M$ given by
    $i(X)(f) = X(f\restrict_{\partial M})$. This realizes
    $T_p \partial M$ as a $n-1$ dimensional vector subspace of $T_p
    M$.
    Exercise: this map is injective.
  }
  \item{
    The cotangent space at $p \in \partial M$ is
    $$
    T_p^\ast M = \mathrm{Hom}(T_p M, \mathbb{R}),
    $$
    and we have $k$-covectors by taking exterior powers of this space,
    and differential $k$-forms as smooth sections of the bundle
    $\Lambda^k(T^\ast M) \to M$.
  }
  \item{
    Orientation of a manifold with boundary is defined by any of the
    following equivalent notions:
    \begin{enumerate}
      \item{
        a pointwise orientation $\mu = \{X_1, \dots, X_n\}$ in terms
        of frames,
      }
      \item{
        a nonvanishing smooth $n$-form $\omega$, where $n = \dim M$,
      }
      \item{
        an oriented atlas.
      }
    \end{enumerate}
  }
\end{itemize}

\begin{xmpl}
$[0, 1]$ is a 1-manifold with boundary $\partial M = \{0, 1\}$. We
have charts $\phi_1: [0, 1) \to [0, 1) \subset \mathcal{H}^1$ given by
$\phi_1(x) = x$ and
$\phi_2: (0, 1] \to [0, 1) \subset \mathcal{H}^1$ by
$\phi_2(x) = 1 - x$. Then $\phi_1^{-1} \circ \phi_2 = 1 - x$, so the
chart $\phi_2$ is not orientation preserving. But if we consider the
chart $\phi_3 : (0, 1] \to (-1, 0] \subset \mathcal{L}^1$ given by $x
\mapsto x - 1$, this is orientation preserving.
\end{xmpl}

We will show next that if $M$ is oriented then there is a canonical
orientation on $\partial M$, by contracting the volume form with an
outward-pointing vector field.
