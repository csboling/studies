\section{Vector bundles}

A vector bundle is informally a family of vector spaces smoothly
parameterized by a smooth manifold.

\begin{xmpl}
We can think of $\mathbb{R}^3$ as a family of copies of $\mathbb{R}^2$
parameterized by the line $x = y = 0$. A plane such as $z = z_0$ is not
a subspace since it does not include the origin, but it has a natural
vector space structure given by
$$
  (x_1, y_1, z_0) + \alpha(x_2, y_2, z_0)
= (x_1 + \alpha x_2, y_1 + \alpha y_2, z_0).
$$
\end{xmpl}

\begin{defn}[Fiber bundle]
Let $\pi: E \to B$ and $\tilde{\pi}: \tilde{E} \to \tilde{B}$ be
smooth surjective maps. Call $E$ the total space and $B$ the base
space. For a point $b \in B$,
we write $E_b = \pi^{-1}(b) = \{e \in E ~\vert~ \pi(e) = b\}$ and call
this the \emph{fiber} of $\pi$ above $b$.

A map $\phi: E \to \tilde{E}$ is \emph{fiber-preserving} if
$\tilde{\pi} \circ \phi = \pi$, i.e.
$\phi\restrict_{E_b} : E_b \to \tilde{E}$ satisfies
$\phi\restrict_{E_b} (E_b) \subset \tilde{E}_b$.

$\pi$ is said to be \emph{locally trivial of rank $n$} when
\begin{enumerate}[(a)]
  \item{
    each fiber $E_b$, $b \in B$ has the structure of a real vector
    space isomorphic to $\mathbb{R}^n$.
  }
  \item{
    for each $b \in B$, there exists a neighborhood $U$ about $b$ and
    a fiber-preserving map
    (to $\tilde{\pi} : U \times \mathbb{R}^n$ given by $\pi(u, v^I) = u)$)
    $\phi : \pi^{-1}(U) \to U \times \mathbb{R}^n$ such that
    $\phi\restrict_{E_u} \to \{ u \} \times \mathbb{R}^n$ is a linear
    isomorphism.
  }
\end{enumerate}
A \emph{rank $n$ vector bundle} is a triple $(\pi, E, B)$ where
$\pi: E \to B$ is a surjective smooth map that is locally trivial of
rank $n$.
\end{defn}

\begin{xmpl}
Let $M$ be a smooth manifold. The \emph{product bundle} is defined by
the map $\pi: M \times \mathbb{R}^n \to M$ given by $\pi(m, v) =
m$. The fiber $E_m = \{m\} \times \mathbb{R}^n$ has a natural vector
space structure. Let $m \in M$ and take the open set $U = M$. Then
$\pi^{-1}(U) = \pi^{-1}(M) = M \times \mathbb{R}^n$, and the local
trivialization is given by
$\mathrm{id}_{M \times \mathbb{R}^n}: \pi^{-1}(U) = M \times \mathbb{R}^n \to \mathbb{M} \times
\mathbb{R}^n$
\end{xmpl}

\begin{xmpl}[Restricting bundles to regular submanifolds]
Let $S \subset B$ be a regular submanifold, and
$\pi: E \to B$ be a vector bundle. Write
$E\restrict_S \triangleq \pi^{-1}(S)$. We can show directly that
$\pi\restrict_{E\restrict_{S}}: E\restrict_S \to S$ is a vector bundle
over $B$. We require $S$ to be embedded so that $E\restrict|_S$ has a
manifold structure.
\end{xmpl}

We could also consider \emph{sub-bundles}, where instead we restrict
each fiber to a subspace.

\subsection{Maps between bundles}
\begin{defn}
Let $\pi: E \to N$ and $\tilde{\pi}: F \to M$ be smooth vector bundles
(of possibly different rank). A \emph{bundle map} is a pair
$(\bar{f}, f)$ of maps $\bar{f}: E \to F$ and $f: N \to M$ such that
$f \circ \pi = \tilde{\pi} \circ \tilde{f}$. In this case
$\bar{f}((E_1)_b) \subset (E_2)_{f(b)}$. For vector bundles we also
require that $\bar{f}: (E_1)_b \to (E_2)_{f(b)}$ is linear.
\end{defn}

\begin{defn}[Isomorphism of vector bundles]
We say that vector bundles $(\pi_1, E_1, B)$ and
$(\pi_2, E_2, B)$ over $B$ are isomorphic if there exists a
diffemorphism $\bar{f}: E_1 \to E_2$ such that
$(\bar{f}, \id_B)$ is a bundle map.
\end{defn}

\begin{defn}
We say that a rank $n$ vector bundle $\pi: E \to B$ is \emph{trivial} if it is
isomorphic to the product bundle $p: B \times \mathbb{R}^n \to B$.
\end{defn}

\begin{xmpl}
The M\"obius bundle is a nontrivial vector bundle of rank 1 over
$S^1$. Take $E = [0, 1] \times \mathbb{R} / (0, t) \sim (1, -t)$. Then
there is a map $E \to S^1 = [0, 1] / 0 \sim 1$
\end{xmpl}

\subsection{Tangent bundle}
\begin{prop}
Let $X$ be a set and $\mathscr{B}$ be a collection of
subsets of $X$ such that $\cup_{B \in \mathscr{B}} B = X$.
Suppose that for any $B_1, B_2 \in \mathscr{B}$, if $p \in B_1 \cap
B_2$, $\exists B \in \mathscr{B}$ with $p \in B \subset B_1 \cap B_2$.
Define $\tau = \{U ~\vert~ U = \cup B_\lambda, B_\lambda \in
\mathscr{B}\}$.
Then $\tau$ is a topology on $X$ and $\mathscr{B}$ is a basis for $\tau$.
\end{prop}

\begin{proof}
$\varnothing = \cup_{B \in \varnothing} B$. $X \in \tau$ by
assumption, and $\tau$ is closed under union. If $B_1, B_2 \in \mathscr{B}$,
then $B_1 \cap B_2 \in \tau$.
Let $U, V \in \tau$ with $U = \cup B_\lambda$,
$V = \cup B_\mu$. Then
$$
  U \cap V
= (\cup B_\lambda) \cap (\cup B_\mu)
= \cup (B_\lambda \cap B_\mu)
\in \tau.
$$

Let $U \in \tau$, $p \in U$. Write
$U = \cup B_\lambda$, $B_\lambda \in \mathscr{B}$. Then
$p \in B_\lambda \subset U$ for some $B_\lambda \in \mathscr{B}$ in
the union.
\end{proof}

\begin{prop}
Let $M$ be a manifold. Then $M$ has a countable basis consisting of
coordinate charts.
\end{prop}

\begin{proof}
Let $\mathcal{A} = \{ (U_\alpha, \phi_\alpha) \}$ be a maximal atlas
of charts. Take $\mathscr{B} = \{ U_i \}$ to be a countable basis for
$M$, since $M$ is second countable.

Let $p \in U$. Then $\exists U_\alpha \in \mathcal{A}$ with
$p \in U_\alpha$. There exists $U_{p, \alpha} \in \mathscr{B}$ with
$p \in U_{p, \alpha} \subset U_\alpha$. Then
$(U_{p, \alpha}, \phi_\alpha)$ is a chart, and
$\{ U_{p, \alpha} \}$, omitting repetitions, is the desired countable
basis of coordinate charts.
\end{proof}

\begin{defn}[Tangent bundle]
As a set, $TM = \coprod_{p \in M} T_p M$, where $T_p M$ is the vector
space of point derivations of $C_p^\infty(M)$. We may also write
$TM = \{(p, v) ~\vert~ p \in M, v \in T_p M \}$. We have a surjective
map $\pi: TM \to M$ given by $\pi(p, v) = p$. As an abuse of notation,
we often see written $v \in TM$, where $\pi(v) = p$ is the
\emph{footpoint} of $v \in TM$.
\end{defn}

We can assign a topology to $TM$. Let
$(U, \phi) = (U, x^1, \dots, x^m)$ be a coordinate chart in $M$, and
let
$(u, w) \in T U = \{ (u, w) ~\vert~ u \in U, w \in T_u U = T_u M \}
 = \pi^{-1}(U)$.
There exist unique $c^i(w)$, for $i = 1, \dots, m$, with
$w = \sum_i c^i(w) \left.\frac{\partial}{\partial x^i}\right|_U$
since we have a basis
$\left\{
  \left.\frac{\partial}{\partial x^1}\right|_u, \dots,
  \left.\frac{\partial}{\partial x^m}\right|_u
\right\}$ of $T_uM$.

Define $\tilde{\phi}: TU \to \phi(U) \times \mathbb{R}^m$ by
$\tilde{\phi}((u, w)) = (\phi(u), c^1(w), \dots, c^m(w))$.
This has an inverse function given by
$$
(\tilde{\phi})^{-1}(\phi(u), c^1, \dots, c^m)
= (u, \sum c^i \left.\frac{\partial}{\partial x^i}\right|_U)
\in TU.
$$
Then we have a bijective map
$TU \to \phi(U) \times \mathbb{R}^m \subset \mathbb{R}^m \times
\mathbb{R}^m$.

\begin{prop}
The tangent bundle is a topological space.
\end{prop}

\begin{proof}
We then topologize $TU$ be declaring $\tilde{\phi}$ to be a
homeomorphism, i.e. by pulling back the topology on
$\mathbb{R}^m \times \mathbb{R}^m$ by $\tilde{\phi}$. Another way to
say this is that $A \subset TU$ is open if and only if
$\phi(A) \subset \phi(U) \times \mathbb{R}^m$ is open.

If $V \subset U$, then $TV \subset TU$. We let this have the topology
given by the chart $(V, \phi\restrict_V)$. This is the same as the
subspace topology, since $V \subset U$ open implies
$\phi(V) \subset \phi(U)$ is open, so
$\phi(V) \times \mathbb{R}^m \subset \phi(U) \times \mathbb{R}^m$ is
open.

Take $\mathscr{B} = \{ A ~\vert~ A \text{ open in $TU$ for some chart
  $(U , \phi)$} \}$. Let $(p, v) \in TM$.
Then there exists a chart $U$ in $M$ about $p$, so $(p, v) \in TU$.
Let $A, B \in \mathscr{B}$. Then $A \subset TU$ for some chart $U$
and $B \subset TV$ for some chart $V$.
But $T(U \cap V) \subset T(U) \cap T(V)$.
Since $T(U \cap V) \subset T(U)$,
$A \cap T(U \cap V) \subset T(U \cap V)$ is open. Similarly
$T(U \cap V) \subset T(V)$, so
$B \cap T(U \cap V) \subset T(U \cap V)$ is open. Then
\begin{align*}
   A \cap B
&= (A \cap T(U)) \cap (B \cap T(V)) \\
&= (A \cap T(U) \cap T(V)) \cap (B \cap T(U) \cap T(V)) \\
&= (A \cap T(U \cap V)) \cap (B \cap T(U \cap V))
\end{align*}
is open in $T(U \cap V)$ because the last two sets are. Therefore
$A \cap B$ is open in $T(U \cap V)$.
\end{proof}

\begin{prop}
The tangent bundle is a manifold.
\end{prop}

\begin{proof}
We have already shown that $TM$ is locally Euclidean, since we have
homeomorphisms $\tilde{\phi}: TU \to \phi(U) \times \mathbb{R}^n$.

Let $(p, v), (q, w) \in TM$. If $p \neq q$, then since $M$ is
Hausdorff there exist charts $U, V$ with $p \in U$, $q \in V$, and
$U \cap V = \varnothing$. Then $TU \cap TV = \varnothing$, and
$(p, v) \in TU, (q, w) \in TV$. If $p = q$, then let $U$ be a chart
containing $p$. Since $TU$ is homeomorphic to
$\phi(U) \times \mathbb{R}^n$, which is open in
$\mathbb{R}^n \times \mathbb{R}^n$, $TU$ is Hausdorff. Therefore there
exist open sets $A, B \in TU$ with $(p, v) \in A$, $(p, w) \in B$, and
$A \cap B = \varnothing$.

From the earlier proposition, $M$ has a countable basis $\{ U_i \}_{i=1}^\infty$ of
coordinate charts. For each $i$, $TU_i$ is homeomorphic to some
$\phi_i(U_i) \times \mathbb{R}^m$, an open subset of Euclidean space,
and hence
$TU_i$ is second countable. Therefore there exists a countable basis
$\{ V_i^j \}_{j=1}^\infty$ for $TU_i$, and $\{V_i^j\}_{i,j =
  1}^\infty$ is a countable basis for $TM$.
\end{proof}
