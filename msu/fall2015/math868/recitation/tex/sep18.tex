\section{Comments}
Recall that $C_p^\infty$ consists of \emph{equivalence classes}, i.e.
$$
[(f, U)] + [(g, V)] = [(f + g, U \cap V)].
$$

\begin{theorem}[Inverse function theorem]
Let $f : O \subset \mathbb{R}^n \to \mathbb{R}^n$ be $C^r$ on $O$, and
let $a \in O$ with $Df(a)$ nonsingular. Then there exist open sets $U,
V$ such that $a \in U$, $f(a) \in V$, and
\begin{enumerate}
  \item{
    $f$ is injective on $U$
  }
  \item{
    there exists a $g$ such that $g(f(x)) = x$ and $g$ is $C^r$.
  }
\end{enumerate}
Here for $D f$ of a function $f: \mathbb{R}^n \to \mathbb{R}^n$ we
mean the Jacobian matrix
$$
  (Df)^i_j
= \left(
    \frac{\partial f^i}
         {\partial x_j}
  \right)
$$
\end{theorem}

\begin{proof}
Last time we showed that for a function
$f: (X, d) \to (x, d)$ between metric spaces that is a contraction
map, i.e. $d(f(x_1), f(x_2)) \leq c d(x_1, x_2)$, then $f$ has a unique
fixed point. It suffices to prove the case when
$Df(a) = \mathrm{Id}_{n  \times n}$, since as long as $D f$ is
nonsingular there is a sequence of linear maps which can be
precomposed with $D f$ to arrive at $\mathrm{Id}_{n \times n}$.

Last time we defined
$$
\varphi_y(x) = x + y - f(x),
$$
which has a fixed point $x$ if and only if $y = f(x)$. We can pick a
neighborhood $U$ which is convex small enough that
$|Df(x) - Df(a)| \leq \frac{1}{4}$. Then for $x_1, x_2 \in U$, we have
$$
|\varphi_y(x_1) - \varphi_y(x_2)| \leq \frac{1}{4}|x_1 - x_2|
$$
(from last time) so
\begin{align*}
      |D\varphi_y(x)|
&=    |D(y + x - f(x))| \\
&=    |D x - D f(x)| \\
&=    |\mathrm{Id} - Df(x)| \\
&\leq \frac{1}{4}.
\end{align*}

$V = f(U)$ is nonempty, so pick $y_0 \in V$.
Then $y_0 = f(x_0)$ for some $x_0$, so pick $r$ small enough that
$\overline{B_r(x_0)} \subset U$. We claim that
$\varphi_y : \overline{B_r(x_0)} \to \overline{B_r(x_0)}$ for every
$y \in B_{\frac{r}{2}}(y_0)$. Let $x \in \overline{B_r(x_0)}$.
Then
\begin{align*}
      |\varphi_y(x) - x_0|
&\leq |\varphi_y(x) - \varphi_y(x_0)|
    + |\varphi_y(x_0) - x_0| \\
&\leq \frac{1}{4}|x - x_0|
    + |x_0 + y - f(x_0) - x_0| \\
&\leq \frac{1}{4}|x - x_0|
    + \frac{1}{2}r \\
&\leq \frac{1}{r}r + \frac{1}{2}r
    < r.
\end{align*}

Since $\overline{B_r(x_0)}$ is a complete metric space,
$\forall y \in B_{\frac{1}{2}r}(y_0)$ there exists an $x$ such that
$\varphi_y(x) = x$. Therefore there exists an $x$ such that $f(x) =
y$. Therefore $B_{\frac{1}{2}r}(y_0) \subset V$, so $V$ is open.

Now define $T(y) = [Df(f^{-1}(y))]^{-1}$. Pick $y, y + k \in V$. Then
there exists an $x, x + h$ such that $f(x) = y$ and $f(x + h) = y +
k$. Then
\begin{align*}
      \frac{|g(y + k) - g(y) - T k|}{|k|}
&=    \frac{|x + h - x - T k|}
           {|k|} \\
&=    \frac{|h - Tk|}
           {|k} \\
&=    \frac{|T (Df) h  - Tk|}
           {|k|} \\
&=    \frac{|T(Df h - K)|}
           {|k|} \\
&\leq \frac{|T||(Df)h - k|}
           {|k|} \\
&\leq \frac{|T||f(x + h) - f(x) - Df \cdot h|}
           {|k|}.
\end{align*}
Since $Df$ is continuous, on a small neighborhood we can guarantee
that $\det Df$ is small, and since taking an inverse matrix is a
continuous process $(Df)^{-1}$ will have bounded operator norm.

Consider
\begin{align*}
      |\varphi_y(x + h) - \varphi_y(x)|
&=    |x + h + y - f(x + h) - (x + y - f(x))| \\
&=    |h - k|
&\leq \frac{1}{4}|h|
\end{proof}
and since $|h - k| \geq ||h| - |k||$ we have
$$
     -\frac{1}{4}|h|
\leq |h| - |k|
\leq \frac{1}{4}|h|
$$
so $|k| \geq \frac{3}{4}|h|$. But this means that
$$
     \frac{|g(y + k) - g(y) - T k|}{|k|}
\leq \frac{4}{3}|T|
     \frac{|f(x + h) - f(x) - Df \cdot h|}
          {|h|}
\to  0
$$
so $g$ is differentiable. Furthermore since
$$
Dg = (\dot)^{-1} \circ Df \circ g,
$$
$Dg$ is continuous, since matrix inversion is a continuous function.

\section{Non-exact forms}
A vector field on $\mathbb{R}^2$ has the form
$$
X = P \frac{\partial}{\partial x} + Q \frac{\partial}{\partial y}.
$$
When is a vector field a gradient vector field, i.e. exact? We want
$X = \nabla f$. On a star-shaped domain, it is necessary and
sufficient that
$\frac{\partial Q}{\partial x} = \frac{\partial P}{\partial y}$ on
$\mathbb{R}^2$.

Consider $\mathbb{R}^2 \setminus \{ 0 \}$. Let
$(P, Q) = \left(\frac{-y}{x^2  + y^2}, \frac{-x}{x^2  + y^2}\right)$,
which has a singularity at the origin, and
$\frac{\partial P}{\partial y} = \frac{\partial Q}{\partial x}$.
We see that
\begin{align*}
   \int_C
     \left(
       \frac{-y}{x^2  + y^2} \dif x
     + \frac{-x}{x^2  + y^2} \dif y
     \right)
&= \int_0^{2\pi}
     (-\sin \theta)(-\sin \theta) + \cos^2 \theta \dif\theta \\
&= \int_0^{2 \pi} 1 \dif \theta = 2 \pi \neq 0
\end{align*}
so $(P, Q)$ cannot be a gradient.
