\chapter{Background}



Laboratory Information Management System



The complex needs of modern research have created several software
markets, and there are now dozens of tools for computerizing various
laboratory management and research tasks.

\section{Requirements}

\subsection{Data provenance}
In many cases, drawing conclusions about a data set relies on
metadata and information about experimental conditions that is
difficult to acquire for every trial and is not obviously
relevant at the outset, forcing researchers to backtrack and repeat
work in order to be confident in their results.


\section{Workflow design software}

Taverna
VisTrails
ReproZip

Defining, composing, and documenting complex procedures is a core
organizational need of many research groups. Workflow editors provide
users with a means of constructing executable tasks by describing how
data moves through them, typically by visually manipulating a directed
graph of processes.

Industry groups have also made several attempts to produce a
standardized data model for business processes and equipment, the
most recent of which is ISO 15926 \cite{West2009}. This model promises
a level of generality that is sufficient to enable interoperability
between businesses in different sectors and countries and relying on
large, varied sets of equipment and software. The still-growing
specification encompasses information as diverse as process
specification and refinement, structural description of organizations
and devices, component lifecycle information and more. ISO 15926's
representation format is based on semantic web technologies such as OWL, which
employs a graph model to describe semantic relationships between
entities, where each entity and relationship has an associated
hyperlink. The standard has been under development for 25 years, but
many specification documents have yet to be published and no software
implementations are currently freely available. The extreme complexity
of the model is also an impediment to adoption by end users as well as
implementation.

\section{Equipment automation tools}

LabVIEW
Zettajs

It is desirable to include
Unfortunately, relatively few LIMS vendors incorporate features for

LabVIEW's G visual programming language arguably has much in common
with some workflow editors,

\section{LIMS}
Some of these software packages provide a means of specifying
experiment procedures and many of them integrate with an electronic
lab notebook, but

Many of these tools are essentially glorified Wiki packages

\section{Electronic lab notebooks}

An electronic lab notebook (ELN) is a software tool for helping
researchers to chronicle their day-to-day investigations and
results. Several surveys of commercially available ELNs have been
published \cite{Rubacha2011, }, but the domain is still evolving
rapidly and many of these programs have begun to integrate
capabilities for experiment specification and execution.

Typically commercial products in this domain offer users compliance
with the FDA's recommendation on electronic recordkeeping \cite{FDA}

In particular, Agilent's OpenLAB suite (formerly Kalabie) offers a
notebook tool which combines data collection, storage, analysis, and
collaboration capabilities. This package is also capable of
integrating with data collected from instruments manufactured by
Agilent and some of its business partners.
 still lacks modularity,
user-customizability, version control

Since many of these programs are proprietary,



\section{Summary}

A number of software tools for automating data collection, analyzing
and comparing data sets, and interdisciplinary scientific
collaboration have emerged in recent years. Moving toward an    will
require the convergence of LIMS software with equipment automation .
Many commercial LIMS are also inadequately flexible to researchers'
rapidly changing data organization requirements