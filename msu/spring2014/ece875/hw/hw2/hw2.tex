\documentclass{article}

\title{ECE 875 - Homework \#2}
\author{Sam Boling}
\date{\today}

\usepackage{enumitem}
\usepackage{amsmath}
\usepackage{amsfonts}
\usepackage{amssymb}
\usepackage{graphicx}

\renewcommand*{\Re}{\operatorname{\mathfrak{Re}}}
\renewcommand*{\Im}{\operatorname{\mathfrak{Im}}}

\newcommand{\horline}
           {\begin{center}
              \noindent\rule{8cm}{0.4pt}
            \end{center}}

\newcommand\scalemath[2]{\scalebox{#1}{\mbox{\ensuremath{\displaystyle #2}}}}

\begin{document}

\maketitle

\section*{Problem \#1.8}
In the conduction band of a semiconductor, it has a lower valley at the center
of the Brillouin zone, and six upper valleys at the zone boundary along [100].
If the effective mass for the lower valley is $0.1m_0$ and that for the upper
valleys is $1.0m_0$, find the ratio of the effective density of states in the
upper valleys to that in the lower valley.
\horline
The effective density of states in the conduction band is given by
$$
N_{C} = 2 \left(\frac{2 \pi m_{de} kT}{h^2}\right)^{\frac{3}{2}} M_C
$$
so that the ratio of effective densities is
$$
\frac{N_{C_{upper}}}{N_{C_{lower}}} 
  = \left(\frac{m_{de_{upper}}}{m_{de_{lower}}}\right)^{\frac{3}{2}}
    \frac{M_{C{upper}}}{M_{C{lower}}}.
$$
We take $m_{de}$ to be the given effective masses since this must be the
average of the effective mass values within each valley.
Since there are 6 upper valleys at the edge of the first Brillouin zone,
there are $M_{C_{upper}} = 3$ upper minima within the zone in the conduction 
band and since the one lower valley is at the center of the Brillouin zone 
this means
$$
\frac{N_{C_{upper}}}{N_{C_{lower}}} = 
  3\left(\frac{0.1 m_0}{1.0 m_0}\right)^{3/2} 
 \approx 0.094868.
$$

\end{document}
