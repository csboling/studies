\documentclass{article}

\title{ECE 875 - Homework \#6}
\author{Sam Boling}
\date{\today}

\usepackage{enumitem}

\usepackage{amsmath}
\usepackage{mathrsfs}
\usepackage{amsfonts}
\usepackage{amssymb}

\usepackage{graphicx}
\usepackage{caption}
\usepackage{rotating}

\renewcommand*{\Re}{\operatorname{\mathfrak{Re}}}
\renewcommand*{\Im}{\operatorname{\mathfrak{Im}}}

\newcommand{\horline}
           {\begin{center}
              \noindent\rule{8cm}{0.4pt}
            \end{center}}

\newcommand\scalemath[2]{\scalebox{#1}{\mbox{\ensuremath{\displaystyle #2}}}}

\begin{document}

\maketitle

\section*{Problem \#3.9}
The photoelectric measurement gives
$$
\phi_{Bn} = 0.65 ~\mathrm{V}
$$
while the C-V line intercept gives
$$
\psi_{bi} = V_{\mathrm{intercept}} + \frac{kT}{q} =  0.5 + 0.0259  ~\mathrm{V}.
$$
But from the C-V measurement approach,
\begin{align*}
\phi_{Bn} &= \psi_{bi} + \phi_n + \frac{kT}{q} - \Delta \phi \\
          &= \psi_{bi} 
           + \frac{kT}{q}\ln\frac{N_C}{N_D}
           + \frac{kT}{q}
           - \Delta \phi \\
\phi_{Bn} - \psi_{bi} + \Delta \phi &= \frac{kT}{q}\left(\ln\frac{N_C}{N_D} + 1\right) \\
\frac{N_C}{N_D} &= \exp\left(\frac{\phi_{Bn} - \psi_{bi} + \Delta \phi}
                                  {\frac{kT}{q}}\right) \\
N_D &= N_C \exp\left(-\frac{\phi_{Bn} - \psi_{bi} + \Delta \phi}{\frac{kT}{q}}\right).
\end{align*}

Neglecting the image force lowering term $\Delta \phi$ at first,
the doping concentration can then be found as
\begin{align*}
N_D &\approx (2.8 \times 10^{19} ~\mathrm{cm}^{-3}) 
             \exp\left(-\frac{0.124 ~\mathrm{V}}{0.0259 ~\mathrm{V}}\right) \\
    &= 2.33 \times 10^{17} ~\mathrm{cm}^{-3},
\end{align*}
which gives a maximum electric field of
\begin{align*}
E_m &= \sqrt{\frac{2 q N_D}{\varepsilon_s}\left(\psi_{bi} - \frac{kT}{q}\right)} \\
    &= \sqrt{\frac{2 (1.6 \times 10^{-19} ~\mathrm{C})
                     (2.33 \times 10^{17} ~\mathrm{cm}^{-3})}
                  {(11.9 \cdot 8.85 \times 10^{-14} ~\mathrm{F}~\mathrm{cm}^{-1})}
             \left(0.5259 - 0.0259 ~\mathrm{V}\right)} \\
    &= 1.86 \times 10^{5} ~\mathrm{V} ~\mathrm{cm}^{-1}
\end{align*}
and thus an image charge lowering of
\begin{align*}
\Delta \phi &= \sqrt{\frac{q E_m}{4 \pi \varepsilon_s}} \\
            &= \sqrt{\frac{(1.6 \times 10^{-19} ~\mathrm{C})
                           (5.31 \times 10^{4} ~\mathrm{V})}
                    {(11.9 \cdot 8.85 \times 10^{-14})}} \\
            &= 0.0476 ~\mathrm{V}.
\end{align*}
Repeating the computation with this $\delta \phi$ value, and iterating this
process until $N_D$ changes by no more than $1 \times 10^{14}$ between iterations,
we get the following values:


\begin{tabular}{c | c c c} 
iteration & $\Delta \phi ~(\mathrm{V})$ & $N_D (\mathrm{cm}^{-3}$ & $E_m (\mathrm{V}~\mathrm{cm}^{-1}$ \\
1         & 0             & $2.32 \times 10^{17}$ & $1.88 \times 10^5$ \\
2         & 0.0456        & $3.69 \times 10^{16}$ & $7.49 \times 10^4$ \\
3         & 0.03          & $7.27 \times 10^{16}$ & $1.05 \times 10^5$ \\
4         & 0.036         & $5.87 \times 10^{16}$ & $9.44 \times 10^4$ \\
5         & 0.037         & $6.31 \times 10^{16}$ & $9.79 \times 10^4$ \\
6         & 0.0344        & $6.16 \times 10^{16}$ & $9.67 \times 10^4$ \\
7         & 0.0342        & $6.21 \times 10^{16}$ & $9.71 \times 10^4$ \\
8         & 0.0343        & $6.19 \times 10^{16}$ & $9.70 \times 10^4$ \\
9         & 0.0342        & $6.19 \times 10^{16}$ & $9.70 \times 10^4$
\end{tabular}

Thus we see that the donor concentration is about $N_D = 6.19 \times 10^{16} ~\mathrm{cm}^{-3}$.

\section*{Problem \#4.1}
\begin{enumerate}
  \item{To make the silicon surface intrinsic requires 
        \begin{align*}
        q\psi_s &= E_i - E_f \\
        \psi_s &= \frac{1}{q}(E_i - E_f) = \frac{kT}{q} \ln \frac{N_A}{n_i} \\
               &= (0.0259 ~\mathrm{V}) \ln \frac{5 \times 10^{17}}{9.65 \times 10^9} \\
               &= 0.46 ~\mathrm{V}
        \end{align*}
        The applied voltage to achieve this surface potential is  
        \begin{align*}
        V &= \psi_s + V_i = \psi_s + \frac{|Q_s|d}{\varepsilon_i} \\
          &= \psi_s + \frac{d}{\varepsilon_i}
                      \left|\frac{\sqrt{2}\varepsilon_s kT}{q L_D} 
                            F\right|
        \end{align*}
        and under the weak inversion condition we may approximate 
        $$
        F \approx \sqrt{\beta \psi_s} 
          = \sqrt{(38.6 ~\mathrm{V}^{-1})(0.46 ~\mathrm{V})} = 4.21
        $$
        and find the Debye length as
        \begin{align*}
        L_D &= \sqrt{\frac{\varepsilon_s}{q N_A \beta}} \\
            &= \sqrt{\frac{11.9 \cdot 8.85 \times 10^{-14} ~\mathrm{F}~\mathrm{cm}^{-1}}
                         {(1.6 \times 10^{-19} ~\mathrm{C})
                          (5 \times 10^{17} ~\mathrm{cm}^{-3})
                          (38.6 ~\mathrm{V}^{-1})}} \\
            &= 5.84 \times 10^{-7} ~\mathrm{cm}
        \end{align*}
         so
        \begin{align*}
        V &= \psi_s + \frac{d}{\varepsilon_i}
                      \left|\frac{\sqrt{2}\varepsilon_s kT}{q L_D} 
                            F\right| \\
          &= \psi_s + d\frac{kT}{q} \frac{\varepsilon_s}{\varepsilon_i}
                      \left|\frac{\sqrt{2}F}{L_D}\right| \\
          &= 0.46 ~\mathrm{V} + (10 \times 10^{-7} ~\mathrm{cm})
                    (0.0259 ~\mathrm{V})
                    \frac{11.9}{3.9} 
                    \left|\frac{\sqrt{2} \cdot 4.21}
                               {5.84 \times 10^{-7} ~\mathrm{cm}}\right| \\
          &= 1.27 ~\mathrm{V}.
        \end{align*}
       }
  \item{The silicon is in strong inversion at the threshold voltage
        \begin{align*}
          V_T &= \frac{|Q_s|}{C_i} + 2 \psi_{Bp} \\
              &= \frac{\sqrt{2 \varepsilon_s q N_A (2 \psi_{Bp})}}{C_i} + 2\psi_{Bp} \\
              &= \frac{d}{\varepsilon_i}\sqrt{2 \varepsilon_s q N_A (2 \psi_{Bp})} + 2\psi_{Bp} \\
              &= \frac{10 \times 10^{-7}}
                      {3.9 \cdot 8.85 \times 10^{-14} ~\mathrm{F}~\mathrm{cm}^{-1}} \\
              &\cdot \sqrt{2(11.9 \cdot 8.85 \times 10^{-14} ~\mathrm{F}~\mathrm{cm}^{-1})
                        (1.6 \times 10^{-19} ~\mathrm{C})} \\
              &\cdot \sqrt{
                        (5 \times 10^{17} ~\mathrm{cm}^{-3})
                        (2 \cdot 0.46 ~\mathrm{V})}
               + 2 \cdot 0.46 ~\mathrm{V} \\
              &= 2.06 ~\mathrm{V}.
        \end{align*}
       }
\end{enumerate}

\pagebreak

\section*{Problem \#4.2}
Plotted in figure \ref{fig:prob4pt2}.

\begin{sidewaysfigure}
  \centering
  \includegraphics[width=\textheight]{prob4pt2plot}
  \caption{Surface charge plot for problem \#4.2. \label{fig:prob4pt2}}
\end{sidewaysfigure}

\section*{Problem \#4.5}
$$N_A = 10^{16} ~\mathrm{cm}^{-3}, d = 10 ~\mathrm{nm}, V_G = 1.77 ~\mathrm{V}$$

$\psi_{Bp}$ is given by
\begin{align*}
\psi_{Bp} &= kT \ln \frac{N_A}{n_i} = (0.0259 ~\mathrm{V})
                                      \ln \frac{10^{16} ~\mathrm{cm}^{-3}}
                                               {9.65 \times 10^9 ~\mathrm{cm}^{-3}} 
                                    = 0.36 ~\mathrm{V},
\end{align*}
so we compute
\begin{align*}
          V_T &= \frac{|Q_s|}{C_i} + 2 \psi_{Bp} \\
              &= \frac{\sqrt{2 \varepsilon_s q N_A (2 \psi_{Bp})}}{C_i} + 2\psi_{Bp} \\
              &= \frac{d}{\varepsilon_i}\sqrt{2 \varepsilon_s q N_A (2 \psi_{Bp})} + 2\psi_{Bp} \\
              &= \frac{10 \times 10^{-7}}
                      {3.9 \cdot 8.85 \times 10^{-14} ~\mathrm{F}~\mathrm{cm}^{-1}} \\
              &\cdot \sqrt{2(11.9 \cdot 8.85 \times 10^{-14} ~\mathrm{F}~\mathrm{cm}^{-1})
                        (1.6 \times 10^{-19} ~\mathrm{C})} \\
              &\cdot \sqrt{
                        (10^{16} ~\mathrm{cm}^{-3})
                        (2 \cdot 0.36 ~\mathrm{V})}
               + 2 \cdot 0.36 ~\mathrm{V} \\
              &= 0.87 ~\mathrm{V} < V_G,
\end{align*}
so the capacitor is in strong inversion. Then we have
\begin{align*}
V_G &= \frac{|Q_s|}{C_i} + \frac{2kT}{q}\ln\frac{N_A}{n_i} \\
|Q_s| &= C_i\left(V_G - \frac{2kT}{q}\ln \frac{N_A}{n_i}\right) \\
      &= \frac{\varepsilon_i}{d}
         \left(V_G - \frac{2kT}{q}\ln\frac{N_A}{n_i}\right) \\
      &= \frac{3.9 \cdot 8.85 \times 10^{-14} ~\mathrm{F}~\mathrm{cm}^{-1}}
              {10 \times 10^{-7} ~\mathrm{cm}}
         \left(1.77 - 2 (0.0259 ~\mathrm{V})\ln\frac{10^{16}}{9.65 \times 10^9}\right) \\
      &= 7.75 \times 10^{-7} ~\mathrm{C}~\mathrm{cm}^{2}.
\end{align*}


%We can find the proportionality constant $\alpha$
%relating the expression $F\left(\beta \psi_s, \frac{n_{p0}}{p_{p0}}\right)$ in the weak
%and strong inversion regions by setting
%\begin{align*}
%\sqrt{\beta 2 \psi_{Bp}} &= \alpha \exp\left(\frac{q}{2kT} 2\psi_{Bp}\right) \\
%\sqrt{\beta \frac{2 kT}{q} \ln \frac{N_A}{n_i}} &= \alpha \exp\left(\ln \frac{N_A}{n_i}\right) \\
%\alpha &= \frac{n_i}{N_A}\sqrt{2 \ln \frac{N_A}{n_i}} \\
%       &= \frac{9.65 \times 10^{9}}{10^{16}}\sqrt{2 \ln \frac{10^{16}}{9.65 \times 10^{9}}} \\
%       &= 5.08 \times 10^{-6}.
%\end{align*}
%Then we have
%\begin{align*}
%V_G &= \frac{|Q_s|}{C_i} + \frac{2 kT}{q}\ln \frac{N_A}{n_i} \\
%    &= \frac{d}{\varepsilon_i} 
%       \frac{\sqrt{2} \varepsilon_s kT}{q L_D} 
%       F\left(\beta \psi_s, \frac{n_{p0}}{p_{p0}}\right) \\
%    &= \frac{d}{\varepsilon_i} 
%       \frac{\sqrt{2} \varepsilon_s kT}{q L_D} 
%       \alpha\frac{N_A}{n_i} \\
%    &= \frac{10 \times 10^{-7} ~\mathrm{cm}}{3.9 \varepsilon_0}
%       \frac{\sqrt{2} 11.9 \varepsilon_0 (0.0259 ~\mathrm{V})}
%            {}
%\end{align*}
%since the Debye length is
%\begin{align*}
%L_D &= \sqrt{\frac{\varepsilon_s}{q N_A \beta}} \\
%    &= \sqrt{\frac{11.9 \times 10^{-14} ~\mathrm{F}~\mathrm{cm}^{-1}}
%                  {(1.6 \times 10^{-19} ~\mathrm{C})
%                   (10^{16} ~\mathrm{cm}^{-3})
%                   (38.6 ~\mathrm{V}^{-1})}} \\
%    &= 
%\end{align*}

%The charge per unit area in the inversion region in an ideal MOS capacitor
%is given by
%\begin{align*}
%Q_s &= \pm \frac{\sqrt{2}\varepsilon_s kT}
%                {q L_D}
%           F\left(\beta \psi_s, \frac{n_{p0}}{p_{p0}}\right) \\
%    &= 
%\end{align*}
%where the Debye length is
%\begin{align*}
%        L_D &= \sqrt{\frac{\varepsilon_s}{q N_A \beta}} \\
%            &= \sqrt{\frac{11.9 \cdot 8.85 \times 10^{-14} ~\mathrm{F}~\mathrm{cm}^{-1}}
%                         {(1.6 \times 10^{-19} ~\mathrm{C})
%                          (10^{16} ~\mathrm{cm}^{-3})
%                          (38.6 ~\mathrm{V}^{-1})}} \\
%            &= 
%        \end{align*}

\pagebreak
\section*{Problem \#4.6}
The minimum capacitance on the CV curve under high-frequency condition is
\begin{align*}
C_{min} &= \frac{\varepsilon_i \varepsilon_s}{\varepsilon_s d + \varepsilon_i W_{Dm}} \\
\end{align*}
where
\begin{align*}
W_{Dm} &\approx \sqrt{\frac{4 \varepsilon_s kT \ln (N_A / n_i)}{q^2 N_A}} \\
       &=       \sqrt{\frac{4 (11.9 \cdot 8.85 \times 10^{-14} ~\mathrm{F}~\mathrm{cm}^{-1})
                              (0.0259 ~\mathrm{V})
                              \ln\left(\frac{10^{16}}{9.65 \times 10^{9}}\right)}
                           {(1.6 \times 10^{-19} ~\mathrm{C})
                            (10^{16} ~\mathrm{cm}^{-3})}} \\
       &= 3.07 \times 10^{-5} ~\mathrm{cm}
\end{align*}
so
\begin{align*}
C_{min} &= \frac{3.9 \cdot 11.9}{11.9 (8 \times 10^{-7} ~\mathrm{cm}) 
                             +   3.9  (3.07 \times 10^{-5} ~\mathrm{cm})}
           (8.85 \times 10^{-14} ~\mathrm{F}~\mathrm{cm}^{-1}) \\
        &= 3.25 \times 10^{-8} ~\mathrm{F} = 32.5 ~\mathrm{nF}.
\end{align*}
\end{document}
