\documentclass{article}

\title{ECE 875 - Homework \#6}
\author{Sam Boling}
\date{\today}

\usepackage{enumitem}

\usepackage{amsmath}
\usepackage{mathrsfs}
\usepackage{amsfonts}
\usepackage{amssymb}

\usepackage{graphicx}
\usepackage{caption}
\usepackage{rotating}

\renewcommand*{\Re}{\operatorname{\mathfrak{Re}}}
\renewcommand*{\Im}{\operatorname{\mathfrak{Im}}}

\newcommand{\horline}
           {\begin{center}
              \noindent\rule{8cm}{0.4pt}
            \end{center}}

\newcommand\scalemath[2]{\scalebox{#1}{\mbox{\ensuremath{\displaystyle #2}}}}

\begin{document}

\maketitle

\section*{Problem \#6.2}
%Z/L = 1 => I_D = 18.7 \mu A at V_D = 0.4, V_G = 3V.
%What width Z will produce 1.6 mA?
%Poly gate is 0.6 \mu m wide, S and D each diffuse 0.05 \mu m under gate.

In the most general idealized case,
\begin{align*}
I_D = \frac{Z}{L} \int_0^L |Q_n(y)|v(y)~dy,
\end{align*}
so increasing the width $Z$ by a multiplicative factor of 
$\frac{1.6 ~\mathrm{mA}}{18.7 ~\mathrm{\mu A}} = 85.56$ or to
$85.5 L$, or
$Z = 85.5 \cdot 0.6 ~\mathrm{\mu m} = 51.34 ~\mathrm{\mu m}$.

\section*{Problem \#6.3}
We have $V_G > V_T$ and $V_D < V_G - V_T$, so in the linear region the channel
conductance is given by
\begin{align*}
g_D &= \frac{dI_D}{dV_D} = \frac{Z}{L} \mu C_{ox} \left(V_G - V_T - V_D\right) \\
    &= \frac{5 ~\mathrm{\mu m}}
            {0.25 ~\mathrm{\mu m}}
            (500 ~\mathrm{cm}^{2}~\mathrm{V}^{-1}~\mathrm{s}^{-1})
            (3.45 \times 10^{-7} ~\mathrm{F}~\mathrm{cm}^{-2}) 
            (1 - 0.5 - 0.1 ~\mathrm{V}) \\
    &= 1.38 ~\mathrm{mS}.
\end{align*}

\section*{Problem \#6.4}
%L = 10 \mu m, I_D = 1 mA, gate current 1 \muA

%find length to reduce gate current to 10^-6 I_D

Assuming the gate contact is ohmic, the gate current is given
by $V_{G} = I_{G} R_{G}$ or $I_{G} = \frac{V_{G}}{R_{G}}$. The 
resistance of the gate is given by
\begin{align*}
R_{G} = \frac{\rho \cdot ~\mathrm{depth}}{\mathrm{width} \cdot \mathrm{height}},
\end{align*}
where $\rho$ is the gate conductivity and the depth is taken in the
direction of current flow and is therefore in this case the gate 
thickness. Therefore
\begin{align*}
R_{G} = \frac{\rho h}{L Z}
\end{align*}
so the gate current is given by
\begin{align*}
I_{G} = \frac{V_{G} L Z}{\rho h}.
\end{align*}
The ratio of this current to the drain current is then
\begin{align*}
\frac{I_{G}}{I_{D}} &= \frac{V_{G} L Z}
                            {\rho h \frac{Z}{L} \mu C_{ox} (V_G - V_T - \frac{V_D}{2})V_D}.
\end{align*}
The ratio of gate current to drain current under the original condition is 
$\frac{1 ~\mathrm{\mu A}}{1 ~\mathrm{mA}} = 10^{-3}$ and in the desired condition is
$10^{-6}$, so we have
\begin{align*}
10^{-3} = \frac{L_1^2}{\alpha}, 10^{-6} = \frac{L_2^2}{\alpha}, \\
\alpha = 10^{3}L_1^2, \\
10^{-6}10^{3} L_1^2 = L_2^2 \\
L_2 = \sqrt{10^-3} L_1 = 0.316 ~\mathrm{\mu m},
\end{align*}
where 
$$
\alpha = \frac{V_D}{V_G} \rho h \frac{Z}{L} \mu C_{ox} (V_G - V_T - \frac{V_D}{2})
$$

\section*{Problem \#6.5}
For an ideal MOSFET, the current $I_D$ does not change in saturation and thus
is equal to $I_{Dsat}$. We then have that
\begin{align*}
I_{Dsat} &= \frac{Z}{2ML}\mu_n C_{ox}(V_G - V_T)^2, \\
50 ~\mathrm{\mu A} &= \alpha(1 - V_T)^2, \\
200  ~\mathrm{\mu A} &= \alpha (3 - V_T)^2, \\
\end{align*}
where $\alpha = \frac{Z}{2ML}\mu_n C_{ox}$, so
\begin{align*}
\alpha &= \frac{50 ~\mathrm{\mu A}}{(1 - V_T)^2}, \\
\frac{50 ~\mathrm{\mu A}}{(1 - V_T)^2} (3 - V_T)^2 &= 200 ~\mathrm{\mu A}, \\
\left(\frac{3 - V_T}{1 - V_T}\right)^2 &= \frac{200}{50},  \\
\pm\frac{3 - V_T}{1 - V_T} &= \sqrt{4} = 2.
\end{align*}
The positive square root yields a negative $V_T$ which cannot be the case 
since the current is increasing with $V_G$ and this is therefore an NMOS. The
negative square root gives
\begin{align*}
V_T - 3 &= 2(1 - V_T), \\
V_T &= \frac{5}{3} ~\mathrm{V} \approx 1.666 ~\mathrm{V}.
\end{align*}

\section*{Problem \#6.7}
%oxide thickness = 15 nm. Find N_A to give V_T = 0.5V for n+ poly gate.

We have that
\begin{align*}
V_T &= V_{FB} - 2\psi_{B} + \frac{\sqrt{2 \varepsilon_s q N_A (2\psi_B)}}{C_{ox}} \\
    &= V_{FB}
     - 2\frac{kT}{q} \ln \frac{N_A}{n_i}
     + \frac{\sqrt{2 \varepsilon_s q N_A (2 \psi_B)}}{C_{ox}}
\end{align*}
where
\begin{align*}
V_{FB} = \psi_{ms} - \frac{Q_{insulator}}{C_{ox}}
\end{align*}
and
\begin{align*}
Q_{insulator} &= Q_m + Q_{ot} + Q_f + Q_{it} = 0
\end{align*}
under the stated assumptions. $N_A$ to provide the given $V_T$ can then be
found iteratively as follows:

\begin{tabular}{c | c c c c}
$N_A$       & $10^{16}$ & $10^{17}$ & $10^{16.9}$ & $\mathrm{cm}^{-3}$\\
\hline
$\psi_{ms}$ & 0.92      & 0.99      & 0.98         & $\mathrm{V}$ \\
$V_T$       & 0.011     & 0.577     & 0.49         & $\mathrm{V}$
\end{tabular}

so the desired doping concentration is about 
$10^{16.9} = 7.94 \times 10^{16} ~\mathrm{cm}^{-3}$.

\end{document}
