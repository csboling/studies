\documentclass{article}

\title{ECE 875 - Homework \#4}
\author{Sam Boling}
\date{\today}

\usepackage{enumitem}
\usepackage{amsmath}
\usepackage{amsfonts}
\usepackage{amssymb}
\usepackage{graphicx}

\renewcommand*{\Re}{\operatorname{\mathfrak{Re}}}
\renewcommand*{\Im}{\operatorname{\mathfrak{Im}}}

\newcommand{\horline}
           {\begin{center}
              \noindent\rule{8cm}{0.4pt}
            \end{center}}

\newcommand\scalemath[2]{\scalebox{#1}{\mbox{\ensuremath{\displaystyle #2}}}}

\begin{document}

\maketitle

\section*{Problem \#2.7}
For a $p$-$n$ junction with the $p$ side doped to $1 \times 10^{17}$ cm$^{-3}$,
the $n$-side doped to $1 \times 10^{19}$ cm$^{-3}$, and a reverse bias of 
$-2$ V, calculate the generation-recombination current density, assuming that 
the effective lifetime is $1 \times 10^{-5}$ s.
\horline
Near the junction, that is, in the depletion region, majority carrier 
concentrations are nearly zero under reverse bias since the applied
electric field acts to oppose the built-in electric field. Therefore 
$np < n_i^2$ so the generation-recombination current density is given by
$$
J = q U W_D = q W_D \left( -\frac{n_i}{\tau_g} \right)
$$
from equation 98 in chapter 1 where $\tau_g$ is the effective lifetime.
$$
J = -q \sqrt{\frac{2 \varepsilon}{q}
             \frac{N_A + N_D}{N_A N_D} (\psi_{bi} - V_{bias})} 
       \frac{n_i}{\tau_g}.
$$
But the builtin potential in nondegenerate semiconductors is given by
\begin{align*}
\psi_{bi} &\approx \frac{kT}{q} \ln \left(\frac{N_D N_A}{n_i^2}\right) \\
          &\approx \frac{4.14 \times 10^{-21}}{1.60 \times 10^{-19}}
                  \ln\left(\frac{10^{36}}{(9.65 \times 10^9)^2}\right) \\
          &\approx 0.955 \mathrm{V}
\end{align*}
in silicon at 300K.

Then
\begin{align*}
W_D &= \sqrt{\frac{2\varepsilon}{q} 
            \frac{N_A + N_D}{N_A N_D}
            (\psi_{bi} - V_{bias})} \\
    &\approx
       \sqrt{\frac{2 \cdot 11.9 \cdot 8.85 \times 10^{-14}}
                  {1.6 \times 10^{-19}}
             \frac{1.01 \times 10^{19}}
                  {10^{36}}
             (0.955 + 2)} \\
    &\approx 1.98 \times 10^{-5} \mathrm{cm}, \\
U   &= -\frac{n_i}{\tau_g} \approx -\frac{9.65 \times 10^{9}}{10^{-5}} \\
    &\approx -9.65 \times 10^{14} \mathrm{cm}^{-3} \mathrm{s}^{-1},
\end{align*}
so
\begin{align*}
J &= q U W_D 
   \approx (1.6 \times 10^{-19})(1.98 \times 10^{-5})(-9.65 \times 10^{14}) \\
  &\approx -3.06 \times 10^{-9} \frac{\mathrm{A}}{\mathrm{cm}^2}.
\end{align*}

\end{document}
