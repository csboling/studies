\documentclass{article}

\title{ECE 875 - Homework \#9}
\author{Sam Boling}
\date{\today}

\usepackage{enumitem}

\usepackage{amsmath}
\usepackage{mathrsfs}
\usepackage{amsfonts}
\usepackage{amssymb}

\usepackage{graphicx}
\usepackage{caption}
\usepackage{rotating}

\renewcommand*{\Re}{\operatorname{\mathfrak{Re}}}
\renewcommand*{\Im}{\operatorname{\mathfrak{Im}}}

\newcommand{\horline}
           {\begin{center}
              \noindent\rule{8cm}{0.4pt}
            \end{center}}

\newcommand\scalemath[2]{\scalebox{#1}{\mbox{\ensuremath{\displaystyle #2}}}}

\begin{document}

\maketitle

\section*{Problem \#6.8}

We have
\begin{align*}
V_T &= V_{FB} 
     + 2\psi_B 
     + \frac{\sqrt{2 \varepsilon_s q N_A (2 \psi_B)}}{C_{ox}} \\
    &= \left(\phi_{ms} - \frac{Q_f}{C_{ox}}\right)       
     + 2\psi_B
     + \frac{\sqrt{4 \varepsilon_s q N_A \psi_{B}}}{C_{ox}}, \\
C_{ox} (V_T  - \phi_{ms} - 2\psi_{B}) &= 2\sqrt{\varepsilon_s q N_A kT \psi_B} - Q_f, \\
C_{ox} &= \frac{2 \sqrt{\varepsilon_s q N_A \psi_B} - Q_f}
               {V_T - \phi_{ms} - 2\psi_B},
\end{align*}
where
\begin{align*}
\phi_{ms} &= -0.98.
\end{align*}
Then since
\begin{align*}
\psi_B &= \frac{kT}{q} \ln \frac{N_A}{n_i} \\
       &= 0.0259 \ln \frac{10^{17} ~\mathrm{cm}^{-3}}{9.65 \times 10^{9} ~\mathrm{cm}^{-3}} \\
       &= 0.4184 ~\mathrm{V}
\end{align*}
this gives
\begin{align*}
C_{ox} &= \frac{2 \sqrt{\varepsilon_s q N_A \psi_B} - Q_f}
               {V_T - \phi_{ms} - 2\psi_B} \\
       &= \frac{2 \sqrt{(11.9 \cdot 8.85 \times 10^{-14} ~\mathrm{F}~\mathrm{cm}^{-1})
                        (1.6 \times 10^{-19} ~\mathrm{C})
                        (10^{17} ~\mathrm{cm}^{-3})
                        (0.4184 ~\mathrm{V})} 
                - (1.6 \times 10^{-19} ~\mathrm{C})
                  (10^{11})}
                       {20 ~\mathrm{V} - (-0.98)~\mathrm{V} - 2(0.4184 ~\mathrm{V})} \\
       &= 7.54 \mathrm{nF}
\end{align*}
and thus an oxide thickness of
\begin{align*}
d &= \frac{\varepsilon_{i}}{C_{ox}} \\
  &= \frac{3.9 \cdot 8.85 \times 10^{-14} ~\mathrm{F}~\mathrm{cm}^{-1}}
          {7.54 \times 10^{-9} ~\mathrm{F}} \\
  &= 0.458 ~\mathrm{\mu m}.
\end{align*}


\section*{Problem \#6.10}
%\Delta V_T = 1 when V_{BS} = -1. \Delta V_T when V_{BS} = -3? q\psi_B = 0.5.

We have that
\begin{align*}
\Delta V_T &= V_T(V_{BS}) - V_T(0) = \frac{\sqrt{2 \varepsilon_s q N_A}}{C_{ox}}
                                    (\sqrt{2 \psi_B - V_{BS}} - \sqrt{2 \psi_B}) \\
           &= \frac{1}{C_{ox}}
              (\sqrt{\varepsilon_s N_A (4q\psi_B - 2q V_{BS})}
            - \sqrt{4\varepsilon_s N_A q \psi_B})
\end{align*}
so
\begin{align*}
\Delta V_T &= \frac{\sqrt{\varepsilon_s N_A}}{C_{ox}}
              (\sqrt{(4q\psi_B - 2q V_{BS})}
            - \sqrt{4q \psi_B}), \\
\frac{\sqrt{\varepsilon_s N_A}}{C_{ox}} &=
    \frac{1}{\sqrt{2 ~\mathrm{eV} + 2 ~\mathrm{eV}} - \sqrt{2 ~\mathrm{eV}}} \\
&=  \frac{1}{2\sqrt{q} - \sqrt{2q} ~\sqrt{\mathrm{V}}} \\
&=  4.27 \times 10^{9} ~\mathrm{eV}^{-1/2}.
\end{align*}
Then at $V_{BS} = -3$ we have
\begin{align*}
  \Delta V_T &= \frac{\sqrt{\varepsilon_s N_A}}{C_{ox}}
                (\sqrt{4 q \psi_B - 2q V_{BS}} - \sqrt{4 q \psi_B}) \\
             &= (4.27 \times 10^{9} ~\mathrm{eV}^{-1/2})
                \sqrt{1.6 \times 10^{-19} ~\mathrm{C}}
                (\sqrt{2 + 6 ~\mathrm{V}}
               - 2\sqrt{0.5 ~\mathrm{V}}) \\
             &= 2.42 ~\mathrm{V}.
\end{align*}

\pagebreak

\section*{Problem \#6.11}
%N_A = 10^{17}, d = 5 nm, V_T = 0.5, subthresh swing 100mV/dec, Id = .1\muA at V_T.
%Find reverse substrate bias to reduce leakage current at V_G = 0 to 10^{-13} A.
The given subthreshold swing means that over the 500 mV from $V_T$ to $0$, $I_D$ will
decrease from $10^{-7} ~\mathrm{A}$ to 
$10^{-7 - \frac{500 ~\mathrm{mV}}{100 ~\mathrm{mV}}} = 10^{-12} ~\mathrm{A}$ when 
$V_G = 0$. Thus to decrease this current to $10^{-13} ~\mathrm{A}$ we wish to increase
the threshold voltage by 0.1 V, or $\Delta V_T = 0.1.$ Then
\begin{align*}
\Delta V_T &= \frac{\sqrt{2 \varepsilon_s q N_A}}{C_{ox}}
              (\sqrt{2 \psi_B - V_{BS}} - \sqrt{2 \psi_B}) \\
\left(\Delta V_T \frac{C_{ox}}{\sqrt{2 \varepsilon_s q N_A}} + \sqrt{2 \psi_B}\right)^2 
  &= 2 \psi_B - V_{BS} \\
V_{BS} &= 2 \psi_B 
        - \left(\Delta V_T \frac{C_{ox}}{\sqrt{2 \varepsilon_s q N_A}} + \sqrt{2 \psi_B}\right)^2,
\end{align*}
where
\begin{align*}
\psi_B &= \frac{kT}{q} \ln \frac{N_A}{n_i} \\
       &= (0.0259 ~\mathrm{V}) \ln \frac{10^{17}}{9.65 \times 10^{9}} \\
       &= 0.418 ~\mathrm{V}
\end{align*}
and
\begin{align*}
C_{ox} &= \frac{\varepsilon_i}{d} \\
       &= \frac{3.9 \cdot 8.85 \times 10^{-14} ~\mathrm{F}~\mathrm{cm}^{-1}}
               {5 \times 10^{-7} ~\mathrm{cm}} \\
       &= 6.9 \times 10^{-7} ~\mathrm{F}
\end{align*}
so
\begin{align*}
V_{BS} &= 2 \psi_B 
        - \left(\Delta V_T \frac{C_{ox}}{\sqrt{2 \varepsilon_s q N_A}} + \sqrt{2 \psi_B}\right)^2 \\
       &= 0.836
        - \left((0.1 \mathrm{V})\frac{6.9 \times 10^{-7} ~\mathrm{F}}
                                     {\sqrt{2(11.9 \cdot 8.85 \times 10^{-14} ~\mathrm{F}~\mathrm{cm}^{-1})
                                             (1.6 \times 10^{-19} ~\mathrm{C})
                                             (10^{17} ~\mathrm{cm}^{-1})}} 
                                    + \sqrt{0.836 ~\mathrm{V}}\right)^2 \\
       &= -0.829 ~\mathrm{V}.
\end{align*} 

\pagebreak

\section*{Problem \#6.13}
%10 nm oxide, N_A = 10^{17}. Find subthresh swing.
The subthreshold swing is given by
\begin{align*}
S &= (\ln 10) \frac{kT}{q} \left(\frac{C_{ox} + C_D}{C_{ox}}\right) 
\end{align*}
We have that
\begin{align*}
C_{ox} &= \frac{\varepsilon_i}{d} \\
       &= \frac{3.9 \cdot 8.85 \times 10^{-14} ~\mathrm{F}~\mathrm{cm}^{-1}}
               {10 \times 10^{-7} ~\mathrm{cm}} \\
       &= 3.45 \times 10^{-7} ~\mathrm{F}
\end{align*}
and since the depletion width is given by
\begin{align*}
W_D &= \sqrt{\frac{2\varepsilon_s \psi_B}{qN_A}} \\
    &= \sqrt{\frac{2\varepsilon_s \frac{kT}{q} \ln \frac{N_A}{n_i}}{qN_A}} \\
    &= \sqrt{\frac{2(11.9 \cdot 8.85 \times 10^{-14} ~\mathrm{F}~\mathrm{cm}^{-1})
                    (0.0259 ~\mathrm{V})\ln\frac{10^{17}}{9.65 \times 10^9}}
                  {(1.6 \times 10^{-19} ~\mathrm{C})
                   (10^{17} ~\mathrm{cm}^{-3})}} \\
    &= 7.42 \times 10^{-6} ~\mathrm{cm},
\end{align*}
we have
\begin{align*}
C_D &= \frac{\varepsilon_s}{W_D} \\
    &= \frac{11.9 \cdot 8.85 \times 10{-14} ~\mathrm{F}~\mathrm{cm}^{-1}}
            {7.42 \times 10^{-6} ~\mathrm{cm}} \\
    &= 1.42 \times 10^{-7} ~\mathrm{F}.
\end{align*}
Thus
\begin{align*}
S &= (\ln 10) \frac{kT}{q} \left(\frac{C_{ox} + C_D}{C_{ox}}\right) \\
  &= (\ln 10) (0.0259 ~\mathrm{V}) \frac{3.45 \times 10^{-7} + 1.12 \times 10^{-7}}{3.45 \times 10^{-7}} \\
  &= 84 ~\mathrm{mV}/~\mathrm{decade}.
\end{align*}


\section*{Problem \#6.18}
% Find Id / Z for two mosfets in saturation
% L = 1\mu, d = 10n, V_D = 5V, Id/Z = 500\muA / \mum. K = 5.

% L, W, d, junc depth shrunk by K. doping ^ by K, all voltages v by K
% thus junction dep width v by about K.
Under velocity saturation condition we have initially
\begin{align*}
\left(\frac{I_{D_{sat}}}{Z}\right)_0 &= (V_G - V_T)C_{ox} v_s \\
                                     &= (V_G - V_T)\frac{\varepsilon_i}{d} v_s.
\end{align*}
For constant voltage scaling we reduce $d$, $Z$ and $L$ dimensions by 
a factor of $K$ while keeping $V_G$ and $V_T$ constant to find
\begin{align*}
\frac{I_{D_{sat}}}{\frac{Z}{K}} &= (V_G - V_T)\frac{\varepsilon_i}{\frac{d}{K}} v_s, \\
\frac{\left(\frac{I_{D_{sat}}}{Z}\right)}
     {\left(\frac{I_{D_{sat}}}{Z}\right)_0} 
 &= \frac{K(V_G - V_T)}{V_G - V_T} \frac{\frac{\varepsilon_i}{\frac{d}{K}}}
                                     {\frac{\varepsilon_i}{d}} \\
 &= \frac{Kd}{\frac{d}{K}} = K^2,
\end{align*}
so
$$
\frac{I_{D_{sat}}}{Z} = K^2\left(\frac{I_{D_{sat}}}{Z}\right)_0 
                      = 25 \cdot 500 ~\mathrm{\mu A} / ~\mathrm{\mu m} 
                      = 12.5 ~\mathrm{m A} / ~\mathrm{\mu m}.
$$

For constant field scaling we additionally scale $V_G$ and $V_T$ by $K$
so that
\begin{align*}
\frac{I_{D_{sat}}}{\frac{Z}{K}} &= \frac{1}{K}(V_G - V_T)\frac{\varepsilon_i}{\frac{d}{K}} v_s,
\frac{\frac{I_{D_{sat}}}{Z}}{\left(\frac{I_{D_{sat}}}{Z}\right)_0}
  &= \frac{K\frac{1}{K}(V_G - V_T) \frac{\varepsilon_i}{\frac{d}{K}}}
          {(V_G - V_T) \frac{\varepsilon_i}{d}} \\
  &= K,
\end{align*}
so
$$
\frac{I_{D_{sat}}}{Z} = K\left(\frac{I_{D_{sat}}}{Z}\right)_0 
                      = 5 \cdot 500 ~\mathrm{\mu A} / ~\mathrm{\mu m}
                      = 2.5 ~\mathrm{mA} / ~\mathrm{\mu m}.
$$

\pagebreak

\section*{Problem \#14.1}
Given two measurements
$$
V_1(T) = \frac{E_g(0)}{q} + \frac{kT}{q} \ln \left(\frac{I_1}{C_2 T^{C_3}})\right)
$$
and
$$
V_2(T) = \frac{E_g(0)}{q} + \frac{kT}{q} \ln \left(\frac{I_2}{C_2 T^{C_3}})\right)
$$
we see that
\begin{align*}
\Delta V(T) = V_1(T) - V_2(T) &=
  \frac{kT}{q}\left[\ln \left(\frac{I_1}{C_2 T^{C_3}}\right)
                  - \ln \left(\frac{I_2}{C_2 T^{C_3}}\right)\right] \\
  &= \frac{kT}{q} \ln \frac{I_1 C_2 T^{C_3}}{I_2 C_2 T^{C_3}} \\
  &= \frac{kT}{q} \ln \frac{I_1}{I_2}.
\end{align*}

\section*{Problem \#14.3}
% doping 10^20, 25C
We have that
\begin{align*}
  G &= \frac{1}{S}\frac{\Delta R}{R} = [1 + 2\nu + P_z]
\end{align*}
and thus
\begin{align*}
P_z &= G - 1 - 2\nu,
\end{align*}
so
\begin{align*}
C_p &= \frac{P_z}{Y} &= \frac{G - 1 - 2\nu}{Y}.
\end{align*}
For the given doping and temperature we have $G \approx 60$, and
in the [100] plane we have $Y = 130 ~\mathrm{GPa}$ and $\nu = 0.28$ so
\begin{align*}
C_p &= \frac{60 - 1 - 0.56}{130 \times 10^{9} ~\mathrm{Pa}} \\
    &= 4.495 \times 10^{-10} ~\mathrm{Pa}^{-1}.
\end{align*}

\pagebreak 

\section*{Problem \#14.8}
% \phi_sol = 5.30, \psi_i = 0.3, \psi_sol = 0.2, L = 1\mu, W=10\mu,
% N_A = 5e16, \mu_n = 800, C_i = 3.45e-7, V_G = 5 (saturation)
We have that
\begin{align*}
I_{sat} &= \frac{\mu C_i W}{2L}(V_G - V_T)^2
\end{align*}
where
\begin{align*}
V_T &= V_{FB} + 2 \psi_B + \frac{\sqrt{2\varepsilon_s q N_A (2 \psi_B)}}{C_i}.
\end{align*}
But
\begin{align*}
\psi_B &= \frac{kT}{q} \ln \frac{N_A}{n_i} \\
       &= (0.0259 ~\mathrm{V})
          \ln \frac{5 \times 10^{16}}{9.65 \times 10^9} \\
       &= 0.4 ~\mathrm{V}
\end{align*}
and
\begin{align*}
V_{FB} &= \phi_{sol} - \phi_s + \psi_i - \psi_{sol} \\
       &= 5.30       - 5.2    + 0.3    - 0.2 \\
       &= 0.2 ~\mathrm{V}
\end{align*}
since
\begin{align*}
\phi_s &= \chi + \frac{E_g}{q} - \frac{E_F - E_i}{q} \\
       &= \chi + \frac{E_g}{q} - \frac{kT}{q} \ln \frac{N_C}{n_i} \\
       &= 4.05 + 1.12 - (0.0259) \ln \frac{2.8 \times 10^9}{9.65 \times 10^9} \\
       &= 5.2 ~\mathrm{V} 
\end{align*}
so
\begin{align*}
V_T &= V_{FB} + 2 \psi_B + \frac{\sqrt{2\varepsilon_s q N_A (2 \psi_B)}}{C_i} \\
    &= 0.2 + 0.8 + \frac{\sqrt{2(11.9 \cdot 8.85 \times 10^{-14} ~\mathrm{F}~\mathrm{cm}^{-1})
                                (1.6 \times 10^{-19} ~\mathrm{C})
                                (5 \times 10^{16} ~\mathrm{cm}^{-3})
                                (0.8 ~\mathrm{V})}}
                        {3.45 \times 10^{-7} ~\mathrm{F}} \\
    &= 1.337 ~\mathrm{V}
\end{align*}
and thus
\begin{align*}
I_{sat} &= \frac{\mu C_i W}{2L}(V_G - V_T)^2 \\
        &= \frac{(800 ~\mathrm{cm}^2 ~\mathrm{V}^{-1} ~\mathrm{s}^{-1})
                 (3.45 \times 10^{-7} ~\mathrm{F})
                 (10 \times 10^{-4}~\mathrm{cm})
                }
                {2(10^{-4} ~\mathrm{cm})}
           (5 - 1.337~\mathrm{V})^2 \\
         &= 18.5 ~\mathrm{mA}.
\end{align*}

\end{document}
