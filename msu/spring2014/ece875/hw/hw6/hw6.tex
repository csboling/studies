\documentclass{article}

\title{ECE 875 - Homework \#6}
\author{Sam Boling}
\date{\today}

\usepackage{enumitem}

\usepackage{amsmath}
\usepackage{mathrsfs}
\usepackage{amsfonts}
\usepackage{amssymb}

\usepackage{graphicx}
\usepackage{caption}
\usepackage{rotating}

\renewcommand*{\Re}{\operatorname{\mathfrak{Re}}}
\renewcommand*{\Im}{\operatorname{\mathfrak{Im}}}

\newcommand{\horline}
           {\begin{center}
              \noindent\rule{8cm}{0.4pt}
            \end{center}}

\newcommand\scalemath[2]{\scalebox{#1}{\mbox{\ensuremath{\displaystyle #2}}}}

\begin{document}

\maketitle

\section*{Problem \#2.4}

Mark the point 0.2 $\mathrm{\mu m}$ from the junction as $x = w$, the given 
doping concentrations as $N_{D2} = 10^{16}$ and $N_{D3} = 10^{15}$, and note that
\begin{align*}
E(w)      &= E(0) + \frac{q N_{D2} w}{\varepsilon_s}, \\
E(W_{dn}) &= E(w) + \frac{q N_{D3} (W_{dn} - w)}{\varepsilon_s}
\end{align*}
so
$$
E(0) = -\frac{q}{\varepsilon_s}(N_{D2} w + N_{D3} (W_{dn} - w)).
$$
Next, we have that
$$
\psi_{bi} = -\int_{-W_{dp}}^{W_{dn}} E(x) ~dx 
          \approx -\int_{0}^{W_{dn}} E(x) ~dx
$$
since the junction is one-sided, and since the shape of the electric field is
known we can compute this integral geometrically as the area of two triangles
and a rectangle:
\begin{align*}
\psi_{bi} &= -\left\{\frac{1}{2}(W_{dn} - w)E(w) 
                  + w E(w) 
                  + \frac{1}{2} w (E(0) - E(w))\right\} \\
          &= -\left\{\frac{1}{2} W_{dn} E(w) 
                  - \frac{1}{2} w E(w) 
                  + w E(w) 
                  + \frac{1}{2} w E(0)
                  - \frac{1}{2} w E(w)\right\} \\
          &= -\frac{1}{2}(W_{dn} E(w) + w E(0)).
\end{align*}

Next, we note that the energy $q \psi_{bi}$ is the size of the gap between 
the conduction band in the second (less-doped) $n$-type region and the 
conduction band in the $p^{+}$ region. From the band diagram it is observed 
that this can also be expressed as $q\psi_{bi} = q\psi_{12} - q\psi_{23}$, 
where we have labelled the $p^{+}$ region as region 1, the first $n$-type 
region as region 2, and the second $n$-type region as region 3, and used 
q $\psi_{ab}$ to denote the energy gap between the conduction band in region
$a$ and region $b$. Next we observe that 
$$
q\psi_{12} = (E_{i1} - E_{F}) + (E_{F} - E_{i2}),
q\psi_{23} = (E_{i2} - E_{F}) - (E_{i3} - E_{F}).
$$
But since region 1 is a heavily doped $p^{+}$ region, this means
$$
E_{i1} - E_{F} \approx E_{i1} - E_{V} 
               \approx \frac{1}{2} E_{g}
               \approx 0.66 ~\mathrm{eV}.
$$
Therefore
$$
q\psi_{12} = \frac{1}{2} E_{g} + kT \ln \frac{N_{D1}}{n_i}
           \approx 0.66 ~\mathrm{eV}
          + (0.259 ~\mathrm{eV}) \ln \frac{10^{16}}{9.65 \times 10^{9}}
           \approx 4.25 ~\mathrm{eV}
$$
and
$$
q\psi_{23} = kT \ln \frac{N_{D1}}{n_i} 
           - kT \ln \frac{N_{D2}}{n_i}
           \approx
             (0.259 ~\mathrm{eV})\left(\ln\frac{10^{16}}{9.65 \times 10^{9}}
                                     - \ln\frac{10^{15}}{9.65 \times 10^{9}}\right) 
           \approx 0.60 ~\mathrm{eV}
$$
so
$$
q \psi_{bi} = q\psi_{12} - q\psi_{23} \approx 3.65 ~\mathrm{eV}
$$
so
$$
\psi_{bi} = 3.65 ~\mathrm{V}.
$$
Then we can combine the equations to see
\begin{align*}
\psi_{bi} &= -\frac{1}{2}(W_{dn} E(w) + w E(0)) \\
          &= -\frac{1}{2}\left[W_{dn}\left(E(0) + \frac{q N_{D2} w}{\varepsilon_s}\right)
                             + w E(0)\right] \\
          &= -\frac{1}{2}\left[W_{dn}\left(\frac{q N_{D2} w}{\varepsilon_s}
                             - \frac{q}{\varepsilon_s}(N_{D2} w + N_{D3}(W_{dn} - w))\right)
                             - \frac{w q}{\varepsilon_s}(N_{D2} w + N_{D3}(W_{dn} - w))\right] \\
          &= -\frac{q}{2\varepsilon_s}\left[N_{D2} W_{dn} w
                                          - N_{D2} W_{dn} w
                                          - N_{D3} W_{dn}^2
                                          + N_{D3} W_{dn} w
                                          - N_{D2} w^2 
                                          - N_{D3} W_{dn} w
                                          + N_{D3} w^2\right] \\
          &= -\frac{q}{2\varepsilon_s}\left[(N_{D3} - N_{D2}) w^2 - N_{D3} W_{dn}^2\right] \\
\frac{2 \varepsilon_s \psi_{bi}}{q} &= (N_{D2} - N_{D3}) w^2 + N_{D3} W_{dn}^2 \\
W_{dn} &= \sqrt{\frac{2 \varepsilon_s \psi_{bi}}{q N_{D3}} 
                + \left(1 - \frac{N_{D2}}{N_{D3}}\right) w^2} \\
       &\approx \sqrt{\frac{2 (11.9 \cdot 8.85 \times 10^{-14} ~\mathrm{F}~\mathrm{cm}^{-1})
                              (3.65 ~\mathrm{V})}
                           {(1.6 \times 10^{-19} ~\mathrm{C})
                            (10^{15} ~\mathrm{cm}^{-3})}
                     + (1 - 10)(0.2 \times 10^{-4} ~\mathrm{cm})^2} \\
       &\approx 2.27 ~\mathrm{\mu m}
\end{align*}
and so
\begin{align*}
E(0) &= -\frac{q}{\varepsilon_s} (N_{D2} w + N_{D3} (W_{dn} - w)) \\
     &= -\frac{1.6 \times 10^{-19} ~\mathrm{C}}
              {8.85 \times 10^{-14} ~\mathrm{F}~\mathrm{cm}^{-1}}
              ((10^{16} ~\mathrm{cm}^{-3})(0.2 \times 10^{-4} ~\mathrm{cm}) \\
           &+  (10^{15} ~\mathrm{cm}^{-3})
               ((2.27 \times 10^{-4} ~\mathrm{cm}) - (0.2 \times 10^{-4} ~\mathrm{cm}))) \\
     &\approx -7.36 \times 10^{5} \frac{\mathrm{V}}{\mathrm{m}}.
\end{align*}

\section*{Problem \#3.1}
Note that
\begin{align*}
q \psi_{bi} &= q \phi_{Bn} - q \phi_n = q \phi_{Bn} - (E_C - E_F) \\
            &= q \phi_{Bn} - kT \ln\frac{N_D}{n_i} 
\end{align*}
so
$$
\psi_{bi} = \phi_{Bn} - \frac{kT}{q} \ln\frac{N_C}{N_D}
$$
and since
$$
W_D = \sqrt{\frac{2 \varepsilon_s}{q N_D} 
            \left(\psi_{bi} - \frac{kT}{q}\right)},
$$
this means the quantities required to plot the band diagrams are as given in 
the following table.
                              
\begin{tabular}{c | c c}
$N_D ~(\mathrm{cm}^{-3})$ & $\psi_{bi} ~(\mathrm{V})$ & $W_D ~(\mathrm{cm})$ \\
\hline
$10^{15}$ & 0.64 & $9.36 \times 10^{-5}$ \\
$10^{17}$ & 0.76 & $1.02 \times 10^{-5}$ \\
$10^{18}$ & 0.82 & $3.37 \times 10^{-6}$
\end{tabular}

This results in the energy band diagrams in figure \ref{fig:prob3-1}.
Note that when the doping concentration exceeds the material's effective
mass density of states the conduction band in the semiconductor falls to
an energy level below the Fermi level of the junction.

\begin{sidewaysfigure}
  \centering
  \includegraphics[width=\textheight]{prob3-1-plots}
  \caption{Energy band diagrams for problem \#3.1. \label{fig:prob3-1}}
\end{sidewaysfigure}

\section*{Problem \#3.8}
\begin{enumerate}
  \item{The built-in potential is given by the intercept of the line on the 
        plot so that $\psi_{bi} = V_{F} \approx .6 ~\mathrm{V}$ in this case.
        The slope of the line is approximately 
        $2.3 \times 10^{-10} ~\mathrm{cm}^{4} ~\mathrm{pF}^{-2} ~\mathrm{V}^{-1}
        = 2.3 \times 10^{-10}\cdot (10^{12})^2 ~\mathrm{cm}^4 ~\mathrm{F}^{-2} ~\mathrm{V}^{-1}$ so the doping
        concentration can be found as
        \begin{align*}
        N_D &= \frac{2}{q \varepsilon_s}\left[
               -\frac{1}{\frac{d}{dV}\left(\frac{1}{C^2}\right)}\right] \\
            &= \frac{2}{(1.6 \times 10^{-19} ~\mathrm{C})
                        (12.9 \cdot 8.85 \times 10^{-14} ~\mathrm{F}~\mathrm{cm}^{-1})}
               \frac{1}{2.3 \times 10^{14} ~\mathrm{cm}^4 ~\mathrm{F}^{-2}~\mathrm{V}^{-1}} \\
            &= 4.76 \times 10^{16} ~\mathrm{cm}^{-3} 
       \end{align*}
       and thus
       \begin{align*}
         \phi_{n} &= \frac{E_{C} - E_{F}}{q} = \frac{kT}{q} \ln \frac{N_C}{N_D} \\
                  &= (0.0259 ~\mathrm{V}) \ln \frac{4.7 \times 10^{17}}{4.76 \times 10^{16}} \\
                  &= 0.059 ~\mathrm{V}.
       \end{align*}
       The barrier height is
       \begin{align*}
         \phi_{Bn} &= \phi_n + \psi_{bi} + \frac{kT}{q} + \Delta \phi 
       \end{align*}
       where the lowering potential $\Delta \phi$ is given by
       $$
       \Delta \phi = \sqrt{\frac{q E_m}{4 \pi \varepsilon_s}} 
                   = \sqrt{\frac{q^2 N_D W_D}{4 \pi \varepsilon_s^2}}
       $$
       and
       \begin{align*}
       W_{D} &= \sqrt{\frac{2 \varepsilon_s}{q N_D}\left(\psi_{bi} - \frac{kT}{q}\right)} \\
             &= \sqrt{\frac{2 (12.9 \cdot 8.85 \times 10^{-14} ~\mathrm{F}~\mathrm{cm}^{-1})}
                           {(1.6 \times 10^{-19} ~\mathrm{C})
                            (4.76 \times 10^{16} ~\mathrm{cm}^{-3})}
                           (0.6 - 0.0259 ~\mathrm{V})} \\
             &\approx .131 ~\mathrm{\mu m},
       \end{align*}
       so
       $$
       \Delta \phi = \sqrt{\frac{(1.6 \times 10^{-19} ~\mathrm{C})^2 
                                 (4.76 \times 10^{16} ~\mathrm{cm}^{-3})
                                 (1.31 \times 10^{-5}  ~\mathrm{cm})}
                                {4 \pi (12.9 \cdot 8.85 \times 10^{-14} ~\mathrm{F}~\mathrm{cm}^{-1})^2}} \approx 0.03 ~\mathrm{V}.
       $$
       Then 
       $$
       \phi_{Bn} \approx 0.059 + 0.6 + 0.0259 + 0.03 \approx 0.71 ~\mathrm{V}.
       $$
   }
   \item{The difference between this result and the effective barrier height 
         is $0.71 - 0.707 = 0.003 ~\mathrm{V}$, giving the lowering potential 
         $\Delta \phi$ under bias condition.
        }
\end{enumerate}


\end{document}
