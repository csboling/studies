\documentclass{article}

\title{ECE 875 - Homework \#1}
\author{Sam Boling}
\date{\today}

\usepackage{enumitem}
\usepackage{amsmath}
\usepackage{amsfonts}
\usepackage{amssymb}
\usepackage{graphicx}

\renewcommand*{\Re}{\operatorname{\mathfrak{Re}}}
\renewcommand*{\Im}{\operatorname{\mathfrak{Im}}}

\newcommand{\horline}
           {\begin{center}
              \noindent\rule{8cm}{0.4pt}
            \end{center}}

\newcommand\scalemath[2]{\scalebox{#1}{\mbox{\ensuremath{\displaystyle #2}}}}

\begin{document}

\maketitle

\section*{Problem 1.4}
\begin{enumerate}[label=(\alph*)]
  \item{Derive an expression for the bond length $d$ in the diamond lattice
        in terms of the lattice constant $a$.
       }
  \item{In a silicon crystal, if a plane has intercepts at 10.86 \r{A},
        16.29 \r{A}, and 21.72 \r{A} along the three Cartesian coordinates,
        find the Miller indices of the plane.
       }
\end{enumerate}

\horline

\begin{enumerate}[label=(\alph*)]
  \item{Note that the bond length is half the diagonal of a cube
        with corners defined by the atom in one corner of the 
        unit cell and the atoms on adjacent faces to that corner.
        The diagonal $x$ of one side of the cube is given by
        $$
        \left(\frac{a}{2}\right)^2 
      + \left(\frac{a}{2}\right)^2
      = x^2,
        $$ 
        and the diagonal $y$ of the cube is given by
        $$
        x^2 + \left(\frac{a}{2}\right)^2 = y^2,
        $$
        so that
        \begin{align*}
        d &= \frac{1}{2} y \\
          &= \frac{1}{2} \sqrt{3 \left(\frac{a}{2}\right)^2} \\
          &= \frac{\sqrt{3}}{4} a.
        \end{align*}
       }
  \item{Silicon has a lattice constant of $a \approxeq 5.431$ \r{A}, 
        so we note that these intercepts can be expressed as
        $(2a, 3a, 4a)$. Then taking the reciprocals of $(2,3,4)$ gives
        $(\frac{1}{2}, \frac{1}{3}, \frac{1}{4}$, and these denominators
        have a least common multiple of 12, so
        $(643)$ is the Miller index for the plane.
       }
\end{enumerate}

\pagebreak

\section*{Problem 1.5}
Show (a) that each vector of the reciprocal lattice is normal to a set of
planes in the direct lattice, and (b) the volume of a unit cell of the 
reciprocal lattice is inversely proportional to the volume of a unit cell
of the direct lattice.
\horline
\begin{enumerate}[label=(\alph*)]
  \item{
  Let $\mathbf{G}$ be a vector of the reciprocal lattice. Then
  \begin{align*}
     \mathbf{G} &= h \mathbf{a^\ast}
                 + k \mathbf{b^\ast} 
                 + l \mathbf{c^\ast} \\
                &= \frac{2\pi}{\mathbf{a} \bullet 
                               \mathbf{b} \times \mathbf{c}}
                     [h (\mathbf{b} \times \mathbf{c})
                    + k (\mathbf{c} \times \mathbf{a})
                    + l (\mathbf{a} \times \mathbf{b})].
  \end{align*}
%  First consider the case that h = k = l = 0. Then 
%  $\mathbf{G} = \mathbf{0}$, so every vector is orthogonal to 
%  $\mathbf{G}$. Then all planes in the direct lattice are 
%  orthogonal to $\mathbf{G}$.

  Let $\mathbf{R}$ be a vector in the direct lattice, so that
  $\mathbf{R} = q \mathbf{a} + r \mathbf{b} + s \mathbf{c}$ for
  some integers $q, r, s$. Then
  \begin{align*}
    \mathbf{G} \bullet \mathbf{R} 
      &= \frac{2\pi}{\mathbf{a} \bullet \mathbf{b} \times \mathbf{c}}
         \left[h(\mathbf{b} \times \mathbf{c}) \bullet \mathbf{R}
             + k(\mathbf{c} \times \mathbf{a}) \bullet \mathbf{R}
             + l(\mathbf{a} \times \mathbf{b}) \bullet \mathbf{R}\right] \\
      &= \frac{2\pi}{\mathbf{a} \bullet \mathbf{b} \times \mathbf{c}}
         \left[hq(\mathbf{a} \bullet \mathbf{b} \times \mathbf{c})
             + kr(\mathbf{b} \bullet \mathbf{c} \times \mathbf{a})
             + ls(\mathbf{c} \bullet \mathbf{a} \times \mathbf{b})\right],
  \end{align*}
  because $\mathbf{v} \bullet \mathbf{v} \times \mathbf{w} = 0$, 
  $\forall \mathbf{v}, \mathbf{w}$ since $\mathbf{v} \times \mathbf{w}$ is
  orthogonal to both $\mathbf{v}$ and $\mathbf{w}$. Then since
  $$
  \mathbf{a} \bullet \mathbf{b} \times \mathbf{c}
   = \mathbf{b} \bullet \mathbf{c} \times \mathbf{a}
   = \mathbf{c} \bullet \mathbf{a} \times \mathbf{b},
  $$
  we have
  $$
  \mathbf{G} \bullet \mathbf{R} = 2 \pi (hq + kr + ls) = 2 \pi n,
  $$
  for some integer $n$, for every $\mathbf{R}$ in the direct lattice.

  Consider a plane wave propagating with wavevector $\mathbf{G}$. Then its
  value at each point $\mathbf{r}$ in space is given by
  $$
  A(\mathbf{r}, t) 
    = \Re\{A_0 e^{i \mathbf{G} \bullet \mathbf{r}} e^{i(\varphi - \omega t)}\},
  $$
  so for any $\mathbf{R}$ in the direct lattice
  \begin{align*}
  A(\mathbf{R}, t)
    &= \Re\{A_0 e^{i \mathbf{G} \bullet \mathbf{R}} e^{i(\varphi - \omega t)}\} \\
    &= \Re\{A_0 e^{i(\varphi - \omega t)}\} \\
    &= A_0 \cos (\varphi - \omega t)
  \end{align*}
  since $e^{i 2 \pi n} = 1$ for any integer $n$.

  For fixed time $t$ this sinusoid has constant phase, so each point 
  $\mathbf{R}$ in the direct lattice sees the same phase. Therefore the direct 
  lattice points lie in surfaces of constant phase for $A(\mathbf{r}, t)$, and 
  since this is a plane wave such surfaces must be planes. Furthermore, these
  planes are orthogonal to the wavevector $\mathbf{G}$. Therefore for any 
  $\mathbf{G}$ in the reciprocal space there exists a set of planes in the 
  direct space orthogonal to $\mathbf{G}$. 

%  Thus any triple $(q,r,s)$ that satisfies this equation for the given
%  $h, k, l$ yields a vector 
%  $\mathbf{P} = q \mathbf{a} + r \mathbf{b} + s \mathbf{c}$ 
%  in the direct lattice that is orthogonal to $\mathbf{G}$.
%  Choose vectors in the direct lattice
%  \begin{align*}
%    \mathbf{R}_1 &= q_1 \mathbf{x} + r_1 \mathbf{y} + s_1 \mathbf{z},\\
%    \mathbf{R}_2 &= q_2 \mathbf{x} + r_2 \mathbf{y} + s_2 \mathbf{z}
%  \end{align*}
%  that meet this criterion, so that
%  \begin{align*}
%    h q_1 + k r_1 + l s_1 &= 0, \\
%    h q_2 + k r_2 + l s_2 &= 0.
%  \end{align*}
%  Since the proof is completed above for the case $h = k = l = 0$, let
%  $l \neq 0$. Then choose any $q_1 \neq 0$ and any $r_1$. The relations 
%  above give
%  $$
%  s_1 = -\frac{hq_1 + kr_1}{l}.
%  $$
%  Then choose $q_2 = mq_1$, $r_2 = nr_1$ for some integers $m, n$. Then
%  $$
%  s_2 = -\frac{hq_2 + kr_2}{l} = -\frac{hmq_1 + knr_1}{l}
%  $$
%  so that
%  \begin{align*}
%  \mathbf{R}_1 \times \mathbf{R}_2 &= \left|\begin{array}{c c c}
%    \mathbf{x} & \mathbf{y} & \mathbf{z} \\
%    q_1  & r_1  & -\frac{hq_1 + kr_1}{l} \\
%    mq_1 & nr_1 & -\frac{hmq_1 + knr_1}{l} 
%  \end{array}\right| \\
%   &= r_1\left(\frac{hnq_1 + knr_1 - hmq_1 - knr_1}{l}\right)\mathbf{x} \\
%   &+ q_1\left(\frac{hmq_1 + knr_1 - hmq_1 - kmr_1}{l}\right)\mathbf{y} \\
%   &+ q_1r_1(n - m)\mathbf{z} \\
%   &= (n - m) (hr_1\mathbf{x} + kq_1\mathbf{y} + q_1 r_1 \mathbf{z}).
%  \end{align*}
%  Therefore as long as $n \neq m$, 
%  $\mathbf{R}_1 \times \mathbf{R}_2 \neq \mathbf{0}$ and are therefore not
%  parallel, so they define a plane in the direct lattice which is orthogonal
%  to $\mathbf{G}$. The same argument may be applied for the cases where 
%  $h \neq 0$ and $k \neq 0$, so for any vector in the reciprocal lattice
%  there exist planes in the direct lattice which are orthogonal to it.
  }
  \item{
    \begin{align*}
      \mathbf{a^\ast} \bullet \mathbf{b^\ast} \times \mathbf{c^\ast} 
        &= \left(2 \pi \frac{\mathbf{b} \times \mathbf{c}}{V_c}\right) 
   \bullet \left(2 \pi \frac{\mathbf{c} \times \mathbf{a}}{V_c}\right)
    \times \left(2 \pi \frac{\mathbf{a} \times \mathbf{b}}{V_c}\right) \\
        &= \left(\frac{2 \pi}{V_c}\right)^3 
             (\mathbf{b} \times \mathbf{c})
             \bullet [(\mathbf{c} \times \mathbf{a}) 
                      \times (\mathbf{a} \times \mathbf{b})].
    \end{align*}
    But the bracketed vector quadruple product can be written as
    \begin{align*}
      (\mathbf{c} \times \mathbf{a}) 
       \times (\mathbf{a} \times \mathbf{b})
      &= (\mathbf{c} \bullet \mathbf{a} \times \mathbf{b})\mathbf{a} 
       - (\mathbf{c} \bullet \mathbf{a} \times \mathbf{a})\mathbf{b} = V_c \mathbf{a}
    \end{align*}
    since $\mathbf{a} \times \mathbf{a} = \mathbf{0}$ and 
    $$
    \mathbf{c} \bullet \mathbf{a} \times \mathbf{b} 
     = \mathbf{a} \bullet \mathbf{b} \times \mathbf{c} = V_c.
    $$
    Thus
    $$
    \mathbf{a^\ast} \bullet \mathbf{b^\ast} \times \mathbf{c^\ast} =
      \left(\frac{2 \pi}{V_c}\right)^3 V_c 
        \mathbf{a} \bullet \mathbf{b} \times \mathbf{c} 
    = \frac{(2\pi)^3}{V_c},
    $$
    so the volume of a primitive cell of the reciprocal lattice is 
    inversely proportional to the volume of a primitive cell of the
    reciprocal lattice.
    }
\end{enumerate}

\pagebreak


\section*{Problem 1.6}
Show that the reciprocal lattice of a body-centered cubic (bcc) lattice with
a lattice constant $a$ is a face-centered cubic (fcc) lattice with the side
of the cubic cell to be $\frac{4\pi}{a}$. 
[Hint: Use a symmetric set of vectors for bcc:
$$
\mathbf{a} = \frac{a}{2}(\mathbf{y} + \mathbf{z} - \mathbf{x}),
\mathbf{b} = \frac{a}{2}(\mathbf{z} + \mathbf{x} - \mathbf{y}),
\mathbf{c} = \frac{a}{2}(\mathbf{x} + \mathbf{y} - \mathbf{z}),
$$
where $a$ is the lattice constant of a conventional primitive cell, and
$\mathbf{x}$, $\mathbf{y}$, $\mathbf{z}$ are unity vectors of a Cartesian
coordinate. For fcc;
$$
\mathbf{a} = \frac{a}{2}(\mathbf{y} + \mathbf{z}),
\mathbf{b} = \frac{a}{2}(\mathbf{x} + \mathbf{z}),
\mathbf{c} = \frac{a}{2}(\mathbf{x} + \mathbf{y}).]
$$
\horline
First compute, for the bcc lattice,
\begin{align*}
  \mathbf{b} \times \mathbf{c} &=
  \left|\begin{array}{r r r}
    \mathbf{x} & \mathbf{y} & \mathbf{z} \\
     \frac{a}{2} & -\frac{a}{2} &  \frac{a}{2} \\
     \frac{a}{2} &  \frac{a}{2} & -\frac{a}{2}
  \end{array}\right| \\
&= \left(\frac{a^2}{4} - \frac{a^2}{4}\right) \mathbf{x} 
 + \left(\frac{a^2}{4} + \frac{a^2}{4}\right) \mathbf{y} 
 + \left(\frac{a^2}{4} + \frac{a^2}{4}\right) \mathbf{z} \\
&= \frac{a^2}{2} (\mathbf{y} + \mathbf{z}), \\
  \mathbf{c} \times \mathbf{a} &=
  \left|\begin{array}{r r r}
    \mathbf{x} & \mathbf{y} & \mathbf{z} \\
     \frac{a}{2} &  \frac{a}{2} & -\frac{a}{2} \\
    -\frac{a}{2} &  \frac{a}{2} &  \frac{a}{2} 
  \end{array}\right| 
= \frac{a^2}{2}(\mathbf{x} + \mathbf{z}), \\
  \mathbf{a} \times \mathbf{b} &=
  \left|\begin{array}{r r r}
    \mathbf{x} & \mathbf{y} & \mathbf{z} \\
    -\frac{a}{2} &  \frac{a}{2} &  \frac{a}{2} \\
     \frac{a}{2} & -\frac{a}{2} &  \frac{a}{2} 
  \end{array}\right| 
= \frac{a^2}{2}(\mathbf{x} + \mathbf{y}),
\end{align*}
and thus
$$
\mathbf{a} \bullet \mathbf{b} \times \mathbf{c} = \frac{a^3}{2}.
$$
Then
\begin{align*}
  \mathbf{a^\ast} &= \frac{2\pi}{\frac{a^3}{2}} \frac{a^2}{2} (\mathbf{y} + \mathbf{z}) \\
  &= \frac{2\pi}{a} (\mathbf{y} + \mathbf{z}) 
   = \frac{\frac{4\pi}{a}}{2} (\mathbf{y} + \mathbf{z}), \\
  \mathbf{b^\ast} &= \frac{\frac{4\pi}{a}}{2} (\mathbf{x} + \mathbf{z}), \\
  \mathbf{c^\ast} &= \frac{\frac{4\pi}{a}}{2} (\mathbf{x} + \mathbf{y}),
\end{align*}
constant $a$ is the basis for the direct lattice of an fcc structure with
lattice constant $\frac{4\pi}{a}$.

\end{document}
