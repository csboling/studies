\documentclass{article}

\title{ECE 863 - Homework \#3}
\author{Sam Boling}
\date{\today}

\usepackage{enumitem}
\usepackage{amsmath}
\usepackage{amsfonts}
\usepackage{amssymb}
\usepackage{graphicx}

\newcommand{\horline}
           {\begin{center}
              \noindent\rule{8cm}{0.4pt}
            \end{center}}

\begin{document}

\maketitle

\section*{Problem 10.1}
For any $k \in \{1, 2, \dots, i-1\}$ and any $\Delta \in \mathbb{Z}$,
we see that
\begin{align*}
  f_{\underline{x}[n_1] \cdots \underline{x}[n_k]}(x_1, \dots, x_k) &=
   \overbrace{\int_{-\infty}^{\infty} \cdots \int_{-\infty}^{\infty}}^{i-k}
     f_{\underline{x}[n_1]\cdots\underline{x}[n_i]}(x_1, \dots, x_i)
      ~dx_1 \cdots ~dx_i \\
&= \overbrace{\int_{-\infty}^{\infty} \cdots \int_{-\infty}^{\infty}}^{i-k}
     f_{\underline{x}[n_1 + \Delta]\cdots\underline{x}[n_i + \Delta]} 
     (x_1, \dots, x_i) ~dx_1 \cdots ~dx_i
\end{align*}
since $\underline{x}$ is stationary to order $i$. But
$$
 \overbrace{\int_{-\infty}^{\infty} \int_{-\infty}^{\infty}}^{i-k}
   f_{\underline{x}[n_1 + \Delta]\cdots\underline{x}[n_i + \Delta]} 
   (x_1, \dots, x_i) ~dx_1 \cdots ~dx_i
 = f_{\underline{x}[n_1 + \Delta]\cdots\underline{x}[n_k + \Delta]}
   (x_1, \dots, x_k),
$$
so $\underline{x}$ is stationary to order $k$ for any 
$k \in \{1, 2, \dots, i-1\}.$


\section*{Problem 10.3}
The properties for autocorrelation:
\begin{itemize}
  \item{
    $$
    r_{\underline{x}}[n_1, n_2] 
      = \int_{-\infty}^{\infty} \int_{-\infty}^{\infty}
        x_1 x_2 f_{\underline{x}[n_1]\underline{x}[n_2]}(x_1,x_2) 
        ~dx_1 ~dx_2.
    $$
    Since $\underline{x}$ is real-valued and since the range of
    $f_{\underline{x}[n_1]\underline{x}[n_2]}$ is $[0,1]$ the integrand is 
    real over the domain of integration and therefore the integral 
    $r_{\underline{x}}[n_1, n_2]$ is real if it exists.
  }
  \item
  {
  \begin{align*}
  r_{\underline{x}}[n_1, n_2] 
    &= \int_{-\infty}^{\infty}\int_{-\infty}^{\infty}
      x_1 x_1 f_{\underline{x}[n_1]\underline{x}[n_1]}(x_1, x_1) 
      ~dx_1 ~dx_2 \\ 
    &= \int_{-\infty}^{\infty}\int_{-\infty}^{\infty}
      x_1^2 f_{\underline{x}[n_1]\underline{x}[n_1]}(x_1, x_1) ~dx_1 ~dx_1,
  \end{align*}
  and since $\underline{x}$ is real, $(\underline{x}(\zeta))^2 \geq 0$ for 
  any $\zeta \in \mathcal{S}_{\underline{x}}$. Therefore the integrand is
  nonnegative over the whole domain of integration, so the integral 
  $r_{\underline{x}}[n_1, n_2]$ is 
  nonnegative if it exists.
  }
  \item
  {For a real random process $\underline{x}$, 
   $r_{\underline{x}}[n_1, n_2]$ is an inner product over a vector field 
   whose underlying field is $\mathbb{R}$ and is therefore symmetric.
  }
  \item{
  \begin{align*}
  |r_{\underline{x}}[n_1, n_2]| &= 
    \left|\int_{-\infty}^{\infty} \int_{-\infty}^{\infty}
      x_1 x_2 f_{\underline{x}[n_1] \underline{x}[n_2]}(x_1, x_2) 
       ~dx_1 ~dx_2 \right| \\ 
   &= \left|\int_{-\infty}^{\infty} 
      (x_1 \sqrt{f_{\underline{x}[n_1] \underline{x}[n_2]}(x_1, x_2)})
      (x_2 \sqrt{f_{\underline{x}[n_1] \underline{x}[n_2]}(x_1, x_2)}) 
      ~dx_1 ~dx_2\right|\\
   &\leq 
    \sqrt{\int_{-\infty}^{\infty}\int_{-\infty}^{\infty} 
    \left|x_1 
          \sqrt{f_{\underline{x}[n_1] \underline{x}[n_2]}(x_1, x_2)}
    \right|^2 
     ~dx_1 ~dx_2} \\
   &\cdot
    \sqrt{\int_{-\infty}^{\infty}\int_{-\infty}^{\infty}
    \left|x_2 
          \sqrt{f_{\underline{x}[n_1] \underline{x}[n_2]}(x_1, x_2)}
    \right|^2 
     ~dx_1 ~dx_2}
  \end{align*}
  by the Schwarz inequality, and since $\underline{x}$ is real this means
  \begin{align*}
  r[n_1, n_2] &\leq
    \sqrt{\int_{-\infty}^{\infty}\int_{-\infty}^{\infty} 
    x_1^2 f_{\underline{x}[n_1] \underline{x}[n_2]}(x_1, x_2)
     ~dx_1 ~dx_2} \\
    & \cdot
    \sqrt{\int_{-\infty}^{\infty}\int_{-\infty}^{\infty}
    x_2^2 f_{\underline{x}[n_1] \underline{x}[n_2]}(x_1, x_2)
     ~dx_1 ~dx_2} \\
   &= \sqrt{r_{\underline{x}}[n_1,n_1]r_{\underline{x}}[n_2,n_2]}.
  \end{align*}
  }
  \item{
  Writing $r_{\underline{x}}[n_1, n_2]$ as the inner product
  $\langle x[n_1], x[n_2] \rangle$, which is symmetric in the real case,
  we see that
  $$
     \langle x[n_1] + x[n_2], x[n_1] + x[n_2] \rangle 
   =  \langle x[n_1], x[n_1] \rangle  
   + 2\langle x[n_1], x[n_2] \rangle 
   +  \langle x[n_2], x[n_2] \rangle
  $$
  and therefore
  $$
     2r_{\underline{x}}[n_1, n_2] 
   = \|x[n_1] + x[n_2]\|^2 - \|x[n_1]\|^2 - \|x[n_2]\|^2
  $$
  so
  \begin{align*}
     2|r_{\underline{x}}[n_1, n_2]| 
  &=    \left|\|x[n_1] + x[n_2]\|^2 - \|x[n_1]\|^2 - \|x[n_2]\|^2\right| \\
  &=    \left|\|x[n_1]\|^2 + \|x[n_2]\|^2 - \|x[n_1] + x[n_2]\|^2\right| \\
  &\leq \left|\|x[n_1]\|^2 + \|x[n_2]\|^2\right| \\
  &=    r_{\underline{x}}[n_1, n_1] + r_{\underline{x}}[n_2, n_2]
  \end{align*}
  since $r_{\underline{x}}[n,n] \geq 0$. Therefore 
  $|r_{\underline{x}}[n_1, n_2] 
  \leq \frac{1}{2}\left[r_{\underline{x}}[n_1, n_1] 
                      + r_{\underline{x}}[n_2, n_2]\right]$.
  }
  \item
  {
    We have that
    $$
    |r_{\underline{x}}[n_1,n_2]| \leq 
    \frac{1}{2}[r_{\underline{x}}[n_1,n_1] + r_{\underline{x}}[n_2,n_2]],
    $$
    but if $\underline{x}$ is WSS, we may set $\eta = n_a - n_b$ 
    and write $r_{\underline{x}}[n_1,n_2] = r_{\underline{x}}[\eta]$,
    $r_{\underline{x}}[n_1,n_1] = r_{\underline{x}}[n_2,n_2] 
    = r_{\underline{x}}[0]$. Then 
    $$
    |r_{\underline{x}}[n_1,n_2]| \leq 
    r_{\underline{x}}[0]
    $$
    as desired.
  }
  \item
  {
    $$
    r_{\underline{x}}[\eta] = r_{\underline{x}}[n_1,n_2]
                            = r_{\underline{x}}[n_2,n_1]
                            = r_{\underline{x}}[-\eta]
    $$
    if $\underline{x}$ is real.
  }
\end{itemize}
It follows immediately that $c_{\underline{x}}[n_1,n_2]$ is real,
$c_{\underline{x}}[n_1,n_1] \geq -\mu_{\underline{x}}^2$, and
$c_{\underline{x}}[n_1,n_2] = c_{\underline[x]}{n_2,n_1}$. From the Schwarz
inequality as above, we see that
\begin{align*}
|c_{\underline{x}}[n_1,n_2]| &\leq 
  \sqrt{
   \langle x[n_1] - \mu_{\underline{x}}[n_1], 
           x[n_1] - \mu_{\underline{x}}[n_1]\rangle}
  \sqrt{
   \langle x[n_2] - \mu_{\underline{x}}[n_2], 
           x[n_2] - \mu_{\underline{x}}[n_2]\rangle}\\
   &= \sqrt{c[n_1,n_1]c[n_2,n_2]}
\end{align*}
and taking $\underline{x}[n] = \underline{y}[n] - \mu_{\underline{y}}[n]$ 
for an arbitrary random process $\underline{y}$ shows that
$$
|c_{\underline{x}}[n_1,n_2]| 
  \leq \frac{1}{2}[c_{\underline{x}}[n_1,n_1] 
                +  c_{\underline{x}}[n_2,n_2]]
$$
as well.

\section*{Problem 10.4}
We see that
\begin{align*}
c_{\underline{x}}[n_1, n_2] 
  &= \mathcal{E}\left\{
     (x[n_1] - \mu_{\underline{x}}[n_1])
     (x[n_2] - \mu_{\underline{x}}[n_2])\right\} \\
  &= \mathcal{E}\left\{
     x[n_1]x[n_2] 
   - \mu_{\underline{x}}[n_2]x[n_1]
   - \mu_{\underline{x}}[n_1]x[n_2]
   + \mu_{\underline{x}}[n_1]\mu_{\underline{x}}[n_2]\right\} \\
  &= \mathcal{E}\left\{x[n_1]x[n_2]\right\}
   - \mathcal{E}\left\{\mu_{\underline{x}}[n_2]x[n_1]\right\}
   - \mathcal{E}\left\{\mu_{\underline{x}}[n_1]x[n_2]\right\}
   + \mathcal{E}\left\{
       \mu_{\underline{x}}[n_1]\mu_{\underline{x}}[n_2]\right\}.
\end{align*}
But
$$
\mathcal{E}\left\{\mu_{\underline{x}}[n_1]x[n_2]\right\} =
\int_{-\infty}^{\infty} \mu_{\underline{x}}[n_1] 
                        x f_{\underline{x}[n_1]}(x) ~dx =
  \mu_{\underline{x}}[n_1]\mu_{\underline{x}}[n_2]
$$
and $\mathcal{E}\left\{\mu_{\underline{x}}[n_1]
                       \mu_{\underline{x}}[n_2]\right\}
     = \mu_{\underline{x}}[n_1]\mu_{\underline{x}}[n_2]$, so this means
$$
c_{\underline{x}}[n_1, n_2] 
 = r_{\underline{x}}[n_1,n_2]
 - \mu_{\underline{x}}[n_1]
   \mu_{\underline{x}}[n_2].
$$

\section*{Problem 10.5}
\begin{enumerate}
  \item{
    We see that
    $$
    \mathcal{E}\left\{\underline{\overline{m}}_x^{\mathcal{N}_+}\right\}
  = \mathcal{E}\left\{
      \frac{1}{|\mathcal{N}_+|}\sum_{n \in \mathcal{N}_+} x[n; \zeta]
    \right\}
  = \frac{1}{|\mathcal{N}_+|} \sum_{n \in \mathcal{N}_+} 
      \mathcal{E}\left\{x[n; \zeta]\right\} = \mu_{\underline{x}},
    $$
    so this is an unbiased estimator (because we have used the one-sided
    sample mean).
  }
  \item
  {
  If $n_a \leq 0$, this sample mean will include erroneous samples in the
  average, biasing the estimation unless the true expected mean
  $\mu_{\underline{x}}$ includes the non-positive indexed points as well.
  }
\end{enumerate}

\section*{Problem 10.6}
\begin{align*}
\mathcal{E}\left\{\underline{\overline{r}}_x[n;\zeta]\right\}
 &= \mathcal{E}\left\{\lim_{\mathcal{N} \to \pm \infty} 
  \frac{1}{|\mathcal{N}|}\sum_{n \in \mathcal{N}}
    x[n]x[n-\eta]\right\} \\
 &= \lim_{\mathcal{N} \to \pm \infty} \frac{1}{|\mathcal{N}|}
    \sum_{n \in \mathcal{N}} \mathcal{E}\left\{x[n]x[n-\eta]\right\} \\
 &= \lim_{\mathcal{N} \to \pm \infty} \frac{1}{|\mathcal{N}|}
    \sum_{n \in \mathcal{N}} r_{\underline{x}}[\eta] = 
    r_{\underline{x}}[\eta],
\end{align*}
so this is an unbiased estimator for the correlation. 

\section*{Problem 10.7}
Since utterances of the same syllable may have different durations,
their means may be different, so speech is not ergodic in the mean. 
Similarly, the process is not wide-sense stationary, and so we have not
defined what it means to be ergodic in the correlation.

\section*{Problem 10.8}
For a wide-sense stationary process, the mean is stationary so the 
autocorrelation and autocovariance are related by a constant. Since the
expectation is a linear operator, this means that if one of these statistics
is ergodic, so is the other.

If the process is not WSS then we have not defined what it means to be 
ergodic in the correlation or covariance.

\section*{Problem 10.9}
If a process is strict-sense stationary, then for any $N \in \mathbb{Z}_+$
it is stationary to order $N$. Therefore it is stationary to order 2.

If a process $\underline{x}$ is 2nd-order stationary, then 
$  f_{\underline{x}[n_1]\underline{x}[n_2]} 
 = f_{\underline{x}[n_1+\Delta]\underline{x}[n_2+\Delta]}$
for any $n_1, n_2, \Delta$, so 
$r_{\underline{x}}[n_1,n_2] = r_{\underline{x}}[n_1 + \Delta, n_2 + \Delta]$
and $\mu_{\underline{x}}[n_1] = \mu_{\underline{x}}[n_1 + \Delta]$ 
$\forall n_1, n_2, \Delta$. Therefore the process is wide-sense stationary.

Example 1028.1 provides a counterexample to the hypothesis that wide-sense
stationarity implies 2nd order stationarity.



\section*{Problem 10.10}
Trivially, if the random process is everywhere zero, it is WSS. A random
process that takes nonzero values for $n > 0$ is not wide-sense 
stationary, since its autocorrelation depends on both the spacing between
sample points and whether the sample points have nonnegative time indices.
In this case the process is still locally WSS for $n \leq 0$, but may or
may not be asymptotically WSS.

\section*{Problem 10.11}
\begin{enumerate}
  \item
  {
  \begin{align*}
  r_{\underline{y}}[n_1,n_2] &= \mathcal{E}\left\{
    (x[n_1] - x[n_1 - n_0])(x[n_2] - x[n_2 - n_0])\right\} \\
  &= \mathcal{E}\{
     x[n_1]x[n_2] - x[n_1]x[n_2-n_0] \\ 
  &- x[n_2]x[n_1-n_0] 
   + x[n_1-n_0]x[n_2-n_0]\} \\
  &= r_{\underline{x}}[n_1,n_2] - r_{\underline{x}}[n_1, n_2-n_0]\\
  &- r_{\underline{x}}[n_2,n_1-n_0] + r_{\underline{x}}[n_1-n_0,n_2-n_0].
  \end{align*}
  Since $\underline{x}$ is WSS, we let $\eta = n_1 - n_2$ and then see that
  this means
  $$
  r_{\underline{y}}[\eta] 
    = 2r_{\underline{x}} 
    - r_{\underline{x}}[n_0 + \eta]
    - r_{\underline{x}}[n_0 - \eta].
  $$
  Furthermore
  $$
  \mathcal{E}\left\{y[n]\right\} 
    = \mu_{\underline{x}}[n] - \mu_{\underline{x}}[n-n_0]
    = \mu_{\underline{x}}-\mu_{\underline{x}} = 0,
  $$
  so $\underline{y}$ is WSS.
  }
  \item
  {
  $$
  r_{\underline{z}}[n_1,n_2] 
    = \mathcal{E}\left\{x[n_1]x[n_2]x[n_1-n_0]x[n_2-n_0\right\},
  $$
  so $\underline{z}$ might be WSS. Note that this means
  \begin{align*}
  r_{\underline{z}}[n_1,n_2]
    = \int_{-\infty}^{\infty}\int_{-\infty}^{\infty}
      \int_{-\infty}^{\infty}\int_{-\infty}^{\infty} &
      x_1 x_{10} x_2 x_{20} \\
      &f_{\underline{x}[n_1]\underline{x}[n_1-n_0]
         \underline{x}[n_2]\underline{x}[n_2-n_0]}(x_1, x_{10}, x_2, x_{20})\\
   &  ~dx_1 ~dx_{10} ~dx_2 ~dx_{20},
  \end{align*}
  and if $\underline{x}$ is 4th-order stationary then
  $$
    f_{\underline{x}[n_1]\underline{x}[n_1-n_0]
       \underline{x}[n_2]\underline{x}[n_2-n_0]} =
    f_{\underline{x}[n_1]\underline{x}[n_1]
       \underline{x}[n_2]\underline{x}[n_2]} 
  $$
  and thus the correlation depends only on the spacing.
  }
\end{enumerate}

\end{document}
