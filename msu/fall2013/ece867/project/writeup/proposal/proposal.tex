\documentclass{paper}

\usepackage{natbib}

\title{Proposal -- Simulating and characterizing rate distortion of a compressed sensing architecture for multichannel neural recordings}
\author{Sam Boling}
\date{11/4/2013}

\begin{document}

\maketitle

Multichannel sensors for biological applications promise to provide a wealth of data for use in both basic science research and clinical engineering applications. In particular, modern manufacturing techniques have enabled construction of increasingly dense electrode arrays for monitoring extracellular neuron behavior in the cortex, improving the spatial resolution available to brain electrophysiologists and allowing new data-driven control techniques for smart prosthetic limbs \cite{hochberg2012}. However, the required bandwidth for such a data acquisition system grows rapidly with the number of electrode channels, and the power constraints of wearable computers therefore limit the number of neurons that can be recorded with a wireless system. For this reason, most extracellular recording equipment uses a cable connection, or tether, from the recording electrodes to a computer with abundant processing resources in order to maximize the number of channels \cite{nicolelis2001}. Tethering interferes with the subject's natural movement and is highly impractical in a long-term clinical setting -- ideally all measurement and processing would be performed by a portable, battery-powered, low-impact device. Achieving this goal will require novel approaches to signal extraction and computation.

The theory of compressed sensing (CS), also known as compressive sampling, suggests an alternative signal representation approach that can allow sampling at a fraction of the Nyquist rate in some cases. Since neural signals of interest are sparse in various wavelet bases \cite{charbiwala2011}, CS indicates that it is possible to accurately reconstruct the time signals of interest from a number of linear measurements per second that is much smaller than the limit given by the signal bandwidth and the sampling theorem. Many examinations of CS in an embedded context perform these projections after sampling with a conventional analog-to-digital converter and thus essentially treat CS as an approach to lossy compression in order to conserve transmission bandwidth. However, some researchers have proposed circuits for performing CS measurements in the analog front end for both single- and multi-channel systems, conserving power and reducing the computational load on the digital subsystem, including approaches that enable expanding the number of channels without affecting the sampling rate \cite{kirolos2006, slavinsky2011}.

I intend to simulate the operation of a CS-based multichannel neural recording device that performs compressive measurements in analog and transmits digital measurements for reconstruction by a remote machine with abundant computational resources.  Following reconstruction, I will use quantitative approaches to examine the rate distortion behavior of the system with the sampling rate, digital quantization, and number of channels as variables. I will use simulated extracellular recording data provided by the authors of \cite{quiroga2004} and create a simulation in Python of  the data collection node, using a multichannel analog scheme described in \cite{slavinsky2011}. A variety of open-source solvers exist to perform reconstruction on the compressively sensed data, such as SPAMS (http://spams-devel.gforge.inria.fr/). I will develop code to compute various information measures between the initial and reconstructed data, and ultimately attempt to use the Blahut-Arimoto algorithm to iteratively find rate distortion functions for the procedure. I hope to demonstrate the viability of this sensing approach for lowering the power and bandwidth requirements of an embedded neural data acquisition system and to quantify its tradeoffs using tools from information theory.

\bibliographystyle{plain}
\bibliography{references}

\end{document}
